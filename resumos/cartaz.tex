Nesta tese, cinco problemas em teoria dos grafos foram estudados do ponto de vista da complexidade computacional.

Foram investigadas três variações  do problema de coloração de vértices: coloração equilibrada, clique e biclique coloração. São descritos diversos algoritmos exatos exponenciais, parâmetrizados e provas de W[1]-dificuldade, que generalizam resultados da literatura através de técnicas consideravelmente mais simples.

Uma generalização natural do problema corte emparelhado - com aplicações em design the redes tolerantes a falhas e imersão de grafos e superfícies - foi definida e estudada.
São apresentados resultados de NP-dificuldade, algoritmos polinomiais, exatos exponenciais, parametrizados, e kernelização.

Por fim, grafos estrela foram introduzidos e caracterizados. São discutidas relações com grafos quadrados, condições necessárias para pertencimento na classe, cotas no tamanho de pre-imagens minimais e uma prova de pertencimento em NP.