\chapter{Equitable, Clique, and Biclique coloring}
\label{ch:coloring}

In a coloring problem, the goal is to partition (i.e. color) the vertices (or edges) of a graph such that each set (color class) of the partition, individually, satisfies some condition.
For the classical \pname{Vertex Coloring} problem, the goal is to partition the graph's vertex set such that, inside each member of the partition there are no two adjacent vertices.
Multiple additional constraints or properties may be added to the desired partition.
By further imposing that the size of each partition member be as close as possible to each other, the \pname{Equitable Coloring} problem, which appears to be much harder to solve even for graph classes where classical vertex coloring is efficiently solvable, is generated.
Another possible modification to \pname{Vertex Coloring} generates the b\pname{-Coloring} problem~\citep{b_coloring}, where a coloring of the vertices such that each color class has at least one vertex with one neighbor in each of the other classes is sought.
Much like \pname{Equitable Coloring} appears to be considerably harder than \pname{Vertex Coloring}, \pname{List Coloring} also exhibits a similar behavior; in this problem each vertex has a list of admissible colors, and the goal is to color the graph respecting these restrictions.
\pname{List Assignment}, however, takes things to a whole different level.
It asks if for a given graph, for every possible choice of list with exactly $k$ colors to each vertex of the graph, it is possible to find a list-coloring.
In fact, this coloring version is not even \NPc\, being $\Pi_2^p$-$\Complete$ even for bipartite graphs~\citep{choosability_complexity}.
One may also impose the constraint that no maximal induced subgraph be entirely contained in a single set of the partition.
For example, it may be required that no maximal clique, biclique (complete bipartite graph), or star of the given graph may be monochromatic generating the problems known as \pname{Clique Coloring},  \pname{Biclique Coloring}, and \pname{Star Coloring}, respectively.

In this chapter, we present results concerning the \pname{Equitable, Clique} and \pname{Biclique Coloring} problems.
We first formalize of many of the concepts we use in our proofs, as well as present some related work on each of the problems and a brief discussion on \pname{Vertex Coloring}.
We then proceed in earnest to our results.
For \pname{Equitable Coloring}, our first results are \W[1]-\Hness\ proofs for some subclasses of chordal graphs; namely, for block graphs of bounded diameter when parameterized by treewidth and maximum number of colors, for $K_{1,4}$-free interval graphs when parameterized by treewidth, maximum number of colors and maximum degree, and for disjoint union of split graphs (which are also complete multipartite) when parameterized by treewidth and maximum number of colors.
We close the subject of \pname{Equitable Coloring} with some algorithms.
We show that the problem admits an \XP\ algorithm for chordal graphs when parameterized by the maximum number of colors, a constructive polynomial time algorithm to equitably color claw-free chordal graphs, and an \FPT\ algorithm parameterized by the treewidth of the complement graph.
We then turn to \pname{Clique Coloring} and \pname{Biclique Coloring}.
The first exact exponential time algorithm for biclique coloring, which builds upon ideas used for clique coloring, is presented.
Afterwards, we give kernelization algorithms for both problems when parameterized by neighborhood diversity; using results on covering problems, an \FPT\ algorithm under the same parameterization is obtained for \pname{Clique Coloring}, which has optimal running time, up to the base of the exponent, unless the Exponential Time Hypothesis fails.
For \pname{Biclique Coloring}, an \FPT\ algorithm is given, but when parameterized by maximum number of colors and neighborhood diversity.

\section{Definitions and Related Work}

A \tdef{k-coloring} of a graph $G$ is a $k$-partition $\varphi = \{\varphi_1, \dots,\varphi_k\}$ of $V(G)$.
Each $\varphi_i$ is a \tdef{color class} and $v \in V(G)$ is \tdef{colored} with color $i$ if and only if $v \in \varphi_i$.
In a slight abuse of notation, we use $\varphi(v)$ to denote the color of $v$ and, for $X \subseteq V(G)$, $\varphi(X) = \bigcup_{v \in X} \{\varphi(v)\}$.

\subsection{Proper Coloring}

A \tdef{proper k-coloring} of $G$ is a $k$-coloring such that each $\varphi_i$ is an independent set.
In the literature, proper coloring is usually referenced to as \tdef{Vertex Coloring}, a convention we also adopt.
If $G$ has a proper $k$-coloring we say that $G$ is \tdef{k-colorable}.
The smallest integer $k$ such that $G$ is $k$-colorable is called the \tdef{chromatic number} $\chi(G)$ of $G$.
The natural decision problem associated with vertex coloring simply asks whether or not a given graph is $k$-colorable.

\problem{Vertex Coloring}{A graph $G$ and a positive integer $k$.}{Is $G$ $k$-colorable?}


\begin{figure}[!htb]
    \centering
    \begin{tikzpicture}[rotate=90,scale=0.6]
            %\draw[help lines] (-5,-5) grid (5,5);
            \GraphInit[vstyle=Simple]
            \SetVertexSimple[Shape=circle, FillColor=black, MinSize=2pt]
            \tikzset{VertexStyle/.append style = {inner sep = \inners, outer sep = \outers}}
            \SetUpVertex[Ldist=3pt]
            \grWheel[prefix=a]{6}
            \AssignVertexLabel{a}{1,2,1,2,3,4}
        \end{tikzpicture}
    \caption{An optimal proper coloring.}
    \label{fig:prop_color}
\end{figure}

Determining if a given instance of \pname{Vertex Coloring} is a $\YES$ instance is a classic problem in both graph theory and algorithmic complexity, being a known $\NPc$ problem.
Some particular cases of \pname{Vertex Coloring} are still $\NPc$.
For instance, even if we fix $k=3$ or restrict the input to $K_3$-free graphs the problem does not get any easier.

It is worth to point out the subtle difference between the parameter $k$ being part of the input or being \tdef{fixed}.
Informally, when $k$ is fixed, we are willing to pay exponential time only on $k$ to solve our problem, whereas when $k$ is part of the input, we are not.
Note that when we fix $k$ and find an $f(k)n^{\bigO{1}}$ time algorithm, we show that the problem is in $\FPT$ when parameterized by $k$.
The fact that $3$-coloring is $\NPc$ is evidence that \pname{Vertex Coloring} parameterized by the number of colors is not in $\FPT$, otherwise we would have an $f(3)n^{\bigO{1}}$ algorithm, which would imply that $\P = \NP$. 

For an unconstrained input, \pname{Vertex Coloring} is hard to approximate to a factor of $n^{1-\epsilon}$, for any $\epsilon > 0$, unless some complexity hypothesis fail (see~\citep{color_zpp} for more on the topic).
On a brighter note, a celebrated theorem due to Brooks in~\citep{brooks_theorem} gives a nice upper bound for general graphs, and gives a natural direction for research on tighter upper bounds on graph classes.

\begin{theorem*}[Brooks' Theorem]
    For every connected graph $G$ which is neither complete nor an odd-cycle $\chi(G) \leq \Delta(G)$.
\end{theorem*}

These results motivated much of the research about \pname{Vertex Coloring}.
There are polynomial time algorithms for a myriad of different classes, including chordal, bipartite and cographs.
More generally, there are known polynomial time algorithms for \tdef{perfect graphs}~\citep{perfect_polynomial}, which is a superclass of the aforementioned ones.
$G$ is perfect if for every induced subgraph $G'$ of $G$, $\chi(G') = \omega(G')$.

More particular cases for \pname{Vertex Coloring} have also been analyzed. For instance, \cite{coloring_art} present some results for graph classes that have two connected five-vertex forbidden induced subgraphs. There are some surveys on the subject, as such we point to~\citep{coloring_survey} and~\citep{coloring_survey2} for more on the classic \pname{Vertex Coloring} problem, since it is not the focus of this thesis.


\subsection{Equitable Coloring}
A $k$-coloring of an $n$ vertex graph is said to be \tdef{equitable} if for every color class $\varphi_i$, $\floor{\frac{n}{k}} \leq |\varphi_i| \leq \ceil{\frac{n}{k}}$ or, equivalently, if for, any two color class $\varphi_i$ and $\varphi_j$, $||\varphi_i| - |\varphi_j|| \leq 1$.
If $G$ admits a proper equitable $k$-coloring, we say that $G$ is \tdef{equitably k-colorable}.
Unlike other coloring variants previously discussed, an equitably $k$-colorable graph is not necessarily equitably $(k+1)$-colorable.

As such, two different parameters are defined: the smallest integer $k$ such that $G$ is equitably $k$-colorable is called the \tdef{equitable chromatic number} $\CN{=}(G)$; the smallest integer $k'$ such that $G$ is equitably $k$-colorable for every $k \geq k'$ is the \textit{equitable chromatic threshold} $\CN{=}^*(G)$ of $G$.

As with the previous coloring problems, we define the \pname{Equitable Coloring} decision problem.

\problem{Equitable Coloring}{A graph $G$ and a positive integer $k$.}{Is $G$ equitably $k$-colorable?}


\begin{figure}[!htb]
    \centering
    \begin{tikzpicture}
        \begin{scope}[rotate=90,scale=0.6]
            %\draw[help lines] (-5,-5) grid (5,5);
            \GraphInit[vstyle=Normal]
            \SetVertexNormal[Shape=circle]
            \tikzset{VertexStyle/.append style = {inner sep = \inners, outer sep = \outers}}
            \SetVertexNoLabel
            \grStar[prefix=a]{6}
            \AssignVertexLabel{a}{2,2,2,2,2,1}
        \end{scope}
    \end{tikzpicture}
    \hfill
    \begin{tikzpicture}
        \begin{scope}[rotate=90,scale=0.6]
            %\draw[help lines] (-5,-5) grid (5,5);
            \GraphInit[vstyle=Normal]
            \SetVertexNormal[Shape=circle]
            \tikzset{VertexStyle/.append style = {inner sep = \inners, outer sep = \outers}}
            \SetVertexNoLabel
            \grStar[prefix=b]{6}
            \AssignVertexLabel{b}{3,3,4,2,2,1}
        \end{scope}
    \end{tikzpicture}
    \hfill
    
    \caption{A proper non-equitable coloring (left) and an equitable coloring (right).}
    \label{fig:eq_color}
\end{figure}


\pname{Equitable Coloring} was first discussed by~\citep{first_equitable}, with an intended application for municipal garbage collection, and later in processor task scheduling~\citep{mutual_exclusion_scheduling} and server load balancing~\citep{domain_decomposition}.

Much of the work done over \pname{Equitable Coloring} aims to prove an analogue of Brooks' theorem, known as the \tdef{Equitable coloring conjecture} (ECC).
In terms of the equitable chromatic threshold, however, we have the \tdef{Hajnal-Szemerédi theorem}~\citep{hajnal_szmeredi_theorem}.

\begin{conjecture*}[ECC]
    For every connected graph $G$ which is neither a complete graph nor an odd-hole, $\CN{=}(G) \leq \Delta(G)$.
\end{conjecture*}

\begin{theorem*}[Hajnal-Szmerédi Theorem]
    Any graph $G$ is equitably $k$-colorable if $k \geq \Delta(G) + 1$. Equivalently, $\CN{=}^*(G) \leq \Delta(G) + 1$.
\end{theorem*}

\cite{e_delta_cc} suggest that a stronger result than the Hajnal-Szmerédi theorem may be achievable, presenting some classes where the \tdef{Equitable $\Delta$-coloring conjecture} (E$\Delta$CC) holds.
Moreover, they prove that if E$\Delta$CC holds for every regular graph, then it holds for every graph.

\begin{conjecture*}[E$\Delta$CC]
    For every connected graph $G$ which is not a complete graph, an odd-hole nor $K_{2n+1, 2n+1}$, for any $n \geq 1$, $\CN{=}^*(G) \leq \Delta(G)$ holds.
\end{conjecture*}

Quite a lot of effort was put into finding classes where E$\Delta$CC holds, even with the knowledge that only proofs for regular graphs are required.
A result given by~\cite{claw_free_de_werra}, combined with Brooks' Theorem, implies that every claw-free graph is equitably $k$-colorable for every $k \geq \chi(G)$.
A very extensive survey on the subject was conducted by~\cite{equitable_survey}, where many of the results of the past 50 years were assembled.
Among the many reported results, the E$\Delta$CC is known to hold for:
bipartite graphs (with the obvious exceptions, where the ECC holds),
trees,
split graphs,
planar graphs,
\tdef{outerplanar graphs} (planar graphs with a drawing such that no vertex is within a polygon formed by other vertices),
\tdef{low degeneracy graphs} (graphs such that every subgraph has a vertex with a small degree),
\tdef{Kneser graphs} (complement of the intersection graph of $F \subset 2^{[n]}$, with every set of $F$ containing exactly $k$ elements),
interval graphs,
random graphs and
some forms of graph products.
For the exact results please refer to the survey.

Almost all complexity results for \pname{Equitable Coloring} arise from a related problem, known as \pname{Bounded Coloring}, an observation given by~\cite{equitable_treewidth}.
A $k$-coloring is said to be \tdef{$l$-bounded} if for every color class $\varphi_i$, $|\varphi_i| \leq l$.
$G$ is \tdef{l-bounded k-colorable} if it admits an $l$-bounded $k$-coloring.

\problem{Bounded Coloring}{A graph $G$ and two positive integers $l$ and $k$.}{Is $G$ $l$-bounded $k$-colorable?}

\begin{observation*}
    A Graph $G$ with $n$ vertices is $l$-bounded $k$-colorable if and only if $G' = G \cup \overline{K_{lk - n}}$ is equitably $k$-colorable.
\end{observation*}


In terms of computational complexity, however, neither problem was nearly as explored as \pname{Vertex Coloring}.
Among the complexity results for \pname{Bounded Coloring} and, consequently, \pname{Equitable Coloring}, we have polynomial time solvability for split graphs~\citep{equitable_split}, complement of interval graphs~\citep{graph_partitioning1}, forests~\citep{mutual_exclusion_scheduling}, trees~\citep{equitable_trees} and complements of bipartite graphs~\citep{graph_partitioning1}.

For cographs, we have a polynomial time algorithm when $k$ is fixed, otherwise the problem is $\NPc$~\citep{graph_partitioning1}, a situation similar to that of bipartite and interval graphs~\citep{graph_partitioning1}.
A consequence of the hardness result for cographs is that \pname{Equitable Coloring} is $\NPc$ for graphs of bounded cliquewidth.

On complements of \tdef{comparability graphs} (graphs representing a valid partial ordering) however, even if we fix $l$, \pname{Bounded Coloring} is still $\NPc$~\citep{chain_antichain}.
\cite{colorful_treewidth} show that, when parameterized by treewidth, \pname{Equitable Coloring} is $\W[1]\text{-}\Hard$.
Also in terms of treewidth, \cite{equitable_treewidth} give a polynomial time algorithm for graphs of bounded treewidth.
Note that for all of the mentioned classes, \pname{Vertex Coloring} is polynomially solvable.

A summary of the known complexities is available in Table~\ref{tab:equitable_complexity}.


\begin{table}[!htb]
    \centering
    \begin{tabular}{l|l|l}
       \hline
       \hline
       Class               &  fixed $k$         & input $k$             \\
       \hline
       Trees                &  $\P$             & $\P$                  \\
       Forests              &  $\P$             & $\P$                  \\
       Bipartite            &  $\NPc$           & $\NPc$                \\
       Co-bipartite         &  $\P$             & $\P$                  \\
       Cographs             &  $\P$             & $\NPc$                \\
       Bounded Cliquewidth  &  $\NPc$                & $\NPc$                \\
       Bounded Treewidth    &  $\P$             & $\P$                  \\
       Chordal              &  $\P^*$           & $\NPc$                \\
       Block                &  $\P^*$           & $\NPc^*$              \\
       Split                &  $\P$             & $\P$                  \\
       Interval             &  $\P$             & $\NPc$                \\
       Co-interval          &  $\P$             & $\P$                  \\
       General case         &  $\NPc$           & $\NPc$                \\
       \hline
       \hline
    \end{tabular}
    \caption{Complexity results for \pname{Equitable Coloring}. Entries marked with a * are results established in this work.}
    \label{tab:equitable_complexity}
\end{table}

\subsection{Clique Coloring}
A \tdef{k-clique-coloring} of $G$ is a $k$-coloring of $G$ such that no maximal clique of $G$ is entirely contained in a single color class.
We say that $G$ is \tdef{k-clique-colorable} if $G$ admits a $k$-clique-coloring.
The smallest integer $k$ such that $G$ is $k$-clique-colorable is called the \tdef{clique chromatic number} $\CN{C}(G)$ of $G$.
Much like \pname{Vertex Coloring}, there is a natural decision problem associated with this coloring variant, which we refer to as \pname{Clique Coloring}.


\problem{Clique Coloring}{A graph $G$ and a positive integer $k$.}{Is $G$ $k$-clique-colorable?}


\begin{figure}[!htb]
    \centering
    \begin{tikzpicture}[rotate=90,scale=0.6]
            %\draw[help lines] (-5,-5) grid (5,5);
            \GraphInit[vstyle=Simple]
            \SetVertexSimple[Shape=circle, FillColor=black, MinSize=2pt]
            \tikzset{VertexStyle/.append style = {inner sep = \inners, outer sep = \outers}}
            \SetUpVertex[Ldist=3pt]
            \grWheel[prefix=a]{6}
            \AssignVertexLabel{a}{1,1,1,1,1,2}
        \end{tikzpicture}
    \caption{An optimal clique coloring.}
    \label{fig:clique_color}
\end{figure}

Research on the topic is much more recent than what was done for \pname{Vertex Coloring}, with the first papers appearing in the early 1990s~\citep{first_clique_color} and interest on the subject rising in the early 2000s.
Even when $k$ is fixed, \pname{Clique Coloring} is known to be $\SiP{2}\text{-}\Complete$, as shown by~\cite{clique_coloring_complexity}, with an $\bigOs{2^n}$ algorithm being proposed by~\cite{clique_color_algorithm}.

As with \pname{Vertex Coloring}, \pname{Clique Coloring} has been studied when restricting the input graph to certain graph classes.
\cite{weakly_clique_color} investigate 2-clique-coloring in terms of \tdef{weakly chordal graphs} (graphs free of any hole or anti-hole with more than 4 vertices), giving a series of results for the general case ($\SiP{2}\text{-}\Complete$) and showing that, for some nested subclasses, there are $\NPc$ and $\P$ instances of the problem.
When dealing with \tdef{unichord-free} graphs (graphs that contain no induced cycle with a unique chord), the problem is solvable in polynomial time~\citep{unichord_coloring}.

\tdef{Circular-arc} graphs (intersection graphs of a set of arcs of a circle) are always 3-clique-colorable, with a polynomial time algorithm to determine if the input is 2-clique-colorable~\citep{clique_circular_arc}.
When the given graph is \tdef{odd-hole-free}, it is $\SiP{2}\text{-}\Complete$ to decide whether it is 2-clique-colorable or not~\citep{clique_oddhole}.
\cite{clique_coloring_few_p4} give a series of bounds on graphs that, in some sense, contain few $P_4$'s, showing that most them are either 2 or 3-clique-colorable.

For \tdef{planar graphs} (graphs that can be drawn on a plane with no crossing edges), \cite{clique_coloring_planar} show that they are 3-clique-colorable, and \cite{clique_color_perfect_np_complete} present a polynomial time algorithm to decide whether a planar graph is 2-clique-colorable or not.

Some of these classes are subclasses of perfect graphs, and a conjecture suggests that every perfect graph is 3-clique-colorable~\citep{maximal_clique_coloring}.
Also in terms of perfect graphs, it is $\NPc$ to decide whether a perfect graph is 2-clique-colorable~\citep{clique_color_perfect_np_complete}.
\cite{clique_oddhole} also give the observation that every \tdef{strongly perfect graph}~\citep{strongly_perfect} is 2-clique-colorable, a superclass of both chordal graphs and cographs.
For a summary of the mentioned results, please refer to Table~\ref{tab:clique_color_complexity}.

\begin{table}[!htb]
    \centering
    \begin{tabular}{c|c|c}
        \hline
        \hline
        Class            & $\CN{C}$             & Complexity \\
        \hline
        Cograph          & $= 2$                & $\P$\\
        Chordal          & $= 2$                & $\P$\\
        Weakly Chordal   & $\leq 3^*$           & $\SiP{2}\text{-}\Complete^\dag$\\
        Unichord-free    & $\leq 3 $            & $\P$ \\
        Circular-arc     & $\leq 3$             & $\P^\dag$ \\
        Odd-hole-free    & $\leq 3^*$           & $\SiP{2}\text{-}\Complete^\dag$\\
        Few $P_4$'s      & $\leq 2$ or $\leq 3$ & $\P$\\
        Planar           & $\leq 3$             & $\P^\dag$\\
        Perfect          & $\leq 3^*$           & $\NPc^\dag$\\ 
        Strongly Perfect & $= 2$                & $\P$\\
        General case     & Unbounded            & $\SiP{2}\text{-}\Complete$\\
        \hline
        \hline
    \end{tabular}
    \caption{Complexity and bounds for \pname{Clique Coloring}. Entries marked with a $*$ are conjectures. $\dag$ indicates results for 2-clique-colorability.}
    \label{tab:clique_color_complexity}
\end{table}

\subsection{Biclique Coloring}
A \tdef{k-biclique-coloring} of $G$ is a $k$-coloring of $G$ such that no maximal biclique of $G$ is entirely contained in a single color class.
We say that $G$ is \tdef{k-biclique-colorable} if $G$ admits a $k$-biclique-coloring.
The smallest integer $k$ such that $G$ is $k$-biclique-colorable is called the \tdef{biclique chromatic number} $\CN{B}(G)$ of $G$.
Much like \pname{Clique Coloring}, there is a natural decision problem associated with this coloring variant, which we refer to as \pname{Biclique Coloring}.

\problem{Biclique Coloring}{A graph $G$ and a positive integer $k$.}{Is $G$ $k$-biclique-colorable?}


\begin{figure}[!htb]
    \centering
    \begin{tikzpicture}[rotate=90,scale=0.6]
            %\draw[help lines] (-5,-5) grid (5,5);
            \GraphInit[vstyle=Simple]
            \SetVertexSimple[Shape=circle, FillColor=black, MinSize=2pt]
            \tikzset{VertexStyle/.append style = {inner sep = \inners, outer sep = \outers}}
            \SetUpVertex[Ldist=3pt]
            \grWheel[prefix=a]{6}
            \AssignVertexLabel{a}{1,2,1,2,3,3}
        \end{tikzpicture}
    \caption{An optimal biclique coloring.}
    \label{fig:biclique_color}
\end{figure}


\pname{Biclique Coloring} is an even more recent research topic than \pname{Clique Coloring}, with the first results being a $\SiP{2}\text{-}\Cness$ proof due to~\cite{biclique_coloring_complexity} and the confirmation that verifying a solution to the problem is a $\coNP\text{-}\Complete$ task~\citep{biclique_coloring_verification}.

In terms of complexity results, very little is known about \pname{Biclique Coloring}.
For unichord-free graphs, \cite{unichord_coloring} give a polynomial time algorithm to compute $\CN{B}$ and show that the biclique chromatic number of unichord-free graphs is either equal to or one greater than the size of the largest true twin class.
\cite{biclique_coloring_verification} present a polynomial time algorithm for powers of cycles and powers of paths.
Finally, \cite{biclique_coloring_complexity} give complexity results for $H$-free graphs, for every $H$ on three vertices, being polynomial for $H \in \{K_3, P_3, \overline{P_3}\}$ and $\NPc$ for $\overline{K_3}$-free graphs;
moreover they show that the problem is $\NPc$ for diamond ($C_4$ plus one chord) free graphs and split graphs, and polynomial for \tdef{threshold graphs} (\{$2K_2,P_4,C_4$\}-free).
We summarize the presented results in Table~\ref{tab:biclique_color_complexity}.

\begin{table}[!htb]
    \centering
    \begin{tabular}{c|c|c}
        \hline
        \hline
        Class            & $\CN{B}$             & Complexity \\
        \hline
        Split            & Unbounded            & $\NPc$\\
        Threshold        & Unbounded            & $\P$\\
        Diamond-free     & Unbounded            & $\NPc$\\
        $C_n^r$          & $\leq 3$             & $\P$\\
        $P_n^r$          & $= 2$                & $\P$\\
        Unichord-free    & Bounded              & $\P$\\
        General case     & Unbounded            & $\SiP{2}\text{-}\Complete$\\
        \hline
        \hline
    \end{tabular}
    \caption{Complexity and bounds for \pname{Biclique Coloring}.}
    \label{tab:biclique_color_complexity}
\end{table}

Both \pname{Clique Coloring} and \pname{Biclique Coloring} are, actually, colorings of the hypergraphs arising from an underlying graph (a coloring of its vertices such that no hyperedge is monochromatic), which is also an $\NPc$ task.
However, in classical hypergraph coloring problems, the hyperedge family is part of the input of the problem and, as such, naively verifying a solution is polynomial on the size of the input.
\section{Hardness of \pname{Equitable Coloring} for subclasses of Chordal Graphs}

\begin{figure}[!tb]
    \centering
    \begin{tikzpicture}[scale=\gscale]
        %\draw[help lines] (-5,-5) grid (5,5);
        \GraphInit[unit=3,vstyle=Normal]
        \SetVertexNormal[Shape=circle, FillColor=black, MinSize=2pt]
        \tikzset{VertexStyle/.append style = {inner sep = \inners, outer sep = \outers}}
        \SetVertexNoLabel
        \Vertex[x=0,y=0,L={y}, Lpos=270, LabelOut, Ldist=3pt]{y}
        \draw[] (0,0) circle (0.4cm);
        \Vertex[a=-5, d=1.7cm]{x11}
        \Vertex[a=-30, d=2.3cm]{x12}
        \Vertex[a=-55, d=1.7cm]{x13}
        
        \Vertex[a=115, d=1.7cm]{x21}
        \Vertex[a=90, d=2.3cm]{x22}
        \Vertex[a=65, d=1.7cm]{x23}
        
        \Vertex[a=235, d=1.7cm]{x31}
        \Vertex[a=210, d=2.3cm]{x32}
        \Vertex[a=185, d=1.7cm]{x33}
        
        \foreach \x in {1,2,3}
            \foreach \y in {1,2,3}
                \Edge(y)(x\x\y);
        
        \Edge(x11)(x12)
        \Edge(x11)(x13)
        \Edge(x12)(x13)
        
        
        \Edge(x21)(x22)
        \Edge(x21)(x23)
        \Edge(x22)(x23)
        
        
        \Edge(x31)(x32)
        \Edge(x31)(x33)
        \Edge(x32)(x33)
        
        
    \end{tikzpicture}
    \hfill
    \begin{tikzpicture}[scale=\gscale]
        \GraphInit[unit=3,vstyle=Normal]
        \SetVertexNormal[Shape=circle, FillColor=black, MinSize=2pt]
        \tikzset{VertexStyle/.append style = {inner sep = \inners, outer sep = \outers}}
        \SetVertexNoLabel
        \grComplete[RA=1]{3}
        \Vertex[a=60,d=1.5]{x}
        \Vertex[a=180,d=1.5]{y}
        \Vertex[a=300,d=1.5]{z}
        \draw[] (60:1.5) circle (0.4cm);
        \draw[] (180:1.5) circle (0.4cm);
        \draw[] (300:1.5) circle (0.4cm);
        
        \Edge(x)(a0)
        \Edge(x)(a1)
        \Edge(x)(a2)
        
        \Edge(y)(a0)
        \Edge(y)(a1)
        \Edge(y)(a2)
        
        \Edge(z)(a0)
        \Edge(z)(a1)
        \Edge(z)(a2)
        
    \end{tikzpicture}
    \hfill
    \begin{tikzpicture}[scale=\gscale]
        \GraphInit[unit=3,vstyle=Normal]
        \SetVertexNormal[Shape=circle, FillColor=black, MinSize=2pt]
        \tikzset{VertexStyle/.append style = {inner sep = \inners, outer sep = \outers}}
        \SetVertexNoLabel
        \Vertex[x=0,y=0]{y1}
        
        \draw[] (0,0) circle (0.4cm);
        
        \Vertex[x=3,y=0]{y2}
        
        \draw[] (3,0) circle (0.4cm);
        
        \begin{scope}[shift={(1.5cm,0)}, rotate=-90]
            \grComplete[RA=1, prefix=a]{2}
        \end{scope}
        
        \begin{scope}[shift={(0cm,-1.5cm)}]
            \grComplete[RA=1, prefix=c]{2}
        \end{scope}
        
        \begin{scope}[shift={(-1.5cm,0)}, rotate=-90]
            \grComplete[RA=1, prefix=b]{2}
        \end{scope}
        
        \begin{scope}[shift={(3cm,-1.5cm)}]
            \grComplete[RA=1, prefix=d]{2}
        \end{scope}
        
        \Edges(a0,y1,a1)
        \Edges(b0,y1,b1)
        \Edges(c0,y1,c1)
        
        \Edges(a0,y2,a1)
        \Edges(d0,y2,d1)
        
        
    \end{tikzpicture}
    \hfill
    
    \caption{A $(2,4)$-flower, a $(2,4)$-antiflower, and a $(2,2)$-trem.}
    \label{fig:flower}
\end{figure}

\begin{figure}[!tb]
    \centering
    \begin{tikzpicture}[scale=\gscale]
            %\draw[help lines] (-8,-8) grid (8,8);
        \begin{scope}[rotate=36,shift={(0cm, 5cm)}]
            \GraphInit[unit=3,vstyle=Normal]
            \SetVertexNormal[Shape=circle, FillColor=black, MinSize=2pt]
            \tikzset{VertexStyle/.append style = {inner sep = \inners, outer sep = \outers}}
            \SetVertexNoLabel
            \Vertex[Math, x=0,y=0,L={y_1}, Lpos=290, LabelOut, Ldist=3pt]{1y}
            \Vertex[a=-5, d=1.7cm]{1x11}
            \Vertex[a=-30, d=2.3cm]{1x12}
            \Vertex[a=-55, d=1.7cm]{1x13}
            
            \Vertex[a=115, d=1.7cm]{1x21}
            \Vertex[a=90, d=2.3cm]{1x22}
            \Vertex[a=65, d=1.7cm]{1x23}
            
            \Vertex[a=235, d=1.7cm]{1x31}
            \Vertex[a=210, d=2.3cm]{1x32}
            \Vertex[a=185, d=1.7cm]{1x33}
            
            \foreach \x in {1,2,3}
                \foreach \y in {1,2,3}
                    \Edge(1y)(1x\x\y);
            
            \Edge(1x11)(1x12)
            \Edge(1x11)(1x13)
            \Edge(1x12)(1x13)
            
            
            \Edge(1x21)(1x22)
            \Edge(1x21)(1x23)
            \Edge(1x22)(1x23)
            
            
            \Edge(1x31)(1x32)
            \Edge(1x31)(1x33)
            \Edge(1x32)(1x33)
        \end{scope}
        \begin{scope}[rotate=-36,shift={(0cm, 5cm)}]
            \GraphInit[unit=3,vstyle=Normal]
            \SetVertexNormal[Shape=circle, FillColor=black, MinSize=2pt]
            \tikzset{VertexStyle/.append style = {inner sep = \inners, outer sep = \outers}}
            \SetVertexNoLabel
            \Vertex[Math, x=0,y=0,L={y_1}, Lpos=290, LabelOut, Ldist=3pt]{2y}
            \Vertex[a=-5, d=1.7cm]{2x11}
            \Vertex[a=-30, d=2.3cm]{2x12}
            \Vertex[a=-55, d=1.7cm]{2x13}
            
            \Vertex[a=115, d=1.7cm]{2x21}
            \Vertex[a=90, d=2.3cm]{2x22}
            \Vertex[a=65, d=1.7cm]{2x23}
            
            \Vertex[a=235, d=1.7cm]{2x31}
            \Vertex[a=210, d=2.3cm]{2x32}
            \Vertex[a=185, d=1.7cm]{2x33}
            
            \foreach \x in {1,2,3}
                \foreach \y in {1,2,3}
                    \Edge(2y)(2x\x\y);
            
            \Edge(2x11)(2x12)
            \Edge(2x11)(2x13)
            \Edge(2x12)(2x13)
            
            
            \Edge(2x21)(2x22)
            \Edge(2x21)(2x23)
            \Edge(2x22)(2x23)
            
            
            \Edge(2x31)(2x32)
            \Edge(2x31)(2x33)
            \Edge(2x32)(2x33)
        \end{scope}
        \begin{scope}[rotate=-108,shift={(0cm, 5cm)}]
            \GraphInit[unit=3,vstyle=Normal]
            \SetVertexNormal[Shape=circle, FillColor=black, MinSize=2pt]
            \tikzset{VertexStyle/.append style = {inner sep = \inners, outer sep = \outers}}
            \SetVertexNoLabel
            \Vertex[Math, x=0,y=0,L={y_1}, Lpos=290, LabelOut, Ldist=3pt]{3y}
            \Vertex[a=-5, d=1.7cm]{3x11}
            \Vertex[a=-30, d=2.3cm]{3x12}
            \Vertex[a=-55, d=1.7cm]{3x13}
            
            \Vertex[a=115, d=1.7cm]{3x21}
            \Vertex[a=90, d=2.3cm]{3x22}
            \Vertex[a=65, d=1.7cm]{3x23}
            
            \Vertex[a=235, d=1.7cm]{3x31}
            \Vertex[a=210, d=2.3cm]{3x32}
            \Vertex[a=185, d=1.7cm]{3x33}
            
            \foreach \x in {1,2,3}
                \foreach \y in {1,2,3}
                    \Edge(3y)(3x\x\y);
            
            \Edge(3x11)(3x12)
            \Edge(3x11)(3x13)
            \Edge(3x12)(3x13)
            
            
            \Edge(3x21)(3x22)
            \Edge(3x21)(3x23)
            \Edge(3x22)(3x23)
            
            
            \Edge(3x31)(3x32)
            \Edge(3x31)(3x33)
            \Edge(3x32)(3x33)
        \end{scope}
        \begin{scope}[rotate=108,shift={(0cm, 5cm)}]
            \GraphInit[unit=3,vstyle=Normal]
            \SetVertexNormal[Shape=circle, FillColor=black, MinSize=2pt]
            \tikzset{VertexStyle/.append style = {inner sep = \inners, outer sep = \outers}}
            \SetVertexNoLabel
            \Vertex[Math, x=0,y=0,L={y_1}, Lpos=290, LabelOut, Ldist=3pt]{4y}
            \Vertex[a=-5, d=1.7cm]{4x11}
            \Vertex[a=-30, d=2.3cm]{4x12}
            \Vertex[a=-55, d=1.7cm]{4x13}
            
            \Vertex[a=115, d=1.7cm]{4x21}
            \Vertex[a=90, d=2.3cm]{4x22}
            \Vertex[a=65, d=1.7cm]{4x23}
            
            \Vertex[a=235, d=1.7cm]{4x31}
            \Vertex[a=210, d=2.3cm]{4x32}
            \Vertex[a=185, d=1.7cm]{4x33}
            
            \foreach \x in {1,2,3}
                \foreach \y in {1,2,3}
                    \Edge(4y)(4x\x\y);
            
            \Edge(4x11)(4x12)
            \Edge(4x11)(4x13)
            \Edge(4x12)(4x13)
            
            
            \Edge(4x21)(4x22)
            \Edge(4x21)(4x23)
            \Edge(4x22)(4x23)
            
            
            \Edge(4x31)(4x32)
            \Edge(4x31)(4x33)
            \Edge(4x32)(4x33)
        \end{scope}
        \begin{scope}[rotate=120,shift={(0cm, 0cm)}]
            \GraphInit[unit=3,vstyle=Normal]
            \SetVertexNormal[Shape=circle, FillColor=black, MinSize=2pt]
            \tikzset{VertexStyle/.append style = {inner sep = \inners, outer sep = \outers}}
            \SetVertexNoLabel
            \Vertex[Math, x=0,y=0,L={y_1}, Lpos=290, LabelOut, Ldist=3pt]{0y}
            \Vertex[a=-15, d=1.7cm]{0x11}
            \Vertex[a=-30, d=2.3cm]{0x12}
            \Vertex[a=-45, d=1.7cm]{0x13}
            
            \Vertex[a=-87, d=1.7cm]{0x21}
            \Vertex[a=-102, d=2.3cm]{0x22}
            \Vertex[a=-117, d=1.7cm]{0x23}
            
            \Vertex[a=-159, d=1.7cm]{0x31}
            \Vertex[a=-174, d=2.3cm]{0x32}
            \Vertex[a=-189, d=1.7cm]{0x33}
            
            \Vertex[a=-231, d=1.7cm]{0x41}
            \Vertex[a=-246, d=2.3cm]{0x42}
            \Vertex[a=-261, d=1.7cm]{0x43}
            
            \Vertex[a=-303, d=1.7cm]{0x51}
            \Vertex[a=-318, d=2.3cm]{0x52}
            \Vertex[a=-333, d=1.7cm]{0x53}
            
            
            \foreach \x in {1,2,3,4,5}
                \foreach \y in {1,2,3}
                    \Edge(0y)(0x\x\y);
                    
                    \Edge(0x11)(0x11)
            \Edge(0x11)(0x12)
            \Edge(0x11)(0x13)
            \Edge(0x12)(0x12)
            \Edge(0x12)(0x13)
            \Edge(0x13)(0x13)
            \Edge(0x21)(0x21)
            \Edge(0x21)(0x22)
            \Edge(0x21)(0x23)
            \Edge(0x22)(0x22)
            \Edge(0x22)(0x23)
            \Edge(0x23)(0x23)
            \Edge(0x31)(0x31)
            \Edge(0x31)(0x32)
            \Edge(0x31)(0x33)
            \Edge(0x32)(0x32)
            \Edge(0x32)(0x33)
            \Edge(0x33)(0x33)
            \Edge(0x41)(0x41)
            \Edge(0x41)(0x42)
            \Edge(0x41)(0x43)
            \Edge(0x42)(0x42)
            \Edge(0x42)(0x43)
            \Edge(0x43)(0x43)
            \Edge(0x51)(0x51)
            \Edge(0x51)(0x52)
            \Edge(0x51)(0x53)
            \Edge(0x52)(0x52)
            \Edge(0x52)(0x53)
            \Edge(0x53)(0x53)
            
        \end{scope}
        \Edge(0y)(1y)
        \Edge(0y)(2y)
        \Edge(0y)(3y)
        \Edge(0y)(4y)
        
        
    \end{tikzpicture}
    
    \caption{\textsc{equitable coloring} instance built on Theorem~\ref{thm:blocks} corresponding to the \pname{Bin Packing} instance $A = \{2,2,2,2\}$, $k=3$ and $B = 4$.}
    \label{fig:super_flower}
\end{figure}

All of our reductions involve the \pname{Bin Packing} problem, which is $\NPH$ in the strong sense~\citep{garey_johnson} and $\W[1]$-$\Hard$ when parameterized by the number of bins~\citep{bin_packing_w1}.
In the general case, the problem is defined as: given a set of positive integers $A = \{a_1, \dots, a_n\}$, called \textit{items}, and two integers $k$ and $B$, can we partition $A$ into $k$ \emph{bins} such that the sum of the elements of each bin is at most $B$?
We shall use a version of \pname{Bin Packing} where each bin sums \emph{exactly} to~$B$.
This second version is equivalent to the first, even from the parameterized point of view; it suffices to add~$kB - \sum_{j \in [n]} a_j$ unitary items to~$A$.
For simplicity, by \pname{Bin Packing} we shall refer to the second version, which we formalize as follows.

\pproblem{bin-packing}{A set of $n$ items $A$ and a bin capacity $B$.}{The number of bins $k$.}{Is there a $k$-partition $\varphi$ of $A$ such that, $\forall i \in [k]$, $\sum_{a_j \in \varphi_i} a_j = B$?}

The idea for the following reductions is to build one gadget for each item~$a_j$ of the given \pname{Bin Packing} instance, perform their disjoint union, and equitably $k$-color the resulting graph.
The color given to the circled vertices in Figure~\ref{fig:flower} control the bin to which the corresponding item belongs to.
Each reduction uses only one of the three gadget types.
Since every gadget is a chordal graph, their treewidth is precisely the size of the largest clique minus one, that is, $k$, which is also the number of desired colors for the built instance of \textsc{equitable coloring}.

\subsubsection{Disjoint union of Split Graphs}

\begin{definition}
    An $(a,k)$-antiflower is the graph $F_-(a,k) = K_{k-1} \oplus \left(\bigcup_{i \in [a+1]} K_1\right)$, that is, it is the graph obtained after performing the disjoint union of $a+1$ $K_1$'s followed by the join with $K_{k-1}$.
\end{definition}

\begin{theorem}
    \label{thm:dis_split}
    \textsc{equitable coloring} of the disjoint union of split graphs parameterized by the number of colors is $\W[1]$-$\Hard$.
\end{theorem}
\begin{proof}
    Let $\langle A,k,B\rangle$ be an instance of \pname{Bin Packing} and $G$ a graph such that~$G = \bigcup_{j \in [n]} F_-(a_j,k)$.
    Note that $|V(G)| = \sum_{j \in [n]} |F_-(a_j, k)| = \sum_{j \in [n]} k + a_j = nk + kB$.
    Therefore, in any equitable $k$-coloring of~$G$, each color class has~$n + B$ vertices.
    Define~$F_j = F_-(a_j,k)$ and let~$C_j$ be the corresponding $K_{k-1}$.
    We show that there is an equitable~$k$-coloring~$\psi$ of~$G$ if and only if~$\varphi = \langle A,k,B\rangle$ is a $\YES$ instance of \pname{Bin Packing}.
    
    Let~$\varphi$ be a solution to \pname{Bin Packing}.
    For each~$a_j \in A$, we do $\psi(C_j) = [k] \setminus \{i\}$ if $a_j \in \varphi_i$.
    We color each vertex of the independent set of~$F_{j}$ with~$i$ and note that all remaining possible proper colorings of the gadget use each color the same number of times.
    Thus, $|\psi_i| = \sum_{j \mid a_j \in \varphi_i} (a_j + 1) + \sum_{j \mid a_j \notin \varphi_i} 1 = \sum_{j \mid a_j \in \varphi_i} (a_j + 1) + \sum_{j \in [n]} 1 - \sum_{j \mid a_j \in \varphi_i} 1 = n + B$.
    
    Now, let~$\psi$ be an equitable $k$-coloring of~$G$.
    Note that~$|\psi_i| = n+B$ and that the independent set of an antiflower is monochromatic.
    For each $j \in [n]$, $a_j \in \varphi_i$ if $i \notin \psi(C_j)$.
    That is, $n + B = |\psi_i| = \sum_{j \mid i \notin C_j} (a_j + 1) + \sum_{j \mid i \in C_j} 1 = \sum_{j \mid i \notin C_j} (a_j + 1) + \sum_{j \in [n]} 1 - \sum_{j \mid i \notin C_j} 1 = \sum_{j \mid i \notin C_j} a_j + n$, from which we conclude that $\sum_{j \mid i \notin C_j} a_j = B$.
\end{proof}

\subsubsection{Block Graphs}

We now proceed to the parameterized complexity of block graphs.
Conceptually, the proof follows a similar argumentation as the one developed in Theorem~\ref{thm:dis_split}; in fact, we are able to show that even restricting the problem to graphs of diameter at least four is not enough to develop an $\FPT$ algorithm, unless $\FPT = \W[1]$.
 
\begin{definition}
    An $(a,k)$-flower is the graph $F(a,k) = K_1 \oplus \left(\bigcup_{i \in [a+1]} K_{k-1}\right)$, that is, it is obtained from the union of $a+1$ cliques of size $k-1$ followed by a join with $K_1$.
\end{definition}

\begin{theorem}
    \label{thm:blocks}
    \textsc{equitable coloring} of block graphs of diameter at least four parameterized by the number of colors and treewidth is $\W[1]$-$\Hard$. 
\end{theorem}

\begin{tproof}
    Let $\langle A,k,B\rangle$ be an instance of \pname{Bin Packing}, $\forall k \in [n]$, $F_j = F(a_j, k+1)$, $F_0 = F(B, k+1)$ and, for $j \in \{0\} \cup [n]$, let $y_j$ be the universal vertex of $F_j$.
    Define a graph $G$ such that $V(G) = V\left(\bigcup_{j \in \{0\} \cup [n]} V(F_j)\right)$ and $E(G) = \{y_0y_j \mid j \in [n]\} \cup E\left(\bigcup_{j \in \{0\} \cup [n]} E(F_j)\right)$.
    Looking at Figure~\ref{fig:super_flower}, it is easy to see that any minimum path between a non-universal vertex of $F_a$ and a non-universal vertex of $F_b$, $a \neq b \neq 0$ has length four.
    We show that $\langle A,k,B\rangle$ is an $\YES$ instance if and only if $G$ is equitably $(k+1)$-colorable.
    \begin{align*}
        |V(G)| &= |V(F_0)| + \sum_{j \in [n]} |V(F_j)| &= &k(B + 1) + 1 + \sum_{j \in [n]} \left(1 + k(a_j + 1)\right)\\
               &= kB + k + n + k^2B + kn + 1           &= &(k+1)(kB + n + 1)\\
    \end{align*}
    
    Given a $k$-partition $\varphi$ of $A$ that solves our instance of \pname{Bin Packing}, we construct a coloring $\psi$ of $G$ such that $\psi(y_j) = i$ if $a_j \in \varphi_i$ and $\psi(y_0) = k+1$.
    Using a similar argument to the previous theorem, after coloring each $y_j$, the remaining vertices of $G$ are automatically colored.
    For $\psi_{k+1}$, note that $|\psi_{k+1}| = 1 + \sum_{j \in [n]} (a_j + 1) = kB + n + 1 = \frac{|V(G)|}{k+1}$.
    It remains to prove that every other color class $\psi_i$ also has $\frac{|V(G)|}{k+1}$ vertices.
    
    \begin{align*}
        |\psi_i| &= B + 1 + \sum_{j \mid y_j \notin \psi_i} (a_j + 1) + \sum_{j \mid y_j \in \psi_i} 1 &= &B + 1 + \sum_{j \in [n]} (a_j + 1) - \sum_{j \mid y_j \in \psi_i} a_j\\
                 &= B + 1 + kB + n - B &= &kB + n + 1\\
    \end{align*}
    
    For the converse we take an equitable $(k+1)$-coloring of $G$ and suppose, without loss of generality, that $\psi(y_0) = k+1$ and, consequently, for every other $y_i$, $\psi(y_i) \neq k+1$.
    To build our $k$-partition $\varphi$ of $A$, we say that $a_j \in \varphi_i$ if $\psi(y_j) = i$.
    The following equalities show that $\sum_{a_j \in \varphi_i} a_j = B$ for every $i$, completing the proof.
    
    \begin{align*}
        |\psi_i|  &= B + 1 + \sum_{j \mid y_j \in \psi_i} 1 + \sum_{j \mid y_j \notin \psi_i} (a_j + 1) &= &B + 1 + \sum_{j \in [n]} (a_j + 1) - \sum_{j \mid y_j \in \psi_i} a_j \\
        kB + n + 1 &= B + 1 + kB + n - \sum_{j \mid y_j \in \psi_i} a_j &\Rightarrow &B =  \sum_{j \mid y_j \in \psi_i} a_j
    \end{align*}
\end{tproof}

\subsubsection{Interval Graphs without some induced stars}


\begin{definition}
    Let $\mathcal{Q} = \{Q_1, Q_1', \dots, Q_a, Q_a'\}$ be a family of cliques such that $Q_i \simeq Q_i' \simeq K_{k-1}$ and $Y = \{y_1, \dots, y_a\}$ be a set of vertices.
    An $(a,k)$-trem is the graph $H(a,k)$ where $V\left(H(a,k)\right) = \mathcal{Q} \cup Y$ and $E\left(H(a,k)\right) = E\left(\bigcup_{i \in [a]} (Q_i \cup Q_i') \oplus y_i\right) \cup E\left(\bigcup_{i \in [a-1]} y_i \oplus Q_{i+1}\right)$.
\end{definition}

\begin{theorem}
    \label{thm:chordal_w1_hard}
    \pname{Equitable Coloring} of $K_{1,4}$ free interval graphs parameterized by treewidth, maximum number of colors and maximum degree is $\W[1]$-$\Hard$.
\end{theorem}

\begin{tproof}
    Once again, let $\langle A,k,B\rangle$ be an instance of \pname{Bin Packing}, define $\forall j \in [n]$, $H_j = H(a_j, k)$ and let $Y_j$ be the set of cut-vertices of $H_j$.
    The graph $G$ is defined as $G = \bigcup_{j \in [n]} V(H_j)$.
    By the definition of an $(a,k)$-trem, we note that the vertices with largest degree are the ones contained in $Y_j \setminus \{y_a\}$, which have degree equal to $3(k-1)$.
    We show that $\langle A,k,B\rangle$ is an $\YES$ instance if and only if $G$ is equitably $k$-colorable, but first note that $|V(G)| = \sum_{j \in [n]} |V(H_j)| = \sum_{j \in [n]} a_j + 2a_j(k-1) = kB + 2(k-1)kB = k(2kB - B)$.
    
    Given a $k$-partition $\varphi$ of $A$ that solves our instance of \pname{Bin Packing}, we construct a coloring $\psi$ of $G$ such that, for each $y \in Y_j$, $\psi(y) = i$ if and only if $a_j \in \varphi_i$.
    Using a similar argument to the other theorems, after coloring each $Y_j$, the remaining vertices of $G$ are automatically colored, and we have  $|\psi_i| = \sum_{j \in \varphi_i} a_j + \sum_{j \notin \varphi_i} 2a_j = B + 2(k-1)B = 2kB - B$.
    
    For the converse we take an equitable $k$-coloring of $G$ and observe that, for every $j \in [n]$, $|\psi(Y_j)| = 1$.
    As such, to build our $k$-partition $\varphi$ of $A$, we say that $a_j \in \varphi_i$ if and only if $\psi(Y_j) = \{i\}$.
    Thus, since $|\psi| = 2kB - B$, we have that $2kB - B = \sum_{j \in \varphi_i} a_j + \sum_{j \notin \varphi_i} 2a_j = \sum_{j \in [n]} a_j - \sum_{j \in \varphi_i} a_j = 2kB - \sum_{j \in \varphi_i} a_j$, from which we conclude that $B = \sum_{j \in \varphi_i} a_j$.
\end{tproof}
\section{Exact algorithms for \pname{Equitable Coloring}}

\subsection{Chordal graphs}

In this section we will make heavy use of the clique tree $\mathcal{T}(G) = (\mathcal{Q}, F)$ of our chordal graph $G$, which we denote by $\mathcal{T}$ for simplicity.
We also assume that $\mathcal{Q} = \{Q_1, \dots, Q_r\}$, $|V(G)| = n$, that $\mathcal{T}$ is rooted at $Q_1$ and that $T_i$ is the subtree of $\mathcal{T}$ rooted at bag $Q_i$.

Our dynamic programming algorithm explores the separability structure inherent to chordal graphs, embodied by the clique tree, to combine every coloring of a subtree that may yield an equitable coloring of the whole graph.
To do so, we must keep track of which bag, say $Q_i$, we are currently exploring and which colors have been used at the separator between $Q_i$ and $\mathcal{T} \setminus Q_i$.

A \tdef{$k$-color counter}, or simply a \tdef{counter} for an $n$ vertex graph is an element $X \in \left(\left[\ceil{\frac{n}{k}}\right] \cup \{0\}\right)^k$, that is, the $k$-th Cartesian power of $\left[\ceil{\frac{n}{k}}\right] \cup \{0\}$.
A counter $X$ is \tdef{equitable} if for every $x_i, x_j \in X$, $|x_i - x_j| \leq 1$.
For simplicity, denote $S(n,k) = \left[\ceil{\frac{n}{k}}\right] \cup \{0\}$.
The \tdef{sum} of two counters $X,Y$ is defined as $X + Y = (x_1 + y_1, \dots, x_k + y_k)$.
We also define the sum of two families $\mathcal{X}, \mathcal{Y}$ of counters as $\mathcal{Z} = \mathcal{X} + \mathcal{Y} = \left\{X + Y \in S(n,k)^k \mid X \in \mathcal{X}, Y \in \mathcal{Y} \right\}$, that is, the sum of all pairs of elements from each family that belong to $S(n,k)^k$.

\begin{observation}
    \label{obs:counter_bound}
    If $\mathcal{X}, \mathcal{Y} \subseteq S(n,k)^k$ and $\mathcal{Z} = \mathcal{X} + \mathcal{Y}$, $|\mathcal{Z}| \leq |S(n,k)|^k $.
\end{observation}

\begin{theorem}
    \label{thm:chordal_exp}
    There is an $\bigO{n^{2k+2}}$ time algorithm for \pname{Equitable Coloring} on chordal graphs where $k$ is the maximum number of colors.
\end{theorem}

\begin{tproof}
    Let $G$ be the input chordal graph of order $n$ and $\mathcal{T}$ its clique tree, rooted at bag $Q_1$. Moreover, we assume that $k \geq \omega(G)$, otherwise the answer is trivially $\NOi$.
    
    For a bag $Q$, denote by $p(Q)$ the parent clique of $Q$ on $\mathcal{T}$, $I(Q) = Q \cap p(Q)$ and by $U(Q) = Q \setminus p(Q)$ the set of vertices that first appeared on the path between $Q$ and the root $Q_1$ (if $Q = Q_1$, $U(Q) = Q$).
    
    Given $Q$ and a list of available colors $L$, define $\Pi(Q,L)$ as the set of all colorings of $U(Q)$ using only the colors of $L$, $\beta(Q,L,\pi)$ the list of colors used by $L$ and $\pi$ to color $I(Q)$ and $Y(Q,\pi)$ as the counter where $y_i = 1$ if and only if some vertex of $U(Q)$ was colored with color $i$ in $\pi$.
    
    We define our dynamic programming state $f(Q_i, L)$ as all colorings of $T_i$ conditioned to the fact that $I(Q_i)$ was colored with $L$. Intuitively, we will try every possible coloring $\pi$ of $U(Q)$ and combine the solutions of each bag adjacent to $Q_i$ given the colors used by $\pi$ and $L$. Finally, there will be an equitable $k$-coloring of $G$ if there is an equitable counter in $f(Q_1, \emptyset)$.
    
    \begin{equation*}
        f(Q_i, L) = \bigcup_{\pi \in \Pi(Q_i, [k] \setminus L)} \left(Y(Q_i, \pi) + \sum_{Q_j \in N_{T_i}(Q_i)} f(Q_j, \beta(Q_j, L, \pi))\right)
    \end{equation*}
    
    To prove the correctness of our algorithm, we will use induction on the size of $\mathcal{T}$.
    For the base case, where $|V(\mathcal{T})| = 1$, trivially, any proper coloring of $G$ will be equitable.
    
    For the general case, suppose that $Q_i$ has at least one child, say $Q_j$.
    Inductively, for any list of colors $R$, $f(Q_j, R)$ holds every proper coloring of $T_j \setminus I(Q_j)$, given that $I(Q_j)$ was colored with $R$.
    In particular, for every $\pi \in \Pi(Q_i, [k] \setminus L)$ we have this guarantee.
    Note that, for each pair of children $Q_j, Q_l$ of $Q_i$, their solutions are completely independent because $Q_j \cap Q_l \subseteq Q_i$ (clique tree property) and $Q_i$ is entirely colored by $\pi$ and $L$.
    This implies that no vertex was counted more than once on each counter, since their color is chosen exactly once for each possible $\pi$.
    Therefore, since the problems are independent for each child of $Q_i$, $\mathcal{Z}(\pi) = \sum_{Q_j \in N_{T_i}(Q_i)} f(Q_j, \beta(Q_j, L, \pi))$ combines every possible coloring of the children of $Q_i$ and $\mathcal{Z}(\pi) + Y(Q_i, \pi)$ is the family of all colorings of $T_i$, given that $Q_i$ was colored with $\pi$ and $L$.
    Finally, $\bigcup_{\pi \in \Pi(Q_i, [k] \setminus L)} Y(Q, \pi) + Z(\pi)$ tries every possible coloring $\pi$ of $Q_i$, guaranteeing that $f(Q_i, L)$ has every possible coloring of $T_i$, given that $I(Q_i)$ used $L$.
    Since $f(Q_i, L)$ has every coloring of $\mathcal{T_i}$ given $L$, $G$ will be equitably $k$-colorable if and only if there is an equitable counter at $f(Q_1, \emptyset)$.
    
    In terms of complexity analysis, we first note that each sum of two counter families $\mathcal{X},\mathcal{Y}$ takes $\bigO{|\mathcal{X}||\mathcal{Y}|}$.
    However, due to Observation~\ref{obs:counter_bound}, the size of $\mathcal{X} + \mathcal{Y}$ is at most $|S(N,k)|^k$ and, therefore, $\sum_{Q_j \in N_{T_i}(Q_i)} f(Q_j, \beta(Q_j, L, \pi))$ takes at most $\bigO{n|S(n,k)|^{2k}} = \bigO{n \left(\ceil{\frac{n}{k}} + 1\right)^{2k}}$ time; moreover, the addition of $Y(Q_i, \pi)$ to $\mathcal{Z}(\pi)$, by the same argument, is $\bigO{|S(n,k)|^k}$.
    
    For the outermost union, we have $\bigO{k!}$ possible colorings $\pi$ for $U(Q_i)$, which implies that computing each $f(Q_i, L)$ takes $\bigO{k!n\left(\ceil{\frac{n}{k}} + 1\right)^{2k}}$ time.
    Since we have $r \leq n$ bags and $k!$ possible lists, there are $\bigO{nk!}$ states, therefore the total complexity of our dynamic programming algorithm is $\bigO{k!^2n^2\left(\ceil{\frac{n}{k}} + 1\right)^{2k}} = \bigO{n^{2k + 2}}$.
\end{tproof}

\begin{corollary}
    \pname{Equitable Coloring} of chordal graphs parameterized by the number of colors is in $\XP$.
\end{corollary}

As shown by Theorem~\ref{thm:chordal_w1_hard}, \pname{Equitable Coloring} of $K_{1,4}$-free interval graphs is \W[1]-\Hard when parameterized by treewidth, maximum number of colors and maximum degree.
The construction of the hard instance, however, has a massive amount of copies of the claw ($K_{1,3}$); determining the complexity of the problem for the class of claw-free chordal graphs is, therefore, a direct question. 
Before answering it, we require a bit more of notation.
Given a partial $k$-coloring $\varphi$ of $G$, let $G[\varphi]$ denote the subgraph of $G$ induced by the vertices colored with $\varphi$, define $\varphi_-$ as the set of colors used $\floor{|V(G[\varphi])|/k}$ times in $\varphi$ and $\varphi_+$ the remaining colors.
If $k$ divides $|V(G[\varphi])|$, we say that $\varphi_+ = \emptyset$.
Our goal is to color $G$ one maximal clique (say $Q$) at a time and keep the invariant that the new vertices introduced by $Q$ can be colored with a subset of the elements of $L_-$.
To do so, we rely on the fact that, for claw-free graphs, the maximal connected components of the subgraph induced by any two colors form either cycles, which cannot happen since $G$ is chordal, or paths.
By carefully choosing which colors to look at, we find odd length paths that can be greedily recolored to restore our invariant.

\begin{lemma}
    \label{lem:claw_free_chordal}
    There is an $\bigO{n^2}$-time algorithm to equitably $k$-color a claw-free chordal graph or determine that no such coloring exists.
\end{lemma}

\begin{proof}
    We proceed by induction on the number $n$ of vertices of $G$, and show that $G$ is equitably $k$-colorable if and only if its maximum clique has size at most $k$.
    The case $n = 1$ is trivial.
    For general $n$, take one of the leaves of the clique tree of $G$, say $Q$, a simplicial vertex $v \in Q$ and define $G' = G \setminus \{v\}$.
    By the inductive hypothesis, there is an equitable $k$-coloring of $G'$ if and only if $k \geq \omega(G')$.
    If $k < \omega(G')$ or $k < |Q|$, $G$ can't be properly colored.
    
    Now, since $k \geq \omega(G) \geq |Q|$, take an equitable $k$-coloring $\varphi'$ of $G'$ and define $Q' = Q \setminus \{v\}$.
    If $|\varphi'_- \setminus \varphi'(Q')| \geq 1$, we can extend $\varphi'$ to $\varphi$ using one of the colors of $\varphi'_- \setminus \varphi'(Q')$ to greedily color $v$.
    Otherwise, note that $\varphi'_+ \setminus \varphi'(Q') \neq \emptyset$ because $k \geq \omega(G')$.
    Now, take some color $c \in \varphi'_- \cap \varphi'(Q')$, $d \in \varphi'_+ \setminus \varphi'(Q')$; by our previous observation, we know that $G'[\varphi_c \cup \varphi_d]$ has $C = \{C_1, \dots, C_l\}$ connected components, which in turn are paths.
    Now, take $C_i \in C$ such that $C_i$ has odd length and both endvertices are colored with $d$; said component must exist since $d \in \varphi'_+$ and $c \in \varphi'_-$.
    Moreover, $C_i \cap Q' = \emptyset$, we can swap the colors of each vertex of $C_i$ and then color $v$ with $d$; neither operation makes an edge monochromatic.
    
    As to the complexity of the algorithm, at each step we may need to select $c$ and $d$ -- which takes $\bigO{k}$ time -- construct $C$, find $C_i$ and perform its color swap, all of which take $\bigO{n}$ time.
    Since we need to color $n$ vertices and $k \leq n$, our total complexity is $\bigO{n^2}$.
\end{proof}

The above algorithm was not the first to solve \pname{Equitable Coloring} for claw-free graphs; this was accomplished by \cite{claw_free_de_werra} which implies that, for any claw-free graph $G$, $\chi_=(G) = \chi_=^*(G) = \chi(G)$.

\begin{theorem}[\cite{claw_free_de_werra}]
    If $G$ is claw-free and $k$-colorable, then $G$ is equitably $k$-colorable.
\end{theorem}

However, \citep{claw_free_de_werra} is not easily accessible, as it is not in any online repository.
Moreover, the given algorithm has no clear time complexity and, as far as we were able to understand the proof, its running time would be $\bigO{k^2n}$, which, for $k = f(n)$, is worse than the algorithm we present in Lemma~\ref{lem:claw_free_chordal}.
Using Lemma~\ref{lem:claw_free_chordal} and Theorem~\ref{thm:chordal_w1_hard} we obtain the following.

\begin{theorem}
     Let $G$ be a $K_1,r$-free chordal graph. If $r \geq 4$, \pname{Equitable Coloring} of $G$ parameterized by treewidth, number of colors and maximum degree is \W[1]\textsf{-hard}.
     Otherwise, the problem is solvable in polynomial time.
\end{theorem}

% \subsection{Distance to Cluster}

% In this section and the one that follows, we adapt the algorithm for \pname{Equitable Coloring} of unipolar graphs~\citep{unipolar_graphs} given by~\cite{matheus_etc}.
% A graph is \textit{unipolar} if its vertex set can be decomposed in cliques $\{C_1, C_2, \dots, C_\ell\}$ such that between $C_i$ and $C_j$, $2 \leq i,j \leq \ell$, there are no edges; $C_1$ is called the \textit{central clique}.
% Essentially, \cite{matheus_etc} construct an auxiliary graph upon which they apply maximum a maximum flow algorithm to obtain the equitable coloring.
% We describe their algorithm in detail, as it will be required to 
% The key insight they rely upon is the fact, on the auxiliary graph, an

\subsection{Clique partitioning}

Since \pname{Equitable Coloring} is $\W[1]$-$\Hard$ when simultaneously parameterized by many parameters, we are led to investigate a related problem.
Much like \pname{Equitable Coloring} is the problem of partitioning $G$ in $k'$ independent sets of size $\ceil{n/k}$ and $k - k'$ independent sets of size $\floor{n/k}$, one can also attempt to partition $\overline{G}$ in cliques of size $\ceil{n/k}$ or $\floor{n/k}$.
A more general version of this problem is formalized as follows:

\problem{clique partitioning}{A graph $G$ and two positive integers $k$ and $r$.}{Can $G$ be partitioned in $k$ cliques of size $r$ and $\frac{n-rk}{r-1}$ cliques of size $ r - 1$?}

We note that both \textsc{maximum matching} (when $k \geq n/2$) and \textsc{triangle packing} (when $k < n/2$) are particular instances of \textsc{clique partitioning}, the latter being $\FPT$ when parameterized by $k$~\citep{triangle_packing}.
As such, we will only be concerned when $r \geq 3$.
To the best of our efforts, we were unable to provide an $\FPT$ algorithm for \textsc{clique partitioning} when parameterized by $k$ and $r$, even if we fix $r = 3$.
However, the situation is different when parameterized by the treewidth of $G$, and we obtain an algorithm running in $2^{\tw(G)}n^{\bigO{1}}$ time for the corresponding counting problem, \textsc{\#clique partitioning}.

The key ideas for our bottom-up dynamic programming algorithm are quite straightforward. First, cliques are formed only when building the tables for forget nodes. Second, for join nodes, we can safely consider only the combination of two partial solutions that have empty intersection on the covered vertices (i.e. that have already been assigned to some clique). Finally, both join and forget nodes can be computed using fast subset convolution~\citep{fourier_mobius}.
For each node $x \in \mathbb{T}$, our algorithm builds the table $f_x(S, k')$, where each entry is indexed by a subset $S \subseteq B_x$ that indicates which vertices of $B_x$ have already been covered, an integer $k'$ recording how many cliques of size $r$ have been used, and stores how many partitions exist in $G_x$ such that only $B_x \setminus S$ is yet uncovered.
If an entry is inconsistent (e.g. $S \nsubseteq B_x$), we say that $f(S, K') = 0$.

\begin{theorem}
    \label{thm:clique_part}
    There is an algorithm that, given a nice tree decomposition of an $n$-vertex graph $G$ of width $\tw$, computes the number of partitions of $G$ in $k$ cliques of size $r$ and $\frac{n - rk}{r-1}$ cliques of size $r-1$ in time $\bigOs{2^{\tw}}$ time.
\end{theorem}

\begin{proof}
    \textit{Leaf node:} Take a leaf node $x \in \mathbb{T}$ with $B_x = \emptyset$. 
    Since the only one way of covering an empty graph is with zero cliques, we compute $f_x$ with:
    \begin{equation*}
        \centering
        \hfill f_x(S, k') =
        \begin{cases}
            1, \text{ if } k' = 0 \text{ and } S = \emptyset \text{;}\\
            0, \text{ otherwise.}\\
        \end{cases}
    \end{equation*}
    
    \emph{Introduce node:} Let $x$ be a an introduce node, $y$ its child and $v \in B_x \setminus B_y$.
    Due to our strategy, introduce nodes are trivial to solve; it suffices to define $f_x(S, k') = f_y(S, k')$. If $v \in S$, we simply define $f_x(S, k') = 0$.
    
    \emph{Forget node:} For a forget node $x$ with child $y$ and forgotten vertex $v$, we formulate the computation of $f_x(S, k')$ as the subset convolution of two functions as follows:
    
    \begin{align*}
        \centering
        f_x(S, k') &= f_y(S \cup \{v\}, k') + \sum_{A \subseteq S} f_y(S \setminus A, k' - 1)g_r(A, v) + \sum_{A \subseteq S} f_y(S \setminus A, k')g_{r-1}(A, v)\\
        g_l(A, v) &=
            \begin{cases}
                1, \text{ if $A$ is a clique of size $l$ contained in $N[v]$ and $v \in A$;}\\
                0, \text{otherwise.}
            \end{cases}
    \end{align*}
    
    The above computes, for every $S \subseteq B_x$ and every clique $A$ (that contains $v$) of size $r$ or $r - 1$ contained in $N[v] \cap B_y \cap S$, if $S \setminus A$ and some $k''$ is a valid entry of $f_y$, or if $v$ had been previously covered by another clique (first term of the sum).
    Directly computing the last two terms of the equation, for each pair $(S, k')$, yields a total running time of the order of $ \sum_{|S| = 0}^\tw \binom{\tw}{|S|}2^{|S|} = (1+2)^\tw = 3^\tw$ for each forget node.
    However, using the fast subset convolution technique described by~\cite{fourier_mobius}, we can compute the above equation in time $\bigOs{2^{|B_x|}} = \bigOs{2^{\tw}}$.
    
    Correctness follows directly from the hypothesis that $f_y$ is correctly computed and that,  for every $A \subseteq B_x$, $g_r(A, v)g_{r-1}(A, v) = 0$.
    For the running time, we can pre-compute both $g_r$ and $g_{r-1}$ in $\bigO{2^\tw r^2}$, so their values can be queried in $\bigO{1}$ time.
    As such, each forget node takes $\bigO{2^\tw\tw^3k}$ time, since we can compute the subset convolutions of $f_y * g_r$ and $f_y * g_{r-1}$ in $\bigO{2^\tw\tw^3}$ time each.
    The additional factor of $k$ comes from the second coordinate of the table index.
    
    \emph{Join node:} Take a join node $x$ with children $y$ and $z$.
    Since we want to partition our vertices, the cliques we use in $G_y$ and $G_z$ must be completely disjoint and, consequently, the vertices of $B_x$ covered in $B_y$ and $B_z$ must also be disjoint.
    As such, we can compute $f_x$ through the equation:
    
    \begin{equation*}
        f_x(S, k') = \sum_{k_y + k_z = k'} \sum_{A \subseteq S} f_y(A, k_y)f_z(S \setminus A, k_z)
    \end{equation*}
    
    Note that we must sum over the integer solutions of the equation $k_y + k_z = k'$ since we do not know how the cliques of size $r$ are distributed in $G_x$.
    To do that, we compute the subset convolution $f_y(\cdot, k_y) * f_z(\cdot, k_z)$.
    The time complexity of $\bigO{2^\tw\tw^3k^2}$ follows directly from the complexity of the fast subset convolution algorithm, the range of the outermost sum and the range of the second parameter of the table index.
    
    For the root $x$, we have $f_x(\emptyset, k) \neq 0$ if and only if $G_x = G$ can be partitioned in $k$ cliques of size $r$ and the remaining vertices in cliques of size $r-1$.
    Since our tree decomposition has $\bigO{n\tw}$ nodes, our algorithm runs in time $\bigO{2^\tw\tw^4k^2n}$.
    
    To recover a solution given the tables $f_x$, start at the root node with $S = \emptyset$, $k' = k$ and let $\mathcal{Q} = \emptyset$ be the cliques in the solution.
    We shall recursively extend $\mathcal{Q}$ in a top-down manner, keeping track of the current node $x$, the set of vertices $S$ and the number $k'$ of $K_r$'s used to cover $G_x$.
    Our goal is to keep the invariant that $f_x(S, k') \neq 0$.
    
    \emph{Introduce node:} Due to the hypothesis that $f_x(S, k') \neq 0$ and the way that $f_x$ is computed, it follows that $f_y(S, k') \neq 0$.
        
    \emph{Forget node:} Since the current entry is non-zero, there must be some $A \subseteq S$ such that exactly one of the products $f_y(S \setminus A, k' - 1)g_r(A, v)$, $f_y(S \setminus A, k')g_{r-1}(A, v)$ is non-zero and, in fact, any such $A$ suffices.
    To find this subset, we can iterate through $2^S$ in $\bigO{2^\tw}$ time and test both products to see if any of them is non-zero.
    Note that the chosen $A \cup \{v\}$ will be a clique of size either $r$ or $r-1$, and thus, we can set $\mathcal{Q} \gets \mathcal{Q} \cup \{A \cup \{v\}\}$.
        
    \emph{Join node:} The reasoning for join nodes is similar to forget nodes, however, we only need to determine which states to look at in the child nodes.
    That is, for each integer solution to $k_y + k_z = k'$ and for each $A \subseteq S$, we check if both $f_y(A, k_y)f_z(S \setminus A, k_z)$ is non-zero; in the affirmative, we compute the solution for both children with the respective entries.
    Any such triple $(A, k_y, k_z)$ that satisfies the condition suffices.
    
    Clearly, retrieving the solution takes $\bigO{2^\tw k}$ time per node, yielding a running time of $\bigOs{2^\tw}$.
\end{proof}

\begin{corollary}
    Equitable coloring is $\FPT$ when parameterized by the treewidth of the complement graph.
\end{corollary}

\section{Clique and biclique coloring}

Both \pname{Clique Coloring} and \pname{Biclique Coloring} are relaxations of the classical \pname{Vertex Coloring} problem, in the sense that monochromatic edges are allowed.
However, this freedom comes at the cost of validating a solution, which becomes a $\coNP\text{-}\Complete$ task in both cases.
One may think of \pname{Vertex Coloring} as the task of covering a graph's vertices using a given number of independent sets.
That is, there cannot be a color class with an edge inside it.
For \pname{Clique Coloring} and \pname{Biclique Coloring}, the idea is quite similar.
We want to forbid not edges, but maximal clique or bicliques, respectively, inside our color classes.
All of the following results establish families of sets that may safely be used to cover the given graph and describe how to compute them.

Much of the following discussion will deal with the clique and biclique hypergraphs $\Hyper{C}(G)$ and $\Hyper{B}(G)$.
As such, we denote by $\trans{C}(G)$ ($\trans{B}(G)$) the \tdef{family of all transversals} of the clique (biclique) hypergraph of $G$ and by $\transs{C}(G)$ ($\transs{B}(G)$) the \tdef{family of complements of transversals}.
Also, denote by $\oblq{C}(G)$ ($\oblq{B}(G)$) the \tdef{family of all obliques} of the clique (biclique) hypergraph of $G$.
Finally, $\clq(G)$ ($\biq(G)$) is the family of \tdef{maximal cliques} (bicliques) of $G$.

In this chapter, we present algorithms that make heavy use of the algorithm described by~\cite{inclusion_exclusion}, which applies the inclusion-exclusion principle to solve a variety of problems in $2^nn^{\bigO{1}}$ time, including \pname{Vertex Coloring}.
Our main results are an $\bigOs{2^n}$ algorithm for \pname{Biclique Coloring}, an $\FPT$ algorithm for \pname{Clique Coloring} parameterized by neighbourhood diversity and an $\FPT$ algorithm for \pname{Biclique Coloring} parameterized by the number of colors and neighbourhood diversity.
To achieve them, we will rely on the following problems and results of the literature.

\begin{lemma}[\cite{clique_color_algorithm}]
    \label{lem:down_closure}
    For any family $\mathcal{F}$, its down closure $\mathcal{F}_{\downarrow} = \left\{X \subseteq V \mid \exists Y \in \mathcal{F},\ X \subseteq Y\right\}$ can be enumerated in $O^*(|\mc{F}_{\downarrow}|)$ time.
\end{lemma}

\begin{lemma}[\cite{clique_color_algorithm}]
    \label{lem:clique_transversal_colorings}
    A $k$-partition $\varphi = \{\varphi_1, \dots, \varphi_k\}$ is a $k$-clique-coloring of $G$ if and only if for every $i$, $\overline{\varphi_i} \in \trans{C}(G)$.
\end{lemma}

\problem{exact cover}{A set $A = \{a_1, \dots, a_n\}$, a covering family $\mathcal{F} \subseteq 2^A$ and an integer $k$.}{Is it possible to $k$-partition  $A$ into $\varphi$ such that $\varphi \subseteq \mathcal{F}$?}

\begin{theorem}[\cite{inclusion_exclusion}]
    \label{thm:inc_exc}
    There is a $\bigOs{2^n}$ time algorithm to solve \pname{exact cover}.
\end{theorem}

\begin{theorem}[\cite{clique_color_algorithm}]
    \label{thm:clique_color_algorithm}
    There is an $\bigOs{2^n}$ time algorithm for \pname{Clique Coloring}.
\end{theorem}


\problem{set multicover}{A set $A = \{a_1, \dots, a_n\}$, a covering family $\mathcal{F} \subseteq 2^A$, an integer $k$ and a coverage demand $c: A \mapsto \mathbb{N}$.}{Is it possible to $k$-cover $A$ with $\varphi \subseteq \mathcal{F}$ and $\forall a_j, |\{i \mid a_j \in \varphi_i\}| \geq c(a_j)$?}

\begin{theorem}[\cite{set_multicover}]
    \label{thm:set_multicover}
    Set multicover can be solved in $O^*((b+2)^n)$, with $b$ the maximum coverage requirement.
\end{theorem}
\section{Exact Algorithm for \pname{Biclique Coloring}}
\label{sec:biclique_exact}

\begin{figure}[!tb]
    \centering
    \begin{tikzpicture}[scale=1]
        \begin{scope}
        %\draw[help lines] (-5,-5) grid (5,5);
            \GraphInit[unit=3,vstyle=Normal]
            \SetVertexNormal[Shape=circle, FillColor=black, MinSize=2pt]
            \tikzset{VertexStyle/.append style = {inner sep = \inners, outer sep = \outers}}
            \SetVertexNoLabel
            \Vertex[x=-1,y=-1]{x-1-1}
            \Vertex[x=-1,y=0]{x-10}
            \Vertex[x=-1,y=1]{x-11}
            \Vertex[x=0,y=-1]{x0-1}
            \Vertex[x=0,y=0]{x00}
            \Vertex[x=0,y=1]{x01}
            \Vertex[x=1,y=-1]{x1-1}
            \Vertex[x=1,y=0]{x10}
            \Vertex[x=1,y=1]{x11}

            
            \Edge(x-1-1)(x-10)
            \Edge(x-1-1)(x0-1)
            \Edge(x0-1)(x00)
            \Edge(x0-1)(x1-1)
            
            \Edge(x-10)(x-11)
            \Edge(x-10)(x00)
            \Edge(x00)(x01)
            \Edge(x00)(x10)
            
            \Edge(x-11)(x01)
            \Edge(x01)(x11)
            
            \Edge(x10)(x11)
            \Edge(x10)(x1-1)
        \end{scope}
    \end{tikzpicture}
    \hfill
    \begin{tikzpicture}[scale=1]
        \begin{scope}
            %\draw[help lines] (-5,-5) grid (5,5);
            \GraphInit[unit=3,vstyle=Normal]
            \SetVertexNormal[Shape=circle, FillColor=white, MinSize=1pt]
            \tikzset{VertexStyle/.append style = {inner sep = \inners, outer sep = \outers}}
            \SetVertexNoLabel
            \Vertex[x=-1,y=-1]{x-1-1}
            \Vertex[x=-1,y=1]{x-11}
            \Vertex[x=1,y=-1]{x1-1}
            \Vertex[x=1,y=1]{x11}
            
            \SetVertexNormal[Shape=circle, FillColor=black, MinSize=1pt]
            \SetVertexNoLabel
            \Vertex[x=0,y=-1]{x0-1}
            \Vertex[x=-1,y=0]{x-10}
            \Vertex[x=1,y=0]{x10}
            \Vertex[x=0,y=0]{x00}
            \Vertex[x=0,y=1]{x01}

            
            \Edge(x-1-1)(x-10)
            \Edge(x-1-1)(x0-1)
            \Edge(x0-1)(x00)
            \Edge(x0-1)(x1-1)
            
            \Edge(x-10)(x-11)
          \Edge(x-10)(x00)
            \Edge(x00)(x01)
            \Edge(x00)(x10)
            
          \Edge(x-11)(x01)
            \Edge(x01)(x11)
            
            \Edge(x10)(x11)
            \Edge(x10)(x1-1)
        \end{scope}
    \end{tikzpicture}
    \hfill
    \begin{tikzpicture}[scale=1]
        \begin{scope}
            %\draw[help lines] (-5,-5) grid (5,5);
            \GraphInit[unit=3,vstyle=Normal]
            \SetVertexNormal[Shape=circle, FillColor=white, MinSize=1pt]
            \tikzset{VertexStyle/.append style = {inner sep = \inners, outer sep = \outers}}
            \SetVertexNoLabel
            \Vertex[x=-1,y=-1]{x-1-1}
            \Vertex[x=-1,y=1]{x-11}
            \Vertex[x=1,y=-1]{x1-1}
            \Vertex[x=1,y=1]{x11}
            \Vertex[x=0,y=0]{x00}
            
            \SetVertexNormal[Shape=circle, FillColor=black, MinSize=1pt]
            \SetVertexNoLabel
            \Vertex[x=0,y=-1]{x0-1}
            \Vertex[x=-1,y=0]{x-10}
            \Vertex[x=1,y=0]{x10}
            \Vertex[x=0,y=1]{x01}

            
            \Edge(x-1-1)(x-10)
            \Edge(x-1-1)(x0-1)
            \Edge(x0-1)(x00)
            \Edge(x0-1)(x1-1)
            
            \Edge(x-10)(x-11)
            \Edge(x-10)(x00)
            \Edge(x00)(x01)
            \Edge(x00)(x10)
            
            \Edge(x-11)(x01)
            \Edge(x01)(x11)
            
            \Edge(x10)(x11)
            \Edge(x10)(x1-1)
        \end{scope}
    \end{tikzpicture}

    \caption{From left to right: a graph, one of its maximal bicliques, and a transversal.}
    \label{fig:transversals}
\end{figure}

Drawing inspiration from~\cite{clique_color_algorithm}, we first show the relationships between hypergraph structures and colorings, and use these to build an $\bigOs{2^n}$-time algorithm for \pname{Biclique Coloring} by stating it as an exact cover instance.
Naturally, the covering family must be carefully chosen such that any solution to the covering problem produces a valid coloring.
We first formalize the observation that, given a color~$i$, every maximal biclique of~$G$ must have one color other than~$i$.
%Our first task is to establish a family of sets that can be safely used to cover~$V(G)$,

\begin{lemma}
    \label{lem:transversal_colorings}
    A $k$-partition $\varphi = \{\varphi_1, \dots, \varphi_k\}$ is a $k$-biclique-coloring of $G$ if and only if for every $i$, $\overline{\varphi_i} \in \trans{B}(G)$.
\end{lemma}

%\begin{tproof}
\begin{proof}
    Suppose that there exists some $\varphi_i$ such that $\overline{\varphi_i} \notin \trans{B}(G)$.
    This implies that there exists some $B \in \biq(G)$ such that $B \cap \overline{\varphi_i} = \emptyset$ and that $B \subseteq \varphi_i$; that is, $|\varphi(B)| = 1$, which is a contradiction, since $\varphi$ is a $k$-biclique-coloring.
    
    For the converse, let $\varphi$ be a $k$-partition of $G$ with $\overline{\varphi_i} \in \trans{B}(G)$, but suppose that $\varphi_k$ is not a $k$-biclique-coloring.
    That is, there exists some maximal biclique $B \in \biq(G)$ such that $B \subseteq \varphi_i$ for some $i$.
    This implies that $B \cap \overline{\varphi_i} = \emptyset$, and, therefore, $\overline{\varphi_i} \notin \trans{B}(G)$, which contradicts the hypothesis.
\end{proof}
%\end{tproof}

Simply testing for each $X \in 2^{V(G)}$ if $X \in \transs{B}(G)$ is a costly task.
A naive algorithm would check, for each $B \in \biq(G)$, if $\overline{X} \cap B \neq \emptyset$.
With $|\biq(G)| \in \bigO{n3^{\frac{n}{3}}}$ (see~\cite{gaspers} for the proof), such algorithm would take $\bigO{n2^n3^{\frac{n}{3}}}$-time.
The next Lemma, along with Lemma~\ref{lem:down_closure}, considerably reduces the complexity of enumerating $\trans{B}(G)$.
We will enumerate $\oblq{B}(G)$ by generating its maximal elements and then use the fact that $\oblq{B}(G)$ is closed under the subset operation.

\begin{lemma}
    \label{lem:complementary_obliques}
    The maximal obliques of $\Hyper{B}(G)$ are exactly the complements of the maximal bicliques of $G$.
\end{lemma}

%\begin{tproof}
\begin{proof}
    Let $X \in \oblq{B}$ be a maximal oblique. By definition, there exists some~$B \in \biq(G)$ such that~$X \cap B = \emptyset$, which implies that $X \subseteq \overline{B}$.
    Note that, if $X \subset \overline{B}$, there is some $v \in \overline{B} \setminus X$, which implies that $(X \cup \{v\}) \cap B = \emptyset$ and that $X$ is not a maximal oblique.
    Let $B \in \biq(G)$. By definition, $\overline{B} \in \oblq{B}$ and must be maximal because $\{B, \overline{B}\}$ is a partition of $V(G)$.
\end{proof}
%\end{tproof}

\begin{corollary}
    \label{col:is_maximal_oblique}
    Given a graph $G = (V, E)$ and a subset $X \subseteq V(G)$, there exists an $O(n(n - |X|))$-time algorithm to determine if $X$ is a maximal oblique.
\end{corollary}

\begin{theorem}
    \label{thm:exact_biclique}
    There is an $\bigOs{2^n}$-time algorithm for \pname{Biclique Coloring}.
\end{theorem}

%\begin{tproof}
\begin{proof}
    Our goal is to make use of Theorem~\ref{thm:inc_exc} to solve an instance of \pname{Exact Cover}, with $A = V(G)$, $\mathcal{F} = \transs{B}(G)$ and $k$ the partition size.
    Lemma~\ref{lem:transversal_colorings} guarantees that there is an answer to our instance of \pname{Biclique Coloring} if and only if there is an answer to the corresponding \pname{Exact Cover} one.
    To compute~$\transs{B}(G)$, for each $X \in 2^{V(G)}$, we use Lemma~\ref{lem:complementary_obliques} and Corollary~\ref{col:is_maximal_oblique} to say whether or not $X$ is a maximal oblique of $\Hyper{B}(G)$.
    Next, we compute $\oblq{B}(G)$ from its maximal elements using Lemma~\ref{lem:down_closure}, and use the fact that $\trans{B}(G) = 2^{V(G)} \setminus \oblq{B}(G)$ and complement each transversal to obtain $\transs{B}(G)$.
    Clearly, this procedure takes $\bigOs{2^n}$-time to construct $\transs{B}(G)$ and an additional $\bigOs{2^n}$-time by Theorem~\ref{thm:inc_exc}.
\end{proof}
%\end{tproof}

\section{Algorithms Parameterized by Neighborhood Diversity}

As previously discussed, a type is a maximal set of vertices that are either true or false twins to each other.
Suppose that we are already given a partition $\{D_1, \dots, D_{\nd(G)}\}$ of $V(G)$ in types.
If $D_i$ is composed of true twins, we say that it is a \tdef{true twin class} $T_i$ and, by definition, $G[D_i]$ is a clique.
Similarly, if $D_i$ is composed of false twins, it is a \tdef{false twin class} $F_i$ and $G[D_i]$ is an independent set.
When $|D_i| = 1$, we treat the class differently depending on the problem.
For the entirety of this section, we assume that $d = \nd(G)$.

\subsection{\pname{Biclique Coloring}}

For \pname{Biclique Coloring}, if there is some $D_i$ with a single vertex we shall treat it as a true twin class.

\begin{observation}
    \label{obs:biclique_true_twins}
    Given $G$ and a true twin class $T_i$ of $G$, any $k$-biclique-coloring~$\varphi$ of $G$ has $|\varphi(T_i)| = |T_i|$.
\end{observation}

\begin{lemma}
    \label{lem:biclique_false_twins}
    Given $G$ and a false twin class $F \subset V(G)$, any  $k$-biclique-coloring~$\varphi'$ of $G$ can be changed into a $k$-biclique-coloring $\varphi$ of $G$ such that $|\varphi(F)| \leq 2$.
\end{lemma}

%\begin{tproof}
\begin{proof}
    If $|\varphi'(F)| \leq 2$, $\varphi = \varphi'$.
    Otherwise, there exists $f_1, f_2, f_3 \in F$ with three different colors. 
    Since every maximal biclique $B$ of $G$ that intercepts $F$ has that $F \subset B$ and thus $|\varphi'(B)| \geq 3$.
    By making $\varphi(f_1) = \varphi'(f_1)$ and $\varphi(f_3) = \varphi(f_2) = \varphi'(f_2)$, we obtain $|\varphi(B)| \geq |\varphi(F)| \geq 2$.
    Repeating this process until $|\varphi(F)| = 2$ does not make any biclique monochromatic and completes the proof.
\end{proof}
%\end{tproof}

The central idea of our parameterized algorithm is to build an induced subgraph $H$ of $G$ and, afterward, use the results established here and in Section~\ref{sec:biclique_exact} to show that the solution to a particular instance of \pname{Set Multicover} derived from $H$ can be transformed in a solution to \pname{Biclique Coloring} of $G$.


\begin{definition}[B-Projection and B-Lifting]
    Let $T_i$ and $F_j$ be as previously discussed.
    We define the following projection rules:
    $\forall t_i^q \in T_i,\ \Proj{B}(t_i^q) = \{t'_i\}$;
    for $f_j^1\in F_j,\ \Proj{B}(f_j^1) = \{f'{_j^1}\}$;
    $\forall f_j^r \in F_j \setminus \{f_j^1\}$, $\Proj{B}(f_j^r) = \{f'{_j^2}\}$
    and $\Proj{B}(X) = \bigcup_{u \in X} \Proj{B}(u)$.
    
    Lifting rules are defined as $\Lift{B}(t'_i) = \{t_i\}$; $\Lift{B}(f'{_j^1}) = \{f_j^1\}$; $\Lift{B}(f'{_j^2}) = F_j \setminus \{f_j^1\}$ and $\Lift{B}(Y) = \bigcup_{u \in Y} \Lift{B}(u)$. Note that $\Proj{B}(\Lift{B}(X)) = X$, $\forall X$.
\end{definition}


\begin{definition}[B-Projected Graph]
    The B-projected graph $H$ of $G$ satisfies $V(H) = \Proj{B}(V(G))$ and $v'_iv'_j \in E(H)$ if and only if there exist $v_i \in \Lift{B}(v'_i)$ and $v_j \in \Lift{B}(v'_j)$ such that $v_iv_j \in E(G)$. $H$ is an induced subgraph of $G$.
\end{definition}

\begin{figure}[!tb]
    \centering
        \begin{tikzpicture}[scale=1]
            \GraphInit[unit=3,vstyle=Normal]
            \SetVertexNormal[Shape=circle, FillColor=black, MinSize=2pt]
            \tikzset{VertexStyle/.append style = {inner sep = \inners, outer sep = \outers}}
            \SetVertexNoLabel
            \Vertex[x=0,y=0]{a}
            \Vertex[x=-1,y=0]{b}
            \Vertex[x=0,y=-1]{c}
            
            
            
            \Vertex[x=1,y=0]{h}
            \Vertex[x=0,y=1]{i}
            \Vertex[a=45, d=1]{j}
            
            \Edge(a)(b)
            \Edge(a)(c)
            
            \Edge(a)(h)
            \Edge(a)(i)
            \Edge(a)(j)
            \begin{scope}[shift={(-1.65cm, 0cm)}]
                \grComplete[RA=0.65]{3}
            \end{scope}
            \begin{scope}[shift={(0cm, -1.65cm)}]
                \grComplete[RA=0.65]{4}
            \end{scope}
        \end{tikzpicture}
    \hfill
        \begin{tikzpicture}[scale=1]
            \GraphInit[unit=3,vstyle=Normal]
            \SetVertexNormal[Shape=circle, FillColor=black, MinSize=2pt]
            \tikzset{VertexStyle/.append style = {inner sep = \inners, outer sep = \outers}}
            \SetVertexNoLabel
            \Vertex[x=0,y=0]{a}
            \Vertex[x=-1,y=0]{b}
            \Vertex[x=0,y=-1]{c}
            \Vertex[x=0,y=-1.65]{d}
            \Vertex[x=-1.65,y=0]{e}
            
            
            
            \Vertex[x=1,y=0]{h}
            \Vertex[x=0,y=1]{i}
            
            \Edge(a)(b)
            \Edge(a)(c)
            
            \Edge(a)(h)
            \Edge(a)(i)
            \Edge(b)(e)
            \Edge(c)(d)
        \end{tikzpicture}
    \hfill
        \begin{tikzpicture}[scale=1]
            \GraphInit[unit=3,vstyle=Normal]
            \SetVertexNormal[Shape=circle, FillColor=black, MinSize=2pt]
            \tikzset{VertexStyle/.append style = {inner sep = \inners, outer sep = \outers}}
            \SetVertexNoLabel
            \Vertex[x=0,y=0]{a}
            \Vertex[x=-1,y=0]{b}
            \Vertex[x=0,y=-1]{c}
            
            
            
            \Vertex[a=45, d=1]{j}
            
            \Edge(a)(b)
            \Edge(a)(c)
            
            \Edge(a)(j)
            \begin{scope}[shift={(-1.65cm, 0cm)}]
                \grComplete[RA=0.65]{3}
            \end{scope}
            \begin{scope}[rotate=90,shift={(-1.65cm, 0cm)}]
                \grComplete[RA=0.65]{3}
            \end{scope}
        \end{tikzpicture}
    \hfill
    
    
    \caption{A graph, its B-projected and C-projected graphs}
    \label{fig:b-projected}
\end{figure}

For the remainder of this section, $G$ will be the input graph to \pname{Biclique Coloring} and $H$ the B-Projected graph of $G$.
Our \pname{Set Multicover} instance consists of $V(H)$ as the ground set, $\transs{B}(H)$ as the covering family, the size $k$ of the cover the same as the coloring of $G$ and $c(t_i') = |T_i|$ for every true twin class $T_i$ and $c(f'{_j^1}) = c(f'{_j^2}) = 1$ for each false twin class $F_j$.
The next observation follows directly from the fact that $\transs{B}(H)$ is closed under the subset operation, while the subsequent results allow us to move freely between $\transs{B}(G)$ and $\transs{B}(H)$.

\begin{observation}
    \label{obs:fatless_multicover}
    If there is a minimum $k$-multicover $\psi$ of $V(H)$ by $\transs{B}(H)$, then there exists a minimum $k$-multicover $\psi' = \{\psi_1, \dots, \psi_k\}$ such that $\left|\left\{j \mid u \in \psi_j\right\}\right| = c(u)$ for every $u \in V(H)$.
\end{observation}

\begin{lemma}
    \label{lem:lift_proj_biclique}
    If $B' \in \biq(H)$ then $B = \Lift{B}(B') \in \biq(G)$. Conversely, if $B \in \biq(G)$ and $B$ is not contained in any true twin class, then $B' = \Proj{B}(B) \in \biq(H)$.
\end{lemma}

%\begin{tproof}
\begin{proof}
    Note that $B$ is a biclique by the definition of $\Lift{B}$ and the fact that $B'$ is a biclique. By the contrapositive, suppose that $B \notin \biq(G)$ and that $u \in V(G)$ is such that $B \cup \{u\}$ is a (not necessarily maximal) biclique of $G$. Note that either:
    (i) if $u \in F_j$ then $F_j \nsubseteq B$ and $\Proj{B}(u) \notin B'$, because $u \notin \Lift{B}(f'{_j^1})$ or $u \notin \Lift{B}(f'{_j^2})$;
    or (ii) if $u \in T_i$ then $T_i \cap B = \emptyset$, which implies that $\Proj{B}(u) \notin B'$.
    %Since $H$ is an induced subgraph of $G$, no new bicliques can be created in $H$ that were not present in $G$.
    Since $B \cup \{u\}$ is a biclique, $\Proj{B}(u)$ is adjacent to only one partition of $B'$.
    The fact that $\Proj{B}(u) \notin B'$ implies that $\Proj{B}(B \cup \{u\}) = \Proj{B}(B) \cup \Proj{B}(u) = B' \cup \Proj{B}(u)$ is a biclique of $H$ and $B'$ is not maximal.
    
    Conversely, by the definition of $\Proj{B}$, $B' = (X, Y)$ must be a biclique of~$H$. By the contrapositive, there is $u' \in V(H)$ such that $B' \cup \{u'\}$ is a (not necessarily maximal) biclique of $H$, and let $u \in \Lift{B}(u')$. By the definition of $\Lift{B}$, it follows that $u$ can only be adjacent to one of partition of $B$, say $\Lift{B}(X)$. Thus, $u' \in Y$ and, for each $v \in \Lift{B}(Y)$, $uv \notin E(G)$, otherwise there would be $v' \in \Proj{B}(v)$ with $u'v' \in E(H)$. Hence, $B \cup \{u\}$ is a biclique of $G$ and $B$ is not maximal.
\end{proof}
%\end{tproof}


\begin{theorem}
    \label{thm:lifted_transversal}
    $X \subseteq V(H)$ is in $\transs{B}(H)$ if and only if $\Lift{B}(X) \in \transs{B}(G)$.
\end{theorem}

%\begin{tproof}
\begin{proof}
    Recall that $X \in \transs{B}(H)$ if and only if no maximal biclique of $H$ is contained in $X$.
    It is clear that, for every $B' \in \biq(H)$, $B' \nsubseteq X$ implies that $\Lift{B}(B') \nsubseteq \Lift{B}(X)$, since no two vertices of $H$ are lifted to the same vertex of $G$, and $\Lift{B}(B') \in \biq(G)$ due to Lemma~\ref{lem:lift_proj_biclique}.
    Moreover, no biclique of $G$ entirely contained in a true twin class can be a subset of $\Lift{B}(X)$. As such, $\Lift{B}(X)$ contains a maximal biclique $B$ only if $\Proj{B}(B) \subseteq X$ and $\Proj{B}(B) \notin \biq(H)$, which is impossible due to Lemma~\ref{lem:lift_proj_biclique} and the assumption that $B$ is maximal.
    
    Taking the contrapositive, $X \notin \transs{B}(H)$ implies that there is some maximal biclique $B'$ of $H$ such that $B' \subseteq X$. This implies that $\Lift{B}(B') \subseteq \Lift{B}(X)$, and, since $\Lift{B}(B')$ is a maximal biclique of $G$ due to Lemma~\ref{lem:lift_proj_biclique}, it holds that $\Lift{B}(X)$ is not a complement of transversal of $G$.
\end{proof}
%\end{tproof}


\begin{theorem}
    \label{thm:lifted_multicover}
    $\psi$ is a $k$-multicover of $H$ if and only if $G$ is $k$-biclique-colorable.
\end{theorem}

%\begin{tproof}
\begin{proof}
    Recall that a $k$-partition is a $k$-biclique-coloring if and only if all elements of the partition belong to $\transs{B}(G)$. By the construction of our set multicover instance, we have that, for each $\psi_i$, $\psi_i \in \transs{B}(H)$. By making $\varphi = \left\{\Lift{B}(\psi_1), \dots, \Lift{B}(\psi_k)\right\}$, and recalling Observation~\ref{obs:fatless_multicover}, we have that each vertex $u \in V(H)$ is covered exactly $c(u)$ times; moreover, since true twins appear multiple times and types are equivalence relations, we can attribute to each $t_i^q$ any of the $|T_i|$ colors available, as long as no two receive the same color.
    Therefore, $\varphi$, after properly allocating the true twin classes, is a $k$-partition of $V(G)$.
    Due to Theorem~\ref{thm:lifted_transversal}, every $\Lift{B}(\psi_i)$ is a complement of transversal and therefore $\varphi$ is a valid $k$-biclique-coloring of $G$.
    
    For the converse, we first make use of Lemma~\ref{lem:biclique_false_twins} to guarantee that every false twin class is in at most two color classes. In particular, if two colors are required we force $f_j^1$ to have the smallest color and $F_j \setminus \{f_j^1\}$ to have the other one.
    Afterwards, for every color class $\varphi_i$, we take $\psi_i = \Proj{B}(\varphi_i)$.
    Note that, each color class has at most one element of each $T_i$. Also, for each $F_j$ and any two distinct color classes $\varphi_l, \varphi_r$, $\Proj{B}(\varphi_l) \cap \Proj{B}(\varphi_r) \cap \Proj{B}(F_j) = \emptyset$, since $f_j^1$ has a different color from $F_j \setminus \{f_j^1\}$.
    These observations guarantee that $\Lift{B}(\psi_i) = \varphi_i$ and, because of Theorem~\ref{thm:lifted_transversal}, $\psi_i \in \transs{B}(H)$.
    Finally, $\psi = \{\psi_1, \dots, \psi_k\}$ will be a valid $k$-multicover of $H$ because every vertex of $V(H)$ will be covered the required amount of times.
\end{proof}
%\end{tproof}

Note that the size of the largest true twin class is exactly the largest coverage requirement $b$ of our \pname{Set Multicover} instance.
Moreover, since we need at least $b$ colors to biclique color $G$, it holds that $b \leq k$.

\begin{theorem}
    \label{thm:fpt_biclique}
    \pname{Biclique Coloring} can be solved in $\bigOs{(k+2)^{2d}}$.
\end{theorem}

%\begin{tproof}
\begin{proof}
        Start by computing the type partition of $G$ in $\bigO{n^3}$ time and building $H$ in $O(n + m)$.
        Afterwards, solve the corresponding \pname{Set Multicover} instance in $\bigOs{(b+2)^{2d}}$ time using Theorem~\ref{thm:set_multicover} and lift the multicover using the construction described in the proof of Theorem~\ref{thm:lifted_multicover} in $\bigO{n}$.
\end{proof}
%\end{tproof}

Another option would be not to contract true twin classes, keeping all such vertices in the projected graph, which would effectively yield a kernel linear on the product $kd$.
The brute force approach would yield a running time of $\bigOs{k^{kd}3^{kd/3}}$: we could verify if one of the $k^{kd}$ possible colorings is a proper $k$-biclique-coloring by checking if none of the $\bigO{3^{kd/3}}$ maximal bicliques is monochromatic.
We could refine our algorithm and use Theorem~\ref{thm:exact_biclique} to solve the problem in $\bigOs{2^{kd}}$, which is no better than $(k+2)^{2d} \approx k^{2d} = 2^{2d\log k}$.
\subsection{\pname{Clique Coloring}}

For \pname{Clique Coloring}, a type class with a single vertex is treated as a false twin class.
Unlike \pname{Biclique Coloring}, both true and false twin classes are well behaved, one of the reasons we get a much better algorithm for this problem.


\begin{lemma}
    \label{lem:clique_false_twins}
    Given $G$ and a false twin class $F \subset V(G)$, any  $k$-clique-coloring~$\varphi'$ can be changed into a $k$-clique-coloring $\varphi$ such that $|\varphi(F)| = 1$.
\end{lemma}

%    \begin{tproof}
\begin{proof}
    If $|\varphi'(F)| = 1$, we are done.
    Otherwise, there exists $f_1, f_2 \in F$ such that $\varphi'(f_1) \neq \varphi'(f_2)$.
    For every maximal clique $C_1$ where $f_1 \in C_1$, define $C' = C \setminus \{f_1\}$ and note that $C_2 = C' \cup \{f_2\}$ is also a maximal clique.
    Since $\varphi'$ is an  coloring $|\varphi'(C') \cup \{\varphi'(f_1)\}| \geq 2$.
    Therefore, making $\varphi(f_2) = \varphi(f_1) = \varphi'(f_1)$ does not make $|\varphi(C_2)| = 1$. 
    Repeating this until $|\varphi(F)| = 1$ does not make any clique that intercepts $F$ monochromatic and completes the proof.
\end{proof}
%    \end{tproof}
    
\begin{lemma}
    \label{lem:clique_true_twins}
    Given $G$ and  a true twin class $T \subseteq V(G)$, any  $k$-clique-coloring $\varphi'$ can be changed into a $k$-clique-coloring $\varphi$ such that $|\varphi(T)| \leq 2$.
\end{lemma}

%\begin{tproof}
\begin{proof}
    If $|\varphi'(T)| \leq 2$, we are done.
    Otherwise, there exists $t_1, t_2, t_3 \in T$ with different colors.
    Note that, for every maximal clique $C$ that intercepts $T$, $C \subseteq T$.
    Therefore, $|\varphi'(C)| \geq |\varphi'(T)| \geq 3$.
    By making $\varphi(t_1) = \varphi'(t_1)$ and $\varphi(t_3) = \varphi(t_2) = \varphi'(t_2)$ we have $|\varphi(C)| \geq |\varphi(T)| \geq 2$.
    Repeating this process until $|\varphi(T)| \leq 2$ does not make any clique that intercepts $T$ monochromatic and the proof follows.
\end{proof}
%\end{tproof}

\begin{definition}[C-Projection and C-Lifting]
    Let $T_i$ be any true twin class and $F_j$ be any false twin class.
    We define the following projection rules:
    for $t_i^1 \in T_i,\ \Proj{C}(t_i^1) = \{t'{_i^1}\}$,
    $\forall t_i^q \in T_i \setminus \{t_i^1\}$, $\Proj{C}(t_i^q) = \{t'{_i^2}\}$,
    $\forall f_j^r \in F_j$, $\Proj{C}(f_j^r) = \{f'_j\}$
    and $\Proj{C}(X) = \bigcup_{u \in X} \Proj{C}(u)$.
    
    Lifting rules are defined as $\Lift{C}(t'{_i^1}) = \{t_i^1\}$,
    $\Lift{C}(t'{_i^2}) = T_i \setminus \{t_i^1\}$,
    $\Lift{C}(f'_j) = \{f_j^1\}$ and $\Lift{C}(Y) = \bigcup_{u' \in Y} \Lift{C}(u')$.
    Note that $\Proj{C}(\Lift{C}(X)) = X$, $\forall X$.
\end{definition}

\begin{definition}[C-Projected Graph]
    The C-projected graph $H$ of $G$ satisfies $V(H) = \Proj{C}(V(G))$ and $v'_iv'_j \in E(H)$ if and only if there exist~$v_i \in \Lift{C}(v'_i)$ and $v_j \in \Lift{C}(v'_j)$ such that $v_iv_j \in E(G)$. $H$ is an induced subgraph of $G$.
\end{definition}


For the remainder of this section, $G$ will be the input graph to \pname{Clique Coloring} and $H$ the C-Projected graph of $G$.
We show, using Lemma~\ref{lem:lift_proj_clique} and Theorem~\ref{thm:projected_clique_coloring}, that \pname{Clique Coloring} parameterized by neighborhood diversity has a linear kernel.
Note that our results imply that $\CN{C}(G) \leq 2d$.
A straightforward brute force approach would yield an $\bigOs{4^dd^{2d}3^{2d/3}}$-time algorithm: for each of the $k^{2d} \leq (2d)^{2d}$ possible colorings of $H$, we check if none of the $\bigO{3^{2d/3}}$ maximal cliques of $H$ are monochromatic, returning $\YES$ if none are, otherwise $\NOi$.
Instead, we use Theorem~\ref{thm:clique_color_algorithm} and obtain an $\bigOs{2^{2d}}$-time algorithm.


\begin{lemma}
    \label{lem:lift_proj_clique}
    If $C' \in \clq(H)$ then $\Lift{C}(C') \in \clq(G)$. Conversely, if $C \in \clq(G)$ then $\Proj{C}(C) \in \clq(H)$.
\end{lemma}

%\begin{tproof}
\begin{proof}
    Note that $C = \Lift{C}(C')$ is a clique due to the definition of $\Lift{C}$ and the fact that $C'$ is a clique.
    By the contrapositive, suppose that $C$ is not a maximal clique.
    In this case, there is some vertex $u \in V(G)$ such that $C \cup \{u\}$ is a (not necessarily maximal) clique of $G$. Note that either:
    (i) if $u \in T_i$, $T_i \nsubseteq C$ and $u \notin \Lift{C}(t'{_i^1})$ or $u \notin \Lift{C}(t'{_i^2})$, thus $\Proj{C}(u) \notin C'$;
    (ii) if $u \in F_j$, $F_j \cap C = \emptyset$ and $\Proj{C}(u) \notin C'$.
    Since no two vertices of $H$ are lifted to the same vertex of $G$ and $\Proj{C}(u) \notin C'$, it follows that  $\Proj{C}(C \cup \{u\}) = \Proj{C}(C) \cup \Proj{C}(u) = C' \cup \Proj{C}(u)$ is a clique by the definition of $\Proj{C}$.
    
    Clearly, $C' = \Proj{C}(C)$ is a clique of $H$, due to the definition of $\Proj{C}$.
    Suppose, however, that $C' \notin \clq(H)$, which implies that there is some $u' \in V(H)$ such that $C' \cup \{u'\}$ is a clique of $H$ and let $u \in \Lift{C}(u')$.
    By the definition of $\Lift{C}$, $C \subseteq N(u)$ if and only if $\Proj{C}(C) \subseteq N(u')$, which implies that $C'$ is not maximal only if $C$ is not maximal.
    A contradiction that completes the proof.
\end{proof}
%\end{tproof}

\begin{theorem}
    \label{thm:projected_clique_coloring}
     $G$ is $k$-clique-colorable if and only if $H$ is $k$-clique-colorable.
\end{theorem}

%\begin{tproof}
\begin{proof}
    Let $\varphi_G$ be a $k$-clique-coloring of $G$ that complies with Lemmas~\ref{lem:clique_false_twins} and~\ref{lem:clique_true_twins}.
    Without loss of generality, for every $T_i$, we color $t_i^1$ with one color and $T_i \setminus \{t_i^1\}$ with the other, if it exists, otherwise color every vertex of $T_i$ with the same color.
    We define the $k$-clique-coloring of $H$ as $\varphi_H(u') = \varphi_G(u \in \Lift{C}(u'))$, for every $u' \in V(H)$.
    Suppose now that there exists some $C' \in \clq(H)$ such that $|\varphi_H(C')| = 1$.
    By Lemma~\ref{lem:lift_proj_clique}, $\Lift{C}(C')$ is a maximal clique of $G$ and, since $|\varphi_G\left(\Lift{C}(C')\right)| = 1$, it holds that $\Lift{C}(C')$ is a monochromatic maximal clique of $G$ and $\varphi_G$ is not a valid $k$-clique-coloring, which contradicts the hypothesis.
        
    Now, let $\varphi_H$ be a $k$-clique-coloring of $H$, and define $\varphi_G(u) = \varphi_H(u' \in \Proj{C}(u))$.
    By assuming that there exists some $C \in \clq(G)$ such that $\left|\varphi_G(C)\right| = 1$ and using Lemma~\ref{lem:lift_proj_clique}, it is clear that $\left|\varphi_H(\Proj{C}(C))\right| = 1$ which is impossible, since $\varphi_H$ is a valid $k$-clique-coloring of $H$.
\end{proof}
%\end{tproof}

\begin{theorem}
    \label{thm:fpt_clique}
    There is an $\bigOs{2^{2d}}$-time algorithm for \pname{Clique Coloring}.
\end{theorem}

%\begin{tproof}
\begin{proof}
    Start by computing the optimal type partition of $G$ in $\bigO{n^3}$-time and building $H$ in $\bigO{n + m}$.
    Then color $H$ in $\bigOs{2^{2d}}$-time by Theorem~\ref{thm:clique_color_algorithm} and lift the coloring using the construction described in Theorem~\ref{thm:projected_clique_coloring} in $\bigO{n}$.
\end{proof}
%\end{tproof}

\subsection{A lower bound under ETH}

We now proceed to show that the algorithm described in Theorem~\ref{thm:fpt_clique} is optimal, up to a constant in the exponent, under the assumption that ETH holds.
Before proceeding, we recall the canonical problem associated with the $\SiP{2}$ class.

\problem{2-quantified satisfability (\pname{QSAT}$_2$)}
{An $n_1 + n_2$ variable 3DNF formula $\varphi(\boldsymbol{x}, \boldsymbol{y})$, on $\boldsymbol{x}$ and $\boldsymbol{y}$.}
{Is there $\boldsymbol{x} \in \{0,1\}^{n_1}$ such that for every $\boldsymbol{y} \in \{0,1\}^{n_2}$, $\varphi(\boldsymbol{x}, \boldsymbol{y}) = 1$?}

\begin{lemma}
    There is no $\bigOs{2^{o(n_1 + n_2)}}$ algorithm for an instance of \pname{QSAT}$_2$ on $n_1 + n_2$ variables if ETH holds.
\end{lemma}

\begin{proof}
    By the counter-positive, suppose that there is an algorithm $\prod$ for \pname{QSAT}$_2$ with complexity $\bigOs{2^{o(n_1 + n_2}}$ and let $\langle\boldsymbol{x}, \boldsymbol{y}, \varphi(\boldsymbol{x},\boldsymbol{y})\rangle$ be an instance of \pname{QSAT}$_2$ as in the definition of \pname{QSAT}$_2$.
    With $\prod$ in hand, we can solve $\neg\left(\exists \boldsymbol{x} \forall \boldsymbol{y}\varphi(\boldsymbol{x}, \boldsymbol{y})\right) \equiv \forall \boldsymbol{x} \exists \boldsymbol{y} \neg\varphi(\boldsymbol{x}, \boldsymbol{y})$ simply by negating the output of $\prod$.
    Note that, since $\varphi(\boldsymbol{x}, \boldsymbol{y})$ is in 3DNF, $\neg\varphi(\boldsymbol{x}, \boldsymbol{y})$ is in 3CNF.
    The case where $n_1 = 0$ is precisely \textsc{3sat}, and we have an algorithm that solves it in $\bigOs{2^{o(n_2)}}$, implying that ETH is false.
\end{proof}

\begin{figure}[!tb]
    \centering
        \begin{tikzpicture}[scale=1]
            \GraphInit[unit=3,vstyle=Normal]
            \SetVertexNormal[Shape=circle, FillColor=black, MinSize=2pt]
            \tikzset{VertexStyle/.append style = {inner sep = \inners, outer sep = \outers}}
            \begin{scope}[shift={(-2.5cm,0)}]
                \Vertex[x=0,y=0,LabelOut,Lpos=90,Ldist=0pt,Math,L={x_1}]{x-1}
                \Vertex[x=1,y=0,LabelOut,Lpos=90,Ldist=0pt,Math,L={\overline{x}_1}]{o-x-1}
                \Vertex[x=0,y=1,LabelOut,Lpos=90,Ldist=0pt,Math,L={x_1'}]{x-1-l}
                \Vertex[x=1,y=1,LabelOut,Lpos=90,Ldist=0pt,Math,L={\overline{x}_1'}]{o-x-1-l}
                \Edges[style={bend left=50}](x-1, x-1-l)
                \Edge(x-1-l)(o-x-1-l)
                \Edges[style={bend right=50}](o-x-1-l, o-x-1)
                \Edge[style={dashed, thin}](x-1)(o-x-1)
            \end{scope}
            \begin{scope}[shift={(-0.5cm,0)}]
                \Vertex[x=0,y=0,LabelOut,Lpos=90,Ldist=0pt,Math,L={x_2}]{x-2}
                \Vertex[x=1,y=0,LabelOut,Lpos=90,Ldist=0pt,Math,L={\overline{x}_2}]{o-x-2}
                \Vertex[x=0,y=1,LabelOut,Lpos=90,Ldist=0pt,Math,L={x_2'}]{x-2-l}
                \Vertex[x=1,y=1,LabelOut,Lpos=90,Ldist=0pt,Math,L={\overline{x}_2'}]{o-x-2-l}
                \Edges[style={bend left=50}](x-2, x-2-l)
                \Edge(x-2-l)(o-x-2-l)
                \Edges[style={bend right=50}](o-x-2-l, o-x-2)
                \Edge[style={dashed, thin}](x-2)(o-x-2)
            \end{scope}
            \begin{scope}[shift={(1.5cm,0)}]
                \Vertex[x=0,y=0,LabelOut,Lpos=90,Ldist=0pt,Math,L={y_1}]{y-1}
                \Vertex[x=1,y=0,LabelOut,Lpos=90,Ldist=0pt,Math,L={\overline{y}_1}]{o-y-1}
                \Vertex[x=0,y=1,LabelOut,Lpos=90,Ldist=0pt,Math,L={y_1'}]{y-1-l}
                \Vertex[x=1,y=1,LabelOut,Lpos=0,Ldist=0pt,Math,L={\overline{y}_1'}]{o-y-1-l}
                \Edges[style={bend left=50}](y-1, y-1-l)
                \Edges[style={bend right=50}](o-y-1-l, o-y-1)
                \Edge[style={dashed, thin}](y-1)(o-y-1)
            \end{scope}
            \begin{scope}[shift={(3.5cm,0)}]
                \Vertex[x=0,y=0,LabelOut,Lpos=90,Ldist=0pt,Math,L={y_2}]{y-2}
                \Vertex[x=1,y=0,LabelOut,Lpos=90,Ldist=0pt,Math,L={\overline{y}_2}]{o-y-2}
                \Vertex[x=0,y=1,LabelOut,Lpos=0,Ldist=0pt,Math,L={y_2'}]{y-2-l}
                \Vertex[x=1,y=1,LabelOut,Lpos=90,Ldist=0pt,Math,L={\overline{y}_2'}]{o-y-2-l}
                \Edges[style={bend left=50}](y-2, y-2-l)
                \Edges[style={bend right=50}](o-y-2-l, o-y-2)
                \Edge[style={dashed, thin}](y-2)(o-y-2)
            \end{scope}
            
            \begin{scope}[shift={(-1.0cm,-2cm)}]
                \Vertex[x=0,y=0,LabelOut,Lpos=270,Ldist=0pt,Math,L={p_1}]{p-1}
                \Vertex[x=1.25,y=0,LabelOut,Lpos=270,Ldist=0pt,Math,L={p_1'}]{p-1-l}
                \Edges(p-1, p-1-l)
            \end{scope}
            \Edge(p-1)(o-x-1)
            \Edge(p-1)(x-2)
            \Edge(p-1)(o-y-1)
            \Edges(y-2,p-1, o-y-2)
            \begin{scope}[shift={(1.5cm,-2cm)}]
                \Vertex[x=0,y=0,LabelOut,Lpos=270,Ldist=0pt,Math,L={p_2}]{p-2}
                \Vertex[x=1.25,y=0,LabelOut,Lpos=270,Ldist=0pt,Math,L={p_2'}]{p-2-l}
                \Edges(p-2, p-2-l)
            \end{scope}
            \Edge(p-2)(x-2)
            \Edge(p-2)(y-1)
            \Edge(p-2)(o-y-2)
            \Edges(x-1,p-2, o-x-1)
            \Edge(p-1-l)(p-2)
            \begin{scope}[shift={(4cm,-2cm)}]
                \Vertex[x=0,y=0,LabelOut,Lpos=270,Ldist=0pt,Math,L={p_3}]{p-3}
                \Vertex[x=-1,y=4,LabelOut,Lpos=90,Ldist=0pt,Math,L={p_3'}]{p-3-l}
            \end{scope}
            \Edge(p-3)(o-x-1)
            \Edge(p-3)(x-2)
            \Edge(p-3)(y-2)
            \Edges(y-1,p-3, o-y-1)
            \Edge(p-2-l)(p-3)
            
            \draw[dashed] (-2.7,-0.15) rectangle (4.7,0.15);
            \Edge(p-3-l)(y-1-l)
            \Edge(p-3-l)(o-y-1-l)
            \Edge(p-3-l)(y-2-l)
            \Edge(p-3-l)(o-y-2-l)
            
            \draw[] (4,-2) arc (-90:120:2.12); 
        
        \end{tikzpicture}
    \caption{Construction for the formula $\varphi(\boldsymbol{x},\boldsymbol{y}) = (\overline{x}_1 \wedge x_2 \wedge \overline{y}_1) \vee (x_2 \wedge y_1 \wedge \overline{y}_2) \vee (\overline{x}_1 \wedge x_2 \wedge y_2)$.}
    \label{fig:reduction}
\end{figure}

\begin{theorem}
    If ETH holds, there is no $\bigOs{2^{o(d)}}$ time algorithm for \textsc{clique 2-coloring} parameterized by the neighborhood diversity $d$ of the graph.
\end{theorem}

\begin{proof}
    Let $\Phi = \langle\boldsymbol{x}, \boldsymbol{y}, \varphi(\boldsymbol{x},\boldsymbol{y})\rangle$ be an instance of \pname{QSAT}$_2$ as in the problem's definition.
    We construct the graph $G$ for \pname{Clique bicoloring} as follows:
    For each $x_i \in \boldsymbol{x}$, $G$ has 4 vertices $x_i,x_i',\overline{x}_i',\overline{x}_i$ and the edges $x_ix_i',x_i'\overline{x}_i',\overline{x}_i'\overline{x}_i$.
    For each $y_j \in \boldsymbol{y}$, $G$ also has 4 vertices $y_j,y_j',\overline{y}_j',\overline{y}_j$ but only the edges $y_jy_j',\overline{y}_j\overline{y}_j'$.
    Vertices $x_i,\overline{x}_i, y_j, \overline{y}_j$ form a clique minus the edges between a literal and its negation.
    For each clause $p_l \in \varphi(\boldsymbol{x}, \boldsymbol{y})$, add two vertices $p_l,p_l'$ to $G$ and an edge between~$p_l$ and $x_i$ ($\overline{x}_i$) if $x_i$ ($\overline{x}_i$) is in clause $p_l$. If neither $x_i$ nor $\overline{x}_i$ are in clause $p_l$, connect $p_l$ to both $x_i$ and $\overline{x}_i$.
    The same is done between $p_l$ and each $y_j$.
    Vertex $p_m'$ is adjacent to every $y_j'$ and every $\overline{y}_j'$;
    furthermore, $p_1p_1'\dots p_mp_m'$ is an induced path of $G$.
    By~\cite{clique_coloring_complexity}, $G$ is a $\YES$ instance if and only if $\Phi$ is also a $\YES$ instance.
    We now show that $\nd(G)$ is linearly bounded by the size of $\Phi$.
    
    Define $\eta = \{x_1, \overline{x}_1, \dots, \overline{x}_{n_1}, y_1, \overline{y}_1, \dots, \overline{y}_{n_2}\}$ and $P = \bigcup_{l \leq m}\{p_l, p_l'\}$, $\eta' = \{x_1', \overline{x}_1', \dots, \overline{x}_{n_1}', y_1', \overline{y}_1', \dots, \overline{y}_{n_2}'\}$, $P = \{p_l \mid l \leq m\}$ and $P' = \{p_l' \mid l \leq m\}$
    For any $\{a,b\} \subseteq \eta \cup \eta'$, it is straightforward to verify that $N(a) \setminus N(b)$, $N[a] \setminus N[b]$, $N(b) \setminus N(a)$ and $N[b] \setminus N[a]$ are non-empty, which implies that $a$ and $b$ are neither false nor true twins.
    For any $a \in \eta'$ and any $b \in P \cup P'$, it is easy to see that $a$ and $b$ cannot be of the same type.
    If $a \in \eta$ and $b \in P$, since $\varphi(\boldsymbol{x}, \boldsymbol{y})$ is in 3DNF, there is at least one variable not adjacent to $b$ which is adjacent to $a$, since $\eta$ induces a clique minus a matching and, consequently, $a$ and $b$ are not of the same type.
    If $a \in \eta \cup P$ and $b \in P'$ or $\{a,b\} \subseteq P'$, it is trivial to verify that $a$ and $b$ are neither true nor false twins.
    For $\{a,b\} \subseteq P$, since no two clauses are equal, it follows that $a$ and $b$ are not of the same type.
    
    As such, we conclude that each vertex of $G$ is in a different type and, consequently, it has $d = \nd(G) = 4(n_1 + n_2) + 2m$ which is $\bigOs{n_1 + n_2 + m}$ and implies that there is no $\bigOs{2^{o(d)}}$ time algorithm for \textsc{clique 2-coloring} parameterized by neighborhood diversity unless ETH fails.
\end{proof}
\section{Concluding remarks}

In this chapter, we investigated three partitioning problems that belong to the class of coloring problems.
Namely, \pname{Equitable Coloring}, \pname{Clique Coloring}, \pname{Biclique Coloring}.
For \pname{Equitable Coloring}, we developed novel parameterized reductions from \pname{Bin Packing}, which is $\W[1]$-$\Hard$ when parameterized by number of bins.
These reductions showed that \pname{Equitable Coloring} is $\W[1]-\Hard$ in three more cases: (i) if we restrict the problem to block graphs and parameterize by the number of colors, treewidth and diameter; (ii) on the disjoint union of split graphs, a case where the connected case is polynomial; (iii) \pname{equitable coloring} of $K_{1,r}$ interval graphs, for any $r \geq 4$, remains hard even if we parameterize by the number of colors, treewidth and maximum degree.
This, along with a previous result by~\cite{claw_free_de_werra}, establishes a dichotomy based on the size of the largest induced star: for $K_{1,r}$-free graphs, the problem is solvable in polynomial time if $r \leq 2$, otherwise it is $\W[1]-\Hard$.
These results significantly improve the ones by~\cite{colorful_treewidth} through much simpler proofs and in very restricted graph classes.
Since the problem remains hard even for many natural parameterizations, we resorted to a more exotic one -- the treewidth of the complement graph.
By applying standard dynamic programming techniques on tree decompositions and the fast subset convolution machinery of~\cite{fourier_mobius}, we obtain an $\FPT$ algorithm when parameterized by the treewidth of the complement graph.
We also presented an \XP\ algorithm parameterized by number of colors when the input graph is known to be chordal.
Natural future research directions include the identification and study of other uncommon parameters that may aid in the design of other $\FPT$ algorithms.
Revisiting \pname{Clique Partitioning} when parameterized by $k$ and $r$ is also of interest, since its a related problem to \pname{Equitable Coloring} and the complexity of its natural parameterization is yet unknown.

As to the other problems, we showed that, much like \pname{Clique Coloring}, \pname{Biclique Coloring} can be solved in $\bigOs{2^{n}}$-time using the inclusion-exclusion principle.
Also of interest is the nice behavior \pname{Clique Coloring} presents when parameterized by neighborhood diversity, which enabled us to apply very simple reduction rules and obtain an $\bigOs{2^{2d}}$-time \FPT algorithm.
Moreover, said algorithm has optimal running time, assuming that \ETH\ holds.
For \pname{Biclique Coloring}, however, we were unable to provide an \FPT\ algorithm when considering solely neighborhood diversity and had to include the size of the largest true twin class -- which is a lower bound to the biclique chromatic number -- to obtain a parameterized algorithm. 
As such, we are led to believe that \pname{Biclique Coloring} parameterized by neighborhood diversity is not in $\FPT$.
Much of the exploratory work on different graph classes and parameters remains to be done for \pname{Biclique Coloring}, and it may be an interesting venue for future work.