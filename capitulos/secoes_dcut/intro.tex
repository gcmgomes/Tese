
In this chapter, we aim at generalizing several of the previously reported results for \pname{Matching Cut}.
First, by using a reduction inspired by Chvátal's~\citep{chvatal_matching_cut}, we show that for every $d \geq 1$, \pname{$d$-Cut} is \NP-hard even when restricted to $(2d+2)$-regular graphs and that, if $\Delta(G) \leq d+2$, finding a $d$-cut can be done in polynomial time. The degree bound in the \NP-hardness result is unlikely to be improved: if we had an \NP-hardness result for \pname{$d$-Cut} restricted to $(2d+1)$-regular graphs, this would disprove the conjecture about the existence of internal partitions on $r$-regular graphs~\citep{DeVos09,internal_partition_regular6,internal_partition_regular3_4} for $r$ odd, unless every problem in $\NP$ could be solved in \textit{constant} time.
Afterwards, we present a simple exact exponential algorithm that, for every $d \geq 1$, runs in time $\bigOs{c_d^n}$ for some constant $c_d < 2$, hence improving over the trivial brute-force algorithm running in time $\bigOs{2^n}$.

We then proceed to analyze the problem in terms of its parameterized complexity.
Section~\ref{sec:param} begins with a proof, using the treewidth reduction technique of \citep{marx_treewidth_reduction}, that \pname{$d$-Cut} is $\FPT$ parameterized by the maximum number of edges crossing the cut.
Afterwards, we present a dynamic programming algorithm for \pname{$d$-Cut} parameterized by treewidth running in time $\bigOs{2^{\tw(G)+1}(d+1)^{2\tw(G) + 2}}$; in particular, for $d=1$ this algorithm runs in time $\bigOs{8^{\tw(G)}}$ and improves the one given by \cite{matching_cut_structural} for \pname{Matching Cut}, running in time  $\bigOs{12^{\tw(G)}}$.
By employing the cross-composition framework of \cite{cross_composition}, and using a reduction similar to the one in~\citep{matching_cut_ipec}, we show that, unless $\NP \subseteq \coNP/\poly$, there is no polynomial kernel for \pname{$d$-Cut} parameterized simultaneously by the number of crossing edges, the maximum degree, and the treewidth of the input graph.
We then present a polynomial kernel and an $\FPT$ algorithm when parameterizing by the distance to cluster.
This polynomial kernel is our main technical contribution, and it is strongly inspired by the technique presented by \cite{matching_cut_ipec} for \pname{Matching Cut}. Finally, we give an $\FPT$ algorithm parameterized by the distance to co-cluster, denoted by $\dcc(G)$.
These results imply fixed-parameter tractability for \pname{$d$-Cut} parameterized by $\tau(G)$. 
