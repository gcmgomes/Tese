\section{Definitions and Related Work}

A \tdef{cut} of a graph $G = (V, E)$ is a bipartition of its vertex set $V(G)$ into two non-empty sets, denoted by $(A,B)$.
The set of all edges with one endpoint in $A$ and the other in $B$ is the \tdef{edge cut}, or the set of \tdef{crossing edges}, of $(A,B)$.
A \tdef{matching cut} is a (possibly empty) edge cut that is a matching, that is, such that its edges are pairwise vertex-disjoint. Equivalently, $(A, B)$ is a matching cut of $G$ if and only if every vertex is incident to at most one crossing edge of $(A, B)$~\citep{matching_cut_graham, chvatal_matching_cut}, that is, it has at most one neighbor across the cut.
The \pname{Matching Cut} problem is, thus, the task of deciding whether a graph admits a matching cut.
Figure~\ref{fig:matching_cut} gives an example of a graph with a matching cut.

\problem{Matching Cut}{A graph $G$.}{Does $G$ have a matching cut?}


\begin{figure}[!htb]
        \centering
        \begin{tikzpicture}[rotate = 0]
                %\draw[help lines] (-5,-5) grid (5,5);
                \GraphInit[unit=3,vstyle=Normal]
                \SetVertexNormal[Shape=circle, FillColor = black, MinSize=3pt]
                \tikzset{VertexStyle/.append style = {inner sep = \inners, outer sep = \outers}}
                \SetVertexNoLabel
                
                \begin{scope}[rotate=90, shift={(0, 2)}]
                    \tikzset{VertexStyle/.append style = {shape = rectangle, inner sep = 2pt}}
                    \grComplete[RA=1]{2}
                \end{scope}
                \begin{scope}[rotate = 90, shift={(0, -1)}]
                    \grComplete[RA=1,prefix=b]{2}
                \end{scope}
                \begin{scope}
                    \tikzset{VertexStyle/.append style = {shape = rectangle, inner sep = 2pt}}
                    \Vertex[x=-1, y=0]{f}
                \end{scope}
                \Vertex[x=0, y=0]{g}
                \Edges(g,b0,a0,f,a1,b1,g,f)
        \end{tikzpicture}
        \caption{Example of a matching cut. Square vertices would be assigned to $A$, circles to $B$.\label{fig:matching_cut}}
    \end{figure}


Motivated by an open question posed by \cite{matching_cut_ipec} during the presentation of their article,  we investigate a natural generalization that arises from this alternative definition
For a positive integer $d \geq 1$, a \tdef{$d$-cut} is a a cut $(A, B)$ such that each vertex has at most $d$ neighbors across the partition, that is, every vertex in $A$ has at most $d$ neighbors in $B$, and vice-versa. Note that a $1$-cut is a matching cut.
As expected, not every graph admits a $d$-cut, and the \pname{$d$-Cut} problem is the problem of, for a fixed integer $d \geq 1$, deciding whether or not an input graph $G$ has a $d$-cut.

\problem{$d$-Cut}{A graph $G$.}{Does $G$ admit a $d$-cut?}

When $d=1$, the problem is known as \pname{Matching Cut}.
Graphs with no matching cut first appeared in Graham's manuscript~\citep{matching_cut_graham} under the name of \textit{indecomposable graphs}, presenting some examples and properties of decomposable and indecomposable graphs, leaving their recognition as an open problem.
In answer to Graham's question, \cite{chvatal_matching_cut} proved that the problem is \NP-hard for graphs of maximum degree at least four and polynomially solvable for graphs of maximum degree at most three; in fact, as shown by \cite{matching_cut_moshi}, every graph of maximum degree three and at least eight vertices has a matching cut.

Chvátal's results spurred a lot of research on the complexity of the problem~\citep{matching_cut_ipec,matching_cut_structural,matching_cut_tcs, matching_cut_diameter, matching_cut_planar, matching_cut_series_parallel, stable_cutset_line_graphs}.
In particular, \cite{matching_cut_planar} showed that \pname{Matching Cut} remains \NP-hard for planar graphs of maximum degree four and for planar graphs of girth five;
\cite{stable_cutset_line_graphs} gave an \NP-hardness reduction for bipartite graphs of maximum degree four;
\cite{matching_cut_diameter} proved that \pname{Matching Cut} is \NP-hard for graphs of diameter at least three, and presented a polynomial-time algorithm for graphs of diameter at most two.
Beyond planar graphs, \cite{matching_cut_planar} also proves that the matching cut property is expressible in monadic second order logic and, by Courcelle's Theorem~\citep{courcelle_theorem}, it follows that \pname{Matching Cut} is $\FPT$ when parameterized by the treewidth of the input graph; he concludes with a proof that the problem admits a polynomial-time algorithm for graphs of bounded cliquewidth.

\cite{matching_cut_tcs} noted that Chv\'atal's original reduction also shows that, unless the Exponential Time Hypothesis fails, there is no algorithm solving \pname{Matching Cut} in time $2^{o(n)}$ on $n$-vertex input graphs.
Also in~\citep{matching_cut_tcs}, the authors provide a first branching algorithm, running in time $\bigOs{2^{n/2}}$, a single-exponential $\FPT$ algorithm when parameterized by the vertex cover number $\tau(G)$, and an algorithm generalizing the polynomial cases of line graphs~\citep{matching_cut_moshi} and claw-free graphs~\citep{matching_cut_planar}.
\cite{matching_cut_tcs} also asked for the existence a single-exponential algorithm parameterized by treewidth.
In response, \cite{matching_cut_structural} provided a $\bigOs{12^{\tw(G)}}$ algorithm for \pname{Matching Cut} using nice tree decompositions, along with $\FPT$ algorithms for other structural parameters, namely neighborhood diversity, twin-cover, and distance to split graph.

The natural parameter -- the number of edges crossing the cut -- has also been considered.
Indeed, \cite{marx_treewidth_reduction} tackled the \pname{Stable Cutset} problem, to which \pname{Matching Cut} can be easily reduced via the line graph, and through a breakthrough technique showed that this problem is $\FPT$ when parameterized by the maximum size of the stable cutset.
Recently, \cite{matching_cut_ipec} improved on the results of \cite{matching_cut_tcs}, providing an exact exponential algorithm for \pname{Matching Cut} running in  time $\bigOs{1.3803^n}$, as well as $\FPT$ algorithms parameterized by the distance to a cluster graph and the distance to a co-cluster graph, which improve the algorithm parameterized by the vertex cover number, since both parameters are easily seen to be smaller than the vertex cover number.
For the distance to cluster parameter, they also presented a quadratic kernel; while for a combination of treewidth, maximum degree, and number of crossing edges, they showed that no polynomial kernel exists unless $\NP \subseteq \coNP/\poly$.

A problem  closely related to \pname{$d$-Cut} is that of \pname{Internal Partition}, first studied by \cite{internal_partition_thomassen}.
In this problem, we seek a bipartition of the vertices of an input graph such that every vertex has at least as many neighbors in its
own part as in the other part. Such a partition is called an \tdef{internal partition}.
Usually, the problem is posed in a more general form: given functions $a,b: V(G) \rightarrow \mathbb{Z}_+$, we seek a bipartition $(A,B)$ of $V(G)$ such that every $v \in A$ satisfies $\dgr_A(v) \geq a(v)$ and every $u \in B$ satisfies $\dgr_B(u) \geq b(u)$, where $\dgr_A(v)$ denotes the number of neighbors of $v$ in the set $A$. Such a partition is called an \tdef{$(a,b)$-internal partition}.

Originally, Thomassen asked in~\citep{internal_partition_thomassen} whether for any pair of positive integers $s,t$, a graph $G$ with $\delta(G) \geq s + t + 1$ has a vertex bipartition $(A,B)$ with $\delta(G[A]) \geq s$ and $\delta(G[B]) \geq t$.
\cite{internal_partition_stiebitz} answered that, in fact, for any graph $G$ and any pair of functions $a,b: V(G) \rightarrow \mathbb{Z}_+$ satisfying $\dgr(v) \geq a(v) + b(v) + 1$ for every $v \in V(G)$, $G$ has an $(a,b)$-internal partition.
Following Stiebitz's work, \cite{internal_partition_triangle_free} showed that if $G$ is triangle-free, then the pair $a,b$ only needs to satisfy $\dgr(v) \geq a(v) + b(v)$.
More recently, \cite{internal_partition_c4_free} proved that, if $G$ is $\{C_4, K_4, \text{diamond}\}$-free, then $\dgr(v) \geq a(v) + b(v) - 1$ is enough.
Furthermore, they also showed, for any pair $a,b$, a family of graphs such that $\dgr(v) \geq a(v) + b(v) - 2$ for every $v \in V(G)$ that do not admit an $(a,b)$-internal partition.

It is conjectured that, for every positive integer $r$, there exists some constant $n_r$ for which every $r$-regular graph with more than $n_r$ vertices has an internal partition~\citep{DeVos09,internal_partition_regular6} (the conjecture for $r$ even appeared first in~\citep{internal_partition_regular3_4}).
The cases $r \in \{3,4\}$ have been settled by \cite{internal_partition_regular3_4}; the case $r=6$ has been verified by \cite{internal_partition_regular6}.
This latter result implies that every 6-regular graph of sufficiently large size has a 3-cut.