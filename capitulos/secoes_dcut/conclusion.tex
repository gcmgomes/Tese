\section{Concluding remarks}


We presented a series of algorithms and complexity results; many questions, however, remain open.
For instance, all of our algorithms have an exponential dependency on $d$ on their running times.
While we believe that such a dependency is an intrinsic property of \pname{$d$-cut}, we have no proof for this claim. Similarly, the existence of a \textit{uniform} polynomial kernel parameterized by the distance to cluster, i.e., a kernel whose degree does not depend on $d$, remains an interesting open question.

Also in terms of running time, we expect the constants in the base of the exact exponential algorithm to be improvable. However, exploring small structures that yield non-marginal gains as branching rules, as done by Komusiewicz et al.~\cite{matching_cut_ipec} for $d=1$ does not seem a viable approach, as the number of such structures appears to rapidly grow along with $d$.

%\ig{uniform kernel?}


The distance to cluster kernel is hindered by the existence of clusters of size between $d+2$ and $2d$, an obstacle that is not present in the \pname{Matching Cut} problem.
Aside from the extremal argument presented, we know of no way of dealing with them.
We conjecture that it should be possible to reduce the total kernel size from $\bigO{d^2\dc(G)^{2d+1}}$ to $\bigO{d^2\dc(G)^{2d}}$, matching the size of the smallest known kernel for \pname{Matching Cut}~\cite{matching_cut_ipec}.

We also leave open to close the gap between the known polynomial and $\NP$-hard cases in terms of maximum degree.
We showed that, if $\Delta(G) \leq d+2$ the problem is easily solvable in polynomial time, while for graphs with $\Delta(G) \geq 2d+2$, it is $\NP$-hard.
But what about the gap $d+3 \leq \Delta(G) \leq 2d+1$?
After much effort, we were unable to settle any of these cases.
In particular, we are very interested in \pname{2-Cut},  which has a single open case, namely when $\Delta(G) = 5$.
After some weeks of computation, we found no graph with more than 18 vertices and maximum degree five that had no $2$-cut, in agreement with the computational findings of Ban and Linial~\cite{internal_partition_regular6}.
Interestingly, all graphs on 18 vertices without a $2$-cut are either 5-regular or have a single pair of vertices of degree 4, which are actually adjacent.
In both cases, the graph is maximal in the sense that we cannot add edges to it while maintaining the degree constraints.
We recall the initial discussion about the \pname{Internal Partition} problem; closing the gap between the known cases for \textsc{$d$-Cut} would yield significant advancements on the former problem.

Finally, the smallest $d$ for which $G$ admits a $d$-cut may be an interesting additional parameter to be considered when more traditional parameters, such as treewidth, fail to provide $\FPT$ algorithms by themselves.
Unfortunately, by Theorem~\ref{thm:regular_nph}, computing this parameter is not even in $\XP$, but, as we have shown, it can be computed in $\FPT$ time under many different parameterizations. 
