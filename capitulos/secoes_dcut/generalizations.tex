\section{Other Generalizations for \pname{Matching Cut}}

In this section, we describe two other generalizations of \pname{Matching Cut} that we have investigated.
For the first, \pname{$\ell$-Nested Matching Cut}, we present an exponential time algorithm and an attempt at using algorithms for \pname{Matching Cut} as a black box for this problem.
Most results given for \pname{$d$-Cut} and \pname{Matching Cut} can be adapted for this problem, but the arguments are very similar and do not appear to provide additional insights on the problem.
The second problem we discuss here has been dubbed as \pname{$p$-way Matching Cut}.
Unlike \pname{$d$-Cut} and \pname{$\ell$-Nested Matching Cut}, this version is much more challenging, and we limit ourselves to some attempts on tackling the problem.

\subsection{Nested Cuts}

We have already discussed \pname{$d$-Cut} at length throughout this chapter, so this section will detail some other directions we attempted to explore.
Recall the definition of a matching cut: we would like each vertex of the graph, on the cut $(A,B)$, to have at most one neighbor across the cut.
This can be rephrased to the following: a cut $(A,B)$ is a matching cut if and only if each vertex has at most one neighbor \textit{outside} of its part.
Through this perspective, there is nothing special about the number of parts we want to partition our graph into.
A cut on $\ell$ parts satisfying the above is called an \textit{$\ell$-nested matching cut}, and the decision problem for this generalization is dubbed the \pname{$\ell$-Nested Matching Cut} problem.

\problem{$\ell$-Nested Matching Cut}{A graph $G$.}{Does $G$ admit an $\ell$-nested matching cut?}

Let $\varphi = (A_1, \dots, A_\ell)$ be a partition of $V(G)$ and $\border(A_i)$ be the vertices of $A_i$ with one neighbor outside of $A_i$.
The following observation gives some intuition as to the structure of the positive instances of \pname{$\ell$-Nested Matching Cut}.

\begin{observation}
	\label{obs:nested_cut}
    Let $G$ be a graph and $\ell \geq 3$ an integer.
    $G$ admits an $\ell$-nested matching cut $\varphi$ if and only if there is an ($\ell-1$)-nested matching cut $\varphi' = \{A'_1, \dots, A'_{\ell - 1})\}$ with one $A'_i$ such that the subgraph induced by the vertices in $A'_i$ admits a matching cut $(B_1, B_2)$ where, for every $v \in \border(A'_i)$, it holds that $N[v] \subseteq B_1$ or $N[v] \subseteq B_2$.
\end{observation}

\begin{proof}
	We shall build $\varphi'$ from $\varphi$ as follows: for every $i \in [\ell - 2]$, $A'_i = A_i$, and $\varphi'_{\ell - 1} = \varphi_{\ell - 1} \cup \varphi_\ell$.
	That $\varphi'$ is an $(\ell - 1)$-nested matching cut of $G$ and the subgraph induced by the vertices of $\varphi'_{\ell - 1}$ has a matching cut is a straightforward observation.
	Now, for each $v \in \border(\varphi'_{\ell - 1})$, note that $v \in \border(\varphi_{\ell - 1}) \cup \border(\varphi_\ell)$.
	Consequently, every neighbor of $v$ is in either the same side of the cut $(\varphi_{\ell - 1}, \varphi_\ell)$ as $v$, as we wanted.
	
	For the converse, it suffices to note that $\border(A') \cap \border(\varphi'_i) = \emptyset$ and $\border(B') \cap \border(\varphi'_i) = \emptyset$ precisely because of the constraint that $N[v] \subseteq A'$ or $N[v] \subseteq B'$.
	As such, we can construct the desired $\ell$-nested matching cut as $\varphi = \{\varphi'_1, \dots, \varphi'_{\ell - 2}, A', B'\}$.
\end{proof}

Observation~\ref{obs:nested_cut} is a first step towards an algorithm for \pname{$\ell$-Nested Matching Cut}.
Ideally,  we would use the algorithms for \pname{Matching Cut} as a black box, and then choose one of the available parts of the cut and repeat the process.
What is problematic is that there may be multiple possible matching cuts at a given step, and testing all of them would be quite expensive.
As such, since we still do not know how to exploit Observation~\ref{obs:nested_cut}, we turn our attention to an exact exponential algorithm, through a similar approach used by \cite{matching_cut_ipec}.
Our algorithm consists of four stopping rules, seven reduction and nine branching rules.
At every step of the algorithm we have the sets $\{A_1, \dots, A_\ell, F\}$ such that $\varphi = (A_1, \dots, A_\ell)$ (unless rule R3 applies) is a $\ell$-nested matching cut of the vertices of $V(G) \setminus F$.
For simplicity, we assume that $\delta(G) \geq 2$.
Most of the arguments presented here work with slight modifications to graphs of minimum degree one, but they would unnecessarily complicate the description of the algorithm.

\begin{itemize}
	\item[S1] If there is some $v \in F$ and $i,j \in [\ell]$ such that $\deg_{A_i}(v) \geq 2$ and $\deg_{A_j}(v) \geq 2$, STOP: there is no $\ell$-nested matching cut extending $\varphi$.
	\item[S2] If there is a vertex $v \in F$ with neighbors in three different parts of $\varphi$, STOP: there is no $\ell$-nested matching cut extending $\varphi$.
	\item[S3] If there is an edge $uv$ with $u \in A_i$ and $v \in A_j$ such that $N(u) \cap N(v) \cap F \neq \emptyset$, STOP: there is no $\ell$-nested matching cut extending $\varphi$.
	\item[S4] If there is some $v \in A_i$ with two neighbors outside of $A_i \cup F$, STOP: there is no $\ell$-nested matching cut extending $\varphi$.
\end{itemize}

\begin{itemize}
	\item[R1] If there exists some $v \in A_i$ such that $N(v) \supseteq \{x,y\}$ and $x,y \in F$ and $xy \in E(G)$, add $x,y$ to $A_i$.
	\item[R2] If there exists $v \in F$ and an unique $i \in [\ell]$ with $\deg_{A_i}(v) \geq 2$, add $v$ to $A_i$.
	\item[R3] For every edge $uv$ with $u \in A_i$ and $v \in A_j$, add $N(u) \cap F$ to $A_i$ and $N(v) \cap F$ to $A_j$.
	\item[R4] If there is a pair $u,v \in F$ with $N(u) = N(v) = \{x, y\}$ with $x \in A_i$ and $y \in A_j$, add $u$ to $A_i$ and $v$ to $A_j$.
	\item[R5] If there is a pair $u,v \in F$ with $N(u) = N(v) = \{x, y\}$ with $x \in A_i$ and $y \in F$, add $x$ to $A_i$.
	\item[R6] If there is a vertex $v \in F$ with $N(v) = \{x,y\}$, $x \in A_i$, $y \in A_j$, $N(x) \subseteq A_i \cup \{v\}$, and $N(y) \subseteq A_j \cup \{v\}$, add $v$ to $A_i$.
	\item[R7] If there are vertices $u,v,w \in F$ with $\deg(u) = \deg(v) = \deg(w)$ arranged as in Figure~\ref{fig:r7}, add $\{u,v\}$ to $A_i$ and $w$ to $A_j$.
	
	\begin{figure}[!htb]
		\centering
		\begin{tikzpicture}[scale=0.5]
		%\draw[help lines] (-5,-5) grid (5,5);
		\GraphInit[unit=3,vstyle=Normal]
		\SetVertexNormal[Shape=circle, FillColor=black, MinSize=2pt]
		\tikzset{VertexStyle/.append style = {inner sep = \inners, outer sep = \outers}}
		\Vertex[x=0, y=0, LabelOut, Lpos=180, Math]{u}
		\SetVertexNoLabel
		\foreach \y in {-1, 1} {
			\foreach \x in {0,1,2} {
				\pgfmathsetmacro{\xp}{1.5+1.5*\x}
				\Vertex[x=\xp, y=\y, LabelOut, Lpos=90]{v\y\x}
			}
			\foreach \x in {0,1} {
				\pgfmathtruncatemacro{\xp}{1+\x}
				\Edge(v\y\x)(v\y\xp)
			}
			\Edge(u)(v\y0)
		}
		\tikzset{AssignStyle/.append style = {above=4pt}}
		\AssignVertexLabel{v1}{$v$,$A_i$,$v'$}
		\tikzset{AssignStyle/.append style = {below=8pt}}
		\AssignVertexLabel{v-1}{$w$,$A_j$,$w'$}
		\end{tikzpicture}
		\caption{\label{fig:r7} Rule 7 configuration.}
	\end{figure}
\end{itemize}

For our branching rules, we follow the configurations given by Figure~\ref{fig:nested_branches}, and always branch on vertex $v_1$.
We set the size of the instance as the size of the set $F$, that is, how many free vertices are assigned to one of the parts.

\begin{itemize}
	\item[B1] If we put $v_1$ in $A_j$, $j \neq i$, we infer that $v_3$, and $v_4$ must also be added to $A_j$, and that $v_2$ must be added to $A_i$.
	Otherwise, $v_1$ is in $A_i$, which does not given us any additional information.
	Our branching vector is, thus, of the form $\{1\} \times \{4\}^{\ell-1}$.
	
	\item[B2] Note that $v_1$ must be placed in either $A_i$ or $A_j$.
	For the first case, we conclude that $v_4$ must also be in $A_i$, while for the later, $v_4$ must be added to $A_j$ and $v_2$ to $A_i$, yielding the branching vector $(2, 3)$ and the branching factor $1.3247$.
	
	\item[B3] Again, $v_1$ is in either $A_i$ or $A_j$.
	In either case, we conclude that $v_2$ must be in the same part as $v_1$, resulting in the branching vector $(2,2)$, which has a branching factor of $\sqrt{2}$.
	
	\item[B4 and B4'] By adding $v_1$ to $A_i$, we conclude that $v_2$ must also be placed in $A_i$; a similar analysis is performed when $v_1$ is added to $A_j$.
	Otherwise, if we add $v_1$ to $A_k$, at most one of $v_2$ and $v_3$ may be added to a set different from $A_k$.
	If both are in $A_k$, we have that $v_2'$ belongs in $A_i$ and $v_3'$ in $A_j$.
	Otherwise, if $v_2$ is added to $A_i$, we conclude that $v_3,v_4$ belong in $A_k$ and that $v_3'$ belongs in $A_j$; similarly if $v_3$ is assigned to $A_j$.
	This results in a branching vector of the form $\{2\}^2 \times \{5\}^{3\ell-6}$, with unique positive real root of the polynomial associated with it satisfying $\alpha_\ell \leq \sqrt[\leftroot{2} 3]{\ell}$.
	Rule B4' clearly has a better branching factor than B4, but rule B4 dominates the running time of B4'.
	
	\item[B5] In case we assign $v_1$ to $A_i$, we have that both $v_2$, and $v_3$ must also be in $A_i$;
	if $v_1$ is assigned to $A_j$, nothing else can be inferred;
	for all other $A_k$, we have that both $v_2$ and $v_3$ must be assigned to $A_k$, and that $v_2'$ and $v_3'$ belong in $A_i$.
	The branching vector for this rule is given by $\{1\} \times \{3\} \times \{5\}^{\ell-2}$, which, for large values of $\ell$, has a branching factor of at most $\sqrt[\leftroot{2} 3]{\ell}$.
	Again, it can be verified that, for each $\ell$, it holds that the branching factor for this rule is $\leq \sqrt[\leftroot{2} 3]{\ell}$.
	
	\item[B6] If $v_1$ is assigned to $A_i$ (resp. $A_j$), we have that both $v_2$, and $v_3$ (resp. $v_4$, and $v_5$) must also be assigned to $A_i$ (resp. $A_j$);
	otherwise, for every other $A_k$, either $\{v_{p}\}_{p \in [5]}$ belongs to $A_k$, in which case the vertices $\{v'_{p}\}_{p \in \{2,3,4,5\}}$ are assigned to the same set as their neighbor, or at most one $v_p \in \{v_2, v_3, v_4, v_5\}$ is not assigned to $A_k$, in which case the set to which $v'_p$ should be assigned is not determined.
	This rule produces a branching vector of the form $\{3\}^2 \times \{8\}^{4\ell - 8} \times \{9\}^{\ell - 2}$, 
	
	\item[B7] Once again, we only have two options for $v_1$.
	So, if $v_1$ is added to $A_i$, we have that $v_3$ must be added to $A_j$, otherwise $v_1$ is added to $A_j$ and $v_2$ to $A_i$.
	This rule's branching vector is $(2, 2)$, with factor equal to $\sqrt{2}$.
	
	\item[B8] If $v_1$ is assigned to $A_i$, we are done; otherwise, if $v_1$ is assigned to $A_k$, with $k \neq i$, we have that $v_2$ belongs in $A_i$ and $v_3$ in $A_i$.
	This yields the branching vector $\{1\} \times \{3\}^{\ell-1}$, and branching factor $\sqrt[\leftroot{2} 3]{\ell} \leq \alpha_\ell \leq \sqrt{\ell}$.
\end{itemize}

\begin{figure}[!htb]
	\centering
	\begin{tikzpicture}[scale=0.5]
	%\draw[help lines] (-5,-5) grid (5,5);
	\GraphInit[unit=3,vstyle=Normal]
	\SetVertexNormal[Shape=circle, FillColor=black, MinSize=2pt]
	\tikzset{VertexStyle/.append style = {inner sep = \inners, outer sep = \outers}}
	\SetVertexLabelOut
	
	\newcommand{\brule}{1}
	\begin{scope}[shift={(-10,5)}]
	\Vertex[x=0,y=0,Lpos=90, Math, L={v_1}]{v_1\brule}
	\Vertex[x=-1.5, y=0,Lpos=90, Math, L={A_i}]{a_i\brule}
	\Vertex[x=-3, y=0,Lpos=90, Math, L={v_2}]{v_2\brule}
	\Vertex[x=1.5, y=1,Lpos=0, Math, L={v_3}]{v_3\brule}
	\Vertex[x=1.5, y=-1,Lpos=0, Math, L={v_4}]{v_4\brule}
	\Edge(v_1\brule)(a_i\brule)
	\Edge(v_1\brule)(v_3\brule)
	\Edge(v_1\brule)(v_4\brule)
	\Edge(a_i\brule)(v_2\brule)
	\node at (-0.5,-1) {(B\brule)};
	\end{scope}
	
	\renewcommand{\brule}{2}
	\begin{scope}[shift={(0,5)}]
	\Vertex[x=0,y=0,Lpos=90, Math, L={v_1}]{v_1\brule}
	\Vertex[x=-1.5, y=0,Lpos=90, Math, L={A_i}]{a_i\brule}
	\Vertex[x=-3, y=0,Lpos=90, Math, L={v_2}]{v_2\brule}
	\Vertex[x=1.5, y=1,Lpos=0, Math, L={A_j}]{a_j\brule}
	\Vertex[x=1.5, y=-1,Lpos=0, Math, L={v_4}]{v_4\brule}
	\Edge(v_1\brule)(a_i\brule)
	\Edge(v_1\brule)(a_j\brule)
	\Edge(v_1\brule)(v_4\brule)
	\Edge(a_i\brule)(v_2\brule)
	\node at (-0.5,-1) {(B\brule)};
	\end{scope}
	
	\renewcommand{\brule}{3}
	\begin{scope}[shift={(8,5)}]
	\Vertex[x=0,y=0,Lpos=90, Math, L={v_1}]{v_1\brule}
	\Vertex[x=-1.5, y=0,Lpos=90, Math, L={v_2}]{v_2\brule}
	\Vertex[x=1.5, y=1,Lpos=0, Math, L={A_i}]{a_i\brule}
	\Vertex[x=1.5, y=-1,Lpos=0, Math, L={A_j}]{a_j\brule}
	\Edge(v_1\brule)(a_i\brule)
	\Edge(v_1\brule)(a_j\brule)
	\Edge(v_1\brule)(v_2\brule)
	\node at (0,-1) {(B\brule)};
	\end{scope}
	
	\renewcommand{\brule}{4}
	\begin{scope}[shift={(-8.5,0)}]
	\Vertex[x=0,y=0,Lpos=90, Math, L={v_1}]{v_1\brule}
	\Vertex[x=1.5,y=0,Lpos=0, Math, L={v_4}]{v_4\brule}
	\Vertex[x=-1.5, y=1, Lpos=90, Math, L={v_2}]{v_2\brule}
	\Vertex[x=-3, y=1, Lpos=90, Math, L={A_i}]{a_i\brule}
	\Vertex[x=-4.5, y=1, Lpos=90, Math, L={v_2'}]{v_2p\brule}
	\Vertex[x=-1.5, y=-1, Lpos=270, Math, L={v_3}]{v_3\brule}
	\Vertex[x=-3, y=-1, Lpos=270, Math, L={A_j}]{a_j\brule}
	\Vertex[x=-4.5, y=-1, Lpos=270, Math, L={v_3'}]{v_3p\brule}
	
	\node at (0.5,-1) {(B\brule)};
	\Edge(v_1\brule)(v_4\brule) \Edges(v_2p\brule,a_i\brule,v_2\brule,v_1\brule,v_3\brule,a_j\brule,v_3p\brule)
	\end{scope}
	
	\renewcommand{\brule}{5}
	\begin{scope}[shift={(0,0)}]
	\Vertex[x=0,y=0,Lpos=90, Math, L={v_1}]{v_1\brule}
	\Vertex[x=1.5,y=0,Lpos=0, Math, L={A_j}]{a_j\brule}
	\Vertex[x=-1.5, y=1, Lpos=90, Math, L={v_2}]{v_2\brule}
	\Vertex[x=-3, y=1, Lpos=90, Math, L={A_i}]{a_i\brule}
	\Vertex[x=-4.5, y=1, Lpos=90, Math, L={v_2'}]{v_2p\brule}
	\Vertex[x=-1.5, y=-1, Lpos=270, Math, L={v_3}]{v_3\brule}
	\Vertex[x=-3, y=-1, Lpos=270, Math, L={A_i}]{a_ip\brule}
	\Vertex[x=-4.5, y=-1, Lpos=270, Math, L={v_3'}]{v_3p\brule}
	
	\node at (0.5,-1) {(B\brule)};
	\Edge(v_1\brule)(a_j\brule) \Edges(v_2p\brule,a_i\brule,v_2\brule,v_1\brule,v_3\brule,a_ip\brule,v_3p\brule)
	\end{scope}
	
	\renewcommand{\brule}{6}
	\begin{scope}[shift={(8.5,0)}]
	\Vertex[x=0,y=0,Lpos=90, Math, L={v_1}]{v_1\brule}
	
	\Vertex[x=-1.5, y=1, Lpos=90, Math, L={v_2}]{v_2\brule}
	\Vertex[x=-3, y=1, Lpos=90, Math, L={A_i}]{a_i\brule}
	\Vertex[x=-4.5, y=1, Lpos=90, Math, L={v_2'}]{v_2p\brule}
	\Vertex[x=-1.5, y=-1, Lpos=270, Math, L={v_3}]{v_3\brule}
	\Vertex[x=-3, y=-1, Lpos=270, Math, L={A_i}]{a_ip\brule}
	\Vertex[x=-4.5, y=-1, Lpos=270, Math, L={v_3'}]{v_3p\brule}
	
	
	\Vertex[x=1.5, y=1, Lpos=90, Math, L={v_4}]{v_4\brule}
	\Vertex[x=3, y=1, Lpos=90, Math, L={A_j}]{a_j\brule}
	\Vertex[x=4.5, y=1, Lpos=90, Math, L={v_4'}]{v_4p\brule}
	\Vertex[x=1.5, y=-1, Lpos=270, Math, L={v_5}]{v_5\brule}
	\Vertex[x=3, y=-1, Lpos=270, Math, L={A_j}]{a_jp\brule}
	\Vertex[x=4.5, y=-1, Lpos=270, Math, L={v_5'}]{v_5p\brule}
	
	\node at (0,-1) {(B\brule)}; \Edges(v_2p\brule,a_i\brule,v_2\brule,v_1\brule,v_3\brule,a_ip\brule,v_3p\brule) \Edges(v_4p\brule,a_j\brule,v_4\brule,v_1\brule,v_5\brule,a_jp\brule,v_5p\brule)
	\end{scope}
	
	\renewcommand{\brule}{4}
	\begin{scope}[shift={(-8.5,-6)}]
	\Vertex[x=0,y=0,Lpos=90, Math, L={v_1}]{v_1\brule}
	\Vertex[x=1.5,y=0,Lpos=0, Math, L={v_4}]{v_4\brule}
	\Vertex[x=-1.5, y=1, Lpos=90, Math, L={v_2}]{v_2\brule}
	\Vertex[x=-3, y=1, Lpos=90, Math, L={A_i}]{a_i\brule}
	\Vertex[x=-4.5, y=1, Lpos=90, Math, L={v_2'}]{v_2p\brule}
	\Vertex[x=-1.5, y=-1, Lpos=270, Math, L={v_3}]{v_3\brule}
	\Vertex[x=-3, y=-1, Lpos=270, Math, L={A_i}]{a_j\brule}
	\Vertex[x=-4.5, y=-1, Lpos=270, Math, L={v_3'}]{v_3p\brule}
	
	\node at (0.5,-1) {(B\brule')};
	\Edge(v_1\brule)(v_4\brule) \Edges(v_2p\brule,a_i\brule,v_2\brule,v_1\brule,v_3\brule,a_j\brule,v_3p\brule)
	\end{scope}
	
	\renewcommand{\brule}{7}
	\begin{scope}[shift={(-1,-6)}]
	\Vertex[x=0,y=0,Lpos=90, Math, L={v_1}]{v_1\brule}
	\Vertex[x=-1.5, y=0,Lpos=90, Math, L={A_i}]{a_i\brule}
	\Vertex[x=1.5, y=0,Lpos=90, Math, L={A_j}]{a_j\brule}
	\Vertex[x=-2.5, y=-1,Lpos=270, Math, L={v_2}]{v_2\brule}
	\Vertex[x=2.5, y=-1,Lpos=270, Math, L={v_3}]{v_3\brule}
	\node at (0,-1) {(B\brule)};
	\Edge(v_1\brule)(a_i\brule)
	\Edge(v_1\brule)(a_j\brule)
	\Edge(v_2\brule)(a_i\brule)
	\Edge(v_3\brule)(a_j\brule)
	\end{scope}
	
	\renewcommand{\brule}{8}
	\begin{scope}[shift={(8.5,-6)}]
	\Vertex[x=0,y=0,Lpos=90, Math, L={v_1}]{v_1\brule}
	\Vertex[x=-1.5, y=0,Lpos=90, Math, L={A_i}]{a_i\brule}
	\Vertex[x=1.5, y=0,Lpos=90, Math, L={v_3}]{v_3\brule}
	\Vertex[x=-2.5, y=-1,Lpos=270, Math, L={v_2}]{v_2\brule}
	\Vertex[x=2.5, y=-1,Lpos=270, Math, L={A_j}]{a_j\brule}
	\node at (0,-1) {(B\brule)};
	\Edge(v_1\brule)(a_i\brule)
	\Edge(v_1\brule)(v_3\brule)
	\Edge(v_2\brule)(a_i\brule)
	\Edge(a_j\brule)(v_3\brule)
	\end{scope}
	\end{tikzpicture}
	\caption{\label{fig:nested_branches} Branching configurations for \pname{$\ell$-Nested Matching Cut}.}
\end{figure}

Given all of the above rules, we must show that, if none of them are applicable, we have an $\ell$-nested matching cut.
In order to do so, we require some additional definitions: let $A_i' = \{v \in A_i \mid \deg_F(v) \geq 2\}$,  $F_i' = F \cap N(A_i')$, $F_i'' = \{v \in F \mid \deg_{F_i'}(v) \geq 2\}$, and $F^* = F \setminus \bigcup_{i \in \ell} (F_i' \cup F_i'')$.
Also, we say that $A_i$ is \tdef{final} if, for all $v \in F_i'$, $\deg(v) = 2$.

\begin{lemma}
	If there is some $A_i$ of $\varphi$ which is not final and no Stopping/ Reduction Rule is applicable, then configurations B1, or B2 exist in the partitioned graph.
\end{lemma}

\begin{proof}
	Let $v_1$ be a degree three vertex of $F_i'$, $a_i$ its neighbor in $A_i'$ and $v_2$ the other neighbor of $A_i'$ in $F$.
	We know that $vv' \notin E(G)$, otherwise rule R1 would be applicable.
	Now, let $v_3,v_4$ be two of the other neighbors of $v$.
	If both are in $F$, we have a configuration B1; otherwise at most one of them is not in $F \cup A_i$, say $v_3$, since we would have applied rule S2 or rule R2 if this observation did not hold, implying that a configuration B2 is present.
\end{proof}

We may now assume that every $A_i$ is final, and that no reduction or stopping rule is applicable.
Our goal is to show that, if none of our branching configurations exist, then $\varphi^* = (A_1 \cup F'_1 \cup F''_1, \dots, A_{\ell - 1} \cup F'_{\ell - 1} \cup F''_{\ell - 1}, A_\ell \cup F'_\ell \cup  F''_\ell \cup F^* )$ is an $\ell$-nested matching cut of $G$.
Before proving that, however, we have to guarantee that the sets $F'_i, F''_i$ are a partition of $F$.

\begin{lemma}
	If there exists $i,j \in [\ell]$ with $F'_i \cap F'_j \neq \emptyset$, then rule B7 is applicable.
\end{lemma}

\begin{proof}
	Let $v_1 \in F'_i \cap F'_j$; since $A'_i$ and $A'_j$ are final, $\deg(v_1) = 2$ and its two neighbors, $a_i, a_j$, have one extra neighbor each, say $v_2$ and $v_3$.
	If $v_2 = v_3$, however, $v_2 \in F'_i$, and has degree equal to two; but this implies that $N(v_1) = N(v_2) = \{a_i, a_j\}$, and rule R4 could have been applied.
	All that remains now is the case where $v_2 \neq v_3$, but this is precisely configuration B7, as desired.
\end{proof}

\begin{lemma}
	If $A_i$ and $A_j$ are final, $F'_i \cap F''_j = \emptyset$.
\end{lemma}

\begin{proof}
	Suppose that there is some $v \in F'_i \cap F''_j$.
	By the definitions of $F_i'$ and $F''_j$, $v$ has degree two, one neighbor in $A_i$, and \textit{two} neighbors in $F'_j$, a contradiction.
\end{proof}

\begin{lemma}
	If there exists $i,j \in [\ell]$ with $F''_i \cap F''_j \neq \emptyset$, rule B6 is applicable.	
\end{lemma}

\begin{proof}
	Let $v_1$ a vertex of $F''_i \cap F''_j$.
	By the previous lemma and the definition of $F''_i$, it is straightforward to check that $v_1$ has four distinct neighbors: $v_2, v_3, v_4, v_5$, such that $v_2, v_3 \subseteq F'_i$ and $v_4, v_5 \subseteq F'_j$.
	Let $a_i$ be the neighbor of $v_2$ in $A_i$, $v_2'$ the other neighbor of $a_i$ in $F$.
	Define $a_i'$ and $v_3'$ similarly for $v_3$; $a_j$ and $v_4'$ for $v_4$; and $a_j'$ and $v_5'$ for $v_5$.
	Note that $v'_2 \neq v'_3$ (resp. $v'_4 \neq v'_5$), or rule R2 would be applicable.
	Consequently, $\{v_1, v_2, a_i, v_2', v_3, a'_i, v'_3, v_4, a_j, v'_4, v_5, a'_j, v'_5\}$ form a configuration B6.
\end{proof}

These last few results prove that $\varphi^*$ is a partition of $V(G)$.
Define $A_i^* = A_i \cup F'_i \cup F''_i$ and $A_\ell^* = A_\ell \cup F'_\ell \cup F''_\ell \cup F^*$.
What remains to be shown is that it is, in fact, an $\ell$-nested matching cut of $G$.

\begin{lemma}
	If no more branching rules are applicable, then for every $i$ and every $v \in A_i$, $\deg_{V(G) \setminus A_i^*}(v) \leq 1$. 
\end{lemma}

\begin{proof}
	First, $v$ has at most one neighbor in $\bigcup_ {j \neq i} A_j$, otherwise rule S4 would have stopped the algorithm.
	The case where $v$ has one neighbor in $A_j$ and one neighbor $u \in F$, since rule S3 is not applicable, by rule R3, $u$ must have been added to $A_i$, and so $u$ does not exist.
	Thus, the only possibility is that $v$ has more than one neighbor in $F$, implying $N_F(v) \subseteq F'_i$, but $F'_i$ is in the same part as $v$.
\end{proof}

\begin{lemma}
	If no more branching rules are applicable, then for every $i$ and every $v \in F'_i$, $\deg_{V(G) \setminus A_i^*}(v) \leq 1$. 
\end{lemma}

\begin{proof}
	Trivial due to the hypothesis that $F'_i$ is final.
\end{proof}


\begin{lemma}
	If no more branching rules are applicable, then for every $i$ and every $v \in F''_i$, $\deg_{V(G) \setminus A_i^*}(v) \leq 1$. 
\end{lemma}


\begin{proof}
	If $v$ has a neighbor in $A_j$, rule B5 is applicable, since $v \in F''_i$.
	On the other hand, if $v$ has a neighbor in $F$, we can apply rule B4' with $v = v_1$.
\end{proof}

\begin{lemma}
	If no more branching rules are applicable, then for every $v \in F^*$, $\deg_{V(G) \setminus A^*_\ell}(v) \leq 1$. 
\end{lemma}

\begin{proof}
	We know that $v$ does not have two neighbors in some $A_j$, but it could be the case that $v$ has neighbors $a_i \in A_i$, $a_j \in A_j$.
	Note that if $v$ $a_i$ cannot have a second  neighbor in $F$, otherwise $v$ would be in $F'_i$.
	As such, if $\deg(v) = 2$, we can still apply rule R6.
	Otherwise, if $\deg(v) \geq 3$, configuration B3 shows up with $v = v_1$.
	This allows us to conclude that $v$ does not have a neighbor in more than one $A_i$.
	Suppose now that $u \in F \setminus F^*$ is a neighbor of $v$, and $a_j \in A_j \cap N(v)$.
	If $u \in F'_i$ ($i$ may be equal to $j$), it follows that rule B8 is applicable with $v = v_3$ and $u = v_1$.
	If, on the other hand, $u \in F''_i$, we have configuration B4' where $u = v_1$ and $v = v_4$.
	Consequently, $v$ has no neighbor in $A_j$, for $j \neq \ell$.
	
	Now, it must be the case that both neighbors $x,y$ of $v$ are in $F \setminus F^*$.
	Note that $\{x,y\} \nsubseteq F'_i$, otherwise $v \in F''_i$.
	For now, suppose that $x \in F'_i$ and $y \in F'_j$.
	If $\deg(v) = 2$, rule R7 may be applied (with $u$ = $v$); so $\deg(v) \geq 3$ and we have configuration B4, again, with $v = v_1$.
	Suppose, then that $x$ is actually in $F''_i$; by the exact same argument, it holds that B4' is applicable with $v = v_4$ and $x = v_1$.
	The case where $x$ and $y$ are in $F''_i$ and $F''_j$, respectively, is identical.
\end{proof}

\begin{theorem}
	If no Stopping, Reduction, or Branching rule is applicable, $\varphi^*$ is an $\ell$-nested matching cut of $G$.
	Moreover, \pname{$\ell$-Nested Matching Cut} can be solved in $\bigOs{\alpha_\ell^n}$, with $\alpha_\ell \leq \sqrt{\ell}$ for a graph on $n$ vertices.
\end{theorem}



\subsection{Multiway Cuts}

The previous section dealt with partitions $\varphi$ such that each vertex has at most one neighbor in another part.
We may relax this constraint, and ask that each vertex has at most one neighbor in \textit{each part other than its own}.
Equivalently, given the integer $p \geq 2$, we want an $p$-partition of the vertices of the graph such that $(A_i, A_j)$ is a matching cut of $G[A_i \cup A_j]$.
A cut that satisfies this property is called a \tdef{$p$-Way matching cut}, with the problem of deciding whether or not a graph admits such a partition defined below.

\problem{$p$-Way Matching Cut}{A graph $G$}{Does $G$ admit a $p$-Way matching cut?}

It is not hard to adapt either of the reductions given by~\cite{matching_cut_chvatal} or us to this generalization, although a bit more of care must be taken when designing color selection gadgets.
The hard part, however, is finding an \FPT\ algorithm for \pname{$p$-Way Matching Cut} when parameterized by the number of edges crossing the cut.
The powerful machinery provided by~\cite{marx_treewidth_reduction} does not appear to be capable of handling the constraint of each pair of parts is a matching cut.
Whenever we attempted to give a graph class that captured this notion, we were either unable to reconstruct the cut on the original graph, or we found counter-examples that the treewidth reduction technique could return that did not represent a $p$-way matching cut of the graph.
Adapting the exact exponential algorithm of~\cite{matching_cut_ipec} also seems a challenging task; while we were successful for nested cuts, the structures used as branching rules appear to explode rapidly with the growth of $p$, a similar phenomenon is observed when trying to devise a kernelization algorithm
Aside from these challenging questions, most parameterized algorithms for \pname{Matching Cut} can be adapted for \pname{$p$-Way Matching Cut}, such as the ones parameterized by treewidth or distance to cluster, without much difficulty.





