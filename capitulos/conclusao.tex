\chapter{Final remarks}
\label{ch:conclusion}

This thesis dealt mainly with graph partitioning problems.
Counted among them, are three coloring problems, multiple generalizations of the matching cut problem, and a broad study on a novel class of intersection graphs.
The presented results vary quite a bit in nature: there are hardness reductions, exact, parameterized and polynomial algorithms, kernels,  characterizations, and structural properties; connections with other problems and graph classes were established, and some previous results in the literature were significantly strengthened.
The final section of the previous chapters gave an overview of the results presented in this thesis.
We summarize in this last chapter some open problems and further research directions.

With respect to coloring problems, we presented a series of \W[1]-\Hness\ reductions for \pname{Equitable Coloring} on different subclasses of chordal graphs, managing to present a hardness result for block graphs of diameter at least four.
The notable difficulty of \pname{Equitable Coloring} was already known, so our results do not come as so surprising.
Problems that are still hard when parameterized by treewidth (e.g. \pname{List Coloring}) usually have constraints beyond structural aspects of the input graph.
The search for parameterizations that allow \FPT\ algorithms is still necessary, and it appears that reasonable such parameterizations require some parameter that captures this non-topological flavor of the problem.
\pname{Clique Coloring} and \pname{Biclique Coloring}, on the other hand, haven't been very explored in the literature, mostly because their placement on the second level of the polynomial hierarchy makes non-parameterized algorithmic analysis quite bleak, offering little to no hope of solving interesting instances in a feasible amount of time.
These last two problems are, consequently, nice targets for further research on parameterized algorithms.
Actually, W[1]-\Hness\ proofs are far from trivial for these problems; by their very definitions, finding a clique or biclique coloring implies on guaranteeing that a quite large, and some time ill behaved, family of subsets of vertices satisfies the desired coloring constraint.

For cut problems, despite adapting many results from \pname{Matching Cut}, and even improving some of them, we leave many open questions related to \pname{$d$-Cut}.
First, we would like to close the gap between the known polynomial and $\NP$-hard cases in terms of maximum degree, i.e., for each graph with maximum degree satisfying $d+3 \leq \Delta(G) \leq 2d+1$, we would like to how hard is it to find a $d$-cut.
After much effort, we were unable to settle any of these cases.
We are particularly interested in \pname{2-Cut}, where the only open case is for graphs of maximum degree equal to five.
We recall the initial discussion about the \pname{Internal Partition} problem; closing the gap between the known cases for \textsc{$d$-Cut} would yield significant advancements on the internal partitions conjecture.
As to the presented algorithms, all of them, in some way or another, have an exponential dependency on $d$.
Answering whether or not this is necessary is interesting by itself, and merits further work; in particular, we would like to know if the kernel we presented can be improved.
For \pname{$\ell$-Nested Matching Cut}, we did not present nearly as many results as we did with \pname{$d$-Cut}, but we do give a non-trivial exact exponential algorithm.
Proving analogous results would be quite nice, but the real challenge in this problem is using the fact that what we are looking for is actually a special matching cut, where at least one of the parts admits even more matching cuts.
Lastly, for \pname{$p$-Way Matching Cut}, we only offer a brief discussion, since most of the techniques we would use to prove results are analogous to the ones we described for \pname{$d$-Cut}.
As the central open question for this problem, we highlight the complexity of the algorithm parameterized by the number of edges crossing the cut, for which we were unable to either provide even an \XP\ algorithm.

Finally, we also discussed the intersection graphs of maximal stars, which we called star graphs.
We presented a series of properties of the class, a Krausz-type characterization, a bound on the size of minimal pre-images, membership of the recognition problem in \NP, and even dabbled, albeit very briefly, on the iterated star operator.
As mentioned in the conclusions of Chapter~\ref{ch:star_graph}, we have two particular interests on future work on this graph class: (i) completely settling the complexity of the recognition problem, which we believe to be \NPc, and (ii) finding a linear upper bound on the size of star-critical pre-images.
This last problem is backed by our computational experiments, where for each value $1 \leq k \leq 8$ of maximal stars, no star-critical graph on more than $2k$ vertices exists.
To achieve this, we require  a more thorough understanding of what star-critically implies and how we can identify non-star-critical vertices.
Aside from these two problems, many other questions remain unanswered.
Iterated biclique graphs have attracted attention recently~\cite{biclique_iterated}, and iterated star graphs might benefit greatly from the reasoning strategies used in these studies.
Also of interest is the study of the star operator under restrictions either on the pre-image or on the image of the operator; as discussed, working on triangle-free pre-images is all about working on square graphs.
On the other hand, if we only consider $C_4$-free pre-images, we are working on (possibly part of) the intersection between biclique and star graphs.






