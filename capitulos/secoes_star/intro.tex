\section{Intersection Graphs}
\label{sec:intersections}

The \tdef{intersection graph} of a multifamily $\mathcal{F} \subseteq 2^S$, denoted by $G = \Omega(\mathcal{F})$ is the graph of order $|\mathcal{F}|$ and, for every $F_u, F_v \in \mathcal{F}$, $uv \in E(G) \Leftrightarrow F_u \cap F_v \neq \emptyset$.
Any $\mathcal{F}$ such that $\Omega(\mathcal{F}) \simeq G$ is a \tdef{set representation} of $G$.
A known theorem states that every graph is the intersection graph of a family of subgraphs of a graph~\citep{intersection_graphs}.

An \tdef{edge clique cover} $\mathcal{Q} = \{Q_1, \dots, Q_n\}$ of a graph $G$ is a (multi)family of cliques of $G$ such that every edge of $G$ is contained in at least one element of $\mathcal{Q}$.
The \tdef{dual edge clique cover} of a set representation $\mathcal{F} = \{F_1, \dots, F_n\}$, with $|\bigcup_{i \leq n} F_i| = m$, is defined as $\mathcal{Q}(\mathcal{F}) = \{Q_1, \dots, Q_m\}$ such that $Q_j = \{i \mid j \in F_i\}$.
The \tdef{dual set representation} of an edge clique cover $\mathcal{Q} = \{Q_1, \dots, Q_m\}$, with $|\bigcup_{j \leq m} Q_j| = n$, is $\mathcal{F}(\mathcal{Q}) = \{F_1, \dots, F_n\}$ with $F_i = \{j \mid i \in Q_j\}$.

Some interesting intersection graphs are usually defined in terms of the intersection of structures of other graphs. For instance, \tdef{line graphs} are precisely the graphs that are the intersection graphs of the edges of a graph; \tdef{clique graphs} are the intersection graphs of the maximal induced cliques of a graph.
Both of these classes, however have nice characterizations in terms of edge clique covers, which are commonly called \tdef{Krausz-type characterizations}.

\begin{class_definition*}[Line Graph]
    $G$ is a line graph if and only if there is an edge clique cover $\mathcal{Q}$ of $G$ such that both conditions hold:
        \begin{itemize}
            \item[(i)] Every vertex of $G$ appears in exactly two members of $\mathcal{Q}$;
            \item[(ii)] Every edge of $G$ is in only one member of $\mathcal{Q}$.
        \end{itemize}
\end{class_definition*}

\begin{class_definition*}[Clique Graph]
    $G$ is a clique graph if and only if it there is an edge clique cover of $G$ satisfying the Helly property.
\end{class_definition*}

\begin{figure}[!htb]
    \centering
        \begin{tikzpicture}
            \begin{scope}[rotate=-90,scale=0.5]
                \def\x{-2}
                \GraphInit[vstyle=Normal]
                \SetVertexNormal[Shape=circle, FillColor=black, LineWidth=1pt,MinSize=2pt,]
                \tikzset{VertexStyle/.append style = {inner sep = \inners, outer sep = \outers}}
                \Vertices[Lpos=270,Ldist=3pt,LabelOut=TRUE,unit=3]{circle}{a,b,c,d}
                \Edge(a)(b)
                \Edge(a)(d)
                \Edge(c)(b)
                \Edge(d)(b)
                \Edge(c)(d)
            \end{scope}
        \end{tikzpicture}
    \hfill
        \begin{tikzpicture}
            \begin{scope}[rotate=-90,scale=0.5]
                \def\x{-2}
                \GraphInit[vstyle=Normal]
                \SetVertexNormal[Shape=circle, FillColor=black, LineWidth=1pt,MinSize=2pt,]
            \tikzset{VertexStyle/.append style = {inner sep = \inners, outer sep = \outers}}
                \Vertices[Lpos=270,Ldist=3pt,LabelOut=TRUE,unit=3]{circle}{abd,bcd}
                \Edge(abd)(bcd)
            \end{scope}
        \end{tikzpicture}
    \hfill
        \begin{tikzpicture}
            \begin{scope}[rotate=315,scale=0.5]
                \def\x{-2}
                \GraphInit[vstyle=Normal]
                \SetVertexNormal[Shape=circle, FillColor=black, LineWidth=1pt,MinSize=2pt,]
            \tikzset{VertexStyle/.append style = {inner sep = \inners, outer sep = \outers}}
                \Vertices[Lpos=315,Ldist=3pt,LabelOut=TRUE,unit=3]{circle}{ab,bc,cd,ad}
                \Vertex[Lpos=270,Ldist=3pt,LabelOut=TRUE,unit=3,x=0,y=0]{bd}
                \Edge(ab)(bc)
                \Edge(ab)(ad)
                \Edge(bc)(cd)
                \Edge(cd)(ad)
                \Edge(bd)(ab)
                \Edge(bd)(ad)
                \Edge(bd)(cd)
                \Edge(bd)(bc)
            \end{scope}
        \end{tikzpicture}
    \hfill
        \begin{tikzpicture}
            \begin{scope}[rotate=90,scale=0.5]
                \GraphInit[vstyle=Normal]
                \SetVertexNormal[Shape=circle, FillColor=black, LineWidth=1pt,MinSize=2pt,]
            \tikzset{VertexStyle/.append style = {inner sep = \inners, outer sep = \outers}}
                \Vertices[Lpos=90,Ldist=3pt,LabelOut=TRUE,unit=3]{circle}{abc,dbc,bd}
                \Edges(abc,dbc,bd,abc)
            \end{scope}
        \end{tikzpicture}
    \hfill
    \caption{A graph, its clique graph, its line graph, and its star graph}
    %  Some graph
    \label{fig:my_label}
\end{figure}

The recognition of line graphs is known to be efficient~\citep{line_dynamic,line_nich,line_naor}.
For clique graphs, however, the situation was not so simple, and the complexity of clique graph recognition was left open for several years, finally being proven to be $\NPc$ by~\cite{clique_recognition} with a quite complicated argument.

Aside from the complexity point of view, many different properties of intersection graphs have been investigated in the literature.
For instance, clique-critical graphs -- graphs whose clique graph is different from the clique graph of all of its proper induced subgraphs -- have different characterizations~\citep{clique_critical_toft} and bounds~\citep{clique_critical_alcon} which were crucial in the proof of the complexity of the recognition problem.
Another common line of investigation on intersection graphs is the behaviour of iterated applications of the operators.
For instance, \cite{clique_iterated}, and \cite{clique_divergent} study iterated applications of the clique operator.
Biclique graphs -- the intersection graph of the maximal induced complete bipartite graphs of a graph -- were first characterized and studied by \cite{biclique_graph}.
Their results, however, are not very useful from the algorithmic point of view, and appear to not yield many insights on the recognition problem.
Nevertheless, they study the behaviour of biclique graphs, showing that every edge is contained either in a diamond or a 3-fan and specialize their general characterization for biclique graphs of bipartite graphs.
As was done for clique graphs, the iterated biclique operator has also been studied by Groshaus et al. in multiple papers~\citep{biclique_iterated, almost_all_biclique}, with results ranging from characterizations of divergence, divergence type verification algorithms, and other structural results.

Regarding stars, previous work handled the intersection graphs of (not necessarily maximal) substars of a tree~\citep{substar_graph} and of a star~\citep{starlike_graph}.
For the first, a minimal infinite family of forbidden induced subgraphs was given, while, for the latter, a series of characterizations were shown (including a finite family of forbidden induced subgraphs).
Stars are a particular case of bicliques, and both the biclique graph and star graph coincide for $C_4$-free graphs.
In fact, this relationship was successfully applied to determine the complexity of biclique coloring~\citep{biclique_coloring_complexity}, using a reduction from \pname{QSAT}$_2$ to star coloring (a coloring of the vertices of a graph such that no maximal star is monochromatic).
To the best of our knowledge, these are the main topics discussed in the literature that involve maximal stars in some way.
However, star graphs appear to be natural generalizations of square graphs~\cite{murty} in the sense that, when applying the squaring operation, for each vertex $v$ only the largest, non-induced star centered at $v$ is selected, and the intersection graph of these stars is generated.
On the other hand, for star graphs, every \textit{induced} maximal star is used in the construction of the intersection graph.
Despite the classes of star graphs and biclique graphs being equivalent when restricting the  pre-image domain to $C_4$-free graphs, we were unable to deepen the study biclique graphs; our efforts were hindered by some of the questions posed and developed upon in this work.

For other classical results in the area we point to~\citep{intersection_graphs}, from where most of the given definitions come from.
