\section{Intersection graphs}
\label{sec:intersections}

Some interesting intersection graphs are usually defined in terms of the intersection of structures of other graphs. For instance, \tdef{line graphs} are precisely the graphs that are the intersection graphs of the edges of a graph; \tdef{clique graphs} are the intersection graphs of the maximal induced cliques of a graph.
Both of these classes, however have nice characterizations in terms of edge clique covers, which are commonly called \tdef{Krausz-type characterizations}.

\begin{class_definition*}[Line Graph]
    $G$ is a line graph if and only if there is an edge clique cover $\mathcal{Q}$ of $G$ such that both conditions hold:
        \begin{itemize}
            \item[(i)] Every vertex of $G$ appears in exactly two members of $\mathcal{Q}$;
            \item[(ii)] Every edge of $G$ is in only one member of $\mathcal{Q}$.
        \end{itemize}
\end{class_definition*}

\begin{class_definition*}[Clique Graph]
    $G$ is a clique graph if and only if it there is an edge clique cover of $G$ satisfying the Helly property.
\end{class_definition*}

\begin{figure}[!htb]
    \centering
        \begin{tikzpicture}
            \begin{scope}[rotate=-90,scale=0.5]
                \def\x{-2}
                \GraphInit[vstyle=Normal]
                \SetVertexNormal[Shape=circle, FillColor=black, LineWidth=1pt,MinSize=2pt,]
                \tikzset{VertexStyle/.append style = {inner sep = \inners, outer sep = \outers}}
                \Vertices[Lpos=270,Ldist=3pt,LabelOut=TRUE,unit=3]{circle}{a,b,c,d}
                \Edge(a)(b)
                \Edge(a)(d)
                \Edge(c)(b)
                \Edge(d)(b)
                \Edge(c)(d)
            \end{scope}
        \end{tikzpicture}
    \hfill
        \begin{tikzpicture}
            \begin{scope}[rotate=-90,scale=0.5]
                \def\x{-2}
                \GraphInit[vstyle=Normal]
                \SetVertexNormal[Shape=circle, FillColor=black, LineWidth=1pt,MinSize=2pt,]
            \tikzset{VertexStyle/.append style = {inner sep = \inners, outer sep = \outers}}
                \Vertices[Lpos=270,Ldist=3pt,LabelOut=TRUE,unit=3]{circle}{abd,bcd}
                \Edge(abd)(bcd)
            \end{scope}
        \end{tikzpicture}
    \hfill
        \begin{tikzpicture}
            \begin{scope}[rotate=315,scale=0.5]
                \def\x{-2}
                \GraphInit[vstyle=Normal]
                \SetVertexNormal[Shape=circle, FillColor=black, LineWidth=1pt,MinSize=2pt,]
            \tikzset{VertexStyle/.append style = {inner sep = \inners, outer sep = \outers}}
                \Vertices[Lpos=315,Ldist=3pt,LabelOut=TRUE,unit=3]{circle}{ab,bc,cd,ad}
                \Vertex[Lpos=270,Ldist=3pt,LabelOut=TRUE,unit=3,x=0,y=0]{bd}
                \Edge(ab)(bc)
                \Edge(ab)(ad)
                \Edge(bc)(cd)
                \Edge(cd)(ad)
                \Edge(bd)(ab)
                \Edge(bd)(ad)
                \Edge(bd)(cd)
                \Edge(bd)(bc)
            \end{scope}
        \end{tikzpicture}
    \hfill
        \begin{tikzpicture}
            \begin{scope}[rotate=90,scale=0.5]
                \GraphInit[vstyle=Normal]
                \SetVertexNormal[Shape=circle, FillColor=black, LineWidth=1pt,MinSize=2pt,]
            \tikzset{VertexStyle/.append style = {inner sep = \inners, outer sep = \outers}}
                \Vertices[Lpos=90,Ldist=3pt,LabelOut=TRUE,unit=3]{circle}{abc,dbc,bd}
                \Edges(abc,dbc,bd,abc)
            \end{scope}
        \end{tikzpicture}
    \hfill
    \caption{A graph, its clique graph, its line graph, and its star graph}
    %  Some graph
    \label{fig:my_label}
\end{figure}

The recognition of line graphs is known to be efficient~\citep{line_dynamic,line_nich,line_naor}.
For clique graphs, however, the situation was not so simple, and the complexity of clique graph recognition was left open for several years, finally being proven to be $\NPc$ by~\cite{clique_recognition} with a quite complicated argument.

Aside from the complexity point of view, many different properties of intersection graphs have been investigated in the literature.
For instance, clique-critical graphs -- graphs whose clique graph is different from the clique graph of all of its proper induced subgraphs -- have different characterizations~\citep{clique_critical_toft} and bounds~\citep{clique_critical_alcon} which were crucial in the proof of the complexity of the recognition problem.
Another common line of investigation on intersection graphs is the behaviour of iterated applications of the operators.
For instance, \cite{clique_iterated}, and \cite{clique_divergent} study iterated applications of the clique operator.
Biclique graphs -- the intersection graph of the maximal induced complete bipartite graphs of a graph -- were first characterized and studied by \cite{biclique_graph}.
Their results, however, are not very useful from the algorithmic point of view, and appear to not yield many insights on the recognition problem.
Nevertheless, they study the behavior of biclique graphs, showing that every edge is contained either in a diamond or a 3-fan and specialize their general characterization for biclique graphs of bipartite graphs.
As was done for clique graphs, the iterated biclique operator has also been studied by Groshaus et al. in multiple papers~\citep{biclique_iterated, almost_all_biclique}, with results ranging from characterizations of divergence, divergence type verification algorithms, and other structural results.

For other classical results in the area we point to~\citep{intersection_graphs}, from where most of the given definitions come from.

\subsection{Maximal stars}
\label{sec:maximal_stars}

Regarding stars, previous work handled the intersection graphs of (not necessarily maximal) substars of a tree~\citep{substar_graph} and of a star~\citep{starlike_graph}.
For the first, a minimal infinite family of forbidden induced subgraphs was given, while, for the latter, a series of characterizations were shown (including a finite family of forbidden induced subgraphs).
Stars are a particular case of bicliques, and both the biclique graph and star graph coincide for $C_4$-free graphs.
In fact, this relationship was successfully applied to determine the complexity of biclique coloring~\citep{biclique_coloring_complexity}, as discussed in Chapter~\ref{ch:coloring}.
To the best of our knowledge, these are the main topics discussed in the literature that involve maximal stars in some way.
The central object of study in this chapter is the intersection graph of maximal stars, which we formally define in Definition~\ref{def:star_graph}.


\begin{definition}\label{def:star_graph}
    Let $\mathcal{G}$ be the set of all finite graphs and $\str(H)$ be the set of all induced maximal stars of a graph $H$. The \tdef{star operator} is the function $\K{S} : \mathcal{G} \mapsto \mathcal{G}$ such that, $\K{S}(H) = \Omega(\str(H))$.
    If $G = \K{S}(H)$, we say that $H$ is a \tdef{pre-image} of $G$ and that $G$ is the \tdef{star graph} of $H$.
    The \textit{iterated star operator} $\K{S}^i$ is defined as $\K{S}^1(H) = \K{S}(H)$ and $\K{S}^i(H) = \K{S}(\K{S}^{i-1}(H))$.
\end{definition}

When detailing which vertices belong to a star $s$, we shall describe it by $s = \{v_1\}\{v_2, \dots, v_{p+1}\}$, with $v_1$ being its center, denoted by $c(s) = v_1$, and the other $p$ vertices its leaves.
Unless noted, $G$ will be our star graph and $H$ the pre-image of $G$. %
%The family of all maximal stars of $G$ is denoted by $\str(G)$.
By Definition~\ref{def:star_graph}, two stars $s_a,s_b$ intersect if they share at least one vertex, with the possible cases being: (i) the centers of $s_a$ and $s_b$ coincide; (ii) the center of $s_a$ is a leaf of $s_b$; or (iii) $s_a$ and $s_b$ share at least one leaf.
Note that conditions (i) and (iii) may be simultaneously satisfied.
For an example of the intersection possibilities, please refer to Figure~\ref{fig:intersection_example}.


\begin{figure}[!tb]
    \centering
        \begin{tikzpicture}[scale=1.2]
            \GraphInit[unit=3,vstyle=Normal]
            \SetVertexNormal[Shape=circle, FillColor = black, MinSize=3pt]
            \tikzset{VertexStyle/.append style = {inner sep = \inners,outer sep = \outers}}
            \SetVertexLabelOut
            \begin{scope}[xshift=-3.5cm]
                \Vertex[x=0,y=0, Math, Lpos=90]{c}
                
                \Vertex[a=135, d=1, Math, Lpos=135]{a}
                \Vertex[a=225, d=1, Math, Lpos=225]{b}
                
                \Vertex[a=315, d=1, Math, Lpos=315]{e}
                \Vertex[a=45, d=1, Math, Lpos=45]{d}
                \node at (0,-1.5) {(i)};
                \Edges(b,c,d,e,c,a,b)
            \end{scope}
            \begin{scope}[xshift=-1, yshift=0.5cm]
                \Vertex[x=0,y=0, Math, Lpos=90, L={u}]{up}
                \Vertex[x=1,y=0, Math, Lpos=90, L={v}]{vp}
                
                \Vertex[x=0,y=-1, Math, Lpos=270]{w}
                \Vertex[x=1,y=-1, Math, Lpos=270]{z}
                \node at (0.5,-2) {(ii)};
                \Edges(w,up,vp,z)
            \end{scope}
            \begin{scope}[xshift=3.5cm]
                \Vertex[x=0,y=0, Math, Lpos=180]{u}
                \Vertex[x=1,y=0, Math, Lpos=0]{v}
                
                \Vertex[x=0.5,y=0.5, Math, NoLabel]{x_1}
                \Vertex[x=0.5,y=1, Math, NoLabel]{x_2}
                \Vertex[x=0.5,y=-0.5, Math, NoLabel]{x_3}
                \Vertex[x=0.5,y=-1, Math, NoLabel]{x_4}
                
                \node at (0.5,-1.5) {(iii)};
                \Edges(u,x_1,v,x_2,u,x_3,v,x_4,u)
            \end{scope}
        \end{tikzpicture}
        \caption{(i) The stars $\{c\}\{a,e\}$ and $\{c\}\{b,d\}$ intersect only at their center; (ii) center of $\{u\}\{w,v\}$ is a leaf of star $\{v\}\{u,z\}$; (iii) the star centered at $u$ intersects the star centered at $v$ only at their leaves.\label{fig:intersection_example}}
\end{figure}


We say that star $s_a$ \textit{absorbs} star $s_b$ if, by removing one leaf of $s_b$, it becomes a substar of $s_a$.
A vertex $v$ is said to be \tdef{star-critical} if its removal changes the resulting star graph; that is, the star graph of $H$ and the star graph of $H \setminus \{v\}$ are not isomorphic.
Similarly to clique-critical graphs~\cite{clique_critical_toft,clique_critical_alcon}, a graph is \textit{star-critical} if all of its vertices are star-critical.
It is not hard to see that the only vertices which may be non-star-critical are simplicial vertices; for example, if there is a class of false twin simplicial vertices, all but one of them are certainly non-star-critical.

When detailing which vertices belong to a star, we shall describe it by $\{v_1\}\{v_2, \dots, v_{n+1}\}$, with $v_1$ being its center and the other $n$ vertices its leaves.
If the star is a single edge, choose one of the vertices to be the center and the other to be the leaf arbitrarily.
Unless noted, $G$ will be our star graph and $H$ the pre-image of $G$.
The family of all maximal stars of $G$ is denoted by $\str(G)$.
For the entirety of this work, we assume that all of our graphs are connected.



\begin{figure}[!htb]
	\centering
	
	\begin{tikzpicture}[rotate=45,scale=\gscale]
	
	\GraphInit[unit=3,vstyle=Normal]
	%\draw[help lines] (-5,-5) grid (5,5);
	\SetVertexNormal[Shape=circle, FillColor=black, MinSize=2pt]
	\tikzset{VertexStyle/.append style = {inner sep = \inners, outer sep = \outers}}
	\begin{scope}[shift={(-2.41cm, 0cm)}]
	\SetVertexNoLabel
	\grEmptyCycle[RA=1.41,prefix=a]{4}
	\Edges(a0,a1,a2,a3,a0)
	\Vertex[x=0,y=2.41]{1a}
	%\Vertex[y=0,x=-2.41]{2a}
	\Vertex[x=0,y=-2.41]{3a}
	\Edge(1a)(a1)
	%\Edge(2a)(a2)
	\Edge(3a)(a3)
	\end{scope}
	\begin{scope}[shift={(2.41cm, 0cm)}]
	\SetVertexNoLabel
	\grEmptyCycle[RA=1.41,prefix=b]{4}
	\Edges(b0,b1,b2,b3,b0)
	\Vertex[x=0,y=2.41]{1b}
	%\Vertex[y=0,x=2.41]{0b}
	\Vertex[x=0,y=-2.41]{3b}
	\Edge(1b)(b1)
	%\Edge(0b)(b0)
	\Edge(3b)(b3)
	\end{scope}
	\Edge(a0)(b2)
	\end{tikzpicture}
	\hfill
	\begin{tikzpicture}[rotate=45,scale=\gscale]
	
	\GraphInit[unit=3,vstyle=Normal]
	%\draw[help lines] (-5,-5) grid (5,5);
	\SetVertexNormal[Shape=circle, FillColor=black, MinSize=2pt]
	\tikzset{VertexStyle/.append style = {inner sep = \inners, outer sep = \outers}}
	\begin{scope}[shift={(-2.41cm, 0cm)}]
	\SetVertexNoLabel
	\grEmptyCycle[RA=1.41,prefix=a]{4}
	\Edges(a0,a1,a2,a3,a0)
	\Vertex[x=0,y=2.41]{1a}
	%\Vertex[y=0,x=-2.41]{2a}
	\Vertex[x=0,y=-2.41]{3a}
	\Edge(1a)(a1)
	%\Edge(2a)(a2)
	\Edge(3a)(a3)
	
	\Edge(a0)(a2)
	\Edge(a1)(a3)
	
	\Edge(1a)(a2)
	\Edge(3a)(a2)
	\Edge(1a)(a0)
	\Edge(3a)(a0)
	%\Edge(2a)(a1)
	%\Edge(2a)(a3)
	\end{scope}
	\begin{scope}[shift={(2.41cm, 0cm)}]
	\SetVertexNoLabel
	\grEmptyCycle[RA=1.41,prefix=b]{4}
	\Edges(b0,b1,b2,b3,b0)
	\Vertex[x=0,y=2.41]{1b}
	%\Vertex[y=0,x=2.41]{0b}
	\Vertex[x=0,y=-2.41]{3b}
	\Edge(1b)(b1)
	%\Edge(0b)(b0)
	\Edge(3b)(b3)
	
	\Edge(b0)(b2)
	\Edge(b1)(b3)
	
	\Edge(1b)(b2)
	\Edge(3b)(b2)
	\Edge(1b)(b0)
	\Edge(3b)(b0)
	%\Edge(0b)(b1)
	%\Edge(0b)(b3)
	\end{scope}
	\Edge(a0)(b2)
	\Edge(a0)(b1)
	\Edge(a0)(b3)
	\Edge(b2)(a1)
	\Edge(b2)(a3)
	\end{tikzpicture}
	\hfill
	\begin{tikzpicture}[shift={(1.41cm,0cm)},rotate=45,scale=\gscale]
	
	\GraphInit[unit=3,vstyle=Normal]
	%\draw[help lines] (-5,-5) grid (5,5);
	\SetVertexNormal[Shape=circle, FillColor=black, MinSize=2pt]
	\tikzset{VertexStyle/.append style = {inner sep = \inners, outer sep = \outers}}
	\begin{scope}[shift={(-2.41cm, 0cm)}]
	\SetVertexNoLabel
	\grEmptyCycle[RA=1.41,prefix=a]{4}
	\Edges(a0,a1,a2,a3,a0)
	\Edge(a0)(a2)
	\Edge(a1)(a3)
	\end{scope}
	\begin{scope}[shift={(2.41cm, 0cm)}]
	\SetVertexNoLabel
	\grEmptyCycle[RA=1.41,prefix=b]{4}
	\Edges(b0,b1,b2,b3,b0)
	\Edge(b0)(b2)
	\Edge(b1)(b3)
	\end{scope}
	\Edge(a0)(b2)
	\Edge(a0)(b1)
	\Edge(a0)(b3)
	\Edge(b2)(a1)
	\Edge(b2)(a3)
	\end{tikzpicture}
	\hfill
	
	
	
	\caption{A triangle-free graph (left), its square (center) and its star graph (right).}
	\label{fig:tri_star}
\end{figure}

Before proceeding to the main results of this chapter, we make the following remark.

\begin{observation}
	Every vertex of degree at least two in a $K_3$-free graph is the center of exactly one maximal star.
\end{observation}

The above observation immediately leads us to the property that every star graph of a triangle-free graph is closely related to the square of one of its induced subgraphs.

\begin{observation}
	If $H$ is a $K_3$-free graph with at least 3 vertices, $D$ are its vertices of degree at least 2 and $G = \K{S}(H)$, it holds that $G \simeq H[D]^2$.
\end{observation}

As such, every hardness result or polynomial time algorithm for the recognition of squares of triangle-free graphs immediately applies to the class of star graphs of triangle-free graphs.
For an illustration of the previous observation, we refer to Figure~\ref{fig:tri_star}.
For a far more complicated star graph, we refer to Figure~\ref{fig:star_example}.



\begin{figure}[!tb]
	\centering
	\begin{tikzpicture}
	\GraphInit[unit=3,vstyle=Normal]
	\SetVertexNormal[Shape=circle, FillColor = black, MinSize=3pt]
	\tikzset{VertexStyle/.append style = {inner sep = \inners,outer sep = \outers}}
	\SetVertexNoLabel
	\begin{scope}[rotate=45]
	\grComplete[RA=1]{4}
	\end{scope}
	\Vertex[x=0, y=-2]{x4}
	\Vertex[x=0, y=-4]{x3}
	\Vertex[x=0, y=-6]{x2}
	\Vertex[x=-1, y=-6]{x1}
	\Edges(a3,x4,a2)
	\Edges(x4,x3,x2,x1)
	\end{tikzpicture}
	\hfill
	\begin{tikzpicture}
	\GraphInit[unit=3,vstyle=Normal]
	\SetVertexNormal[Shape=circle, FillColor = black, MinSize=3pt]
	\tikzset{VertexStyle/.append style = {inner sep = \inners,outer sep = \outers}}
	\SetVertexNoLabel
	\Vertex[x = 0, y = 0.707]{v78}
	\Vertex[x = 2, y = -0.707]{v854}
	\Vertex[x = 1, y = -0.707]{v754}
	
	\Vertex[x = -1, y = -0.707]{v764}
	\Vertex[x = -2, y = -0.707]{v864}
	\Vertex[x = 0, y = 0]{v65}
	
	\Vertex[x = 1, y = -2]{v543}
	
	\Vertex[x = -1, y = -2]{v643}
	
	\Vertex[x = 0, y = -4]{v432}
	
	\Vertex[x = 0, y = -6]{v321}
	
	
	
	\Edge(v65)(v764)
	\Edge(v643)(v854)
	\Edge(v643)(v864)
	\Edge(v321)(v432)
	\Edge(v754)(v854)
	\Edge(v643)(v754)
	\Edge(v543)(v65)
	\Edge(v764)(v78)
	\Edge(v864)(v78)
	\Edge(v543)(v854)
	\Edge(v764)(v754)
	\Edge(v764)(v864)
	\Edge(v432)(v764)
	\Edge(v643)(v65)
	\Edge(v65)(v754)
	\Edge[style = bend right](v864)(v854)
	\Edge(v432)(v643)
	\Edge(v321)(v543)
	\Edge(v643)(v764)
	\Edge(v643)(v543)
	\Edge(v864)(v754)
	\Edge(v754)(v78)
	\Edge(v854)(v78)
	\Edge(v543)(v864)
	\Edge(v543)(v764)
	\Edge(v321)(v643)
	\Edge(v432)(v543)
	\Edge(v65)(v854)
	\Edge(v543)(v754)
	\Edge(v764)(v854)
	\Edge[style=bend right](v432)(v854)
	\Edge[style=bend left](v432)(v864)
	\Edge(v432)(v754)
	\Edge(v65)(v864)
	\end{tikzpicture}
	\hfill
	
	
	
	\caption{A graph (left) and its star graph (right).}
	\label{fig:star_example}
\end{figure}

However, star graphs appear to be natural generalizations of square graphs~\citep{murty} in the sense that, when applying the squaring operation, for each vertex $v$ only the largest, non-induced star centered at $v$ is selected, and the intersection graph of these stars is generated.
On the other hand, for star graphs, every maximal \textit{induced} star is used in the construction of the intersection graph.
Despite the classes of star graphs and biclique graphs being equivalent when restricting the  pre-image domain to $C_4$-free graphs, we were unable to deepen the study of biclique graphs; our efforts were hindered by some of the questions posed and developed upon in this work.


