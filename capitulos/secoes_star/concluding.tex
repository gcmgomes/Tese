\section{Concluding remarks}

This chapter introduced the class of star graphs -- the intersection graphs of the induced maximal stars of some graph.
We presented various results, such as a Krausz-type characterization for the class, a quadratic bound on the size of potential pre-images, membership of the recognition problem in \textsf{NP} and a monotonicity theorem for graphs which are not star-critical.
We also presented a series of properties the members of the class must satisfy, such as being biconnected and that every edge must belong to some triangle.
We leave two main open question.
The first, and perhaps more challenging of the two, is the complexity of the recognition problem; for example, the complexity of the clique graph recognition problem was left open for many decades, only being settled recently~\cite{clique_recognition} through a series of non-intuitive gadgets and other novel characterizations.
The second is a complete characterization of both star-critical and non-star-critical vertices; in particular, non-star-critical vertices seem the biggest obstacle one must overcome to achieve a linear bound on the size of star-critical pre-images.

Despite our special interests in the above questions, many other directions are available for investigation.
In terms of the class of all star graphs, our best membership checking tool at the moment is generating pre-image candidates and apply Corollary~\ref{cor:frontier} of Theorem~\ref{thm:monotonicity} to prune the search space; if our hypothesis that the recognition problem is \textsf{NP-hard} is indeed true, and thus unlikely solvable in polynomial time, then what is the best way to verify membership?
In a more general context, is there a polynomial delay algorithm that generates all star graphs of a certain order?
We have also only begun the study of the iterated star operator, and various inquiries can be made about its properties, such as convergence/divergence criteria or other structural parameters, like maximum/minimum degree and connectivity.
A strongly related but significantly different open topic is that of edge-star graphs, i.e., the edge-intersection graph of the maximal stars.
Edge-biclique graphs have very recently been studied by Legay and Montero~\cite{edge_biclique} and present significant differences from the vertex-intersection biclique graph of Groshaus et al.~\cite{biclique_graph};
Perhaps the interplay between the edge-star and edge-biclique graphs can yield useful observations for both classes.