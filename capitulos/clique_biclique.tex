\chapter{Clique and Biclique Coloring}
\label{ch:cb_coloring}

Both \textsc{clique coloring} and \textsc{biclique coloring} are relaxations of the classical \textsc{vertex coloring} problem, in the sense that monochromatic edges are allowed.
However, this freedom comes at the cost of validating a solution, which becomes a $\coNP\text{-}\Complete$ task in both cases.
One may think of \textsc{vertex coloring} as the task of covering a graph's vertices using a given number of independent sets.
That is, there cannot be a color class with an edge inside it.
For \textsc{clique coloring} and \textsc{biclique coloring}, the idea is quite similar.
We want to forbid not edges, but maximal clique or bicliques, respectively, inside our color classes.
All of the following results establish families of sets that may safely be used to cover the given graph and describe how to compute them.

Much of the discussion will deal with the clique and biclique hypergraphs $\Hyper{C}(G)$ and $\Hyper{B}(G)$.
As such, we define by $\trans{C}(G)$ ($\trans{B}(G)$) the \tdef{family of all transversals} of the clique (biclique) hypergraph of $G$ and by $\transs{C}(G)$ ($\transs{B}(G)$) the \tdef{family of complements of transversals}.
Also, denote by $\oblq{C}(G)$ ($\oblq{B}(G)$) the \tdef{family of all obliques} of the clique (biclique) hypergraph of $G$.
Finally, $\clq(G)$ ($\biq(G)$) is the family of \tdef{maximal cliques} (bicliques) of $G$.

In this chapter, we present algorithms that make heavy use of the algorithm described by~\citep{inclusion_exclusion}, which applies the inclusion-exclusion principle to solve a variety of problems in $2^nn^{\bigO{1}}$ time, including \textsc{vertex coloring}.
We denote complexities of the form $f(n)n^{\bigO{1}}$, where $f(n)$ is an exponential function on $n$, by $\bigOs{f(n)}$.

Our main results are an $\bigOs{2^n}$ algorithm for \textsc{biclique coloring}, an $\FPT$ algorithm for \textsc{clique coloring} parameterized by neighbourhood diversity and an $\FPT$ algorithm for \textsc{biclique coloring} parameterized by the number of colors and neighbourhood diversity.
To achieve them, we will rely on the following problems and results of the literature.

\begin{lemma}[\cite{clique_color_algorithm}]
    \label{lem:down_closure}
    For any family $\mathcal{F}$, its down closure $\mathcal{F}_{\downarrow} = \left\{X \subseteq V \mid \exists Y \in \mathcal{F},\ X \subseteq Y\right\}$ can be enumerated in $O^*(|\mc{F}_{\downarrow}|)$ time.
\end{lemma}

\begin{lemma}[\cite{clique_color_algorithm}]
    \label{lem:clique_transversal_colorings}
    A $k$-partition $\varphi = \{\varphi_1, \dots, \varphi_k\}$ is a $k$-clique-coloring of $G$ if and only if for every $i$, $\overline{\varphi_i} \in \trans{C}(G)$.
\end{lemma}

\problem{exact cover}{A set $A = \{a_1, \dots, a_n\}$, a covering family $\mathcal{F} \subseteq 2^A$ and an integer $k$.}{Is it possible to $k$-partition  $A$ into $\varphi$ such that $\varphi \subseteq \mathcal{F}$?}

\begin{theorem}[\cite{inclusion_exclusion}]
    \label{thm:inc_exc}
    There is a $\bigOs{2^n}$ time algorithm to solve \textsc{exact cover}.
\end{theorem}

\begin{theorem}[\cite{clique_color_algorithm}]
    \label{thm:clique_color_algorithm}
    There is an $\bigOs{2^n}$ time algorithm for \textsc{clique coloring}.
\end{theorem}


\problem{set multicover}{A set $A = \{a_1, \dots, a_n\}$, a covering family $\mathcal{F} \subseteq 2^A$, an integer $k$ and a coverage demand $c: A \mapsto \mathbb{N}$.}{Is it possible to $k$-cover $A$ with $\varphi \subseteq \mathcal{F}$ and $\forall a_j, |\{i \mid a_j \in \varphi_i\}| \geq c(a_j)$?}

\begin{theorem}[\cite{set_multicover}]
    \label{thm:set_multicover}
    Set multicover can be solved in $O^*((b+2)^n)$, with $b$ the maximum coverage requirement.
\end{theorem}

\section{Exact algorithm for \textsc{biclique coloring}}
\label{sec:biclique_exact}

\begin{figure}[!htb]
    \centering
    \begin{tikzpicture}[scale=1]
        \begin{scope}
         %\draw[help lines] (-5,-5) grid (5,5);
            \GraphInit[unit=3,vstyle=Normal]
            \SetVertexNormal[Shape=circle, FillColor=black, MinSize=2pt]
            \tikzset{VertexStyle/.append style = {inner sep = \inners, outer sep = \outers}}
            \SetVertexNoLabel
            \Vertex[x=-1,y=-1]{x-1-1}
            \Vertex[x=-1,y=0]{x-10}
            \Vertex[x=-1,y=1]{x-11}
            \Vertex[x=0,y=-1]{x0-1}
            \Vertex[x=0,y=0]{x00}
            \Vertex[x=0,y=1]{x01}
            \Vertex[x=1,y=-1]{x1-1}
            \Vertex[x=1,y=0]{x10}
            \Vertex[x=1,y=1]{x11}

            
            \Edge(x-1-1)(x-10)
            \Edge(x-1-1)(x0-1)
            \Edge(x0-1)(x00)
            \Edge(x0-1)(x1-1)
            
            \Edge(x-10)(x-11)
            \Edge(x-10)(x00)
            \Edge(x00)(x01)
            \Edge(x00)(x10)
            
            \Edge(x-11)(x01)
            \Edge(x01)(x11)
            
            \Edge(x10)(x11)
            \Edge(x10)(x1-1)
        \end{scope}
    \end{tikzpicture}
    \hfill
    \begin{tikzpicture}[scale=1]
        \begin{scope}
            %\draw[help lines] (-5,-5) grid (5,5);
            \GraphInit[unit=3,vstyle=Normal]
            \SetVertexNormal[Shape=circle, FillColor=white, MinSize=1pt]
            \tikzset{VertexStyle/.append style = {inner sep = \inners, outer sep = \outers}}
            \SetVertexNoLabel
            \Vertex[x=-1,y=-1]{x-1-1}
            \Vertex[x=-1,y=1]{x-11}
            \Vertex[x=1,y=-1]{x1-1}
            \Vertex[x=1,y=1]{x11}
            
            \SetVertexNormal[Shape=circle, FillColor=black, MinSize=1pt]
            \SetVertexNoLabel
            \Vertex[x=0,y=-1]{x0-1}
            \Vertex[x=-1,y=0]{x-10}
            \Vertex[x=1,y=0]{x10}
            \Vertex[x=0,y=0]{x00}
            \Vertex[x=0,y=1]{x01}

            
            \Edge(x-1-1)(x-10)
            \Edge(x-1-1)(x0-1)
            \Edge(x0-1)(x00)
            \Edge(x0-1)(x1-1)
            
            \Edge(x-10)(x-11)
            \Edge(x-10)(x00)
            \Edge(x00)(x01)
            \Edge(x00)(x10)
            
            \Edge(x-11)(x01)
            \Edge(x01)(x11)
            
            \Edge(x10)(x11)
            \Edge(x10)(x1-1)
        \end{scope}
    \end{tikzpicture}
    \hfill
    \begin{tikzpicture}[scale=1]
        \begin{scope}
            %\draw[help lines] (-5,-5) grid (5,5);
            \GraphInit[unit=3,vstyle=Normal]
            \SetVertexNormal[Shape=circle, FillColor=white, MinSize=1pt]
            \tikzset{VertexStyle/.append style = {inner sep = \inners, outer sep = \outers}}
            \SetVertexNoLabel
            \Vertex[x=-1,y=-1]{x-1-1}
            \Vertex[x=-1,y=1]{x-11}
            \Vertex[x=1,y=-1]{x1-1}
            \Vertex[x=1,y=1]{x11}
            \Vertex[x=0,y=0]{x00}
            
            \SetVertexNormal[Shape=circle, FillColor=black, MinSize=1pt]
            \SetVertexNoLabel
            \Vertex[x=0,y=-1]{x0-1}
            \Vertex[x=-1,y=0]{x-10}
            \Vertex[x=1,y=0]{x10}
            \Vertex[x=0,y=1]{x01}

            
            \Edge(x-1-1)(x-10)
            \Edge(x-1-1)(x0-1)
            \Edge(x0-1)(x00)
            \Edge(x0-1)(x1-1)
            
            \Edge(x-10)(x-11)
            \Edge(x-10)(x00)
            \Edge(x00)(x01)
            \Edge(x00)(x10)
            
            \Edge(x-11)(x01)
            \Edge(x01)(x11)
            
            \Edge(x10)(x11)
            \Edge(x10)(x1-1)
        \end{scope}
    \end{tikzpicture}
    \hfill
    
    
    \caption{A graph (left), one of its maximal bicliques in black (center), and a transversal in black (right).}
    \label{fig:transversals}
\end{figure}


Our first task is to establish a family of sets that can be safely used to cover $V(G)$, which we accomplish as follows.


\begin{lemma}
    \label{lem:transversal_colorings}
    A $k$-partition $\varphi = \{\varphi_1, \dots, \varphi_k\}$ is a $k$-biclique-coloring of $G$ if and only if for every $i$, $\overline{\varphi_i} \in \trans{B}(G)$.
\end{lemma}

\begin{tproof}
$(\Rightarrow)$ Suppose that there exists some $\varphi_i$ such that $\overline{\varphi_i} \notin \trans{B}(G)$.
This implies that there exists some $B \in \biq(G)$ such that $B \cap \overline{\varphi_i} = \emptyset$ and that $B \subseteq \varphi_i$; that is, $|\varphi(B)| = 1$, which is a contradiction, since $\varphi$ is a $k$-biclique-coloring.

$(\Leftarrow)$ Let $\varphi$ be a $k$-partition of $G$ with $\overline{\varphi_i} \in \trans{B}(G)$, but suppose that $\varphi_k$ is not a $k$-biclique-coloring.
That is, there exists some maximal biclique $B \in \biq(G)$ such that $B \subseteq \varphi_i$ for some $i$.
This implies that $B \cap \overline{\varphi_i} = \emptyset$, and, therefore, $\overline{\varphi_i} \notin \trans{B}(G)$, which contradicts the hypothesis.
\end{tproof}

Simply testing for each $X \in 2^{V(G)}$ if $X \in \transs{B}(G)$ is a costly task.
A naive algorithm would check, for each $B \in \biq(G)$, if $\overline{X} \cap B \neq \emptyset$.
With $|\biq(G)| \in \bigO{n3^{\frac{n}{3}}}$ (see~\citep{gaspers} for the proof), such algorithm would take $\bigO{n2^n3^{\frac{n}{3}}}$ time.

The next Lemma, along with Lemma~\ref{lem:down_closure}, considerably reduce the complexity of enumerating $\trans{B}(G)$.
Effectively, we will enumerate $\oblq{B}(G)$ by first generating its maximal elements and then use the fact that $\oblq{B}(G)$ is closed under the subset operation, which takes $\bigOs{2^n}$ time.

\begin{lemma}
    \label{lem:complementary_obliques}
    The maximal obliques of $\Hyper{B}(G)$ are exactly the complements of the maximal bicliques of $G$.
\end{lemma}

\begin{tproof}
    $(\Rightarrow)$ Let $X \in \oblq{B}$ be a maximal oblique. By definition, there exists some $B \in \biq(G)$ such that $X \cap B = \emptyset$, which implies that $X \subseteq \overline{B}$.
    Note that, if $X \subset \overline{B}$, there is some $v \in \overline{B} \setminus X$, which implies that $(X \cup \{v\}) \cap B = \emptyset$ and that $X$ is not a maximal oblique.
    
    $(\Leftarrow)$ Let $B \in \biq(G)$. By definition, $\overline{B} \in \oblq{B}$ and must be maximal because $\{B, \overline{B}\}$ is a partition of $V(G)$.
\end{tproof}

\begin{corollary}
    \label{col:is_maximal_oblique}
    Given a graph $G = (V, E)$ and a subset $X \subseteq V(G)$, there exists an $O(n(n - |X|))$-time algorithm to determine if $X$ is a maximal oblique.
\end{corollary}

\begin{theorem}
    There is an $\bigOs{2^n}$ time algorithm for \textsc{biclique coloring}.
\end{theorem}

\begin{tproof}
    Our goal is to make use of Theorem~\ref{thm:inc_exc} to solve an instance of \textsc{exact cover}, with $A = V(G)$, $\mathcal{F} = \transs{B}(G)$ and $k$ the partition size.
    Lemma~\ref{lem:transversal_colorings} guarantees that there is an answer to our instance of \textsc{biclique coloring} if and only if there is an answer to the corresponding \textsc{exact cover} instance.
    
    To compute $\transs{B}(G)$, for each $X \in 2^{V(G)}$, we use Lemma~\ref{lem:complementary_obliques} and Corollary~\ref{col:is_maximal_oblique} to say whether or not $X$ is a maximal oblique of $\Hyper{B}(G)$.
    
    Afterwards, we compute $\oblq{B}(G)$ from its maximal elements using Lemma~\ref{lem:down_closure}, and use the fact that $\trans{B}(G) = 2^{V(G)} \setminus \oblq{B}(G)$ and complement each transversal to obtain $\transs{B}(G)$.
    Clearly, this procedure takes $\bigOs{2^n}$ to construct $\transs{B}(G)$ and an additional $\bigOs{2^n}$ to use Theorem~\ref{thm:inc_exc}.
\end{tproof}

\section{Algorithms Parameterized by Neighbourhood Diversity}

\begin{definition}[Neighbourhood Diversity]
    A graph $G = (V,E)$ has neighbourhood diversity $\nd(G) = d$ if it can be $d$-partitioned in $\{D_1, \dots, D_d\}$, with each $D_i$ being a type class of $G$.
\end{definition}

As previously discussed, a type is a maximal set of vertices that are either true or false twins to each other.
If $D_i$ is composed of true twins, we say that it is a \tdef{true twin class} $T_i$ and, clearly, $G[D_i]$ is clique.
Similarly, if $D_i$ is composed of false twins, it is a \tdef{false twin class} $F_i$ and $G[D_i]$ is an independent set.
When $|D_i| = 1$, we treat the class differently depending on the problem.

\subsection{Biclique Coloring}

For \textsc{biclique coloring}, if there is some $D_i$ with a single vertex, we shall treat it as a true twin class.

\begin{observation}
    \label{obs:biclique_true_twins}
    Given $G$ and a true twin class $T_i$ of $G$, any $k$-biclique-coloring $\varphi$ of $G$ has $|\varphi(T_i)| = |T_i|$.
\end{observation}

\begin{lemma}
    \label{lem:biclique_false_twins}
    Given $G$ and a false twin class $F \subset V(G)$, any  $k$-biclique-coloring $\varphi'$ of $G$ can be changed into a $k$-biclique-coloring $\varphi$ of $G$ such that $|\varphi(F)| \leq 2$.
\end{lemma}

\begin{tproof}
    If $|\varphi'(F)| \leq 2$, $\varphi = \varphi'$.
    Otherwise, there exists $f_1, f_2, f_3 \in F$ with three different colors. 
    Since every maximal biclique $B$ of $G$ with at least one element of $F$ has that $F \subset B$ and thus $|\varphi'(B)| \geq 3$.
    By making $\varphi(f_1) = \varphi'(f_1)$ and $\varphi(f_3) = \varphi(f_2) = \varphi'(f_2)$, we obtain $|\varphi(B)| \geq |\varphi(F)| \geq 2$.
    Repeating this process until $|\varphi(F)| = 2$ does not make any biclique monochromatic and completes the proof.
\end{tproof}

The central idea of our parameterized algorithm is to build an induced subgraph $H$ of $G$ and, afterward, use the results established here and in Section~\ref{sec:biclique_exact} to show that the solution to a particular instance of \textsc{set multicover} derived from $H$ can be transformed in a solution to \textsc{biclique coloring} of $G$.


\begin{definition}[B-Projection and B-Lifting]
    Let $T_i$ and $F_j$ be as previously discussed.
    We define the following projection rules:
    $\forall t_i^q \in T_i,\ \Proj{B}(t_i^q) = \{t'_i\}$;
    for $f_j^1\in F_j,\ \Proj{B}(f_j^1) = \{f'{_j^1}\}$;
    $\forall f_j^r \in F_j \setminus \{f_j^1\}$, $\Proj{B}(f_j^r) = \{f'{_j^2}\}$
    and $\Proj{B}(X) = \bigcup_{u \in X} \Proj{B}(u)$.
    
    Lifting rules are defined as $\Lift{B}(t'_i) = \{t_i\}$; $\Lift{B}(f'{_j^1}) = \{f_j^1\}$; $\Lift{B}(f'{_j^2}) = F_j \setminus \{f_j^1\}$ and $\Lift{B}(Y) = \bigcup_{u \in Y} \Lift{B}(u)$. Note that $\Proj{B}(\Lift{B}(X)) = X$, for any $X$.
\end{definition}


\begin{definition}[B-Projected Graph]
    The B-projected graph $H$ of $G$ is defined as $V(H) = \Proj{B}(V(G))$ and $v'_iv'_j \in E(H)$ if and only if there exist $v_i \in \Lift{B}(v'_i)$ and $v_j \in \Lift{B}(v'_j)$ such that $v_iv_j \in E(G)$. $H$ is an induced subgraph of $G$.
\end{definition}

\begin{figure}[!htb]
    \centering
        \begin{tikzpicture}[scale=1]
            \GraphInit[unit=3,vstyle=Normal]
            \SetVertexNormal[Shape=circle, FillColor=black, MinSize=2pt]
            \tikzset{VertexStyle/.append style = {inner sep = \inners, outer sep = \outers}}
            \SetVertexNoLabel
            \Vertex[x=0,y=0]{a}
            \Vertex[x=-1,y=0]{b}
            \Vertex[x=0,y=-1]{c}
            
            
            
            \Vertex[x=1,y=0]{h}
            \Vertex[x=0,y=1]{i}
            \Vertex[a=45, d=1]{j}
            
            \Edge(a)(b)
            \Edge(a)(c)
            
            \Edge(a)(h)
            \Edge(a)(i)
            \Edge(a)(j)
            \begin{scope}[shift={(-1.65cm, 0cm)}]
                \grComplete[RA=0.65]{3}
            \end{scope}
            \begin{scope}[shift={(0cm, -1.65cm)}]
                \grComplete[RA=0.65]{4}
            \end{scope}
        \end{tikzpicture}
    \hfill
        \begin{tikzpicture}[scale=1]
            \GraphInit[unit=3,vstyle=Normal]
            \SetVertexNormal[Shape=circle, FillColor=black, MinSize=2pt]
            \tikzset{VertexStyle/.append style = {inner sep = \inners, outer sep = \outers}}
            \SetVertexNoLabel
            \Vertex[x=0,y=0]{a}
            \Vertex[x=-1,y=0]{b}
            \Vertex[x=0,y=-1]{c}
            \Vertex[x=0,y=-1.65]{d}
            \Vertex[x=-1.65,y=0]{e}
            
            
            
            \Vertex[x=1,y=0]{h}
            \Vertex[x=0,y=1]{i}
            
            \Edge(a)(b)
            \Edge(a)(c)
            
            \Edge(a)(h)
            \Edge(a)(i)
            \Edge(b)(e)
            \Edge(c)(d)
        \end{tikzpicture}
    \hfill
    
    
    \caption{A graph (left), and its B-projected graph (right).}
    \label{fig:b-projected}
\end{figure}


For the remainder of this section, $G$ will be the input graph to \textsc{biclique coloring} and $H$ the B-Projected graph of $G$.
Our \textsc{set multicover} instance consists of $V(H)$ as the ground set, $\transs{B}(H)$ as the covering family, the size $k$ of the cover the same as the coloring of $G$ and $c(t_i') = |T_i|$ for every true twin class $T_i$ and $c(f'{_j^1}) = c(f'{_j^2}) = 1$ for each false twin class $F_j$.

The next observation follows directly from the fact that $\transs{B}(H)$ is closed under the subset operation, while the subsequent results allow us to move freely between $\transs{B}(G)$ and $\transs{B}(H)$.

\begin{observation}
    \label{obs:fatless_multicover}
    If there is a minimum $k$-multicover $\psi$ of $V(H)$ by $\transs{B}(H)$, then there exists a minimum $k$-multicover $\psi' = \{\psi_1, \dots, \psi_k\}$ such that $\left|\left\{j \mid u \in \psi_j\right\}\right| = c(u)$ for every $u \in V(H)$.
\end{observation}

\begin{lemma}
    \label{lem:lift_proj_biclique}
    If $B' \in \biq(H)$ then $B = \Lift{B}(B') \in \biq(G)$. Conversely, if $B \in \biq(G)$ and $B$ is not contained in any true twin class, then $B' = \Proj{B}(B) \in \biq(H)$.
\end{lemma}

\begin{tproof}
    ($\Rightarrow$) Note that $B$ is a biclique by the definition of $\Lift{B}$ and the fact that $B'$ is a biclique. By the contrapositive, suppose that $B \notin \biq(G)$ and that $u \in V(G)$ is such that $B \cup \{u\}$ is a (not necessarily maximal) biclique of $G$. Note that either:
    (i) if $u \in F_j$ then $F_j \nsubseteq B$ and $\Proj{B}(u) \notin B'$, because $u \notin \Lift{B}(f'{_j^1})$ or $u \notin \Lift{B}(f'{_j^2})$;
    or (ii) if $u \in T_i$ then $T_i \cap B = \emptyset$, which implies that $\Proj{B}(u) \notin B'$.
    %Since $H$ is an induced subgraph of $G$, no new bicliques can be created in $H$ that were not present in $G$.
    Since $B \cup \{u\}$ is a biclique, $\Proj{B}(u)$ is adjacent to only one partition of $B'$.
    The fact that $\Proj{B}(u) \notin B'$ implies that $\Proj{B}(B \cup \{u\}) = \Proj{B}(B) \cup \Proj{B}(u) = B' \cup \Proj{B}(u)$ is a biclique of $H$ and $B'$ is not maximal.
    
    ($\Leftarrow$) By the definition of $\Proj{B}$, $B' = (X, Y)$ must be a biclique of $H$. By the contrapositive, there is $u' \in V(H)$ such that $B' \cup \{u'\}$ is a (not necessarily maximal) biclique of $H$, and let $u \in \Lift{B}(u')$. By the definition of $\Lift{B}$, it follows that $u$ can only be adjacent to one of partition of $B$, say $\Lift{B}(X)$. Therefore, $u' \in Y$ and, for each $v \in \Lift{B}(Y)$, $uv \notin E(G)$, otherwise there would be $v' \in \Proj{B}(v)$ with $u'v' \in E(H)$. Therefore, $B \cup \{u\}$ is a biclique of $G$ and $B$ is not maximal.
\end{tproof}


\begin{theorem}
    \label{thm:lifted_transversal}
    $X \subseteq V(H)$ is in $\transs{B}(H)$ if and only if $\Lift{B}(X) \in \transs{B}(G)$.
\end{theorem}

\begin{tproof}
    ($\Rightarrow$) Recall that $X \in \transs{B}(H)$ if and only if no maximal biclique of $H$ is contained in $X$.
    It is clear that, for every $B' \in \biq(H)$, $B' \nsubseteq X$ implies that $\Lift{B}(B') \nsubseteq \Lift{B}(X)$, since no two vertices of $H$ are lifted to the same vertex of $G$, and $\Lift{B}(B') \in \biq(G)$ due to Lemma~\ref{lem:lift_proj_biclique}.
    Moreover, no biclique of $G$ entirely contained in a true twin class can be a subset of $\Lift{B}(X)$. As such, $\Lift{B}(X)$ contains a maximal biclique $B$ only if $\Proj{B}(B) \subseteq X$ and $\Proj{B}(B) \notin \biq(H)$, which is impossible due to Lemma~\ref{lem:lift_proj_biclique} and the assumption that $B$ is maximal.
    
    ($\Leftarrow$) Taking the contrapositive, $X \notin \transs{B}(H)$ implies that there is some maximal biclique $B'$ of $H$ such that $B' \subseteq X$. This implies that $\Lift{B}(B') \subseteq \Lift{B}(X)$, and, since $\Lift{B}(B')$ is a maximal biclique of $G$ due to Lemma~\ref{lem:lift_proj_biclique}, it holds that $\Lift{B}(X)$ is not a complement of transversal of $G$.
\end{tproof}


\begin{theorem}
    \label{thm:lifted_multicover}
    $\psi$ is a $k$-multicover of $H$ if and only if $G$ is $k$-biclique-colorable.
\end{theorem}

\begin{tproof}
    Recall that a $k$-partition is a $k$-biclique-coloring if and only if all elements of the partition belong to $\transs{B}(G)$. By the construction of our set multicover instance, we have that, for each $\psi_i$, $\psi_i \in \transs{B}(H)$. By making $\varphi = \left\{\Lift{B}(\psi_1), \dots, \Lift{B}(\psi_k)\right\}$, and recalling Observation~\ref{obs:fatless_multicover}, we have that each vertex $u \in V(H)$ is covered exactly $c(u)$ times; moreover, since true twins appear multiple times and types are equivalence relations, we can attribute to each $t_i^q$ any of the $|T_i|$ colors available, as long as no two receive the same color.
    Therefore, $\varphi$, after properly allocating the true twin classes, is a $k$-partition of $V(G)$.
    Due to Theorem~\ref{thm:lifted_transversal}, every $\Lift{B}(\psi_i)$ is a complement of transversal and therefore $\varphi$ is a valid $k$-biclique-coloring of $G$.
    
    For the converse, we first make use of Lemma~\ref{lem:biclique_false_twins} to guarantee that every false twin class is in at most two color classes. In particular, if two colors are required we force $f_j^1$ to have the smallest color and $F_j \setminus \{f_j^1\}$ to have the other one.
    Afterwards, for every color class $\varphi_i$, we take $\psi_i = \Proj{B}(\varphi_i)$.
    Note that, each color class has at most one element of each $T_i$. Also, for each $F_j$ and any two distinct color classes $\varphi_l, \varphi_r$, $\Proj{B}(\varphi_l) \cap \Proj{B}(\varphi_r) \cap \Proj{B}(F_j) = \emptyset$, since $f_j^1$ has a different color from $F_j \setminus \{f_j^1\}$.
    These observations guarantee that $\Lift{B}(\psi_i) = \varphi_i$ and, because of Theorem~\ref{thm:lifted_transversal}, $\psi_i \in \transs{B}(H)$.
    Finally, $\psi = \{\psi_1, \dots, \psi_k\}$ will be a valid $k$-multicover of $H$ because every vertex of $V(H)$ will be covered the required amount of times.
\end{tproof}

Note that the size of the largest true twin class is exactly the largest coverage requirement $b$ of our \textsc{set multicover} instance. Moreover, since we need at least $b$ colors to biclique color $G$, it holds that $b \leq k$.

\begin{theorem}
    \label{thm:fpt_biclique}
    There exists an $\bigOs{(k+2)^{2\nd(G)}}$ time algorithm to verify if $G$ is $k$-biclique-colorable.
\end{theorem}

\begin{tproof}
        Start by computing the type partition of $G$ in $\bigO{n^3}$ time and building $H$ in $O(n + m)$.
        Afterwards, solve the corresponding \textsc{set multicover} instance in $\bigOs{(k+2)^{2d}}$ time using Theorem~\ref{thm:set_multicover} and lift the multicover using the construction described in the proof of Theorem~\ref{thm:lifted_multicover} in $\bigO{n}$.
\end{tproof}

\subsection{Clique Coloring}

For \textsc{clique coloring}, a type class with a single vertex is treated as a false twin class.
Unlike \textsc{biclique coloring}, both true and false twin classes are well behaved, one of the reasons we get a much better algorithm for this problem.


\begin{lemma}
    \label{lem:clique_false_twins}
    Given $G$ and a false twin class $F \subset V(G)$, any  $k$-clique-coloring $\varphi'$ can be changed into a $k$-clique-coloring $\varphi$ such that $|\varphi(F)| = 1$.
\end{lemma}

    \begin{tproof}
    If $|\varphi'(F)| = 1$, we are done.
    Otherwise, there exists $f_1, f_2 \in F$ such that $\varphi'(f_1) \neq \varphi'(f_2)$.
    For every maximal clique $C_1$ where $f_1 \in C_1$, define $C' = C \setminus \{f_1\}$ and note that $C_2 = C' \cup \{f_2\}$ is also a maximal clique.
    Since $\varphi'$ is an  coloring $|\varphi'(C') \cup \{\varphi'(f_1)\}| \geq 2$.
    Therefore, making $\varphi(f_2) = \varphi(f_1) = \varphi'(f_1)$ does not make $|\varphi(C_2)| = 1$. 
    Repeating this until $|\varphi(F)| = 1$ does not make any clique that intercepts $F$ monochromatic and completes the proof.
    \end{tproof}
    
\begin{lemma}
    \label{lem:clique_true_twins}
    Given $G$ and  a true twin class $T \subseteq V(G)$, any  $k$-clique-coloring $\varphi'$ can be changed into a $k$-clique-coloring $\varphi$ such that $|\varphi(T)| \leq 2$.
\end{lemma}

\begin{tproof}
    If $|\varphi'(T)| \leq 2$, we are done.
    Otherwise, there exists $t_1, t_2, t_3 \in T$ with different colors.
    Note that, for every maximal clique $C$ that intercepts $T$, $C \subseteq T$.
    Therefore, $|\varphi'(C)| \geq |\varphi'(T)| \geq 3$.
    By making $\varphi(t_1) = \varphi'(t_1)$ and $\varphi(t_3) = \varphi(t_2) = \varphi'(t_2)$ we have $|\varphi(C)| \geq |\varphi(T)| \geq 2$.
    Repeating this process until $|\varphi(T)| \leq 2$ does not make any clique that intercepts $T$ monochromatic and completes the proof.
\end{tproof}

We use sightly different projection and lifting rules for this problem.
However, they still rely on the fact that a type has a constant bound on the number of colors.

\begin{definition}[C-Projection and C-Lifting]
    Let $T_i$ be any true twin class and $F_j$ be any false twin class.
    We define the following projection rules:
    for $t_i^1 \in T_i,\ \Proj{C}(t_i^1) = \{t'{_i^1}\}$,
    $\forall t_i^q \in T_i \setminus \{t_i^1\}$, $\Proj{C}(t_i^q) = \{t'{_i^2}\}$,
    $\forall f_j^r \in F_j$, $\Proj{C}(f_j^r) = \{f'_j\}$
    and $\Proj{C}(X) = \bigcup_{u \in X} \Proj{C}(u)$.
    
    Lifting rules are defined as $\Lift{C}(t'{_i^1}) = \{t_i^1\}$,
    $\Lift{C}(t'{_i^2}) = T_i \setminus \{t_i^1\}$,
    $\Lift{C}(f'_j) = \{f_j^1\}$ and $\Lift{C}(Y) = \bigcup_{u' \in Y} \Lift{C}(u')$. Note that $\Proj{C}(\Lift{C}(X)) = X$, for any $X$.
\end{definition}

\begin{definition}[C-Projected Graph]
    The C-projected graph $H$ of $G$ is defined as $V(H) = \Proj{C}(V(G))$ and $v'_iv'_j \in E(H)$ if and only if $\Lift{C}(v'_i)$ and $\Lift{C}(v'_j)$ are neighbours in $G$. Moreover, $H$ is an induced subgraph of $G$.
\end{definition}

\begin{figure}[!htb]
    \centering
        \begin{tikzpicture}[scale=1]
            \GraphInit[unit=3,vstyle=Normal]
            \SetVertexNormal[Shape=circle, FillColor=black, MinSize=2pt]
            \tikzset{VertexStyle/.append style = {inner sep = \inners, outer sep = \outers}}
            \SetVertexNoLabel
            \Vertex[x=0,y=0]{a}
            \Vertex[x=-1,y=0]{b}
            \Vertex[x=0,y=-1]{c}
            
            
            
            \Vertex[x=1,y=0]{h}
            \Vertex[x=0,y=1]{i}
            \Vertex[a=45, d=1]{j}
            
            \Edge(a)(b)
            \Edge(a)(c)
            
            \Edge(a)(h)
            \Edge(a)(i)
            \Edge(a)(j)
            \begin{scope}[shift={(-1.65cm, 0cm)}]
                \grComplete[RA=0.65]{3}
            \end{scope}
            \begin{scope}[shift={(0cm, -1.65cm)}]
                \grComplete[RA=0.65]{4}
            \end{scope}
        \end{tikzpicture}
    \hfill
        \begin{tikzpicture}[scale=1]
            \GraphInit[unit=3,vstyle=Normal]
            \SetVertexNormal[Shape=circle, FillColor=black, MinSize=2pt]
            \tikzset{VertexStyle/.append style = {inner sep = \inners, outer sep = \outers}}
            \SetVertexNoLabel
            \Vertex[x=0,y=0]{a}
            \Vertex[x=-1,y=0]{b}
            \Vertex[x=0,y=-1]{c}
            
            
            
            \Vertex[a=45, d=1]{j}
            
            \Edge(a)(b)
            \Edge(a)(c)
            
            \Edge(a)(j)
            \begin{scope}[shift={(-1.65cm, 0cm)}]
                \grComplete[RA=0.65]{3}
            \end{scope}
            \begin{scope}[rotate=90,shift={(-1.65cm, 0cm)}]
                \grComplete[RA=0.65]{3}
            \end{scope}
        \end{tikzpicture}
    \hfill
    
    
    \caption{A graph (left), and its C-projected graph (right).}
    \label{fig:c-projected}
\end{figure}


For the remainder of this section, $G$ will be the input graph to \textsc{clique coloring} and $H$ the C-Projected graph of $G$.


\begin{lemma}
    \label{lem:lift_proj_clique}
    If $C' \in \clq(H)$ then $\Lift{C}(C') \in \clq(G)$. Conversely, if $C \in \clq(G)$ then $\Proj{C}(C) \in \clq(H)$.
\end{lemma}

\begin{tproof}
    ($\Rightarrow$) Note that $C = \Lift{C}(C')$ is a clique due to the definition of $\Lift{C}$ and the fact that $C'$ is a clique.
    By the contrapositive, suppose that $C$ is not a maximal clique.
    In this case, there is some vertex $u \in V(G)$ such that $C \cup \{u\}$ is a (not necessarily maximal) clique of $G$. Note that either:
    (i) if $u \in T_i$, $T_i \nsubseteq C$ and $u \notin \Lift{C}(t'{_i^1})$ or $u \notin \Lift{C}(t'{_i^2})$, thus $\Proj{C}(u) \notin C'$;
    (ii) if $u \in F_j$, $F_j \cap C = \emptyset$ and $\Proj{C}(u) \notin C'$.
    Since no two vertices of $H$ are lifted to the same vertex of $G$ and $\Proj{C}(u) \notin C'$, it follows that  $\Proj{C}(C \cup \{u\}) = \Proj{C}(C) \cup \Proj{C}(u) = C' \cup \Proj{C}(u)$ is a clique by the definition of $\Proj{C}$.
    
    ($\Leftarrow$) Clearly, $C' = \Proj{C}(C)$ is a clique of $H$, due to the definition of $\Proj{C}$.
    Suppose, however, that $C' \notin \clq(H)$, which implies that there is some $u' \in V(H)$ such that $C' \cup \{u'\}$ is a clique of $H$ and let $u \in \Lift{C}(u')$.
    By the definition of $\Lift{C}$, $C \subseteq N(u)$ if and only if $\Proj{C}(C) \subseteq N(u')$, which implies that $C'$ is not maximal only if $C$ is not maximal.
    A contradiction that completes the proof.
\end{tproof}

\begin{theorem}
    \label{thm:projected_clique_coloring}
     $G$ is $k$-clique-colorable if and only if $H$ is $k$-clique-colorable.
\end{theorem}

\begin{tproof}
    Let $\varphi_G$ be a $k$-clique-coloring of $G$ that complies with Lemmas~\ref{lem:clique_false_twins} and~\ref{lem:clique_true_twins}.
    Without loss of generality, for every $T_i$, we color $t_i^1$ with one color and $T_i \setminus \{t_i^1\}$ with the other, if it exists, otherwise color every vertex of $T_i$ with the same color.
    We define the $k$-clique-coloring of $H$ as $\varphi_H(u') = \varphi_G(u \in \Lift{C}(u'))$, for every $u' \in V(H)$.
    Suppose now that there exists some $C' \in \clq(H)$ such that $|\varphi_H(C')| = 1$.
    By Lemma~\ref{lem:lift_proj_clique}, $\Lift{C}(C')$ is a maximal clique of $G$ and, since $|\varphi_G\left(\Lift{C}(C')\right)| = 1$, it holds that $\Lift{C}(C')$ is a monochromatic maximal clique of $G$ and $\varphi_G$ is not a valid $k$-clique-coloring, which contradicts the hypothesis.
        
    Now, let $\varphi_H$ be a $k$-clique-coloring of $H$, and define $\varphi_G(u) = \varphi_H(u' \in \Proj{C}(u))$.
    By assuming that there exists some $C \in \clq(G)$ such that $\left|\varphi_G(C)\right| = 1$ and using Lemma~\ref{lem:lift_proj_clique}, it is clear that $\left|\varphi_H(\Proj{C}(C))\right| = 1$ which is impossible, since $\varphi_H$ is a valid $k$-clique-coloring of $H$.
\end{tproof}

\begin{theorem}
    \label{thm:fpt_clique}
    There exists an $\bigOs{2^{2\nd(G)}}$ time algorithm for \textsc{clique coloring}
\end{theorem}

\begin{tproof}
    Start by computing the optimal type partition of $G$ in $\bigO{n^3}$ time and building $H$ in $\bigO{n + m}$.
    Afterwards, color $H$ in $\bigOs{2^{2d}}$ time using Theorem~\ref{thm:clique_color_algorithm} and lift the coloring using the construction described in the proof of Theorem~\ref{thm:projected_clique_coloring} in $\bigO{n}$.
\end{tproof}