\chapter{Star Graph}
\label{ch:star_graph}
Section~\ref{sec:intersections} discussed the basics on intersection graph theory, highlighting clique and line graphs, and pointing towards recent work on biclique graphs.
In this chapter, our goal is to study another intersection graph of maximal structures of a graph, the \tdef{star graph}.
As the name implies, members of this class are exactly the intersection graphs of the maximal stars of a graph.
Intuitively, stars appear much more manageable than bicliques, as one member of the partition is a single vertex.
Moreover, the star graph and the biclique graph of $C_4$-free graphs coincide.
This study is, then, an effort to determine both the complexity of biclique and star graph recognition by tackling the latter.

The \tdef{star operator} $\K{S}$ applied on $H$ is the intersection graph of the maximal stars of $H$.
The result of $\K{S}$ is called the \tdef{image graph}, or simply \tdef{image}.
Given a star graph $G$, any $H$ such that $\K{S}(H) \simeq G$ is called a \tdef{pre-image} of $G$.
The \tdef{center} of a star $s = K_{1,n}$ is the vertex of the partition of size one.
A \tdef{leaf} of a star $K_{1,n}$ is one of the vertices of degree one.

When detailing which vertices belong to a star, we shall describe it by $\{v_1\}\{v_2, \dots, v_{n+1}\}$, with $v_1$ being its center and the other $n$ vertices its leaves.
If the star is a single edge, choose one of the vertices to be the center and the other to be the leaf.
Unless noted, $G$ will be our star graph and $H$ the pre-image of $G$.
The family of all maximal stars of $G$ is denoted by $\str(G)$.
For the entirety of this chapter, we assume that all of our graphs are connected.



\begin{figure}[!htb]
    \centering
    
        \begin{tikzpicture}[rotate=90,scale=0.9]
            
                \GraphInit[unit=3,vstyle=Normal]
                %\draw[help lines] (-5,-5) grid (5,5);
                \SetVertexNormal[Shape=circle, FillColor=black, MinSize=2pt]
                \tikzset{VertexStyle/.append style = {inner sep = \inners, outer sep = \outers}}
                \begin{scope}[shift={(-2.41cm, 0cm)}]
                    \SetVertexNoLabel
                    \grEmptyCycle[RA=1.41,prefix=a]{4}
                    \Edges(a0,a1,a2,a3,a0)
                \end{scope}
                \begin{scope}[shift={(2.41cm, 0cm)}]
                    \SetVertexNoLabel
                    \grEmptyCycle[RA=1.41,prefix=b]{4}
                    \Edges(b0,b1,b2,b3,b0)
                \end{scope}
                \Edge(a0)(b2)
        \end{tikzpicture}
        \hfill
        \begin{tikzpicture}[rotate=90,scale=0.9]
            
                \GraphInit[unit=3,vstyle=Normal]
                %\draw[help lines] (-5,-5) grid (5,5);
                \SetVertexNormal[Shape=circle, FillColor=black, MinSize=2pt]
                \tikzset{VertexStyle/.append style = {inner sep = \inners, outer sep = \outers}}
                \begin{scope}[shift={(-2.41cm, 0cm)}]
                    \SetVertexNoLabel
                    \grEmptyCycle[RA=1.41,prefix=a]{4}
                    \Edges(a0,a1,a2,a3,a0)
                    \Edge(a0)(a2)
                    \Edge(a1)(a3)
                \end{scope}
                \begin{scope}[shift={(2.41cm, 0cm)}]
                    \SetVertexNoLabel
                    \grEmptyCycle[RA=1.41,prefix=b]{4}
                    \Edges(b0,b1,b2,b3,b0)
                    \Edge(b0)(b2)
                    \Edge(b1)(b3)
                \end{scope}
                \Edge(a0)(b2)
                \Edge(a0)(b1)
                \Edge(a0)(b3)
                \Edge(b2)(a1)
                \Edge(b2)(a3)
        \end{tikzpicture}
        \hfill
    
    
    
    \caption{A graph (left) and its star graph (right).}
    \label{fig:star}
\end{figure}


\section{Characterization}

Much of the following discussion will be about edge clique covers, a central piece on the characterization of many intersection graph classes.
We denote this family of subsets of $G$ by $\mathcal{Q} = \{Q_1, \dots, Q_n\}$.
The usual strategy in these constructions is to use each clique as a vertex of the pre-image; this is also our approach.
Since each vertex $a \in V(G)$ must be a star in $H$, it is reasonable to partition each clique as $Q_i \sim \{Q_i^c, Q_i^f\}$, that is, the vertices $a \in Q_i^c$ correspond to the stars of $G$ with center in $v_i \in V(H)$, while the vertices $a \in Q_i^f$ correspond to the stars of $G$ where $v_i \in V(H)$ is a leaf.
We call such an edge clique cover a \tdef{star-partitioned edge clique cover} of $\mathcal{Q}$.

To simplify our notation, for each $a \in V(G)$, we denote its \tdef{center} by $c(a) = i \mid a \in Q_i^c$, its \tdef{leaf set} by $F(a) = \{i \mid a \in Q_i^f\}$ and its \tdef{cover} by $Q(a) = F(a) \cup \{c(a)\}$. For each pair of cliques $Q_i, Q_j \in \mathcal{Q}$, their \tdef{leaf-leaf intersection} is given by $\ff(i,j) = Q_i^f \cap Q_j^f$ and its \tdef{center-leaf intersection} by $\cf(i,j) = \left(Q_i^c \cap Q_j^f\right) \cup \left(Q_i^f \cap Q_j^c\right)$.

\begin{definition}[Star-compatibility]
    Given a graph $G$ and a star-partitioned edge clique cover $\mathcal{Q}$ of $G$, we say that $\mathcal{Q}$ is \tdef{star-compatible} if, for every $a \in V(G)$, $|Q(a)| \geq 2$, $\exists!\ i$ such that $a \in Q_i^c$ and if, for every $Q_i, Q_j \in \mathcal{Q}$, if $Q_i \cap Q_j \neq \emptyset$, either $\cf(i,j) = \emptyset$ or $\ff(i,j) = \emptyset$.
\end{definition}

\begin{definition}[Star-differentiability]
    \label{def:differentiability}
    Given a graph $G$ and a star-partitioned edge clique cover $\mathcal{Q}$ of $G$, we say that $\mathcal{Q}$ is \tdef{star-differentiable} if for every $Q_i \in \mathcal{Q}$ and for every pair $\{a, a'\} \subseteq Q_i$ the following conditions hold:
    \begin{enumerate}
        \item If $\{a, a'\} \subseteq Q_i^c$, there exists $Q_j, Q_k \in \mathcal{Q}$ such that $a \in Q_j^f$, $a' \in Q_k^f$, $a \notin Q_k^f$, $a \notin Q_j^f$ and $\cf(j,k) \neq \emptyset$. Moreover, if $Q_i^c \cap Q_j^f \cap Q_k^f = \emptyset$, $\cf(j,k) \neq \emptyset$.
        \item If $a \in Q_i^c$, $a' \in Q_k^c$ and $a \notin Q_k^f$, then there is some $j \in F(a)$ with $\cf(j,k) \neq \emptyset$, $j \notin Q(a')$ and, for every $j' \in F(a)$ with $\cf(j',k) = \emptyset$, $Q_i^c \cap \bigcap_{j'} \ff(j',k) \neq \emptyset$.
        \item If $a \in Q_i^c$, $a' \in Q_k^c$ and $a \in Q_k^f$, for every $j \in F(a) \setminus \{k\}$, $\cf(j,k) = \emptyset$.
        \item If $\{a, a'\} \subseteq Q_i^f$ and $j = c(a) \neq c(a') = k$, then either $Q_i^c \cap \ff\left(j,k\right) \neq \emptyset$ or $\cf\left(j,k\right) \neq \emptyset$.
    \end{enumerate}
\end{definition}

Figures~\ref{fig:diff_cases} and~\ref{fig:diff_cases2} show the four cases of Definition~\ref{def:differentiability} as seen on the pre-image of the star graph we shall build from $\mathcal{Q}$.

\begin{figure}[!htb]
    \centering
            \begin{tikzpicture}[scale=1]
                %\draw[help lines] (-5,-5) grid (5,5);
                \begin{scope}[shift={(0cm,1cm)}]
                    \GraphInit[unit=3,vstyle=Normal]
                    \SetVertexNormal[Shape=circle, FillColor=black, MinSize=2pt]
                    \tikzset{VertexStyle/.append style = {inner sep = \inners, outer sep = \outers}}
                    \Vertex[Math,Ldist=3pt,Lpos=90,LabelOut,L={i},x=0,y=0]{i1}
                    \Vertex[Math,Ldist=3pt,Lpos=180,LabelOut,L={j},a=150, d=2]{j1}
                    \Vertex[Math,Ldist=3pt,Lpos=0,LabelOut,L={k},a=30, d=2]{k1}
                    \node[] at (-0.6,-0.35) {$a$};
                    \node[] at (0.6,-0.31) {$a'$};
                    \Vertex[Math,x=0,y=-2,Ldist=3pt,Lpos=270,LabelOut,L={j'}]{r1};
                    
                    \Edge(i1)(j1)
                    \Edge(i1)(r1)
                    \Edge(i1)(k1)
                    \Edge(j1)(k1)
                    \Edge[style=dotted](j1)(r1)
                    \Edge[style=dotted](k1)(r1)
                \end{scope}
            \end{tikzpicture}
    \hfill
            \begin{tikzpicture}[scale=1]
                %\draw[help lines] (-5,-5) grid (5,5);
                \begin{scope}[shift={(0cm,0cm)}]
                    
                    \GraphInit[unit=3,vstyle=Normal]
                    \SetVertexNormal[Shape=circle, FillColor=black, MinSize=2pt]
                    \tikzset{VertexStyle/.append style = {inner sep = \inners, outer sep = \outers}}
                    \Vertex[Math,Ldist=3pt,Lpos=110,LabelOut,L={i},x=0,y=0]{i2}
                    \Vertex[Math,Ldist=3pt,Lpos=180,LabelOut,L={j},a=150, d=2]{j2}
                    \Vertex[Math,Ldist=3pt,Lpos=0,LabelOut,L={k},a=30, d=2]{k2}
                    \node[] at (-0.6,-0.35) {$a$};
                    \node[] at (1.23,1.50) {$a'$};
                    \Vertex[Math,Ldist=3pt,Lpos=270,LabelOut,L={j'},x=0,y=-2]{j'2};
                    \SetVertexNoLabel
                    \Vertex[x=1.73,y=3]{r2}
                    
                    \Edge(i2)(j2)
                    \Edge(i2)(j'2)
                    \Edge(i2)(k2)
                    \Edge(j2)(k2)
                    \Edge(r2)(k2)
                    \Edge[style=dotted](j2)(j'2)
                    \Edge[style=dotted](k2)(j'2)
                    \Edge[style=dotted](i2)(r2)
                \end{scope}
            \end{tikzpicture}
    \hfill
            \begin{tikzpicture}[rotate=270,scale=1]
                %\draw[help lines] (-5,-5) grid (5,5);
                \begin{scope}[shift={(0cm,0cm)}]
                    \GraphInit[unit=3,vstyle=Normal]
                    \SetVertexNormal[Shape=circle, FillColor=black, MinSize=2pt]
                    \tikzset{VertexStyle/.append style = {inner sep = \inners, outer sep = \outers}}
                    \Vertex[Math,Ldist=3pt,Lpos=270,LabelOut,L={i},x=0,y=0]{i3}
                    \Vertex[Math,Ldist=3pt,Lpos=180,LabelOut,L={j},a=150, d=2]{j3}
                    \Vertex[Math,Ldist=3pt,Lpos=270,LabelOut,L={k},a=30, d=2]{k3}
                    \SetVertexNoLabel
                    \Vertex[Math,Ldist=3pt,Lpos=0,LabelOut,L={k},a=30, d=4]{r3}
                    \Vertex[Math,Ldist=3pt,Lpos=0,LabelOut,L={k},x=0, y=2]{l3}
                    \node[] at (0,0.4) {$a$};
                    \node[] at (1.73,1.4) {$a'$};
                    
                    \Edge(i3)(j3)
                    \Edge(i3)(k3)
                    \Edge[style=dotted](j3)(k3)
                    \Edge(k3)(r3)
                    \Edge(k3)(l3)
                    \Edge[style=dotted](r3)(l3)
                \end{scope}
            \end{tikzpicture}
    \hfill
    
    
    
    \caption{The first three cases of Definition~\ref{def:differentiability}. The first (left), second (center) and third (right).}
    \label{fig:diff_cases}
\end{figure}


\begin{figure}[!htb]
    \centering
    
        \begin{tikzpicture}[scale=1]
            
            \begin{scope}[shift={(-4cm,-1cm)}]
                \GraphInit[unit=3,vstyle=Normal]
                \SetVertexNormal[Shape=circle, FillColor=black, MinSize=2pt]
                \tikzset{VertexStyle/.append style = {inner sep = \inners, outer sep = \outers}}
                \Vertex[Math,Ldist=3pt,Lpos=90,LabelOut,L={i},x=0,y=0]{i4}
                \Vertex[Math,Ldist=3pt,Lpos=270,LabelOut,L={j_1},a=150, d=2]{j4}
                \Vertex[Math,Ldist=3pt,Lpos=270,LabelOut,L={k_1},a=30, d=2]{k4}
                
                
                \SetVertexNoLabel
                \Vertex[Math,Ldist=3pt,Lpos=0,LabelOut,L={k},a=30, d=4]{r4}
                \Vertex[Math,Ldist=3pt,Lpos=0,LabelOut,L={k},x=0.4, y=2]{l4}
                \Vertex[Math,Ldist=3pt,Lpos=0,LabelOut,L={k},a=150, d=4]{m4}
                \Vertex[Math,Ldist=3pt,Lpos=0,LabelOut,L={k},x=-0.4, y=2]{n4}
                
                
                \node[] at (-1.77,1.5) {$a_1$};
                \node[] at (1.77,1.5) {$a'_1$};
                
                \Edge(i4)(j4)
                \Edge(i4)(k4)
                \Edge[style=dotted](j4)(k4)
                \Edge(k4)(r4)
                \Edge(k4)(l4)
                \Edge[style=dotted](r4)(l4)
                \Edge(j4)(n4)
                \Edge(j4)(m4)
                \Edge[style=dotted](m4)(n4)
                
                
            \end{scope}
        \end{tikzpicture}
        \hfill
        \begin{tikzpicture}[scale=1]
            \begin{scope}[shift={(4cm,1cm)}]
                \GraphInit[unit=3,vstyle=Normal]
                \SetVertexNormal[Shape=circle, FillColor=black, MinSize=2pt]
                \tikzset{VertexStyle/.append style = {inner sep = \inners, outer sep = \outers}}
                \Vertex[Math,Ldist=3pt,Lpos=90,LabelOut,L={i},x=0,y=0]{i5}
                \Vertex[Math,Ldist=3pt,Lpos=90,LabelOut,L={j_2},a=210, d=2]{j5}
                \Vertex[Math,Ldist=3pt,Lpos=90,LabelOut,L={k_2},a=-30, d=2]{k5}
                
                
                
                \SetVertexNoLabel
                
                
                \Vertex[Math,Ldist=3pt,Lpos=0,LabelOut,L={k},a=-30, d=4]{r5}
                \Vertex[Math,Ldist=3pt,Lpos=0,LabelOut,L={k},x=0.4, y=-2]{l5}
                \Vertex[Math,Ldist=3pt,Lpos=0,LabelOut,L={k},a=210, d=4]{m5}
                \Vertex[Math,Ldist=3pt,Lpos=0,LabelOut,L={k},x=-0.4, y=-2]{n5}
                
                \node[] at (-1.77,-1.5) {$a_2$};
                \node[] at (1.77,-1.5) {$a'_2$};
                
                
                \Edge(i5)(j5)
                \Edge(i5)(k5)
                \Edge(j5)(k5)
                \Edge(k5)(r5)
                \Edge(k5)(l5)
                \Edge[style=dotted](r5)(l5)
                \Edge(j5)(n5)
                \Edge(j5)(m5)
                \Edge[style=dotted](m5)(n5)
                
            \end{scope}
        \end{tikzpicture}
    
    
    
    \caption{The fourth case 
    Definition~\ref{def:differentiability}.}
    \label{fig:diff_cases2}
\end{figure}



We emphasize that: (i) star-compatibility translates the structural properties of stars; and (ii) star-differentiability enumerates the possible ways that two stars that share at least one vertex are different.
Note that, the \say{missing case}, where $\{a,a'\} \in Q_i^f$ and $c(a) = c(a') = k$ is exactly the same case as 1, but with $\{a,a'\} \in Q_k^c$ instead of $Q_i^c$.

\begin{lemma}
    \label{lem:star_maximality}
    Let $G$ be a graph and $\mathcal{Q}$ a star-partitioned edge clique cover of $G$. If $\mathcal{Q}$ is star-compatible and star-differentiable then, for every pair $\{a, a'\} \subseteq V(G)$, $Q(a) \nsubseteq Q(a')$ and $Q(a') \nsubseteq Q(a)$.
\end{lemma}

\begin{tproof}
    If $a$ and $a'$ do not share any clique, the result is trivial.
    
    Otherwise they do share some clique, say $Q_i$.
    For properties 1, 2 and 4 of Definition~\ref{def:differentiability}, we conclude that $i \in Q(a) \cap Q(a')$ and that $j \in Q(a)$, $k \in Q(a')$ but $j \notin Q(a')$ and $k \notin Q(a)$, which implies $Q(a) \nsubseteq Q(a')$ and $Q(a') \nsubseteq Q(a)$.
    
    For property 3, however, we first conclude that there is some $j \in Q(a)$ but $j \notin Q(a')$, otherwise we would have $\cf(j,k) \neq \emptyset$ and $\ff(j,k) \neq \emptyset$.
    Consequently, $Q(a) \nsubseteq Q(a')$.
    For the converse, we note that $\{a, a'\} \subseteq Q_k$ and, following the same argument, we conclude that there is $j' \in Q(a')$ but $j' \notin Q(a)$ and, finally, that $Q(a') \nsubseteq Q(a)$.
\end{tproof}

\begin{theorem}
    \label{thm:star_characterization}
    $G$ is the star graph for some $H$ if and only if there is a star-compatible and star-differentiable star-partitioned edge clique cover $\mathcal{Q}$ of $G$.
\end{theorem}

\begin{tproof}
    In this proof, we assume that $H$ has $n$ vertices ($v_i$). For simplicity, a star $s_a \in \str(H)$ corresponds to the vertex $a \in V(G)$.
    
    For each $v_i \in V(H)$, let $S(v_i) = \{s_a \in \str(H) \mid v_i \in s_a\}$, that is, the maximal stars of $H$ that contain $v_i$.
    Clearly, we can partition these sets as $S(v_i) \sim \{S^c(v_i), S^f(v_i)\}$, that is, the stars where $v_i$ is the center and where it is a leaf, respectively.
    Our goal is to show that $\mathcal{Q} = \{Q_1, \dots, Q_n\}$, with $Q_i^c = S^c(v_i)$ and $Q_i^f = S^f(v_i)$ is a star-partitioned edge clique cover of $G$ satisfying star-compatibility and star-differentiability.
    
    To verify that $\mathcal{Q}$ is a star-partitioned edge clique cover of $G$, first note that every $Q_i$ is a clique of $G$, since the corresponding stars share at least $v_i \in V(H)$.
    For the coverage part, every $aa' \in E(G)$ has two corresponding stars $s_a, s_{a'} \in \str(H)$, which share at least one vertex, say $v_i \in V(H)$, since $G \simeq \K{S}(H)$.
    By the construction of $\mathcal{Q}$, there is some $Q_i \in \mathcal{Q}$ which corresponds to every maximal star that contains $v_i$; this guarantees that $aa'$ is covered by at least one clique of $\mathcal{Q}$.
    
    For the other properties, first take two vertices $v_i,v_j \in V(H)$ with $v_iv_j \notin E(H)$ but $S(v_i) \cap S(v_j) \neq \emptyset$.
    Clearly, no star in $S(v_i) \cap S(v_j)$ may have $v_i$ and $v_j$ in different sides of its bipartition, thus $S(v_i) \cap S(v_j) = S^f(v_i) \cap S^f(v_j)$.
    Now, suppose that $v_iv_j \in E(H)$; since they are adjacent, any star in $S(v_i) \cap S(v_j)$ must have $v_i$ and $v_j$ in opposite sides of the bipartition and, thus, we have that $S(v_i) \cap S(v_j) = \left(S^c(v_i) \cap S^f(v_j)\right) \cup \left(S^f(v_i) \cap S^c(v_j)\right)$.
    Since each star has a single center, the above analysis shows that $\mathcal{Q}$ satisfies star-compatibility.
    
    For star-differentiability, let $\{s_a, s_{a'}\} \subseteq S(v_i)$. We break our analysis in the same order as the one given in Definition~\ref{def:differentiability}.
    \begin{enumerate}
        \item If $\{s_a, s_{a'}\} \subseteq S^c(v_i)$ there must be at least one leaf in each star, say $v_j$ and $v_k$, respectively, not in the other and these leaves must be adjacent to each other, otherwise at least one of the stars would not be maximal.
        That is, $\{a, a'\} \in Q_i^c$ imply that there is $Q_j,Q_k \in \mathcal{Q}$ with $a \in Q_j^f$, $a' \in Q_k^f$, $a \notin Q_k^f$, $a \notin Q_j^f$ and $\cf(j,k) \neq \emptyset$.
        \item If $s_a \in S(v_i)^c$, $s_{a'} \in S(v_k)^c$ and $s_a \notin S(v_k)^f$, $v_iv_k \in E(H)$ and to keep $v_k$ from being a leaf of $s_a$, one leaf of $s_a$, say $v_j$, must also be adjacent to $v_k$ and not a leaf of $s_{a'}$, since $v_i$ is.
        Now, for every $v_{j'} \in s_a$ and not adjacent to $v_k$, there is a clear $P_3 = v_kv_iv_{j'}$, which must be part of some maximal star.
        Moreover, the set of all $v_{j'}$ non-adjacent to $v_k$ will form a maximal star centered around $v_i$ along with $v_k$.
        Thus, $a \in Q_i^c$, $a' \in Q_k^c$ and $a \notin Q_k^f$, imply that there is some $j \in F(a)$ with $\cf(j,k) \neq \emptyset$, $j \notin Q(a')$ and, for $j' \in F(a)$ with $\cf(j', k) = \emptyset$, $Q_i^c \cap \bigcap_{j'} \ff(j',k) \neq \emptyset$.
        \item If $s_a \in S(v_i)^c$, $s_{a'} \in S(v_k)^c$ and $s_a \in S(v_k)^f$, we know that $s_a = \{v_i\}\{v_k, \dots\}$ and, since $v_k$ is not adjacent to any other leaf $v_j$ of $s_a$, we know that $S(v_j) \cap S(v_k) = S^f(v_j) \cap S^f(v_k)$ and, since $v_k$ is the center of $s_{a'}$, $v_j$ is not one of its leaves.
        Therefore, $a \in Q_i^c$, $a' \in Q_k^c$ and $a \in Q_k^f$, implies that for every $j \in F(a) \setminus \{k\}$, $\cf(j,k) = \emptyset$.
        \item If $\{s_a, s_{a'}\} \subseteq Q_i^f$ and $s_a \in S^c(v_j)$, $s_{a'} \in S^c(v_k)$, either $v_jv_k \notin E(H)$, which induces the existence a star $\{v_i\}\{v_j, v_k, \dots\}$, or $v_jv_k \in E(H)$, which must be part of a star with either $v_j$ or $v_k$ as center and the other as a leaf.
        Hence, $\{a, a'\} \subseteq Q_i^f$ and $j = c(a) \neq c(a') = k$, implies that either $Q_i^c \cap \ff\left(j,k\right) \neq \emptyset$ or $\cf\left(j,k\right) \neq \emptyset$.
    \end{enumerate}
    The above shows that $\mathcal{Q}$ is also star-differentiable, which completes this part of the proof.
    
    For the converse, take $\mathcal{Q}$ a star-partitioned edge clique cover of $G$ satisfying star-compatibility and star-differentiability and let $H$ be a graph with $V(H) = \{v_i \mid Q_i \in \mathcal{Q}\}$ and $E(H) = \{v_iv_j \mid \cf(i,j) \neq \emptyset\}$ and let us prove that $G \simeq \K{S}(H)$.
    
    Take $a \in V(G)$ with $c(a) = i$.
    Due to star-compatibility and the construction of $H$, we know that $H[\{v_j \mid j \in F(a)\}]$ is an independent set of $H$ and that $s_a = \{v_i\}\{v_j \mid j \in F(a)\}$ is a star of $H$.
    Suppose, however, that $s_a$ is not maximal, that is, there is some $v_k \in V(H)$ such that $v_iv_k \in E(H)$ and $s_b = s_a \cup \{v_k\}$ is a star of $H$.
    By the construction of $H$, either there is some $a' \in V(G)$ such that $Q(a) \subseteq Q(a')$, which is impossible due to Lemma~\ref{lem:star_maximality}, or some $a' \in \cf(i,k)$, which we analyze below.
    The following is based on the first two cases of Definition~\ref{def:differentiability}; the other two are impossible, since $k \notin Q(a)$ and $a \in Q_i^c$.
    
    \begin{enumerate}
        \item If $a' \in Q_i^c$, there is some $Q_j \in \mathcal{Q}$ such that $a \in Q_j^f$ and $\cf(j,k) \neq \emptyset$, which implies that $v_jv_k \in E(H)$ and $s_b$ is not a star of $H$. 
        \item If $a' \in Q_k^c$ and $a \notin Q_k^f$, at least one $j \in F(a)$ satisfies $\cf(j,k) \neq \emptyset$ and $j \notin Q(a')$.
        This gives us that $v_jv_k \in E(H)$ and $s_b$ is not a star of $H$.
    \end{enumerate}
    
    Therefore, we conclude that $a'$ cannot exist, that $s_a$ is maximal and, consequently that $V(G) \subseteq V(\K{S}(H))$.
    
    To show that $V(\K{S}(H)) \subseteq V(G)$, take $s = \{v_i\}L$, with $s \in \str(H)$, and suppose that there is some $j,k \in L$ and that for every pair $a \in \cf(i,j)$ and $a' \in \cf(i,k)$, $a \notin Q_k$ and $a' \notin Q_j$.
    That is, $Q_i \cap Q_j \cap Q_k = \emptyset$, due to star-compatibility and the hypothesis that $jk \notin E(H)$.
    Once again, we analyze the possibilities in terms of Definition~\ref{def:differentiability}.
    
    \begin{enumerate}
        \item If $c(a) = c(a') = i$, we have that $\cf(j,k)  \neq \emptyset$, implying that $v_jv_k \in E(H)$, contradicting the hypothesis that $s$ exists.
        \item If $c(a) = i$ and $c(a') = k$, there is some $j' \in Q(a)$ with $\cf(j,k) \neq \emptyset$. To conclude that $j = j'$, we note that, if $j \neq j'$, it would be required that $Q_i^c \cap \ff(j,k) \neq \emptyset$, which is impossible since $Q_i \cap Q_j \cap Q_k = \emptyset$.
        Once again, contradicting the hypothesis that such an $s$ exists.
        \item Trivially impossible since $Q_i \cap Q_j \cap Q_k = \emptyset$.
        \item If $j = c(a) \neq c(a') = k$, either $Q_i^c \cap \ff(j,k) \neq \emptyset$, which is impossible since $Q_i \cap Q_j \cap Q_k = \emptyset$, or $\cf(j,k) \neq \emptyset$, implies that $v_jv_k \in E(H)$ and that $s$ is not a star.
    \end{enumerate}
    
    The above allows us to conclude that there is no $s \in \str(H)$ generated by cliques not pairwise intersecting.
    Such intersection has a unique vertex of $G$ in it due to Lemma~\ref{lem:star_maximality}, which allows us to conclude $V(\K{S}(H)) \subseteq V(G)$ and, consequently, that $V(\K{S}(H)) = V(G)$.
    
    To show that $E(G) \subseteq E(\K{S}(H))$, we first take an edge $ab \in E(G)$.
    Since $\mathcal{Q}$ is a star-partitioned edge clique cover of $G$, there is some $i$ such that $\{a,b\} \subseteq Q_i$ and, because $V(G) = V(\K{S}(H))$ and the construction of $H$, there are corresponding stars $s_a, s_b \in \str(H)$ with $v_i \in s_a \cap s_b$ which guarantee that $ab \in E(\K{S}(H))$.
    For $E(\K{S}(H)) \subseteq E(G)$, take two intercepting stars $s_a,s_b \in \str(H)$ and note that, since $a,b \in \K{S}(H) = V(G)$ and $\mathcal{Q}$ is a star-partitioned edge clique cover of $G$, $ab \in E(G)$ and we conclude that $E(G) = E(\K{S}(H))$, completing the proof.
\end{tproof}

We now pose a version of the decision problem for star graph recognition, which we call \textsc{star graph}, and will be the subject of further study.
We will further require that the output for any algorithm for \textsc{star graph} is already star-partitioned.

\problem{star graph}{A graph $G$.}{Is there a star-partitioned edge clique cover $\mathcal{Q}$ of $G$ satisfying star-compatibility and star-differentiability?}

Theorem~\ref{thm:star_characterization} does not appear to yield many useful properties, but it gives a straightforward verification algorithm to check if a star-partitioned edge clique cover is star-compatible and star-differentiable.

\begin{theorem}
    Given a graph $G$ of order $m$, there is an $\bigO{\max\{m^2n, n^2\}m^2n}$ to decide if a star-partitioned family $\mathcal{Q} \subseteq 2^{V(G)}$ of size $n$ is an edge clique cover of $G$ satisfying star-compatibility and star-differentiability. 
\end{theorem}

\begin{tproof}
    The first task is to determine whether or not $\mathcal{Q}$ is a star-partitioned edge clique cover of $G$.
    The usual $m^2$ algorithm that tests if each $Q_i$ is a clique suffices.
    To check if $\mathcal{Q}$ is an edge clique cover, for each of the $\bigO{m^2}$ edges, we test if one of the $n$ cliques contains it. 
    This simple test takes $\bigO{m^2n}$ times.
    
    To check for star-compatibility: first, for each vertex $a$ of $G$ and each clique $Q_i$, verify if there is a single $i$ such that $a \in Q_i^c$ and at least one $j$ with $a \in Q_j^f$;
    afterwards, for each pair of intercepting cliques $Q_i, Q_j$, test if $\cf(i,j) = \emptyset$ or $\ff(i,j) = \emptyset$.
    The entire process takes $\bigO{mn^2}$ time.
    
    For star-differentiability, we assume that every pairwise intersection of $\mathcal{Q}$ has already been computed in time $\bigO{mn^2}$, and each query $\cf(j,k)$ and $\ff(j,k)$ takes $\bigO{1}$ time.
    Now, for each clique $Q_i$ and for each pair of vertices $\{a, a'\} \in Q_i$, we must check one of the four conditions as follows.
    \begin{enumerate}
        \item If $c(a) = c(a') = i$, for each pair $j \in Q(a)$, $k \in Q(a')$, check if $a' \notin Q_j^f$, $a \notin Q_k^f$ and $\cf(j,k) \neq \emptyset$; this case takes $\bigO{n^2}$.
        \item If $c(a) = i, c(a') = k$ and $a \notin Q_k^f$, for each $j \in F(a)$, check if either $\cf(j,k) \neq \emptyset$ and $j \notin Q(a')$ or $\cf(j,k) = \emptyset$ and $Q_i^c \cap \ff(j,k) \neq \emptyset$; this case takes $\bigO{m^2n}$ time.
        \item If $c(a) = i, c(a') = k$ and $a \in Q_k^f$, check for each $j \in F(a) \setminus \{k\}$, if $\cf(j,k) = \emptyset$, taking $\bigO{n}$ time.
        \item If $j = c(a) \neq c(a') = k$, we check if $Q_i^c \cap \ff(j,k) \neq \emptyset$ in $\bigO{m}$ time, and if $\cf(j,k)$ in $\bigO{1}$ time.
    \end{enumerate}
    In the worst case scenario, we will spend $\bigO{\max{m^2n, n^2}}$ time for each $Q_i$ and each pair $\{a, a'\} \subseteq Q_i$, of which there are $\bigO{m^2n}$ combinations, and conclude that the whole algorithm takes no more than $\bigO{\max\{m^2n, n^2\}m^2n}$ time.
\end{tproof}

Clearly, the problem of the above algorithm is that we have no known bound on the size of $\mathcal{Q}$ with respect to the size of the input graph $G$.
That is, we do not know if \textsc{star graph} is in $\NP$.
We believe that a characterization of \tdef{useless vertices}, that is, vertices whose removal from $H$ does not change $\K{S}(H)$, is essential to determine membership of \textsc{star graph} in $\NP$; we do not have this result at the moment, and leave it as important future work on the topic.
Another possible direction is a new characterization that may give a natural bound on $\mathcal{Q}$, but it is one that does not seem to be a viable choice.

\section{Properties}

The next theorem uses the known result, due to~\cite{moon_moser}, that a graph of order $n$ has at most $3^{n/3}$ maximal independent sets.

\begin{theorem}
    If $G$ is the star graph of a $n$ vertex graph $H$, $|V(G)| \leq n3^{\Delta(H)/3}$.
\end{theorem}

\begin{tproof}
    For every $v \in V(H)$, define $H_v = H[N(v)]$ and note that each maximal independent set of $H_v$ might induce a maximal star of $H$ centered around $v$.
    Since $|V(H_v)| \leq \Delta(H)$, we have that the $H_v$ has at most $3^{\Delta(H)/3}$ maximal independent sets and, therefore, $H$ has at most $3^{\Delta(H)/3}$ maximal stars centered around $v$.
    Summing for every $v \in V(H)$ we arrive at the $n3^{\Delta(H)/3}$ bound.
\end{tproof}

\begin{theorem}
    \label{thm:star_cutless}
    If $G$ is a star graph, no vertex of $G$ is a cut-vertex.
\end{theorem}

\begin{tproof}
    If $|V(G)| \leq 4$, we are done as there are only 5 graphs that satisfy these constraints and none of them contain a cut-vertex.
    They are $K_1$, $K_2$, $K_3$, $K_4$ and $K_4$ with one missing edge.
    
    For graphs with 5 or more vertices, suppose that there is some cut-vertex $x \in G$, that $A,B$ are two of the connected components obtained after removing $x$ from $G$ and take a pair of vertices $a \in V(A) \cap N(x)$, $b \in V(B) \cap N(x)$.
    Suppose now that $G = \K{S}(H)$ for some $H$ and take the stars $s_a, s_b, s_x$ corresponding to $a, b, x$, respectively.
    Since $ab \notin E(G)$ and $ax, bx \in E(G)$, it holds that $s_a \cap s_x \neq \emptyset$ and $s_b \cap s_x \neq \emptyset$ but $s_a \cap s_b = \emptyset$.
    
    If $c(a) = c(x) = i$ and $k = c(b) \neq c(x)$, $s_x$ and $s_b$ share at least on leaf, say $v_j$, since they intercept at some vertex, and $v_j \notin s_a$.
    However, there is no leaf $v_{j'} \in s_a$ adjacent to $v_j$, otherwise there would be an edge $v_jv_{j'} \in E(H)$ and, consequently, some star $s_y$, corresponding to vertex $y \in V(G)$, that keeps $A,B$ connected and intercepts $s_a, s_b, s_x$.
    Therefore, we conclude that no leaf of $s_a$ is adjacent to $v_j$ and, since $c(a) = c(x)$ and $v_iv_j \in E(H)$, we conclude that $v_j \in s_a$, otherwise it would not be maximal, and, consequently, $v_j \in s_a \cap s_b$ and $ab \in E(H)$, which contradicts the hypothesis that $A,B$ are disconnected after removing $x$.
    The case where $c(x) = c(b) \neq c(a)$ follows the exact same argument.
    
    Now if $c(a) \neq c(x) = i$ and $c(x) \neq c(b)$, it is easy to see that $v_i$ cannot be a leaf of both $s_a$ and $s_b$ simultaneously, otherwise $v_i \in s_a \cap s_b$ and $ab \in E(H)$.
    So we have two cases to analyze:
    \begin{enumerate}
        \item If $v_i$ is a leaf of $s_a$, $v_j \in s_x \cap s_b$ and $k = c(b)$, clearly $s_y = \{v_j\}\{v_i, v_k, \dots \}$ is a maximal star of $H$ that intercepts $s_a, s_b, s_x$, keeping $A,B$ from being disconnected.
        The case where $v_i$ is a leaf $s_b$ is the same, and we omit it for brevity.
        \item If $v_i$ not a leaf of neither $s_a$ nor $s_b$, $c(a) = j$ and $c(b) = k$, we have leaves $v_{j'} \in s_a \cap s_x$ , $v_{k'} \in s_b \cap s_x$ which form at least two intercepting maximal stars, $s_{a'} = \{v_{j'}\}\{v_i, v_j, \dots\}$ and $s_{b'} = \{v_{k'}\}\{v_i, v_k, \dots\}$, such that $s_{a'} \cap s_a \cap s_x \neq \emptyset$ and $s_{b'} \cap s_b \cap s_x \neq \emptyset$.
    \end{enumerate}
    
    These cases allow us to conclude that $A,B$ remains connected no matter the configuration of the intersection of the corresponding stars in $H$.
    Consequently, $x$ cannot exist and we complete the proof.
\end{tproof}

The above proof suggests a stronger result, given by the following theorem.

\begin{theorem}
    Every edge of a star graph $G$ is contained in at least one triangle if $|V(G)| \geq 3$.
\end{theorem}

\begin{tproof}
    The only connected star graph with 3 vertices is $K_3$, so take $G$ with $|V(G)| \geq 4$.
    Take a pre-image $H$ of $G$, $ab \in E(G)$, $s_a, s_b \in \str(H)$ the corresponding stars to $a, b$, and assume that $ab$ is not contained in any triangle of $G$.
    Since $G$ is connected, there is at least one $x \in V(G)$ adjacent to (w.l.o.g) $a$, but not to $b$, and a corresponding maximal star $s_x$ of $H$.
    Below, we analyze the possible intersections between $s_a$ and $s_b$ and conclude that there is always some star $s_y$ that shares one vertex with $s_a$ and $s_b$.
    \begin{enumerate}
        \item If $c(a) = c(b) = i$ and the center of $s_x$ is a leaf of $s_a$, clearly $v_i$ is not a leaf of $s_x$, otherwise $s_x \cap s_b \neq \emptyset$, therefore there is some leaf $v_j \in s_x$ with $v_iv_j \in E(H)$, which must be part of at least one maximal star $s_y$ of $H$, from which we conclude that $s_a \cap s_b \cap s_y \neq \emptyset$, $s_a \cap s_x \cap s_y \neq \emptyset$ and both $ab$ and $ax$ are in a triangle of $G$.
        \item If $c(a) = c(b) = i$ and a leaf $v_j$ of $s_x$ is a leaf of $s_a$, either the center $v_k$ of $s_x$ is adjacent to $v_i$, in which case $v_iv_k \in E(H)$ and we follow the same argument as in the previous case, or they are not adjacent, implying that there is a maximal star $s_y = \{v_j\}\{v_i, v_k, \dots\}$ which intercepts $s_a, s_b, s_x$, which allows us to conclude that $s_a, s_b, s_x, s_y$ is a clique of $G$.
        \item if $i = c(a) \neq c(b) = k$, there is some leaf $v_j \in s_a \cap s_b$. Clearly, if $v_iv_k \in E(H)$, there is a star that intercepts both $s_a$ and $s_b$;
        otherwise, $v_iv_k \notin E(H)$ and we conclude that $s_y = \{v_j\}\{v_i, v_k, \dots\} \in \str(H)$ intercepts $s_a$ and $s_b$ and creates a triangle that contains $ab$.
    \end{enumerate}
\end{tproof}

\section{Star graph of triangle-free graphs}


\begin{figure}[!htb]
    \centering
    
        \begin{tikzpicture}[rotate=90,scale=0.9]
            
                \GraphInit[unit=3,vstyle=Normal]
                %\draw[help lines] (-5,-5) grid (5,5);
                \SetVertexNormal[Shape=circle, FillColor=black, MinSize=2pt]
                \tikzset{VertexStyle/.append style = {inner sep = \inners, outer sep = \outers}}
                \begin{scope}[shift={(-2.41cm, 0cm)}]
                    \SetVertexNoLabel
                    \grEmptyCycle[RA=1.41,prefix=a]{4}
                    \Edges(a0,a1,a2,a3,a0)
                    \Vertex[x=0,y=2.41]{1a}
                    %\Vertex[y=0,x=-2.41]{2a}
                    \Vertex[x=0,y=-2.41]{3a}
                    \Edge(1a)(a1)
                    %\Edge(2a)(a2)
                    \Edge(3a)(a3)
                \end{scope}
                \begin{scope}[shift={(2.41cm, 0cm)}]
                    \SetVertexNoLabel
                    \grEmptyCycle[RA=1.41,prefix=b]{4}
                    \Edges(b0,b1,b2,b3,b0)
                    \Vertex[x=0,y=2.41]{1b}
                    %\Vertex[y=0,x=2.41]{0b}
                    \Vertex[x=0,y=-2.41]{3b}
                    \Edge(1b)(b1)
                    %\Edge(0b)(b0)
                    \Edge(3b)(b3)
                \end{scope}
                \Edge(a0)(b2)
        \end{tikzpicture}
        \hfill
        \begin{tikzpicture}[rotate=90,scale=0.9]
            
                \GraphInit[unit=3,vstyle=Normal]
                %\draw[help lines] (-5,-5) grid (5,5);
                \SetVertexNormal[Shape=circle, FillColor=black, MinSize=2pt]
                \tikzset{VertexStyle/.append style = {inner sep = \inners, outer sep = \outers}}
                \begin{scope}[shift={(-2.41cm, 0cm)}]
                    \SetVertexNoLabel
                    \grEmptyCycle[RA=1.41,prefix=a]{4}
                    \Edges(a0,a1,a2,a3,a0)
                    \Vertex[x=0,y=2.41]{1a}
                    %\Vertex[y=0,x=-2.41]{2a}
                    \Vertex[x=0,y=-2.41]{3a}
                    \Edge(1a)(a1)
                    %\Edge(2a)(a2)
                    \Edge(3a)(a3)
                    
                    \Edge(a0)(a2)
                    \Edge(a1)(a3)
                    
                    \Edge(1a)(a2)
                    \Edge(3a)(a2)
                    \Edge(1a)(a0)
                    \Edge(3a)(a0)
                    %\Edge(2a)(a1)
                    %\Edge(2a)(a3)
                \end{scope}
                \begin{scope}[shift={(2.41cm, 0cm)}]
                    \SetVertexNoLabel
                    \grEmptyCycle[RA=1.41,prefix=b]{4}
                    \Edges(b0,b1,b2,b3,b0)
                    \Vertex[x=0,y=2.41]{1b}
                    %\Vertex[y=0,x=2.41]{0b}
                    \Vertex[x=0,y=-2.41]{3b}
                    \Edge(1b)(b1)
                    %\Edge(0b)(b0)
                    \Edge(3b)(b3)
                    
                    \Edge(b0)(b2)
                    \Edge(b1)(b3)
                    
                    \Edge(1b)(b2)
                    \Edge(3b)(b2)
                    \Edge(1b)(b0)
                    \Edge(3b)(b0)
                    %\Edge(0b)(b1)
                    %\Edge(0b)(b3)
                \end{scope}
                \Edge(a0)(b2)
                \Edge(a0)(b1)
                \Edge(a0)(b3)
                \Edge(b2)(a1)
                \Edge(b2)(a3)
        \end{tikzpicture}
        \hfill
        \begin{tikzpicture}[shift={(1.41cm,0cm)},rotate=90,scale=0.9]
            
                \GraphInit[unit=3,vstyle=Normal]
                %\draw[help lines] (-5,-5) grid (5,5);
                \SetVertexNormal[Shape=circle, FillColor=black, MinSize=2pt]
                \tikzset{VertexStyle/.append style = {inner sep = \inners, outer sep = \outers}}
                \begin{scope}[shift={(-2.41cm, 0cm)}]
                    \SetVertexNoLabel
                    \grEmptyCycle[RA=1.41,prefix=a]{4}
                    \Edges(a0,a1,a2,a3,a0)
                    \Edge(a0)(a2)
                    \Edge(a1)(a3)
                \end{scope}
                \begin{scope}[shift={(2.41cm, 0cm)}]
                    \SetVertexNoLabel
                    \grEmptyCycle[RA=1.41,prefix=b]{4}
                    \Edges(b0,b1,b2,b3,b0)
                    \Edge(b0)(b2)
                    \Edge(b1)(b3)
                \end{scope}
                \Edge(a0)(b2)
                \Edge(a0)(b1)
                \Edge(a0)(b3)
                \Edge(b2)(a1)
                \Edge(b2)(a3)
        \end{tikzpicture}
        \hfill
    
    
    
    \caption{A triangle-free graph (left), its square (center) and its star graph (right).}
    \label{fig:tri_star}
\end{figure}



In this section we will interpret star graphs in terms of powers of graphs and give an $\NPcness$ reduction for the recognition of a subclass of star graphs.

\begin{observation}
    Every vertex of degree at least two in a triangle free graph is the center of exactly one maximal star.
\end{observation}

\begin{theorem}
    \label{thm:tri_free_star}
    If $H$ is a $K_3$-free graph with at least 3 vertices, $D$ are its vertices of degree at least 2 and $G = \K{S}(H)$, it holds that $G \simeq H[D]^2$.
\end{theorem}

\begin{tproof}
    There are only 4 graphs $H$ with at most 4 vertices that are triangle free, namely $P_3, K_{1,3}, P_4, C_4$, which have star graphs equal to $K_1, K_1, K_2, K_4$, respectively, and clearly satisfy the property.
    We restrict our analysis to graphs of order at least 5, and define $L = V(H) \setminus D$.
    This constraint implies that every star has at least 3 vertices.
    
    The previous observation trivially guarantees that $|V(G)| = |D| = |V(H)| - |L| = |V(H^2)| - |L|$, since the only vertices of $H$ that do not have a maximal star centered around them are precisely the ones with degree 1 and every other is the center of exactly 1 maximal star.
    Consequently, we can safely build a bijection $f : D \mapsto V(G)$, where $f(v_i) = a$ if and only if the center of $s_a$, the maximal star of $H$ correspondent to $a$, is $v_i$.
    
    First, to show that $E(G) \subseteq E(H[D]^2)$, take an edge $ab \in E(G)$ and the corresponding stars $s_a, s_b \in \str(H)$.
    By the definition of $G$, $s_a \cap s_b \neq \emptyset$. If $|s_a \cap s_b| = 2$, clearly $f(a)^{-1}f(b)^{-1} \in E(H)$ and it follows that $f^{-1}(a)f^{-1}(b) \in E(H[D]^2)$.
    Otherwise, since there is only one maximal star centered on each vertex of $H$, we have that $v_i = f^{-1}(a) \neq f^{-1}(b) = v_k$ and there is at least one $v_j \in s_a \cap s_b$.
    Therefore, $v_j \in N_H(v_i) \cap N_H(v_k)$ and $v_iv_k \notin E(H)$ since $H$ is $K_3$-free, from where we conclude that $v_iv_jv_k$ is an induced $P_3$ of $H$ and, consequently, $\dist(v_i, v_k) = 2$, which implies that $v_iv_k \in E(H[D]^2)$, that is, $f^{-1}(a)f^{-1}(b) \in E(H[D]^2)$.
    
    For $E(H[D]^2) \subseteq E(G)$, take an edge $v_iv_k \in E(H[D]^2)$ but $v_iv_k \notin E(H)$.
    Because $v_i,v_k \in D$, there is $s_a \in \str(H)$ corresponding to $a = f(v_i)$ and $s_b \in \str(H)$ corresponding to $b = f(v_k)$ with $v_j \in s_a \cap s_b$ and, consequently, $ab \in E(G)$, that is, $f(v_i)f(v_k) \in E(G)$. Otherwise, if $v_iv_k \in E(H)$, $f(v_i)f(v_k) \in E(G)$ trivially holds since $\{v_i,v_k\} = s_a \cap s_b$.
\end{tproof}

\begin{corollary}
    If $H$ is $K_3$-free with  $\delta(H) \geq 2$, then $\K{S}(H) \simeq H^2$.
\end{corollary}

The above characterization gives a nice way to think about star graphs as the square of triangle free graphs with minimum degree 2.
If the pre-image is also $C_4$-free, we have an analogous characterization for the biclique graph.

\begin{corollary}
    If $H$ is $\{K_3, C_4\}$-free, $\delta(H) \geq 2$ and $G = \K{B}(H)$, $G \simeq H^2$.
\end{corollary}

\citep{girth_powers} extensively studied powers of graphs of constrained girth, giving a polynomial time algorithm for 

\problem{square of graph with girth $\geq 6$}{A graph $G$}{Is there an $H$ with $\girth(H) \geq 6$ such that $G \simeq H^2$?}

and an $\NPcness$ proof for

\problem{square of graph with girth $\leq 4$}{A graph $G$}{Is there an $H$ with $\girth(H) \leq 4$ such that $G \simeq H^2$?}

The given proof shows that \textsc{square of graph with girth $= 4$} is $\NPc$, and, using the fact that \textsc{square of graph with girth $= 3$} is $\NPc$, a result given by~\citep{hard_roots}, it is concluded that \textsc{square of graph with girth $\leq 4$} is $\NPc$.

As such, using Theorem~\ref{thm:tri_free_star} and the polynomial time algorithm for \textsc{square of graph with girth $\geq 6$}, we obtain the following result.

\begin{theorem}
    Given a star graph $G$, there is a polynomial time algorithm that checks if there exists a triangle-free graph $H$ with girth at least 6 such that $G \simeq \K{S}(H)$.
\end{theorem}

Since graphs of girth exactly four are triangle-free, these hardness results point toward the $\NPcness$ of \textsc{star graph} of triangle-free graphs.
We state a particular case of \textsc{star graph} in terms of girth of the pre-image and establish its $\NPcness$ in the following theorem.

\problem{star graph of graph with girth $= 4$}{A graph $G$}{Is there an $H$ with $\girth(H) = 4$ such that $G \simeq \K{S}(H)$?}

\begin{theorem}
    \textsc{star graph of graph with girth $= 4$} is $\NPc$.
\end{theorem}

\begin{tproof}
    Due to Theorem~\ref{thm:tri_free_star}, membership in $\NP$ is easily verified.
    For our reduction, we will use \textsc{square of graph with girth $= 4$} and suppose that all graphs used in this proof are connected.
    We denote by $G$ and $H$ the input and output of \textsc{square of graph with girth $= 4$}, respectively, and by $G'$ and $H'$ the input to \textsc{star graph of graph with girth $= 4$}.
    Our instance $G$ is exactly as $G'$, that is, $G = G'$.
    
    First, take a solution $H$ to \textsc{square of graph with girth $= 4$} and build $H'$ with $V(H') = V(H) \cup \{u' \mid u \in V(H)\}$ and $E(H') = E(H) \cup \{uu' \mid u \in V(H)\}$.
    That is, $H$ is an induced subgraph of $H'$ obtained by removing every vertex $u'$ of degree one.
    We denote such set by $U(H')$ and the remaining vertices by $D(H') = V(H') \setminus U(H') = V(H)$.
    Note that $H'$ also has girth 4 because $H$ has girth 4 and no $u' \in U(H')$ is contained in a cycle of $H'$ nor induces a triangle.
    Using Theorem~\ref{thm:tri_free_star}, we conclude that $G' = G = H^2 = \left(H' \setminus U(H')\right)^2 = H'[D(H')]^2 = \K{S}(H')$.
    
    For the converse, take a solution $H'$ to \textsc{star graph of graph with girth $= 4$}, denote by $D(H')$ its vertices of degree at least 2 and take $H = H'[D(H')]$.
    Clearly, $H$ has girth 4 and remains connected.
    Using Theorem~\ref{thm:tri_free_star} we conclude that $G = G' = \K{S}(H') = H'[D(H')]^2 = H^2$.
\end{tproof}


\citep{girth_powers} conjectures that \textsc{square of graph with girth $= 5$} is polynomially solvable, which partially reduces the motivation to study the star graph of $\{K_3, C_4\}$-free graphs, as it may not aid in a proof for the complexity of biclique graph recognition.
It remains to be proven that verifying a solution to \textsc{star graph} is a polynomial task, and, for now, leave it as future work.