\chapter{On the intersection graph of maximal stars}
\label{ch:star_graph}


Intersection graphs form the basis for much of the theory on graph classes.
For instance, the class of chordal graphs, which is one of the most fundamental and broadly studied classes~\cite{classes_survey}, can be defined as precisely the family of intersection graphs of all subtrees of some tree.
Interval graphs, in turn, are defined as the family of intersection graphs of subpaths of some path.
Line graphs are the intersection graphs of the edges of some graph.
Unlike chordal graphs, there are known characterizations for line graphs that make use of a finite family of forbidden induced subgraphs~\citep{line_nich}.
Moreover, line graphs were one of the first classes to be characterized in terms of edge clique covers that satisfy some properties pertinent to the intersection definition; results of this form are known as \tdef{Krausz-type characterizations}.

All of the aforementioned classes are easily recognizable in polynomial time~\cite{classes_survey,line_naor}.
The complexity of recognizing clique graphs -- the intersection graphs of the maximal cliques of some graph -- was an open problem for decades, with a very complicated argument, due to~\cite{clique_recognition}, showing that the problem is $\NPc$.
Many other aspects of clique graphs have been investigated in the literature.
Such is the case for, clique-critical graphs -- graphs whose clique graph is different from the clique graph of all of its proper induced subgraphs.
This graph class has its own characterizations~\cite{clique_critical_toft} and bounds~\cite{clique_critical_alcon} which were central in the proof of the complexity of the clique graph recognition problem.
Another common line of investigation on intersection graphs is the study of iterated intersection graphs, i.e., of the behavior of a graph that undergoes the operation multiple consecutive times.
Results of this flavor usually come in the form of convergence, divergence, and periodicity theorems, relating properties of the input graph to the behavior of the limit graph (after applying the operator an infinite number of times).
A closely related intersection class to star graphs is that of biclique graphs -- the intersection graph of the maximal induced complete bipartite graphs of a graph.
The introductory paper by~\cite{biclique_graph} gives a Krausz-type characterization of the class and some properties of its members; these results, however, are not very useful from the algorithmic point of view, and appear to not yield many insights on the recognition problem.

Star graphs and biclique graphs coincide for $C_4$-free graphs and our initial hope was that results on the former would yield advancements on the latter.
While we were unable to achieve our original goal, we present an introductory study of the intersection graphs of maximal stars, providing answers to some difficulties we encountered when working with the class.
After some standard definitions of the theory of intersection graphs, we begin the discussion with a bound on the number of vertices of star-critical graphs by a quadratic function of the size of its set of maximal stars.
Afterwards, we give a Krausz-type characterization, which, when combined to the previous result, shows that the recognition problem belongs to \NP.
We then shift the focus, to properties of star graphs.
In particular, we show that they are biconnected, that every edge belongs to at least one triangle, we characterize the structures that the pre-image must have in order to generate degree two vertices, and bound the diameter of the star graph with respect to the diameter of its pre-image is given.
Finally, we give a monotonicity theorem, which is used to generate all star graphs on no more than eight vertices and prove that the class of star graphs and square graphs are not properly contained in each other.


\section{Intersection Graphs}
\label{sec:intersections}

The \tdef{intersection graph} of a multifamily $\mathcal{F} \subseteq 2^S$, denoted by $G = \Omega(\mathcal{F})$ is the graph of order $|\mathcal{F}|$ and, for every $F_u, F_v \in \mathcal{F}$, $uv \in E(G) \Leftrightarrow F_u \cap F_v \neq \emptyset$.
Any $\mathcal{F}$ such that $\Omega(\mathcal{F}) \simeq G$ is a \tdef{set representation} of $G$.
A known theorem states that every graph is the intersection graph of a family of subgraphs of a graph~\citep{intersection_graphs}.

An \tdef{edge clique cover} $\mathcal{Q} = \{Q_1, \dots, Q_n\}$ of a graph $G$ is a (multi)family of cliques of $G$ such that every edge of $G$ is contained in at least one element of $\mathcal{Q}$.
The \tdef{dual edge clique cover} of a set representation $\mathcal{F} = \{F_1, \dots, F_n\}$, with $|\bigcup_{i \leq n} F_i| = m$, is defined as $\mathcal{Q}(\mathcal{F}) = \{Q_1, \dots, Q_m\}$ such that $Q_j = \{i \mid j \in F_i\}$.
The \tdef{dual set representation} of an edge clique cover $\mathcal{Q} = \{Q_1, \dots, Q_m\}$, with $|\bigcup_{j \leq m} Q_j| = n$, is $\mathcal{F}(\mathcal{Q}) = \{F_1, \dots, F_n\}$ with $F_i = \{j \mid i \in Q_j\}$.

Some interesting intersection graphs are usually defined in terms of the intersection of structures of other graphs. For instance, \tdef{line graphs} are precisely the graphs that are the intersection graphs of the edges of a graph; \tdef{clique graphs} are the intersection graphs of the maximal induced cliques of a graph.
Both of these classes, however have nice characterizations in terms of edge clique covers, which are commonly called \tdef{Krausz-type characterizations}.

\begin{class_definition*}[Line Graph]
    $G$ is a line graph if and only if there is an edge clique cover $\mathcal{Q}$ of $G$ such that both conditions hold:
        \begin{itemize}
            \item[(i)] Every vertex of $G$ appears in exactly two members of $\mathcal{Q}$;
            \item[(ii)] Every edge of $G$ is in only one member of $\mathcal{Q}$.
        \end{itemize}
\end{class_definition*}

\begin{class_definition*}[Clique Graph]
    $G$ is a clique graph if and only if it there is an edge clique cover of $G$ satisfying the Helly property.
\end{class_definition*}

\begin{figure}[!htb]
    \centering
        \begin{tikzpicture}
            \begin{scope}[rotate=-90,scale=0.5]
                \def\x{-2}
                \GraphInit[vstyle=Normal]
                \SetVertexNormal[Shape=circle, FillColor=black, LineWidth=1pt,MinSize=2pt,]
                \tikzset{VertexStyle/.append style = {inner sep = \inners, outer sep = \outers}}
                \Vertices[Lpos=270,Ldist=3pt,LabelOut=TRUE,unit=3]{circle}{a,b,c,d}
                \Edge(a)(b)
                \Edge(a)(d)
                \Edge(c)(b)
                \Edge(d)(b)
                \Edge(c)(d)
            \end{scope}
        \end{tikzpicture}
    \hfill
        \begin{tikzpicture}
            \begin{scope}[rotate=-90,scale=0.5]
                \def\x{-2}
                \GraphInit[vstyle=Normal]
                \SetVertexNormal[Shape=circle, FillColor=black, LineWidth=1pt,MinSize=2pt,]
            \tikzset{VertexStyle/.append style = {inner sep = \inners, outer sep = \outers}}
                \Vertices[Lpos=270,Ldist=3pt,LabelOut=TRUE,unit=3]{circle}{abd,bcd}
                \Edge(abd)(bcd)
            \end{scope}
        \end{tikzpicture}
    \hfill
        \begin{tikzpicture}
            \begin{scope}[rotate=315,scale=0.5]
                \def\x{-2}
                \GraphInit[vstyle=Normal]
                \SetVertexNormal[Shape=circle, FillColor=black, LineWidth=1pt,MinSize=2pt,]
            \tikzset{VertexStyle/.append style = {inner sep = \inners, outer sep = \outers}}
                \Vertices[Lpos=315,Ldist=3pt,LabelOut=TRUE,unit=3]{circle}{ab,bc,cd,ad}
                \Vertex[Lpos=270,Ldist=3pt,LabelOut=TRUE,unit=3,x=0,y=0]{bd}
                \Edge(ab)(bc)
                \Edge(ab)(ad)
                \Edge(bc)(cd)
                \Edge(cd)(ad)
                \Edge(bd)(ab)
                \Edge(bd)(ad)
                \Edge(bd)(cd)
                \Edge(bd)(bc)
            \end{scope}
        \end{tikzpicture}
    \hfill
        \begin{tikzpicture}
            \begin{scope}[rotate=90,scale=0.5]
                \GraphInit[vstyle=Normal]
                \SetVertexNormal[Shape=circle, FillColor=black, LineWidth=1pt,MinSize=2pt,]
            \tikzset{VertexStyle/.append style = {inner sep = \inners, outer sep = \outers}}
                \Vertices[Lpos=90,Ldist=3pt,LabelOut=TRUE,unit=3]{circle}{abc,dbc,bd}
                \Edges(abc,dbc,bd,abc)
            \end{scope}
        \end{tikzpicture}
    \hfill
    \caption{A graph, its clique graph, its line graph, and its star graph}
    %  Some graph
    \label{fig:my_label}
\end{figure}

The recognition of line graphs is known to be efficient~\citep{line_dynamic,line_nich,line_naor}.
For clique graphs, however, the situation was not so simple, and the complexity of clique graph recognition was left open for several years, finally being proven to be $\NPc$ by~\cite{clique_recognition} with a quite complicated argument.

Aside from the complexity point of view, many different properties of intersection graphs have been investigated in the literature.
For instance, clique-critical graphs -- graphs whose clique graph is different from the clique graph of all of its proper induced subgraphs -- have different characterizations~\citep{clique_critical_toft} and bounds~\citep{clique_critical_alcon} which were crucial in the proof of the complexity of the recognition problem.
Another common line of investigation on intersection graphs is the behaviour of iterated applications of the operators.
For instance, \cite{clique_iterated}, and \cite{clique_divergent} study iterated applications of the clique operator.
Biclique graphs -- the intersection graph of the maximal induced complete bipartite graphs of a graph -- were first characterized and studied by \cite{biclique_graph}.
Their results, however, are not very useful from the algorithmic point of view, and appear to not yield many insights on the recognition problem.
Nevertheless, they study the behavior of biclique graphs, showing that every edge is contained either in a diamond or a 3-fan and specialize their general characterization for biclique graphs of bipartite graphs.
As was done for clique graphs, the iterated biclique operator has also been studied by Groshaus et al. in multiple papers~\citep{biclique_iterated, almost_all_biclique}, with results ranging from characterizations of divergence, divergence type verification algorithms, and other structural results.

For other classical results in the area we point to~\citep{intersection_graphs}, from where most of the given definitions come from.

\subsection{Maximal Stars}

Regarding stars, previous work handled the intersection graphs of (not necessarily maximal) substars of a tree~\citep{substar_graph} and of a star~\citep{starlike_graph}.
For the first, a minimal infinite family of forbidden induced subgraphs was given, while, for the latter, a series of characterizations were shown (including a finite family of forbidden induced subgraphs).
Stars are a particular case of bicliques, and both the biclique graph and star graph coincide for $C_4$-free graphs.
In fact, this relationship was successfully applied to determine the complexity of biclique coloring~\citep{biclique_coloring_complexity}, using a reduction from \pname{QSAT}$_2$ to star coloring (a coloring of the vertices of a graph such that no maximal star is monochromatic).
To the best of our knowledge, these are the main topics discussed in the literature that involve maximal stars in some way.


The \tdef{star operator} $\K{S}$ applied on $H$ is the intersection graph of the induced maximal stars of $H$.
The result of $\K{S}$ is called the \tdef{image graph}, or simply \tdef{image}.
A graph $G$ is a \textit{star graph} if there is some graph $H$ such that $G \simeq \K{S}(H)$. 
Given a star graph $G$, any $H$ such that $\K{S}(H) \simeq G$ is called a \tdef{pre-image} of $G$.
The \tdef{center} of a star $s = K_{1,n}$ is the vertex of the partition of size one.
A \tdef{leaf} of a star $K_{1,n}$ is one of the vertices of degree one.
We say that star $s_a$ \textit{absorbs} star $s_b$ if, by removing one leaf of $s_b$, it becomes a substar of $s_a$.
The \textit{iterated star operator} $\K{S}^i$ is defined as $\K{S}^1(G) = \K{S}^1(G)$ and $\K{S}^i(G) = \K{S}(\K{S}^{i-1}(G))$.

A vertex $v$ is said to be \tdef{star-critical} if its removal changes the resulting star graph; that is, the star graph of $H$ and the star graph of $H \setminus \{v\}$ are not isomorphic.
Similarly to clique-critical graphs~\citep{clique_critical_toft,clique_critical_alcon}, a graph is \textit{star-critical} if all of its vertices are star-critical.
It is not hard to see that the only vertices which may be non-star-critical are simplicial vertices; for example, if there is a class of false twin simplicial vertices, all but one of them are certainly non-star-critical.

When detailing which vertices belong to a star, we shall describe it by $\{v_1\}\{v_2, \dots, v_{n+1}\}$, with $v_1$ being its center and the other $n$ vertices its leaves.
If the star is a single edge, choose one of the vertices to be the center and the other to be the leaf arbitrarily.
Unless noted, $G$ will be our star graph and $H$ the pre-image of $G$.
The family of all maximal stars of $G$ is denoted by $\str(G)$.
For the entirety of this work, we assume that all of our graphs are connected.



\begin{figure}[!htb]
	\centering
	
	\begin{tikzpicture}[rotate=45,scale=\gscale]
	
	\GraphInit[unit=3,vstyle=Normal]
	%\draw[help lines] (-5,-5) grid (5,5);
	\SetVertexNormal[Shape=circle, FillColor=black, MinSize=2pt]
	\tikzset{VertexStyle/.append style = {inner sep = \inners, outer sep = \outers}}
	\begin{scope}[shift={(-2.41cm, 0cm)}]
	\SetVertexNoLabel
	\grEmptyCycle[RA=1.41,prefix=a]{4}
	\Edges(a0,a1,a2,a3,a0)
	\Vertex[x=0,y=2.41]{1a}
	%\Vertex[y=0,x=-2.41]{2a}
	\Vertex[x=0,y=-2.41]{3a}
	\Edge(1a)(a1)
	%\Edge(2a)(a2)
	\Edge(3a)(a3)
	\end{scope}
	\begin{scope}[shift={(2.41cm, 0cm)}]
	\SetVertexNoLabel
	\grEmptyCycle[RA=1.41,prefix=b]{4}
	\Edges(b0,b1,b2,b3,b0)
	\Vertex[x=0,y=2.41]{1b}
	%\Vertex[y=0,x=2.41]{0b}
	\Vertex[x=0,y=-2.41]{3b}
	\Edge(1b)(b1)
	%\Edge(0b)(b0)
	\Edge(3b)(b3)
	\end{scope}
	\Edge(a0)(b2)
	\end{tikzpicture}
	\hfill
	\begin{tikzpicture}[rotate=45,scale=\gscale]
	
	\GraphInit[unit=3,vstyle=Normal]
	%\draw[help lines] (-5,-5) grid (5,5);
	\SetVertexNormal[Shape=circle, FillColor=black, MinSize=2pt]
	\tikzset{VertexStyle/.append style = {inner sep = \inners, outer sep = \outers}}
	\begin{scope}[shift={(-2.41cm, 0cm)}]
	\SetVertexNoLabel
	\grEmptyCycle[RA=1.41,prefix=a]{4}
	\Edges(a0,a1,a2,a3,a0)
	\Vertex[x=0,y=2.41]{1a}
	%\Vertex[y=0,x=-2.41]{2a}
	\Vertex[x=0,y=-2.41]{3a}
	\Edge(1a)(a1)
	%\Edge(2a)(a2)
	\Edge(3a)(a3)
	
	\Edge(a0)(a2)
	\Edge(a1)(a3)
	
	\Edge(1a)(a2)
	\Edge(3a)(a2)
	\Edge(1a)(a0)
	\Edge(3a)(a0)
	%\Edge(2a)(a1)
	%\Edge(2a)(a3)
	\end{scope}
	\begin{scope}[shift={(2.41cm, 0cm)}]
	\SetVertexNoLabel
	\grEmptyCycle[RA=1.41,prefix=b]{4}
	\Edges(b0,b1,b2,b3,b0)
	\Vertex[x=0,y=2.41]{1b}
	%\Vertex[y=0,x=2.41]{0b}
	\Vertex[x=0,y=-2.41]{3b}
	\Edge(1b)(b1)
	%\Edge(0b)(b0)
	\Edge(3b)(b3)
	
	\Edge(b0)(b2)
	\Edge(b1)(b3)
	
	\Edge(1b)(b2)
	\Edge(3b)(b2)
	\Edge(1b)(b0)
	\Edge(3b)(b0)
	%\Edge(0b)(b1)
	%\Edge(0b)(b3)
	\end{scope}
	\Edge(a0)(b2)
	\Edge(a0)(b1)
	\Edge(a0)(b3)
	\Edge(b2)(a1)
	\Edge(b2)(a3)
	\end{tikzpicture}
	\hfill
	\begin{tikzpicture}[shift={(1.41cm,0cm)},rotate=45,scale=\gscale]
	
	\GraphInit[unit=3,vstyle=Normal]
	%\draw[help lines] (-5,-5) grid (5,5);
	\SetVertexNormal[Shape=circle, FillColor=black, MinSize=2pt]
	\tikzset{VertexStyle/.append style = {inner sep = \inners, outer sep = \outers}}
	\begin{scope}[shift={(-2.41cm, 0cm)}]
	\SetVertexNoLabel
	\grEmptyCycle[RA=1.41,prefix=a]{4}
	\Edges(a0,a1,a2,a3,a0)
	\Edge(a0)(a2)
	\Edge(a1)(a3)
	\end{scope}
	\begin{scope}[shift={(2.41cm, 0cm)}]
	\SetVertexNoLabel
	\grEmptyCycle[RA=1.41,prefix=b]{4}
	\Edges(b0,b1,b2,b3,b0)
	\Edge(b0)(b2)
	\Edge(b1)(b3)
	\end{scope}
	\Edge(a0)(b2)
	\Edge(a0)(b1)
	\Edge(a0)(b3)
	\Edge(b2)(a1)
	\Edge(b2)(a3)
	\end{tikzpicture}
	\hfill
	
	
	
	\caption{A triangle-free graph (left), its square (center) and its star graph (right).}
	\label{fig:tri_star}
\end{figure}

Before proceeding to the main results of this chapter, we make the following remark.

\begin{observation}
	Every vertex of degree at least two in a triangle free graph is the center of exactly one maximal star.
\end{observation}

The above observation immediately leads us to the property that every star graph of a triangle-free graph is closely related to the square of one of its induced subgraphs.

\begin{observation}
	If $H$ is a $K_3$-free graph with at least 3 vertices, $D$ are its vertices of degree at least 2 and $G = \K{S}(H)$, it holds that $G \simeq H[D]^2$.
\end{observation}

As such, every hardness result or polynomial time algorithm for the recognition of squares of triangle-free graphs immediately applies to the class of star graphs of triangle-free graphs.
For an illustration of the previous observation, we refer to Figure~\ref{fig:tri_star}.
For a far more complicated star graph, we refer to Figure~\ref{fig:star_example}.



\begin{figure}[!tb]
	\centering
	\begin{tikzpicture}
	\GraphInit[unit=3,vstyle=Normal]
	\SetVertexNormal[Shape=circle, FillColor = black, MinSize=3pt]
	\tikzset{VertexStyle/.append style = {inner sep = \inners,outer sep = \outers}}
	\SetVertexNoLabel
	\begin{scope}[rotate=45]
	\grComplete[RA=1]{4}
	\end{scope}
	\Vertex[x=0, y=-2]{x4}
	\Vertex[x=0, y=-4]{x3}
	\Vertex[x=0, y=-6]{x2}
	\Vertex[x=-1, y=-6]{x1}
	\Edges(a3,x4,a2)
	\Edges(x4,x3,x2,x1)
	\end{tikzpicture}
	\hfill
	\begin{tikzpicture}
	\GraphInit[unit=3,vstyle=Normal]
	\SetVertexNormal[Shape=circle, FillColor = black, MinSize=3pt]
	\tikzset{VertexStyle/.append style = {inner sep = \inners,outer sep = \outers}}
	\SetVertexNoLabel
	\Vertex[x = 0, y = 0.707]{v78}
	\Vertex[x = 2, y = -0.707]{v854}
	\Vertex[x = 1, y = -0.707]{v754}
	
	\Vertex[x = -1, y = -0.707]{v764}
	\Vertex[x = -2, y = -0.707]{v864}
	\Vertex[x = 0, y = 0]{v65}
	
	\Vertex[x = 1, y = -2]{v543}
	
	\Vertex[x = -1, y = -2]{v643}
	
	\Vertex[x = 0, y = -4]{v432}
	
	\Vertex[x = 0, y = -6]{v321}
	
	
	
	\Edge(v65)(v764)
	\Edge(v643)(v854)
	\Edge(v643)(v864)
	\Edge(v321)(v432)
	\Edge(v754)(v854)
	\Edge(v643)(v754)
	\Edge(v543)(v65)
	\Edge(v764)(v78)
	\Edge(v864)(v78)
	\Edge(v543)(v854)
	\Edge(v764)(v754)
	\Edge(v764)(v864)
	\Edge(v432)(v764)
	\Edge(v643)(v65)
	\Edge(v65)(v754)
	\Edge[style = bend right](v864)(v854)
	\Edge(v432)(v643)
	\Edge(v321)(v543)
	\Edge(v643)(v764)
	\Edge(v643)(v543)
	\Edge(v864)(v754)
	\Edge(v754)(v78)
	\Edge(v854)(v78)
	\Edge(v543)(v864)
	\Edge(v543)(v764)
	\Edge(v321)(v643)
	\Edge(v432)(v543)
	\Edge(v65)(v854)
	\Edge(v543)(v754)
	\Edge(v764)(v854)
	\Edge[style=bend right](v432)(v854)
	\Edge[style=bend left](v432)(v864)
	\Edge(v432)(v754)
	\Edge(v65)(v864)
	\end{tikzpicture}
	\hfill
	
	
	
	\caption{A graph (left) and its star graph (right).}
	\label{fig:star_example}
\end{figure}

However, star graphs appear to be natural generalizations of square graphs~\citep{murty} in the sense that, when applying the squaring operation, for each vertex $v$ only the largest, non-induced star centered at $v$ is selected, and the intersection graph of these stars is generated.
On the other hand, for star graphs, every \textit{induced} maximal star is used in the construction of the intersection graph.
Despite the classes of star graphs and biclique graphs being equivalent when restricting the  pre-image domain to $C_4$-free graphs, we were unable to deepen the study biclique graphs; our efforts were hindered by some of the questions posed and developed upon in this work.



\section{A bound for star-critical pre-images}

Our first result is an upper bound on the number of vertices of a star-critical graph in terms of its number of maximal stars.
For a graph $H$ the difference $|V(H)| - |\str(H)|$ could be arbitrarily large, but some vertices of $H$ would have to be non-star-critical for such a property to occur (e.g. if $H \simeq K_{1,r}$ there are $r-1$ non-star-critical vertices).
In a sense, star-critical graphs are minimal with respect to the star graph obtained with the application of the star operator.
Recall that a maximal star $s_a$ absorbs a maximal star $s_b$ if, by removing one leaf of $s_b$, it becomes a substar of $s_a$.

\begin{theorem}
    \label{thm:bound}
    If $H$ is an $n$-vertex star-critical graph, $n \leq \frac{1}{2}\left(3|\str(H)|^2 - |\str(H)|\right)$.
\end{theorem}

\begin{proof}
    We begin by partitioning $V(H)$ in $K = \bigcup_{s_a \in \str(H)} \{c(s_a)\}$ and $I = V(H) \setminus K$, which is a subset of its simplicial vertices.
    Note that $I$ is an independent set of $H$, otherwise there would be an edge with endpoints $\{u,v\} \subseteq I$ and either $u$ or $v$ would be in $K$.
    $I$ is further partitioned in $I_A$ and $I_E$: a vertex is in $I_A$ if its removal causes the absorption of at least one star, while the removal of a vertex in $I_E$ causes the disappearance of at least one edge of the star graph.
    
    Note that $|K| \leq |\str(H)|$ holds because each maximal star has a center.
    To bound $|I|$, we divide the analysis in the two situations where a vertex is star-critical.
    \begin{enumerate}
        \item Suppose that the removal of some $z \in I_A$ causes $s_a$, with $u = c(s_a)$, to be absorbed by $s_b$.
        One of two possibilities arise: if $z$ has only one neighbor then $z$ is the only neighbor of $u$ with this property; therefore there are at most $|K|$ such vertices.
        Otherwise, if $z$ has at least two neighbors, there is some $v \in N(z) \cap N(u)$ with $v \in s_b \setminus s_a$. However, since $I$ is an independent set, $v \in K$ and, moreover, $u,z$ are the only neighbors of $v$ in $s_a$, otherwise $s_a \setminus \{z\}$ cannot be a substar of $s_b$. Therefore, for each maximal star $s_a$, since $H$ is star-critical, there is at most one different $z \in I_A$ for each $v \in (N(u) \cap N(z) \cap K) \setminus s_a \subseteq K$ preventing $v$ from being added to $s_a$.
        This implies that the number of vertices required to avoid absorption is at most $|\str(H)|(|K \setminus \{u\}|) \leq 2\binom{|\str(H)|}{2}$.
        \item For the other condition, each $z \in I_E$ could be responsible for the intersection of a different pair of stars of $H$; i.e., there exists $s_a,s_b \in \str(H)$ such that $s_a \cap s_b = \{z\}$.
        Since we have $\binom{|\str(H)|}{2}$ pairs, we may have as many vertices in $I_E$.
    \end{enumerate}
    Summing both cases, we have $|I| \leq 3\binom{|\str(H)|}{2}$ and since $n = |K| + |I|$, it holds that $n \leq \frac{3|\str(H)|^2 - |\str(H)|}{2}$.
\end{proof}

\begin{corollary}
    If $H$ is star-critical and has no simplicial vertex, $|V(H)| \leq |\str(H)|$.
    If the only simplicial vertices of $H$ are leaves, $|V(H)| \leq 2|\str(H)|$.
\end{corollary}

\begin{proof}
    The first statement follows directly from the case where $|I|$ is empty in the proof of Theorem~\ref{thm:bound}.
    The second statement is a consequence of the hypothesis that every vertex of $I_A$ has degree one and $I_E = \emptyset$.
\end{proof}

Improvements to the bound given by Theorem~\ref{thm:bound} appear to require a complete characterization of non-star-critical vertices.
Also, a better understanding of vertices that are required only for the intersection of some stars to be non-empty seems necessary in order to approach the problem through induction.
We believe that the bound on the size of the pre-image is actually linear, however our current analysis falls short of it.
In fact, we conjecture that the constant is actually two, as formalized below.

\begin{conjecture}
    If $H$ is an $n$-vertex star-critical graph, then $n \leq 2|\str(H)|$.
\end{conjecture}

If this result indeed holds, it would configure an important difference from other intersection graphs.
For instance, there are clique graphs which require a clique-critical pre-image with a quadratic number of vertices~\cite{clique_critical_alcon}.
\section{Characterization}

Throughout this section, we shall denote an edge clique cover of $G$ by $\mathcal{Q} = \{Q_1, \dots, Q_n\}$.
The usual strategy in a Krausz-type characterization is to use each clique as a vertex of the pre-image; this is also our approach.
Since each vertex $a \in V(G)$ must be a star in $H$, it is reasonable to partition each clique as $Q_i \sim \{Q_i^c, Q_i^f\}$, that is, the vertices $a \in Q_i^c$ correspond to the stars of $G$ with center in $v_i \in V(H)$, while the vertices $a \in Q_i^f$ correspond to the stars of $G$ where $v_i \in V(H)$ is a leaf.
We call such an edge clique cover a \tdef{star-partitioned edge clique cover} of $\mathcal{Q}$.

To simplify our notation, with a slight abuse, for each $a \in V(G)$, we denote its \tdef{center} by $c(a)$, i.e. $c(a)$ is the unique $i$ such that $a \in Q_i^c$, its \tdef{leaf set} by $F(a) = \{i \mid a \in Q_i^f\}$ and its \tdef{cover} by $Q(a) = F(a) \cup \{c(a)\}$. For each pair of cliques $Q_i, Q_j \in \mathcal{Q}$, their \tdef{leaf-leaf intersection} is given by $\ff(i,j) = Q_i^f \cap Q_j^f$ and its \tdef{center-leaf intersection} by $\cf(i,j) = \left(Q_i^c \cap Q_j^f\right) \cup \left(Q_i^f \cap Q_j^c\right)$.

\begin{definition}[Star-compatibility]
    Given a graph $G$ and a star-partitioned edge clique cover $\mathcal{Q}$ of $G$, we say that $\mathcal{Q}$ is \tdef{star-compatible} if, for every $a \in V(G)$, $|Q(a)| \geq 2$, $\exists!\ i$ such that $a \in Q_i^c$ and if, for every $Q_i, Q_j \in \mathcal{Q}$, if $Q_i \cap Q_j \neq \emptyset$, either $\cf(i,j) = \emptyset$ or $\ff(i,j) = \emptyset$.
\end{definition}

\begin{definition}[Star-differentiability]
    \label{def:differentiability}
    Given a graph $G$ and a star-partitioned edge clique cover $\mathcal{Q}$ of $G$, we say that $\mathcal{Q}$ is \tdef{star-differentiable} if for every $Q_i \in \mathcal{Q}$ and for every pair $\{a, a'\} \subseteq Q_i$ the following conditions hold:
    \begin{enumerate}
        \item If $\{a, a'\} \subseteq Q_i^c$, there exists $Q_j, Q_k \in \mathcal{Q}$ such that $a \in Q_j^f$, $a' \in Q_k^f$, $a \notin Q_k^f$, $a' \notin Q_j^f$ and $\cf(j,k) \neq \emptyset$. Moreover, if $Q_i^c \cap Q_j^f \cap Q_k^f = \emptyset$, $\cf(j,k) \neq \emptyset$.
        \item If $a \in Q_i^c$, $a' \in Q_k^c$ and $a \notin Q_k^f$, then there is some $j \in F(a)$ with $\cf(j,k) \neq \emptyset$, $j \notin Q(a')$ and, for every $j' \in F(a)$ with $\cf(j',k) = \emptyset$, $Q_i^c \cap \bigcap_{j'} \ff(j',k) \neq \emptyset$.
        \item If $a \in Q_i^c$, $a' \in Q_k^c$ and $a \in Q_k^f$, for every $j \in F(a) \setminus \{k\}$, $\cf(j,k) = \emptyset$.
        \item If $\{a, a'\} \subseteq Q_i^f$ and $j = c(a) \neq c(a') = k$, then either $Q_i^c \cap \ff\left(j,k\right) \neq \emptyset$ or $\cf\left(j,k\right) \neq \emptyset$.
    \end{enumerate}
\end{definition}

Figures~\ref{fig:diff_cases} and~\ref{fig:diff_cases2} show the four cases of Definition~\ref{def:differentiability} as seen on the pre-image of the star graph we build from $\mathcal{Q}$ during the proof of Theorem~\ref{thm:star_characterization}.

\begin{figure}[!htb]
    \centering
            \begin{tikzpicture}[scale=\gscale]
                %\draw[help lines] (-5,-5) grid (5,5);
                \begin{scope}[shift={(0cm,1cm)}]
                    \GraphInit[unit=3,vstyle=Normal]
                    \SetVertexNormal[Shape=circle, FillColor=black, MinSize=2pt]
                    \tikzset{VertexStyle/.append style = {inner sep = \inners, outer sep = \outers}}
                    \Vertex[Math,Ldist=3pt,Lpos=90,LabelOut,L={i},x=0,y=0]{i1}
                    \Vertex[Math,Ldist=3pt,Lpos=180,LabelOut,L={j},a=150, d=2]{j1}
                    \Vertex[Math,Ldist=3pt,Lpos=0,LabelOut,L={k},a=30, d=2]{k1}
                    \node[] at (-0.6,-0.35) {$a$};
                    \node[] at (0.6,-0.31) {$a'$};
                    \Vertex[Math,x=0,y=-2,Ldist=3pt,Lpos=270,LabelOut,L={j'}]{r1};
                    
                    \Edge(i1)(j1)
                    \Edge(i1)(r1)
                    \Edge(i1)(k1)
                    \Edge(j1)(k1)
                    \Edge[style=dotted](j1)(r1)
                    \Edge[style=dotted](k1)(r1)
                \end{scope}
            \end{tikzpicture}
    \hfill
            \begin{tikzpicture}[scale=\gscale]
                %\draw[help lines] (-5,-5) grid (5,5);
                \begin{scope}[shift={(0cm,0cm)}]
                    
                    \GraphInit[unit=3,vstyle=Normal]
                    \SetVertexNormal[Shape=circle, FillColor=black, MinSize=2pt]
                    \tikzset{VertexStyle/.append style = {inner sep = \inners, outer sep = \outers}}
                    \Vertex[Math,Ldist=3pt,Lpos=110,LabelOut,L={i},x=0,y=0]{i2}
                    \Vertex[Math,Ldist=3pt,Lpos=180,LabelOut,L={j},a=150, d=2]{j2}
                    \Vertex[Math,Ldist=3pt,Lpos=0,LabelOut,L={k},a=30, d=2]{k2}
                    \node[] at (-0.6,-0.35) {$a$};
                    \node[] at (1.23,1.50) {$a'$};
                    \Vertex[Math,Ldist=3pt,Lpos=270,LabelOut,L={j'},x=0,y=-2]{j'2};
                    \SetVertexNoLabel
                    \Vertex[x=1.73,y=3]{r2}
                    
                    \Edge(i2)(j2)
                    \Edge(i2)(j'2)
                    \Edge(i2)(k2)
                    \Edge(j2)(k2)
                    \Edge(r2)(k2)
                    \Edge[style=dotted](j2)(j'2)
                    \Edge[style=dotted](k2)(j'2)
                    \Edge[style=dotted](i2)(r2)
                \end{scope}
            \end{tikzpicture}
    \hfill
            \begin{tikzpicture}[rotate=180,scale=\gscale]
                %\draw[help lines] (-5,-5) grid (5,5);
                \begin{scope}[shift={(0cm,0cm)}]
                    \GraphInit[unit=3,vstyle=Normal]
                    \SetVertexNormal[Shape=circle, FillColor=black, MinSize=2pt]
                    \tikzset{VertexStyle/.append style = {inner sep = \inners, outer sep = \outers}}
                    \Vertex[Math,Ldist=3pt,Lpos=90,LabelOut,L={i},x=0,y=0]{i3}
                    \Vertex[Math,Ldist=3pt,Lpos=90,LabelOut,L={j},a=150, d=2]{j3}
                    \Vertex[Math,Ldist=3pt,Lpos=90,LabelOut,L={k},a=30, d=2]{k3}
                    \SetVertexNoLabel
                    \Vertex[Math,Ldist=3pt,Lpos=0,LabelOut,L={k},a=30, d=4]{r3}
                    \Vertex[Math,Ldist=3pt,Lpos=0,LabelOut,L={k},x=0, y=2]{l3}
                    \node[] at (0,0.6) {$a$};
                    \node[] at (1.73,1.6) {$a'$};
                    
                    \Edge(i3)(j3)
                    \Edge(i3)(k3)
                    \Edge[style=dotted](j3)(k3)
                    \Edge(k3)(r3)
                    \Edge(k3)(l3)
                    \Edge[style=dotted](r3)(l3)
                \end{scope}
            \end{tikzpicture}
    \hfill
    
    
    
    \caption{The first three cases of Definition~\ref{def:differentiability}, from left (first) to right (third).}
    \label{fig:diff_cases}
\end{figure}


\begin{figure}[!htb]
    \centering
    
        \begin{tikzpicture}[scale=\gscale]
            
            \begin{scope}[shift={(-4cm,-1cm)}, rotate=180]
                \GraphInit[unit=3,vstyle=Normal]
                \SetVertexNormal[Shape=circle, FillColor=black, MinSize=2pt]
                \tikzset{VertexStyle/.append style = {inner sep = \inners, outer sep = \outers}}
                \Vertex[Math,Ldist=3pt,Lpos=90,LabelOut,L={i},x=0,y=0]{i4}
                \Vertex[Math,Ldist=3pt,Lpos=90,LabelOut,L={k_1},a=150, d=2]{j4}
                \Vertex[Math,Ldist=3pt,Lpos=90,LabelOut,L={j_1},a=30, d=2]{k4}
                
                
                \SetVertexNoLabel
                \Vertex[Math,Ldist=3pt,Lpos=0,LabelOut,L={k},a=30, d=4]{r4}
                \Vertex[Math,Ldist=3pt,Lpos=0,LabelOut,L={k},x=0.4, y=2]{l4}
                \Vertex[Math,Ldist=3pt,Lpos=0,LabelOut,L={k},a=150, d=4]{m4}
                \Vertex[Math,Ldist=3pt,Lpos=0,LabelOut,L={k},x=-0.4, y=2]{n4}
                
                
                \node[] at (-1.77,1.5) {$a_1'$};
                \node[] at (1.77,1.5) {$a_1$};
                
                \Edge(i4)(j4)
                \Edge(i4)(k4)
                \Edge[style=dotted](j4)(k4)
                \Edge(k4)(r4)
                \Edge(k4)(l4)
                \Edge[style=dotted](r4)(l4)
                \Edge(j4)(n4)
                \Edge(j4)(m4)
                \Edge[style=dotted](m4)(n4)
                
                
            \end{scope}
        \end{tikzpicture}
        \hfill
        \begin{tikzpicture}[scale=\gscale]
            \begin{scope}[shift={(4cm,1cm)}]
                \GraphInit[unit=3,vstyle=Normal]
                \SetVertexNormal[Shape=circle, FillColor=black, MinSize=2pt]
                \tikzset{VertexStyle/.append style = {inner sep = \inners, outer sep = \outers}}
                \Vertex[Math,Ldist=3pt,Lpos=90,LabelOut,L={i},x=0,y=0]{i5}
                \Vertex[Math,Ldist=3pt,Lpos=90,LabelOut,L={j_2},a=210, d=2]{j5}
                \Vertex[Math,Ldist=3pt,Lpos=90,LabelOut,L={k_2},a=-30, d=2]{k5}
                
                
                
                \SetVertexNoLabel
                
                
                \Vertex[Math,Ldist=3pt,Lpos=0,LabelOut,L={k},a=-30, d=4]{r5}
                \Vertex[Math,Ldist=3pt,Lpos=0,LabelOut,L={k},x=0.4, y=-2]{l5}
                \Vertex[Math,Ldist=3pt,Lpos=0,LabelOut,L={k},a=210, d=4]{m5}
                \Vertex[Math,Ldist=3pt,Lpos=0,LabelOut,L={k},x=-0.4, y=-2]{n5}
                
                \node[] at (-1.77,-1.5) {$a_2$};
                \node[] at (1.77,-1.5) {$a'_2$};
                
                
                \Edge(i5)(j5)
                \Edge(i5)(k5)
                \Edge(j5)(k5)
                \Edge(k5)(r5)
                \Edge(k5)(l5)
                \Edge[style=dotted](r5)(l5)
                \Edge(j5)(n5)
                \Edge(j5)(m5)
                \Edge[style=dotted](m5)(n5)
                
            \end{scope}
        \end{tikzpicture}
    
    
    
    \caption{The fourth case of Definition~\ref{def:differentiability}.}
    \label{fig:diff_cases2}
\end{figure}


We emphasize that: (i) star-compatibility translates the structural properties of stars; and (ii) star-differentiability enumerates the possible ways that two stars that share at least one vertex are different.
Note that, the \say{missing case}, where $\{a,a'\} \in Q_i^f$ and $c(a) = c(a') = k$ is exactly the same case as 1, but with $\{a,a'\} \in Q_k^c$ instead of $Q_i^c$.

\begin{lemma}
    \label{lem:star_maximality}
    Let $G$ be a graph and $\mathcal{Q}$ a star-partitioned edge clique cover of $G$. If $\mathcal{Q}$ is star-compatible and star-differentiable then, for every pair $\{a, a'\} \subseteq V(G)$, $Q(a) \nsubseteq Q(a')$ and $Q(a') \nsubseteq Q(a)$.
\end{lemma}

\begin{tproof}
    If $a$ and $a'$ do not share any clique, the statement holds.
    Otherwise they do share some clique, say $Q_i$.
    If the pair $a,a'$ satisfies properties 1, 2, or 4 of Definition~\ref{def:differentiability}, since $i \in Q(a) \cap Q(a')$, we conclude that there exists $j \in Q(a)$, $k \in Q(a')$ but $j \notin Q(a')$ and $k \notin Q(a)$, implying $Q(a) \nsubseteq Q(a')$ and $Q(a') \nsubseteq Q(a)$.
    
    For property 3, however, we first conclude that there is some $j \in Q(a)$ but $j \notin Q(a')$, otherwise we would have $\cf(j,k) \neq \emptyset$ and $\ff(j,k) \neq \emptyset$.
    Consequently, $Q(a) \nsubseteq Q(a')$.
    To see that $Q(a') \nsubseteq Q(a)$, note that $\{a, a'\} \subseteq Q_k$ and, following the same argument, we conclude that there is some $j' \in Q(a')$ but $j' \notin Q(a)$, completing the proof.
\end{tproof}

We now present a Krausz-type characterization for the class of star graphs.

\begin{theorem}
    \label{thm:star_characterization}
    An $n$-vertex graph $G$ is the star graph of some graph $H$ if and only if there is a star-compatible and star-differentiable star-partitioned edge clique cover $\mathcal{Q}$ of $G$ with at most $\frac{1}{2}(3n^2 - n)$ cliques.
\end{theorem}

\begin{tproof}
    In this proof, we assume that $H$ has $m$ vertices, denoted by $v_i$, and that star $s_a \in \str(H)$ corresponds to the vertex $a \in V(G)$.
    
    For the first direction of the statement, assume $H$ is a star-critical pre-image of $G$.
    For each $v_i \in V(H)$, let $S(v_i) = \{s_a \in \str(H) \mid v_i \in s_a\}$, that is, the maximal stars of $H$ that contain $v_i$.
    Clearly, we can partition these sets as $S(v_i) \sim \{S^c(v_i), S^f(v_i)\}$, that is, the stars where $v_i$ is the center and where it is a leaf, respectively.
    Our goal is to show that $\mathcal{Q} = \{Q_1, \dots, Q_m\}$, with $Q_i^c = S^c(v_i)$ and $Q_i^f = S^f(v_i)$ is a star-partitioned edge clique cover of $G$ satisfying star-compatibility and star-differentiability which.
    By Theorem~\ref{thm:bound}, this is all that remains is to be proven, since $|\mathcal{Q}| = |V(H)| \leq \frac{1}{2}(3n^2 - n)$.
    
    To verify that $\mathcal{Q}$ is a star-partitioned edge clique cover of $G$, first note that every $Q_i$ is a clique of $G$, since the corresponding stars share at least $v_i \in V(H)$.
    For the coverage part, every $aa' \in E(G)$ has two corresponding stars $s_a, s_{a'} \in \str(H)$, which share at least one vertex, say $v_i \in V(H)$, since $G \simeq \K{S}(H)$.
    By the construction of $\mathcal{Q}$, there is some $Q_i \in \mathcal{Q}$ which corresponds to every maximal star that contains $v_i$; this guarantees that $aa'$ is covered by at least one clique of $\mathcal{Q}$.
    
    For the other properties, first take two vertices $v_i,v_j \in V(H)$ with $v_iv_j \notin E(H)$ but $S(v_i) \cap S(v_j) \neq \emptyset$.
    Clearly, no star in $S(v_i) \cap S(v_j)$ may have $v_i$ and $v_j$ in different sides of its bipartition, thus $S(v_i) \cap S(v_j) = S^f(v_i) \cap S^f(v_j)$.
    Now, suppose that $v_iv_j \in E(H)$; since they are adjacent, any star in $S(v_i) \cap S(v_j)$ must have $v_i$ and $v_j$ in opposite sides of the bipartition and, thus, we have that $S(v_i) \cap S(v_j) = \left(S^c(v_i) \cap S^f(v_j)\right) \cup \left(S^f(v_i) \cap S^c(v_j)\right)$.
    Since each star has a single center, the above analysis shows that $\mathcal{Q}$ satisfies star-compatibility.
    
    For star-differentiability, let $\{s_a, s_{a'}\} \subseteq S(v_i)$. We break our analysis in the same order as the one given in Definition~\ref{def:differentiability}.
    \begin{enumerate}
        \item If $\{s_a, s_{a'}\} \subseteq S^c(v_i)$ there must be at least one leaf in each star, say $v_j$ and $v_k$, respectively, not in the other and these leaves must be adjacent to each other, otherwise at least one of the stars would not be maximal.
        That is, $\{a, a'\} \in Q_i^c$ imply that there is $Q_j,Q_k \in \mathcal{Q}$ with $a \in Q_j^f$, $a' \in Q_k^f$, $a \notin Q_k^f$, $a \notin Q_j^f$ and $\cf(j,k) \neq \emptyset$.
        \item If $s_a \in S(v_i)^c$, $s_{a'} \in S(v_k)^c$ and $s_a \notin S(v_k)^f$, $v_iv_k \in E(H)$ and to keep $v_k$ from being a leaf of $s_a$, one leaf of $s_a$, say $v_j$, must also be adjacent to $v_k$ and not a leaf of $s_{a'}$, since $v_i$ is.
        Now, for every $v_{j'} \in s_a$ and not adjacent to $v_k$, there is a clear $P_3 = v_kv_iv_{j'}$, which must be part of some maximal star.
        Moreover, the set of all $v_{j'}$ non-adjacent to $v_k$ will form a maximal star centered around $v_i$ along with $v_k$.
        Thus, $a \in Q_i^c$, $a' \in Q_k^c$ and $a \notin Q_k^f$, imply that there is some $j \in F(a)$ with $\cf(j,k) \neq \emptyset$, $j \notin Q(a')$ and, for $j' \in F(a)$ with $\cf(j', k) = \emptyset$, $Q_i^c \cap \bigcap_{j'} \ff(j',k) \neq \emptyset$.
        \item If $s_a \in S(v_i)^c$, $s_{a'} \in S(v_k)^c$ and $s_a \in S(v_k)^f$, we know that $s_a = \{v_i\}\{v_k, \dots\}$ and, since $v_k$ is not adjacent to any other leaf $v_j$ of $s_a$, we know that $S(v_j) \cap S(v_k) = S^f(v_j) \cap S^f(v_k)$ and, since $v_k$ is the center of $s_{a'}$, $v_j$ is not one of its leaves.
        Therefore, $a \in Q_i^c$, $a' \in Q_k^c$ and $a \in Q_k^f$, implies that for every $j \in F(a) \setminus \{k\}$, $\cf(j,k) = \emptyset$.
        \item If $\{s_a, s_{a'}\} \subseteq Q_i^f$ and $s_a \in S^c(v_j)$, $s_{a'} \in S^c(v_k)$, either $v_jv_k \notin E(H)$, which induces the existence a star $\{v_i\}\{v_j, v_k, \dots\}$, or $v_jv_k \in E(H)$, which must be part of a star with either $v_j$ or $v_k$ as center and the other as a leaf.
        Hence, $\{a, a'\} \subseteq Q_i^f$ and $j = c(a) \neq c(a') = k$, implies that either $Q_i^c \cap \ff\left(j,k\right) \neq \emptyset$ or $\cf\left(j,k\right) \neq \emptyset$.
    \end{enumerate}
    The above shows that $\mathcal{Q}$ is also star-differentiable, which completes this part of the proof.
    
    For the converse, take $\mathcal{Q}$ a star-partitioned edge clique cover of $G$ satisfying star-compatibility and star-differentiability  of size at most $\frac{1}{2}(3n^2 - n)$ and let $H$ be a graph with $V(H) = \{v_i \mid Q_i \in \mathcal{Q}\}$ and $E(H) = \{v_iv_j \mid \cf(i,j) \neq \emptyset\}$ and let us prove that $G \simeq \K{S}(H)$.
    
    Take $a \in V(G)$ with $c(a) = i$.
    Due to star-compatibility and the construction of $H$, we know that $H[\{v_j \mid j \in F(a)\}]$ is an independent set of $H$ and that $s_a = \{v_i\}\{v_j \mid j \in F(a)\}$ is a star of $H$.
    Suppose, however, that $s_a$ is not maximal, that is, there is some $v_k \in V(H)$ such that $v_iv_k \in E(H)$ and $s_b = s_a \cup \{v_k\}$ is a star of $H$.
    By the construction of $H$, either there is some $a' \in V(G)$ such that $Q(a) \subseteq Q(a')$, which is impossible due to Lemma~\ref{lem:star_maximality}, or some $a' \in \cf(i,k)$, which we analyze below.
    The following is based on the first two cases of Definition~\ref{def:differentiability}; the other two are impossible, since $k \notin Q(a)$ and $a \in Q_i^c$.
    
    \begin{enumerate}
        \item If $a' \in Q_i^c$, there is some $Q_j \in \mathcal{Q}$ such that $a \in Q_j^f$ and $\cf(j,k) \neq \emptyset$, which implies that $v_jv_k \in E(H)$ and $s_b$ is not a star of $H$. 
        \item If $a' \in Q_k^c$ and $a \notin Q_k^f$, at least one $j \in F(a)$ satisfies $\cf(j,k) \neq \emptyset$ and $j \notin Q(a')$.
        This gives us that $v_jv_k \in E(H)$ and $s_b$ is not a star of $H$.
    \end{enumerate}
    
    Therefore, we conclude that $a'$ cannot exist, that $s_a$ is maximal and, consequently that $V(G) \subseteq V(\K{S}(H))$.
    
    To show that $V(\K{S}(H)) \subseteq V(G)$, take $s = \{v_i\}L$, with $s \in \str(H)$, and suppose that there is some $j,k \in L$ and that for every pair $a \in \cf(i,j)$ and $a' \in \cf(i,k)$, $a \notin Q_k$ and $a' \notin Q_j$.
    That is, $Q_i \cap Q_j \cap Q_k = \emptyset$, due to star-compatibility and the hypothesis that $jk \notin E(H)$.
    Once again, we analyze the possibilities in terms of Definition~\ref{def:differentiability}.
    
    \begin{enumerate}
        \item If $c(a) = c(a') = i$, we have that $\cf(j,k)  \neq \emptyset$, implying that $v_jv_k \in E(H)$, contradicting the hypothesis that $s$ exists.
        \item If $c(a) = i$ and $c(a') = k$, there is some $j' \in Q(a)$ with $\cf(j,k) \neq \emptyset$. To conclude that $j = j'$, we note that, if $j \neq j'$, it would be required that $Q_i^c \cap \ff(j,k) \neq \emptyset$, which is impossible since $Q_i \cap Q_j \cap Q_k = \emptyset$.
        Once again, contradicting the hypothesis that such an $s$ exists.
        \item Trivially impossible since $Q_i \cap Q_j \cap Q_k = \emptyset$.
        \item If $j = c(a) \neq c(a') = k$, either $Q_i^c \cap \ff(j,k) \neq \emptyset$, which is impossible since $Q_i \cap Q_j \cap Q_k = \emptyset$, or $\cf(j,k) \neq \emptyset$, implies that $v_jv_k \in E(H)$ and that $s$ is not a star.
    \end{enumerate}
    
    The above allows us to conclude that there is no $s \in \str(H)$ generated by cliques not pairwise intersecting.
    Such intersection has a unique vertex of $G$ in it due to Lemma~\ref{lem:star_maximality}, which allows us to conclude $V(\K{S}(H)) \subseteq V(G)$ and, consequently, that $V(\K{S}(H)) = V(G)$.
    
    To show that $E(G) \subseteq E(\K{S}(H))$, we first take an edge $ab \in E(G)$.
    Since $\mathcal{Q}$ is a star-partitioned edge clique cover of $G$, there is some $i$ such that $\{a,b\} \subseteq Q_i$ and, because $V(G) = V(\K{S}(H))$ and the construction of $H$, there are corresponding stars $s_a, s_b \in \str(H)$ with $v_i \in s_a \cap s_b$ which guarantee that $ab \in E(\K{S}(H))$.
    For $E(\K{S}(H)) \subseteq E(G)$, take two intersecting stars $s_a,s_b \in \str(H)$ and note that, since $a,b \in \K{S}(H) = V(G)$ and $\mathcal{Q}$ is a star-partitioned edge clique cover of $G$, $ab \in E(G)$ and we conclude that $E(G) = E(\K{S}(H))$, completing the proof.
\end{tproof}

We now pose a version of the decision problem for star graph recognition, which we call \textsc{Star Graph Recognition}.
We will further require that the output for any algorithm for \textsc{Star Graph Recognition} is already star-partitioned.

\problem{Star Graph Recognition}{A graph $G$.}{Is there a star-partitioned edge clique cover $\mathcal{Q}$ of $G$ satisfying star-compatibility and star-differentiability?}

Theorem~\ref{thm:verification_alg} provides a straightforward verification algorithm to check if a star-partitioned edge clique cover is star-compatible and star-differentiable.

\begin{theorem}
    \label{thm:verification_alg}
    Given a graph $G$ of order $n$, there is an $\bigO{\max\{n^2m, m^2\}n^2m}$ algorithm to decide if a star-partitioned family $\mathcal{Q} \subseteq 2^{V(G)}$ of size $m$ is an edge clique cover of $G$ satisfying star-compatibility and star-differentiability. 
\end{theorem}

\begin{tproof}
    The first task is to determine whether or not $\mathcal{Q}$ is a star-partitioned edge clique cover of $G$.
    The usual $n^2$ algorithm that tests if each $Q_i$ is a clique suffices.
    To check if $\mathcal{Q}$ is an edge clique cover, for each of the $\bigO{n^2}$ edges, we test if one of the $n$ cliques contains it. 
    This simple test takes $\bigO{n^2m}$ time.
    
    To check for star-compatibility: first, for each vertex $a$ of $G$ and each clique $Q_i$, verify if there is a single $i$ such that $a \in Q_i^c$ and at least one $j$ with $a \in Q_j^f$;
    afterwards, for each pair of intersecting cliques $Q_i, Q_j$, test if $\cf(i,j) = \emptyset$ or $\ff(i,j) = \emptyset$.
    The entire process takes $\bigO{nm^2}$ time.
    
    For star-differentiability, we assume that every pairwise intersection of $\mathcal{Q}$ has already been computed in time $\bigO{nm^2}$, and each query $\cf(j,k)$ and $\ff(j,k)$ takes $\bigO{1}$ time.
    Now, for each clique $Q_i$ and for each pair of vertices $\{a, a'\} \in Q_i$, we must check one of the four conditions as follows.
    \begin{enumerate}
        \item If $c(a) = c(a') = i$, for each pair $j \in Q(a)$, $k \in Q(a')$, check if $a' \notin Q_j^f$, $a \notin Q_k^f$ and $\cf(j,k) \neq \emptyset$; this case takes $\bigO{n^2}$.
        \item If $c(a) = i, c(a') = k$ and $a \notin Q_k^f$, for each $j \in F(a)$, check if either $\cf(j,k) \neq \emptyset$ and $j \notin Q(a')$ or $\cf(j,k) = \emptyset$ and $Q_i^c \cap \ff(j,k) \neq \emptyset$; this case takes $\bigO{n^2m}$ time.
        \item If $c(a) = i, c(a') = k$ and $a \in Q_k^f$, check for each $j \in F(a) \setminus \{k\}$, if $\cf(j,k) = \emptyset$, taking $\bigO{n}$ time.
        \item If $j = c(a) \neq c(a') = k$, we check if $Q_i^c \cap \ff(j,k) \neq \emptyset$ in $\bigO{n}$ time, and if $\cf(j,k)$ in $\bigO{1}$ time.
    \end{enumerate}
    In the worst case scenario, we will spend $\bigO{\max{n^2m, m^2}}$ time for each $Q_i$ and each pair $\{a, a'\} \subseteq Q_i$, of which there are $\bigO{n^2m}$ combinations, and conclude that the whole algorithm takes no more than $\bigO{\max\{n^2m, m^2\}n^2m}$ time.
\end{tproof}

Together with Theorems~\ref{thm:bound} and~\ref{thm:star_characterization}, Theorem~\ref{thm:verification_alg} implies that deciding whether or not a graph is a star graph is in $\mathsf{NP}$.

\begin{theorem}
    \textsc{Star Graph Recognition} is in $\mathsf{NP}$.
\end{theorem}
\section{Properties}
The next theorem uses the known result, due to~\cite{moon_moser}, that a graph of order $n$ has at most $3^{n/3}$ maximal independent sets.

\begin{theorem}
    If $G$ is the star graph of a $n$ vertex graph $H$, then $|V(G)| \leq n3^{\Delta(H)/3}$.
\end{theorem}

\begin{tproof}
    For every $v \in V(H)$, define $H_v = H[N(v)]$ and note that each maximal independent set of $H_v$ might induce a maximal star of $H$ centered around $v$.
    Since $|V(H_v)| \leq \Delta(H)$, we have that $H_v$ has at most $3^{\Delta(H)/3}$ maximal independent sets and, therefore, $H$ has at most $3^{\Delta(H)/3}$ maximal stars centered around $v$.
    Summing for every $v \in V(H)$ we arrive at the $n3^{\Delta(H)/3}$ bound.
\end{tproof}

\begin{theorem}
    \label{thm:star_cutless}
    If $G$ is a connected star graph, $G$ has no cut-vertex.
\end{theorem}

\begin{tproof}
    If $|V(G)| \leq 4$, we are done as there are only 5 graphs that satisfy these constraints and none of them contain a cut-vertex.
    They are $K_1$, $K_2$, $K_3$, $K_4$ and $K_4$ with one missing edge (the diamond).
    The first three are trivial, while the last two are shown in Figure~\ref{fig:four_star}
    
    For graphs with 5 or more vertices, suppose that there is some cut-vertex $x \in G$, that $A,B$ are two of the connected components obtained after removing $x$ from $G$ and take a pair of vertices $a \in V(A) \cap N(x)$, $b \in V(B) \cap N(x)$.
    Suppose now that $G = \K{S}(H)$ for some $H$ and take the stars $s_a, s_b, s_x$ corresponding to $a, b, x$, respectively.
    Since $ab \notin E(G)$ and $ax, bx \in E(G)$, it holds that $s_a \cap s_x \neq \emptyset$ and $s_b \cap s_x \neq \emptyset$ but $s_a \cap s_b = \emptyset$.
    
    If $c(a) = c(x) = i$ and $k = c(b) \neq c(x)$, $s_x$ and $s_b$ share at least one leaf, say $v_j$, since they intercept at some vertex, and $v_j \notin s_a$.
    However, there is no leaf $v_{j'} \in s_a$ adjacent to $v_j$, otherwise there would be an edge $v_jv_{j'} \in E(H)$ and, consequently, some star $s_y$, corresponding to vertex $y \in V(G)$, that keeps $A,B$ connected and intercepts $s_a, s_b, s_x$.
    Therefore, we conclude that no leaf of $s_a$ is adjacent to $v_j$ and, since $c(a) = c(x)$ and $v_iv_j \in E(H)$, we conclude that $v_j \in s_a$, otherwise it would not be maximal, and, consequently, $v_j \in s_a \cap s_b$ and $ab \in E(H)$, which contradicts the hypothesis that $A,B$ are disconnected after removing $x$.
    The case where $c(x) = c(b) \neq c(a)$ follows the exact same argument.
    
    Now if $c(a) \neq c(x) = i$ and $c(x) \neq c(b)$, it is easy to see that $v_i$ cannot be a leaf of both $s_a$ and $s_b$ simultaneously, otherwise $v_i \in s_a \cap s_b$ and $ab \in E(H)$.
    So we have two cases to analyze:
    \begin{enumerate}
        \item If $v_i$ is a leaf of $s_a$, $v_j \in s_x \cap s_b$ and $k = c(b)$, clearly $s_y = \{v_j\}\{v_i, v_k, \dots \}$ is a maximal star of $H$ that intercepts $s_a, s_b, s_x$, keeping $A,B$ from being disconnected.
        The case where $v_i$ is a leaf $s_b$ is the same, and we omit it for brevity.
        \item If $v_i$ not a leaf of neither $s_a$ nor $s_b$, $c(a) = j$ and $c(b) = k$, we have leaves $v_{j'} \in s_a \cap s_x$ , $v_{k'} \in s_b \cap s_x$ which form at least two intercepting maximal stars, $s_{a'} = \{v_{j'}\}\{v_i, v_j, \dots\}$ and $s_{b'} = \{v_{k'}\}\{v_i, v_k, \dots\}$, such that $s_{a'} \cap s_a \cap s_x \neq \emptyset$ and $s_{b'} \cap s_b \cap s_x \neq \emptyset$.
    \end{enumerate}
    
    These cases allow us to conclude that $A,B$ remains connected no matter the configuration of the intersection of the corresponding stars in $H$.
    Consequently, $x$ cannot exist and we complete the proof.
\end{tproof}


\begin{theorem}
    Every edge of a star graph $G$ is contained in at least one triangle if $|V(G)| \geq 3$.
\end{theorem}

\begin{tproof}
    The only connected star graph with 3 vertices is $K_3$, so take $G$ with $|V(G)| \geq 4$.
    Take a pre-image $H$ of $G$, $ab \in E(G)$, $s_a, s_b \in \str(H)$ the corresponding stars to $a, b$, and assume that $ab$ is not contained in any triangle of $G$.
    Since $G$ is connected, there is at least one $x \in V(G)$ adjacent to (w.l.o.g) $a$, but not to $b$, and a corresponding maximal star $s_x$ of $H$.
    Below, we analyze the possible intersections between $s_a$ and $s_b$ and conclude that there is always some star $s_y$ that shares one vertex with $s_a$ and $s_b$.
    \begin{enumerate}
        \item If $c(a) = c(b) = i$ and the center of $s_x$ is a leaf of $s_a$, clearly $v_i$ is not a leaf of $s_x$, otherwise $s_x \cap s_b \neq \emptyset$, therefore there is some leaf $v_j \in s_x$ with $v_iv_j \in E(H)$, which must be part of at least one maximal star $s_y$ of $H$, from which we conclude that $s_a \cap s_b \cap s_y \neq \emptyset$, $s_a \cap s_x \cap s_y \neq \emptyset$ and both $ab$ and $ax$ are in a triangle of $G$.
        \item If $c(a) = c(b) = i$ and a leaf $v_j$ of $s_x$ is a leaf of $s_a$, either the center $v_k$ of $s_x$ is adjacent to $v_i$, in which case $v_iv_k \in E(H)$ and we follow the same argument as in the previous case, or they are not adjacent, implying that there is a maximal star $s_y = \{v_j\}\{v_i, v_k, \dots\}$ which intercepts $s_a, s_b, s_x$, which allows us to conclude that $s_a, s_b, s_x, s_y$ is a clique of $G$.
        \item if $i = c(a) \neq c(b) = k$, there is some leaf $v_j \in s_a \cap s_b$. Clearly, if $v_iv_k \in E(H)$, there is a star that intercepts both $s_a$ and $s_b$;
        otherwise, $v_iv_k \notin E(H)$ and we conclude that $s_y = \{v_j\}\{v_i, v_k, \dots\} \in \str(H)$ intercepts $s_a$ and $s_b$ and creates a triangle that contains $ab$.
    \end{enumerate}
\end{tproof}

The previous theorem implies that the minimum degree of any star graph on at least three vertices is at least two.
A natural question arises about the vertices of degree two and the structures on the pre-image that generate them.

A \textit{pending-$P_4$} $\{u, v, w, z\}$ is an induced path on four vertices that satisfies $d(u) = 1$, $N(v) = \{u, w\}$, $N(w) = \{v, z\}$ and $N(z)$ is an independent set of $H$.
A \textit{terminal triangle} is a set $\{u, v, z\}$ such that $N[u] = N[v] = \{u,v,z\}$, and no other pair of vertices in $N(z)$ is adjacent.
In both cases, $z$ is called the \textit{anchor} of the structure.
Our next result shows that for nearly all star graphs, their degree two vertices are either generated by pending-$P_4$'s or terminal triangles.

\begin{lemma}
    If $H$ is star-critical, $G = \K{S}(H)$, and $G$ is not isomorphic to a diamond, then every vertex of degree two of $G$ is generated by a pending-$P_4$,
    or by a terminal triangle.
    Moreover, for every degree two vertex $a \in V(G)$, it holds that $a$ has a vertex $a'$ not adjacent to another vertex of degree two.
\end{lemma}

\begin{proof}
    If $|V(G)| \leq 3$ the result holds, so suppose $|V(G)| \geq 4$, let $a$ be a degree two vertex of $G$ with $N(a) = \{b,d\}$, $s_a$ be the corresponding maximal star of $H$, $c(s_a) = v$, and $u \in s_a$ be one of its leaves.
    Since $d_G(a) = 2$, neither $v$ nor any of its leave can be contained in any other maximal star of $H$, aside possibly from $b$ and $d$.
    
    Suppose that $s_a = \{u,v\}$.
    In this case, we have that both $u$ and $v$ are true twins, with $w \in N_H(u)$.
    If $u$ is simplicial, $|N_H(u)| = 2$, otherwise $a$ would have more than two neighbors.
    If $N_H(u) \setminus \{v\}$ is not an independent set, it has at least two adjacent vertices $w,z$ forming a $K_4$ with $u$ and $v$; regardless of the neighborhood of $w$ and $z$, at least four distinct maximal stars contain either $u$ or $v$, implying $d_G(a) \geq 3$.
    If $N_H(u) \setminus \{v\}$ is an independent set, we have two options:
    \begin{enumerate}
        \item $N_H(u) \supseteq \{v,w,z\}$, in which case at least one of $w$ or $z$, say $w$, has a neighbor other than $u$ and $v$, since $H$ is star-critical. This configuration, however, generates a $K_5$ in $G$: two stars centered at $w$, $s_a$, one centered at $v$ containing all its neighbors, except $u$, and one centered at $u$ with all its neighbors except $v$.
        \item Otherwise, $N_H(u) = \{v,w\}$. Since $|V(G)| \geq 4$, $w$ necessarily has an additional neighbor. If $N_H(w) \setminus \{u,v\}$ is not an independent set of $H$, $w$ has a pair of adjacent neighbors $x,y$, which are not adjacent to $u$ nor $v$.
        However, note that there are at least four stars centered at $w$ that intersect $s_a$, contradicting the hypothesis that $a$ has only two neighbors in $G$.
   \end{enumerate}
   From the above, we conclude that if $|s_a| = 2$, it corresponds to an edge of a terminal triangle.
   
   On the other hand, suppose now that $s_a \supseteq \{u,v,w\}$, and that $c(s_a) = v$.
   Towards showing that $N_H(v)$ is an independent set, suppose that $v$ has at least one edge in its neighborhood.
   \begin{enumerate}\addtocounter{enumi}{2}
       \item If no such edge is incident to $u$ or $w$, then there are two maximal stars centered at $v$ containing $\{u,v,w\}$ but, in this case, neither $u$ nor $w$ may have a star centered at it and, consequently, one of them is non-star-critical. 
       \item If $w$ is adjacent to some $z \in N_H(v)$ but $zu \notin E(H)$, again there are two stars centered at $v$ ($s_a$ and another one containing $\{u,v,z\}$) both being adjacent to any star that includes the edge $wz$.
       For $s_a$ to have only two neighbors, neither $u$, nor $v,$ nor $w$ may be in another other star.
       Since $G$ is connected and has at least four vertices, $z$ must have another neighbor $x$.
       We subdivide our analysis on the neighborhood of $x$:
       \begin{enumerate}
            \item If $xv \in E(H)$, either $x$ or $z$ must be part of another maximal star; actually, $x$ cannot be the center of another star (note that $x$ is part of $s_a$, or it would be in another star that intersects $s_a$), so $z$ must be part of another star, that is, it has a neighbor $y$ not adjacent to $x$; but, in this case, $\{z,x,y\}$ intersects $s_a$, increasing the degree of $a$ to at least three.
            \item So $xv \notin E(H)$ and there is a star centered at $z$ containing $\{v,z,x\}$ which does not contain $w$, this implies that $s_a$ intersects at least three stars.
       \end{enumerate}
       \item If $z$ is adjacent to both $u$ and $w$, $s_a$ already intersects two maximal stars -- one containing $vz$ and another containing $\{u,z,w\}$.
       Note that neither $w$ nor $u$ may have another neighbor, as that would inevitably generate a third star that intersects $s_a$.
       The only possibility would be that $z$ is part of a maximal star that does not contain neither $u$, nor $v$, nor $w$.
       That is, every star that contains $z$ has it as one of its leaves (otherwise we would have leaves adjacent to $u$, $v$, and $w$).
       This implies that $N_H(z) \setminus \{u,v,w\}$ is an independent set.
       However, either $u$ or $w$ is non-star-critical, since its removal does not change the intersection graph, a contradiction.
   \end{enumerate}
   To realize that $N_H(v) = \{u,w\}$, note that at most two of the neighbors of $v$ may have a single star centered at each of them, all others would be of degree one and, consequently, non-star-critical.
   
   We now show that one of the neighbors of $v$ has degree one.
   To see that this is the case, note that, if neither has degree one, both have at least one neighbor not adjacent to $v$ and, thus, centers of maximal stars containing $v$.
   However, neither may be in \textit{any} other star, as this would increase the degree of $s_a$ to more than two, but this is impossible, since at least one of $u$ and $w$ must be in another star for $G$ to have at least four vertices and remain connected.
   For the remainder of the proof, suppose that $u$ has degree one.
   Together with the fact that $v$ only has two neighbors, we conclude that $w$ must be in precisely two maximal stars, one of them with $w$ being its center, since no neighbor of $w$ may be adjacent to $v$.
   This implies that $N_H(w)$ is either an independent set or that it has at most one edge.
   If $N_H(w)$ has an edge $xy$, however, neither $x$ nor $y$ may have a neighbor not adjacent to $w$, otherwise we would have another star containing $w$ and $s_a$ would intersect three stars.
   In this case, $G$ would be precisely a diamond.
   So now we have that $N_H(w)$ is also an independent set and, furthermore, $N_H(w) = \{z, v\}$, since $H$ is star-critical.
   Now, the only way for $w$ to be in more than two stars is if there is more than one star centered at $z$ containing $w$; which is possible only if $z$ is part of a triangle; so we also conclude that $N(z)$ is an independent set.
   This configuration is precisely a pending-$P_4$.
   
   For the second assertion, to show that every vertex $a \in V(G)$ of degree two has a neighbor not adjacent to another vertex of degree two, suppose that this is not the case, and let $a'$ be such that $N(a) = N(a')$.
   We have three possible cases:
   \begin{enumerate}\addtocounter{enumi}{5}
       \item If both $a$ and $a'$ belong to pending-$P_4$'s. Note that, if $s_a$ is generated by the $P_4$ $\{u,v,w,z\}$, we have that $s_{a'}$ must be formed by the $P_4$ $\{u',v', z, w\}$, since $s_{a'}$ must intersect the star centered at $z$ and the star centered at $w$.
       However, this implies that $H$ is isomorphic to $P_6$, since the degree of every vertex, except $u$ and $u'$, is two, and we have that $G$ is a diamond, contradicting the hypothesis.
       \item If $s_a$ belongs to a pending-$P_4$ $\{u,v,w,z\}$ and $s_{a'}$ belongs to a terminal triangle, the anchor $z$ of the pending-$P_4$ cannot be the same as the anchor of the terminal triangle, as it would violate the requirement that $N(z)$ is an independent set.
       This, however, makes it impossible for $a'$ to be adjacent to the neighborhood of $a$.
       \item If both stars belong to terminal triangles, a similar analysis as the previous case follows.
   \end{enumerate}
   Finally, we conclude that at most one of the neighbors of a degree two vertex has another degree two neighbor.
\end{proof}

\begin{corollary}
    Let $E_2(G) = \{uv \in E(G) \mid d_G(u) = 2\ \text{or}\ d_G(v) = 2\}$ be the set of edges incident to at least one degree two vertex of the star graph $G$. Unless $G$ is isomorphic to a diamond or a triangle $E_2(G) \leq \min\left\{|V(G)| - 1, \frac{4}{7}|E(G)|\right\}$.
    The bounds are tight.
\end{corollary}

\begin{proof}
    Let $V_2(G) = \{v \in V(G) \mid d_G(v) = 2\}$.
    By the previous Lemma, we have that for each vertex of degree two there is another vertex non-adjacent to another degree two vertex.
    As such, each pair of edges of $E_2(G)$ with a common endpoint is in one-to-one correspondence with a degree two vertex and its exclusive neighbor, i.e., $|E_2(G)| \leq 2|V_2(G)| \leq |V(G)| - 1$.
    For the second case, for each degree two vertex $v$, its exclusive neighbor has at least two other edges, otherwise the non-exclusive neighbor would be a cut-vertex, but these edges may be between exclusive neighbors.
    As such, we have that $|E_2(G)| + \frac{1}{2}|E_2(G)| + \frac{1}{4}|E_2(G)| \leq |E(G)|$, implying $|E_2(G)| \leq \frac{4}{7}|E(G)|$.
    For the tightness of the bounds, the star graph of $P_7$, the gem, satisfies both conditions.
\end{proof}

We conclude this section with a result about the diameter of a star graph.
In fact, when considering the iterated star operator, it appears that the diameter converges to either three or four, depending on the graph from which the process began, even though the graph itself does not seem to converge.
We highlight that the bound of Theorem~\ref{thm:diameter} is tight, as shown by the example of Figure~\ref{fig:diameter}.


\begin{figure}[!htb]
        \centering
        \begin{tikzpicture}[rotate = 0]
                %\draw[help lines] (-5,-5) grid (5,5);
            \GraphInit[unit=3,vstyle=Normal]
            \SetVertexNormal[Shape=circle, FillColor = black, MinSize=3pt]
            \tikzset{VertexStyle/.append style = {inner sep = \inners, outer sep = \outers}}
            \SetVertexNoLabel
            \begin{scope}
                \begin{scope}[shift={(-1.5, 0)}]
                    \grComplete[RA=1, prefix=a]{3}
                \end{scope}
                \begin{scope}[shift={(1.5, 0)}]
                    \begin{scope}[rotate=180]
                        \grComplete[RA=1, prefix=b]{3}
                    \end{scope}
                \end{scope}
                \Edge(b0)(a0)
            \end{scope}
            
            \begin{scope}[shift={(5,0)}]
                    \begin{scope}[rotate=-135]
                        \grComplete[RA=1, prefix=c]{4}
                    \end{scope}
                    \Vertex[a=0, d=1.5]{y1}
                    \Vertex[a=180, d=1.5]{y2}
                    \Edge(y2)(c0)
                    \Edge(y1)(c1)
                    \Edge(y1)(c2)
                    \Edge(y2)(c3)
            \end{scope}
                
        \end{tikzpicture}
        \caption{Problematic case of Theorem~\ref{thm:diameter}. The pre-image on the left and star graph on the right.\label{fig:diameter}}
\end{figure}

\begin{theorem}
    \label{thm:diameter}
    If $H$ is a graph with diameter $d$ and its star graph $G$ is not a clique, then it holds that the diameter of $G$ is at most $\floor{\frac{d}{2}} + 2$. 
\end{theorem}

\begin{proof}
    Let $P_G = \{s_1, \dots, s_{k+1}\}$ be a diametrical path of $G$ and $P_G' = P_G \setminus \{s_1, s_{k+1}\}$.
    For the following argumentation, we need to guarantee that the endpoints of the path in $G$ have at least two vertices in a shortest path between their corresponding centers in $H$.
    Note that, in the case presented in Figure~\ref{fig:diameter}, neither of the degree two stars of the star graph satisfy the aforementioned condition.
    Let $u = c(s_2)$, $v = c(s_k)$, $P_H$ be a shortest path between $u$ and $v$ with length, say $r$.
    If $r \geq 2k - 3$, we are done, as we would certainly have a path in $G$ between $u$ and $v$ of length at least $\ceil{\frac{2k-3}{2}} = k-2$ and, by adding stars $s_1$ and $s_{k+1}$, we would have a path of length at least $k$.
    Otherwise, $r < 2k - 3$, which directly implies that there is a path between $s_2$ and $s_k$ of length at most $k-3$, contradicting the hypothesis that $P_G$ is a diametrical path.
\end{proof}

\begin{corollary}
	If $H$ is a graph of diameter $d$ and $G_k = \K{S}^k(H)$, for every $k \geq \ceil{\log (d + 4)}$, the diameter of $G_k$ is either three or four, unless $G_{k-1}$ is a clique, in which case the diameter of $G$ is one.
\end{corollary}

\section{Small star graphs}


By the observations made in Section~\ref{sec:prelim}, star graphs and square graphs are quite similar, and even coincide under specific conditions, which is the case when the pre~image is triangle-free.
A natural question that thus arises is if the classes are actually the same.
To see that there are star graphs which are not square graphs, Figure~\ref{fig:star_not_square} presents a small example of such a graph.

\begin{figure}[!htb]
        \centering
        \begin{tikzpicture}[rotate = 0]
                %\draw[help lines] (-5,-5) grid (5,5);
            \GraphInit[unit=3,vstyle=Normal]
            \SetVertexNormal[Shape=circle, FillColor = black, MinSize=3pt]
            \tikzset{VertexStyle/.append style = {inner sep = \inners, outer sep = \outers}}
            \SetVertexNoLabel
            \begin{scope}[shift={(-2, 0)}]
                \grComplete[RA=1]{4}
            \end{scope}
            \begin{scope}[shift={(2, 0)}]
                \begin{scope}[rotate=45]
                    \grCycle[RA=1, prefix=t]{4}
                \end{scope}
                \Vertex[x=0, y=-0.3]{x}
                \Vertex[x=0, y=0.3]{y}
                \Edge(t0)(y)
                \Edge(t1)(y)
                \Edge(t2)(y)
                \Edge(t3)(y)
                \Edge(t0)(x)
                \Edge(t1)(x)
                \Edge(t2)(x)
                \Edge(t3)(x)
            \end{scope}
                
        \end{tikzpicture}
        \caption{The star graph of $K_4$ is not a square graph.\label{fig:star_not_square}}
\end{figure}

Despite not coinciding for many classes of pre-images, it could be the case that every square graph also is a star graph, albeit for a different pre-image.
The smallest example we found of a square graph which is not a star graph is shown in Figure~\ref{fig:square_not_star}: the square of the net.
When attempting to show such a fact, without additional tools the combinatorial explosion of possible cases rapidly becomes intractable.
At the same time, testing all graphs up to the bound given by Theorem~\ref{thm:bound} in search of a pre-image would be completely unfeasible, as we would need to test all connected graphs with up to \textit{51 vertices}.
However, Theorem~\ref{thm:bound} presents what we believe is a very loose value for the size of a pre-image, a claim we support with Theorem~\ref{thm:monotonicity}, its corollary, and some experiments we performed.


\begin{figure}[!htb]
        \centering
        \begin{tikzpicture}[rotate = 0]
                %\draw[help lines] (-5,-5) grid (5,5);
            \GraphInit[unit=3,vstyle=Normal]
            \SetVertexNormal[Shape=circle, FillColor = black, MinSize=3pt]
            \tikzset{VertexStyle/.append style = {inner sep = \inners, outer sep = \outers}}
            \SetVertexNoLabel
            \begin{scope}[shift={(-2, 0)}]
                \begin{scope}[rotate=90]
                    \grComplete[RA=.7,prefix=c1]{3}
                    \grEmptyCycle[RA=1.6, prefix=o1]{3}
                    \foreach \i in {0,1,2} {
                        \Edge(c1\i)(o1\i)
                    }
                \end{scope}
            \end{scope}
            \begin{scope}[shift={(2, 0)}]
                \begin{scope}[rotate=90]
                    \grComplete[RA=.7,prefix=c2]{3}
                    \grEmptyCycle[RA=1.6, prefix=o2]{3}
                    \foreach \i in {0,1,2} {
                        \foreach \j in {0,1,2} {
                            \Edge(c2\i)(o2\j)
                        }
                    }
                \end{scope}
            \end{scope}
                
        \end{tikzpicture}
        \caption{The square of the net is not a star graph.}
        \label{fig:square_not_star}
\end{figure}

\begin{theorem}
    \label{thm:monotonicity}
    Let $H$ be an $n$-vertex graph, with $|\str(H)| = k$, and at least one non-star-critical vertex.
    For any $H'$ with $n+1$ vertices such that $H$ is an induced subgraph of $H'$, at least one of the following holds:
    \begin{enumerate}
        \item $H'$ has non-star-critical vertices; or
        \item $|\str(H')| \geq k+1$.
    \end{enumerate}
\end{theorem}

\begin{proof}
    Since $H$ is a proper induced subgraph of $H'$, let $y$ be the vertex in $V(H') \setminus V(H)$.
    If $y$ is non-simplicial, or $y$ is non-star-critical, or some vertex $x \in V(H)$ remains non-star-critical in $H'$, we are done.
    The only cases that matter have $y$ as a simplicial.
    Suppose that the statement is false, i.e. that every vertex of $H'$ is useful and $|\str(H')| = k$; in particular, the vertex $x$, which is non-star-critical on $H$, becomes  star-critical.
    Before proceeding, note that if $yx \in E(H')$, at least one of $y$ or $x$ must be the center of a star containing this edge and, therefore, we have a new star in $H'$.
    Moreover, $y$ is not a false twin of $x$, otherwise $y$ is non-star-critical.
    For the remainder of the proof, let $H'' = H \setminus \{x\}$ and $H^* = H'' \cup \{y\}$.
    \begin{enumerate}
        \item The removal of $x$ from $H'$ causes the absorption of $s_a' \in \str(H')$, such that $s_a' \supseteq s_a \in \str(H)$; note that this last assertion is true, otherwise $s_a'$ is a new star generated by the addition of $y$.
            \subitem(a) If $s_a \supseteq \{x, z, w\}$ (centered at $z$, only $w$ critical) we have that $(s_a \setminus \{x\}) \in \str(H'')$, because $x$ is non-star-critical, implying that $w$ and $z$ form a pair of true twins, otherwise $s_a \setminus \{x\}$ would be absorbed by a star centered at $w$ containing edge $wz$.
            However, $(s_a \setminus \{x\}) \notin \str(H^*)$ and $y \notin s_a'$; to see that this is the case, $y \in s_a'$ implies that $yz \in E(H')$, $yx,yw \notin E(H')$, and, thus, that $x$ remains non-star-critical.
            Therefore, $(s_a \cup \{y\} \setminus \{x\}) \in \str(H')$, implying that $yx \in E(H')$, a contradiction.
            \subitem(b) If $s_a \supseteq \{x,z,w,\ell\}$ (centered at $z$, $w,\ell$ critical), we have that $(s_a \setminus \{x\}) \in \str(H'')$, but $(s_a \setminus \{x\}) \notin \str(H^*)$.
            Since $s_a'$ has at least two useful vertices, whichever star absorbs $s_a'$ must be centered at $z$; moreover, $y$ must be in the star that absorbs $s_a' \setminus \{x\}$, otherwise, we would have that $s_a \setminus \{x\}$ is absorbed by a star that \textit{already existed} in $H''$; consequently, we have $yx \in E(H')$ and a new star is formed by the addition of $y$.
        \item There are two stars $s_a', s_b' \in \str(H')$ such that $s_a' \cap s_b' = \{x\}$.
        Note that, at least one of $s_a'$ and $s_b'$ contains $y$, say $s_b'$, and we have that $(s_b' \setminus \{y\}) \notin \str(H)$.
        Therefore, $s_b'$ is a new star that depends on $y$ to exist and we have that $|\str(H')| \geq |\str(H)| + 1 = k + 1$.
    \end{enumerate}
\end{proof}

To the best of our knowledge, analogous results to Theorem~\ref{thm:monotonicity} are not known for clique or biclique graphs.
These types of monotonicity properties are particularly useful when looking for small examples; the following statement is a direct corollary.

\begin{corollary}
    Let $G$ be a $k$-vertex graph and $\mathcal{H}_n(k)$ be the set of all graphs on $n$ vertices that has $k$ maximal stars.
    If $G$ is not isomorphic to the star graph of any star-critical $H \in \mathcal{H}_r(k)$, for any $r < n$ and every $H \in \mathcal{H}_n(k)$ is non-star-critical, then $G$ is not a star graph.
\end{corollary}

The above results allowed us to implement a procedure using McKay's Nauty package~\cite{nauty}.
Instead of only looking for the square of the net graph, we generated every single star graph on $k \leq 8$ vertices.
In fact, for each $k$, no graph in $\mathcal{H}_{2k+1}(k)$ was star-critical.
Figures~\ref{fig:four_star}, \ref{fig:five_star}, and \ref{fig:six_star} present every star graph on four, five and six vertices, respectively.
There are 46 star graphs on seven vertices, and 201 star graphs on eight vertices.
Let $\mathcal{H}^*(k)$ denote the set of all star-critical pre-images for star graphs on $k$ vertices.
Our procedure also listed $\mathcal{H}^*(k)$ for every $k \leq 8$.
In particular, there are 190 graphs in $\mathcal{H}^*(4)$, 1056 in $\mathcal{H}^*(5)$, 8876 in $\mathcal{H}^*(6)$, 76320 in $\mathcal{H}^*(7)$, and 892170 in $\mathcal{H}^*(8)$.

\begin{corollary}
	There are square graphs which are not star graphs (e.g. the square of the net), and there are star graphs which are not square graphs (e.g. the star graph of $K_4$).
\end{corollary}

\begin{figure}
    \centering
    \begin{subfigure}{.5\textwidth}
      \centering
      \begin{tikzpicture}
\GraphInit[unit=3,vstyle=Normal]
\SetVertexNormal[Shape=circle, FillColor = black, MinSize=3pt]
\tikzset{VertexStyle/.append style = {inner sep = \inners,outer sep = \outers}}
\SetVertexNoLabel
\grCycle[RA=1]{4}
\Edge(a0)(a2)
\end{tikzpicture}

    \end{subfigure}%
    \begin{subfigure}{.5\textwidth}
      \centering
      \begin{tikzpicture}
\GraphInit[unit=3,vstyle=Normal]
\SetVertexNormal[Shape=circle, FillColor = black, MinSize=3pt]
\tikzset{VertexStyle/.append style = {inner sep = \inners,outer sep = \outers}}
\SetVertexNoLabel
\grComplete[RA=1]{4}
\end{tikzpicture}

    \end{subfigure}
    \caption{The two four vertex star graphs. \label{fig:four_star}}
\end{figure}


\begin{figure}
    \centering
    \begin{subfigure}{.5\textwidth}
      \centering
      \begin{tikzpicture}[rotate=135]
\GraphInit[unit=3,vstyle=Normal]
\SetVertexNormal[Shape=circle, FillColor = black, MinSize=3pt]
\tikzset{VertexStyle/.append style = {inner sep = \inners,outer sep = \outers}}
\SetVertexNoLabel
\grComplete[RA=1]{4}
\Vertex[a=45, d=1.5]{v}
\Edge(v)(a0)
\Edge(v)(a1)
\end{tikzpicture}

    \end{subfigure}%
    \begin{subfigure}{.5\textwidth}
      \centering
      \begin{tikzpicture}
\GraphInit[unit=3,vstyle=Normal]
\SetVertexNormal[Shape=circle, FillColor = black, MinSize=3pt]
\tikzset{VertexStyle/.append style = {inner sep = \inners,outer sep = \outers}}
\SetVertexNoLabel
\grPath[RA=1.44, RS=1.2, prefix=a]{3}
\begin{scope}[xshift=0.72cm]
    \grPath[RA=1.44, RS=0, prefix=b]{2}
\end{scope}
\Edges(a0,b0,a1,b1,a2)
\end{tikzpicture}

    \end{subfigure}
    \par\bigskip
    \begin{subfigure}{.5\textwidth}
      \centering
      \begin{tikzpicture}[rotate=90]
\GraphInit[unit=3,vstyle=Normal]
\SetVertexNormal[Shape=circle, FillColor = black, MinSize=3pt]
\tikzset{VertexStyle/.append style = {inner sep = \inners,outer sep = \outers}}
\SetVertexNoLabel
\grCycle[RA=1]{5}
\Edges(a4,a2,a0,a3,a1)
\end{tikzpicture}

    \end{subfigure}%
    \begin{subfigure}{.5\textwidth}
      \centering
      \begin{tikzpicture}[rotate=90]
\GraphInit[unit=3,vstyle=Normal]
\SetVertexNormal[Shape=circle, FillColor = black, MinSize=3pt]
\tikzset{VertexStyle/.append style = {inner sep = \inners,outer sep = \outers}}
\SetVertexNoLabel
\grComplete[RA=1]{5}
\end{tikzpicture}

    \end{subfigure}
    \caption{The four five vertex star graphs.\label{fig:five_star}}
\end{figure}



\begin{figure}
    \centering
    \begin{subfigure}{.33\textwidth}
      \centering
      \begin{tikzpicture}
\GraphInit[unit=3,vstyle=Normal]
\SetVertexNormal[Shape=circle, FillColor = black, MinSize=3pt]
\tikzset{VertexStyle/.append style = {inner sep = \inners,outer sep = \outers}}
\SetVertexNoLabel
\Vertex[x = 2.365, y = 0.71044]{v1}
\Vertex[x = 1.19352, y = 0.036]{v0}
\Vertex[x = 0.97988, y = 1.19874]{v3}
\Vertex[x = 2.655, y = 2.0248]{v2}
\Vertex[x = 0.036, y = 2.367]{v5}
\Vertex[x = 1.4641, y = 1.90164]{v4}
\Edge(v0)(v3)
\Edge(v4)(v5)
\Edge(v2)(v3)
\Edge(v3)(v5)
\Edge(v0)(v4)
\Edge(v1)(v4)
\Edge(v1)(v2)
\Edge(v3)(v4)
\Edge(v0)(v1)
\Edge(v1)(v3)
\Edge(v2)(v4)
\end{tikzpicture}

    \end{subfigure}%
    \begin{subfigure}{.33\textwidth}
      \centering
      \begin{tikzpicture}
\GraphInit[unit=3,vstyle=Normal]
\SetVertexNormal[Shape=circle, FillColor = black, MinSize=3pt]
\tikzset{VertexStyle/.append style = {inner sep = \inners,outer sep = \outers}}
\SetVertexNoLabel
\Vertex[x = 2.2458, y = 0.88158]{v1}
\Vertex[x = 1.30628, y = 0.036]{v0}
\Vertex[x = 2.8674, y = 2.2288]{v3}
\Vertex[x = 0.48494, y = 0.9946]{v2}
\Vertex[x = 1.42008, y = 1.80488]{v5}
\Vertex[x = 0.036, y = 2.4068]{v4}
\Edge(v4)(v5)
\Edge(v0)(v2)
\Edge(v0)(v5)
\Edge(v3)(v5)
\Edge(v1)(v2)
\Edge(v1)(v5)
\Edge(v0)(v1)
\Edge(v2)(v5)
\Edge(v1)(v3)
\Edge(v2)(v4)
\end{tikzpicture}

    \end{subfigure}
    \begin{subfigure}{.33\textwidth}
      \centering
      \begin{tikzpicture}
\GraphInit[unit=3,vstyle=Normal]
\SetVertexNormal[Shape=circle, FillColor = black, MinSize=3pt]
\tikzset{VertexStyle/.append style = {inner sep = \inners,outer sep = \outers}}
\SetVertexNoLabel
\Vertex[x = 1.93282, y = 0.90976]{v1}
\Vertex[x = 0.40616, y = 0.137682]{v0}
\Vertex[x = 1.15636, y = 1.68636]{v3}
\Vertex[x = 1.4162, y = 0.036]{v2}
\Vertex[x = 0.036, y = 2.6908]{v5}
\Vertex[x = 0.2489, y = 1.21628]{v4}
\Edge(v0)(v3)
\Edge(v2)(v3)
\Edge(v0)(v2)
\Edge(v4)(v5)
\Edge(v3)(v5)
\Edge(v1)(v3)
\Edge(v0)(v4)
\Edge(v1)(v4)
\Edge(v3)(v4)
\Edge(v0)(v1)
\Edge(v1)(v2)
\Edge(v2)(v4)
\end{tikzpicture}

    \end{subfigure}
    \par\bigskip
    \begin{subfigure}{.33\textwidth}
      \centering
      \begin{tikzpicture}
\GraphInit[unit=3,vstyle=Normal]
\SetVertexNormal[Shape=circle, FillColor = black, MinSize=3pt]
\tikzset{VertexStyle/.append style = {inner sep = \inners,outer sep = \outers}}
\SetVertexNoLabel
\Vertex[x = 1.82262, y = 0.55208]{v1}
\Vertex[x = 0.036, y = 0.95892]{v0}
\Vertex[x = 0.138028, y = 2.5836]{v3}
\Vertex[x = 0.84994, y = 0.036]{v2}
\Vertex[x = 0.70724, y = 1.4778]{v5}
\Vertex[x = 1.5197, y = 1.74048]{v4}
\Edge(v0)(v3)
\Edge(v4)(v5)
\Edge(v0)(v2)
\Edge(v0)(v5)
\Edge(v3)(v5)
\Edge(v0)(v4)
\Edge(v1)(v4)
\Edge(v3)(v4)
\Edge(v1)(v5)
\Edge(v0)(v1)
\Edge(v2)(v5)
\Edge(v1)(v2)
\Edge(v2)(v4)
\end{tikzpicture}

    \end{subfigure}%
    \begin{subfigure}{.33\textwidth}
      \centering
      \begin{tikzpicture}
\GraphInit[unit=3,vstyle=Normal]
\SetVertexNormal[Shape=circle, FillColor = black, MinSize=3pt]
\tikzset{VertexStyle/.append style = {inner sep = \inners,outer sep = \outers}}
\SetVertexNoLabel
\Vertex[x = 1.30832, y = 1.51662]{v1}
\Vertex[x = 1.28532, y = 0.036]{v0}
\Vertex[x = 0.036, y = 0.84724]{v3}
\Vertex[x = 1.3539, y = 3.0298]{v2}
\Vertex[x = 0.097124, y = 3.835]{v5}
\Vertex[x = 0.08617, y = 2.357]{v4}
\Edge(v0)(v3)
\Edge(v4)(v5)
\Edge(v1)(v4)
\Edge(v1)(v2)
\Edge(v3)(v4)
\Edge(v0)(v1)
\Edge(v2)(v5)
\Edge(v1)(v3)
\Edge(v2)(v4)
\end{tikzpicture}

    \end{subfigure}
    \begin{subfigure}{.33\textwidth}
      \centering
      \begin{tikzpicture}
\GraphInit[unit=3,vstyle=Normal]
\SetVertexNormal[Shape=circle, FillColor = black, MinSize=3pt]
\tikzset{VertexStyle/.append style = {inner sep = \inners,outer sep = \outers}}
\SetVertexNoLabel
\Vertex[x = 2.4532, y = 1.17998]{v1}
\Vertex[x = 1.25934, y = 2.0774]{v0}
\Vertex[x = 2.2534, y = 2.5008]{v3}
\Vertex[x = 1.47314, y = 0.2163]{v2}
\Vertex[x = 0.94582, y = 1.22042]{v5}
\Vertex[x = 0.036, y = 0.036]{v4}
\Edge(v0)(v3)
\Edge(v4)(v5)
\Edge(v0)(v2)
\Edge(v0)(v5)
\Edge(v1)(v3)
\Edge(v1)(v5)
\Edge(v0)(v1)
\Edge(v2)(v5)
\Edge(v1)(v2)
\Edge(v3)(v5)
\Edge(v2)(v4)
\end{tikzpicture}

    \end{subfigure}
    \par\bigskip
    \begin{subfigure}{.33\textwidth}
      \centering
      \begin{tikzpicture}
\GraphInit[unit=3,vstyle=Normal]
\SetVertexNormal[Shape=circle, FillColor = black, MinSize=3pt]
\tikzset{VertexStyle/.append style = {inner sep = \inners,outer sep = \outers}}
\SetVertexNoLabel
\Vertex[x = 1.89708, y = 1.76094]{v1}
\Vertex[x = 0.036, y = 1.08372]{v0}
\Vertex[x = 0.38382, y = 0.036]{v3}
\Vertex[x = 1.56226, y = 2.8144]{v2}
\Vertex[x = 1.39586, y = 0.93704]{v5}
\Vertex[x = 0.526, y = 1.92696]{v4}
\Edge(v0)(v3)
\Edge(v4)(v5)
\Edge(v0)(v5)
\Edge(v3)(v5)
\Edge(v0)(v4)
\Edge(v1)(v4)
\Edge(v3)(v4)
\Edge(v1)(v5)
\Edge(v0)(v1)
\Edge(v2)(v5)
\Edge(v1)(v2)
\Edge(v2)(v4)
\end{tikzpicture}

    \end{subfigure}%
    \begin{subfigure}{.33\textwidth}
      \centering
      \begin{tikzpicture}
\GraphInit[unit=3,vstyle=Normal]
\SetVertexNormal[Shape=circle, FillColor = black, MinSize=3pt]
\tikzset{VertexStyle/.append style = {inner sep = \inners,outer sep = \outers}}
\SetVertexNoLabel
\Vertex[x = 1.34148, y = 1.45118]{v1}
\Vertex[x = 0.8426, y = 0.036]{v0}
\Vertex[x = 0.036, y = 2.6392]{v3}
\Vertex[x = 0.141848, y = 1.35614]{v2}
\Vertex[x = 0.51116, y = 4.056]{v5}
\Vertex[x = 1.23518, y = 2.742]{v4}
\Edge(v2)(v3)
\Edge(v0)(v2)
\Edge(v4)(v5)
\Edge(v3)(v5)
\Edge(v1)(v3)
\Edge(v1)(v4)
\Edge(v3)(v4)
\Edge(v0)(v1)
\Edge(v1)(v2)
\Edge(v2)(v4)
\end{tikzpicture}

    \end{subfigure}
    \begin{subfigure}{.33\textwidth}
      \centering
      \begin{tikzpicture}
\GraphInit[unit=3,vstyle=Normal]
\SetVertexNormal[Shape=circle, FillColor = black, MinSize=3pt]
\tikzset{VertexStyle/.append style = {inner sep = \inners,outer sep = \outers}}
\SetVertexNoLabel
\Vertex[x = 1.7815, y = 1.37864]{v1}
\Vertex[x = 0.71638, y = 0.24446]{v0}
\Vertex[x = 1.02986, y = 1.60574]{v3}
\Vertex[x = 1.4757, y = 0.036]{v2}
\Vertex[x = 0.036, y = 1.08744]{v5}
\Vertex[x = 2.4698, y = 0.54924]{v4}
\Edge(v0)(v3)
\Edge(v2)(v3)
\Edge(v0)(v2)
\Edge(v0)(v5)
\Edge(v1)(v3)
\Edge(v0)(v4)
\Edge(v1)(v4)
\Edge(v3)(v4)
\Edge(v1)(v5)
\Edge(v0)(v1)
\Edge(v2)(v5)
\Edge(v1)(v2)
\Edge(v3)(v5)
\Edge(v2)(v4)
\end{tikzpicture}

    \end{subfigure}
    \par\bigskip
    \begin{subfigure}{.33\textwidth}
      \centering
      \begin{tikzpicture}
\GraphInit[unit=3,vstyle=Normal]
\SetVertexNormal[Shape=circle, FillColor = black, MinSize=3pt]
\tikzset{VertexStyle/.append style = {inner sep = \inners,outer sep = \outers}}
\SetVertexNoLabel
\Vertex[x = 1.2723, y = 2.8958]{v1}
\Vertex[x = 0.036, y = 0.2646]{v0}
\Vertex[x = 1.57788, y = 1.18194]{v3}
\Vertex[x = 2.3758, y = 2.254]{v2}
\Vertex[x = 0.78102, y = 1.79834]{v5}
\Vertex[x = 1.23758, y = 0.036]{v4}
\Edge(v0)(v3)
\Edge(v4)(v5)
\Edge(v2)(v3)
\Edge(v1)(v3)
\Edge(v0)(v4)
\Edge(v1)(v2)
\Edge(v1)(v5)
\Edge(v3)(v4)
\Edge(v2)(v5)
\Edge(v3)(v5)
\end{tikzpicture}

    \end{subfigure}%
    \begin{subfigure}{.33\textwidth}
      \centering
      \begin{tikzpicture}
\GraphInit[unit=3,vstyle=Normal]
\SetVertexNormal[Shape=circle, FillColor = black, MinSize=3pt]
\tikzset{VertexStyle/.append style = {inner sep = \inners,outer sep = \outers}}
\SetVertexNoLabel
\Vertex[x = 1.51282, y = 0.14127]{v1}
\Vertex[x = 0.25296, y = 0.036]{v0}
\Vertex[x = 1.29558, y = 2.7186]{v3}
\Vertex[x = 0.26328, y = 1.33302]{v2}
\Vertex[x = 1.28896, y = 1.42044]{v5}
\Vertex[x = 0.036, y = 2.6104]{v4}
\Edge(v2)(v3)
\Edge(v0)(v2)
\Edge(v4)(v5)
\Edge(v0)(v5)
\Edge(v3)(v5)
\Edge(v3)(v4)
\Edge(v1)(v5)
\Edge(v0)(v1)
\Edge(v2)(v5)
\Edge(v1)(v2)
\Edge(v2)(v4)
\end{tikzpicture}

    \end{subfigure}
    \begin{subfigure}{.33\textwidth}
      \centering
      \begin{tikzpicture}
\GraphInit[unit=3,vstyle=Normal]
\SetVertexNormal[Shape=circle, FillColor = black, MinSize=3pt]
\tikzset{VertexStyle/.append style = {inner sep = \inners,outer sep = \outers}}
\SetVertexNoLabel
\Vertex[x = 1.44128, y = 0.97014]{v1}
\Vertex[x = 0.036, y = 0.86946]{v0}
\Vertex[x = 0.72938, y = 2.5378]{v3}
\Vertex[x = 0.7919, y = 0.036]{v2}
\Vertex[x = 0.056102, y = 1.6691]{v5}
\Vertex[x = 1.43066, y = 1.62654]{v4}
\Edge(v0)(v3)
\Edge(v4)(v5)
\Edge(v0)(v2)
\Edge(v0)(v5)
\Edge(v1)(v3)
\Edge(v0)(v4)
\Edge(v3)(v4)
\Edge(v1)(v5)
\Edge(v0)(v1)
\Edge(v2)(v5)
\Edge(v1)(v2)
\Edge(v3)(v5)
\Edge(v2)(v4)
\end{tikzpicture}

    \end{subfigure}
    \par\bigskip
    \begin{subfigure}{.33\textwidth}
      \centering
      \begin{tikzpicture}
\GraphInit[unit=3,vstyle=Normal]
\SetVertexNormal[Shape=circle, FillColor = black, MinSize=3pt]
\tikzset{VertexStyle/.append style = {inner sep = \inners,outer sep = \outers}}
\SetVertexNoLabel
\Vertex[x = 1.46994, y = 0.989]{v1}
\Vertex[x = 0.054778, y = 0.91242]{v0}
\Vertex[x = 0.67678, y = 2.545]{v3}
\Vertex[x = 0.80808, y = 0.036]{v2}
\Vertex[x = 1.40618, y = 1.66]{v5}
\Vertex[x = 0.036, y = 1.58814]{v4}
\Edge(v0)(v3)
\Edge(v4)(v5)
\Edge(v0)(v2)
\Edge(v0)(v5)
\Edge(v1)(v3)
\Edge(v1)(v4)
\Edge(v3)(v4)
\Edge(v0)(v1)
\Edge(v2)(v5)
\Edge(v1)(v2)
\Edge(v3)(v5)
\Edge(v2)(v4)
\end{tikzpicture}

    \end{subfigure}%
    \begin{subfigure}{.33\textwidth}
      \centering
      \begin{tikzpicture}
\GraphInit[unit=3,vstyle=Normal]
\SetVertexNormal[Shape=circle, FillColor = black, MinSize=3pt]
\tikzset{VertexStyle/.append style = {inner sep = \inners,outer sep = \outers}}
\SetVertexNoLabel
\Vertex[x = 1.61832, y = 0.51324]{v1}
\Vertex[x = 0.06159, y = 0.44664]{v0}
\Vertex[x = 0.7835, y = 1.82894]{v3}
\Vertex[x = 0.86098, y = 0.036]{v2}
\Vertex[x = 0.036, y = 1.3396]{v5}
\Vertex[x = 1.57708, y = 1.4073]{v4}
\Edge(v0)(v3)
\Edge(v2)(v3)
\Edge(v0)(v2)
\Edge(v4)(v5)
\Edge(v0)(v5)
\Edge(v1)(v3)
\Edge(v0)(v4)
\Edge(v1)(v4)
\Edge(v3)(v4)
\Edge(v1)(v5)
\Edge(v0)(v1)
\Edge(v2)(v5)
\Edge(v1)(v2)
\Edge(v3)(v5)
\Edge(v2)(v4)
\end{tikzpicture}

    \end{subfigure}
    \caption{The fourteen six vertex star graphs.}
    \label{fig:six_star}
\end{figure}
\section{Concluding Remarks}

This chapter introduced the class of star graphs -- the intersection graphs of the induced maximal stars of some graph.
We presented various results, beginning with a quadratic bound on the size of minimal pre-images.
Then, a Krausz-type characterization for the class was presented, which yielded membership of the recognition problem in \textsf{NP}.
We also presented a series of properties the members of the class must satisfy, such as being biconnected, that every edge must belong to some triangle, and present a bound on the diameter of the star graph based on the diameter of the pre-image, which implies that the diameter of the iterated star graph converges to either three or four;  when restricting the analysis to star-critical pre-images, we completely characterize the structures that the pre-image must have in order for the star graph to have a degree two vertex.
We present a monotonicity theorem for star-critical pre-images, which allowed us to compute all 267 star graphs on at most eight vertices, as well as all 978612 star-critical pre-images of these graphs.

We leave two main open questions.
The first, the complexity of the recognition problem, is perhaps the most challenging; for example, the complexity of the clique graph recognition problem was left open for many decades, only being settled recently~\citep{clique_recognition} through a series of non-intuitive gadgets.
The second is the optimality of the quadratic bound, or if it can be improved.
A complete characterization of both critical and non-critical vertices seems the biggest obstacle to obtain a linear bound on the size of critical pre-images; all our attempts to solve this task were thwarted by the amount of cases the analysis usually boils down to.
Despite our special interests on these questions, many different directions are available for investigation, such as on iterated applications of the star operator or relationships with other classes.