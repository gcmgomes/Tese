\section{Concluding Remarks}

In this chapter, we investigated three partitioning problems that belong to the class of coloring problems.
Namely, \pname{Equitable Coloring}, \pname{Clique Coloring}, \pname{Biclique Coloring}.
For \pname{Equitable Coloring}, we developed novel parameterized reductions from \pname{Bin Packing}, which is $\W[1]$-$\Hard$ when parameterized by number of bins.
These reductions showed that \pname{Equitable Coloring} is $\W[1]-\Hard$ in three more cases: (i) if we restrict the problem to block graphs and parameterize by the number of colors, treewidth and diameter; (ii) on the disjoint union of split graphs, a case where the connected case is polynomial; (iii) \pname{equitable coloring} of $K_{1,r}$ interval graphs, for any $r \geq 4$, remains hard even if we parameterize by the number of colors, treewidth and maximum degree.
This, along with a previous result by~\cite{claw_free_de_werra}, establishes a dichotomy based on the size of the largest induced star: for $K_{1,r}$-free graphs, the problem is solvable in polynomial time if $r \leq 2$, otherwise it is $\W[1]-\Hard$.
These results significantly improve the ones by~\cite{colorful_treewidth} through much simpler proofs and in very restricted graph classes.
Since the problem remains hard even for many natural parameterizations, we resorted to a more exotic one -- the treewidth of the complement graph.
By applying standard dynamic programming techniques on tree decompositions and the fast subset convolution machinery of~\cite{fourier_mobius}, we obtain an $\FPT$ algorithm when parameterized by the treewidth of the complement graph.
We also presented an \XP\ algorithm parameterized by number of colors when the input graph is known to be chordal.
Natural future research directions include the identification and study of other uncommon parameters that may aid in the design of other $\FPT$ algorithms.
Revisiting \pname{Clique Partitioning} when parameterized by $k$ and $r$ is also of interest, since its a related problem to \pname{Equitable Coloring} and the complexity of its natural parameterization is yet unknown.

As to the other problems, we showed that, much like \pname{Clique Coloring}, \pname{Biclique Coloring} can be solved in $\bigOs{2^{n}}$-time using the inclusion-exclusion principle.
Also of interest is the nice behavior \pname{Clique Coloring} presents when parameterized by neighborhood diversity, which enabled us to apply very simple reduction rules and obtain an $\bigOs{2^{2d}}$-time \FPT algorithm.
Moreover, said algorithm has optimal running time, assuming that \ETH\ holds.
For \pname{Biclique Coloring}, however, we were unable to provide an \FPT\ algorithm when considering solely neighborhood diversity and had to include the size of the largest true twin class -- which is a lower bound to the biclique chromatic number -- to obtain a parameterized algorithm. 
As such, we are led to believe that \pname{Biclique Coloring} parameterized by neighborhood diversity is not in $\FPT$.
Much of the exploratory work on different graph classes and parameters remains to be done for \pname{Biclique Coloring}, and it may be an interesting venue for future work.