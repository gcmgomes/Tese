\section{Exact Algorithm for \textsc{biclique coloring}}
\label{sec:biclique_exact}

\begin{figure}[!htb]
    \centering
    \begin{tikzpicture}[scale=1]
        \begin{scope}
         %\draw[help lines] (-5,-5) grid (5,5);
            \GraphInit[unit=3,vstyle=Normal]
            \SetVertexNormal[Shape=circle, FillColor=black, MinSize=2pt]
            \tikzset{VertexStyle/.append style = {inner sep = \inners, outer sep = \outers}}
            \SetVertexNoLabel
            \Vertex[x=-1,y=-1]{x-1-1}
            \Vertex[x=-1,y=0]{x-10}
            \Vertex[x=-1,y=1]{x-11}
            \Vertex[x=0,y=-1]{x0-1}
            \Vertex[x=0,y=0]{x00}
            \Vertex[x=0,y=1]{x01}
            \Vertex[x=1,y=-1]{x1-1}
            \Vertex[x=1,y=0]{x10}
            \Vertex[x=1,y=1]{x11}

            
            \Edge(x-1-1)(x-10)
            \Edge(x-1-1)(x0-1)
            \Edge(x0-1)(x00)
            \Edge(x0-1)(x1-1)
            
            \Edge(x-10)(x-11)
            \Edge(x-10)(x00)
            \Edge(x00)(x01)
            \Edge(x00)(x10)
            
            \Edge(x-11)(x01)
            \Edge(x01)(x11)
            
            \Edge(x10)(x11)
            \Edge(x10)(x1-1)
        \end{scope}
    \end{tikzpicture}
    \hfill
    \begin{tikzpicture}[scale=1]
        \begin{scope}
            %\draw[help lines] (-5,-5) grid (5,5);
            \GraphInit[unit=3,vstyle=Normal]
            \SetVertexNormal[Shape=circle, FillColor=white, MinSize=1pt]
            \tikzset{VertexStyle/.append style = {inner sep = \inners, outer sep = \outers}}
            \SetVertexNoLabel
            \Vertex[x=-1,y=-1]{x-1-1}
            \Vertex[x=-1,y=1]{x-11}
            \Vertex[x=1,y=-1]{x1-1}
            \Vertex[x=1,y=1]{x11}
            
            \SetVertexNormal[Shape=circle, FillColor=black, MinSize=1pt]
            \SetVertexNoLabel
            \Vertex[x=0,y=-1]{x0-1}
            \Vertex[x=-1,y=0]{x-10}
            \Vertex[x=1,y=0]{x10}
            \Vertex[x=0,y=0]{x00}
            \Vertex[x=0,y=1]{x01}

            
            \Edge(x-1-1)(x-10)
            \Edge(x-1-1)(x0-1)
            \Edge(x0-1)(x00)
            \Edge(x0-1)(x1-1)
            
            \Edge(x-10)(x-11)
            \Edge(x-10)(x00)
            \Edge(x00)(x01)
            \Edge(x00)(x10)
            
            \Edge(x-11)(x01)
            \Edge(x01)(x11)
            
            \Edge(x10)(x11)
            \Edge(x10)(x1-1)
        \end{scope}
    \end{tikzpicture}
    \hfill
    \begin{tikzpicture}[scale=1]
        \begin{scope}
            %\draw[help lines] (-5,-5) grid (5,5);
            \GraphInit[unit=3,vstyle=Normal]
            \SetVertexNormal[Shape=circle, FillColor=white, MinSize=1pt]
            \tikzset{VertexStyle/.append style = {inner sep = \inners, outer sep = \outers}}
            \SetVertexNoLabel
            \Vertex[x=-1,y=-1]{x-1-1}
            \Vertex[x=-1,y=1]{x-11}
            \Vertex[x=1,y=-1]{x1-1}
            \Vertex[x=1,y=1]{x11}
            \Vertex[x=0,y=0]{x00}
            
            \SetVertexNormal[Shape=circle, FillColor=black, MinSize=1pt]
            \SetVertexNoLabel
            \Vertex[x=0,y=-1]{x0-1}
            \Vertex[x=-1,y=0]{x-10}
            \Vertex[x=1,y=0]{x10}
            \Vertex[x=0,y=1]{x01}

            
            \Edge(x-1-1)(x-10)
            \Edge(x-1-1)(x0-1)
            \Edge(x0-1)(x00)
            \Edge(x0-1)(x1-1)
            
            \Edge(x-10)(x-11)
            \Edge(x-10)(x00)
            \Edge(x00)(x01)
            \Edge(x00)(x10)
            
            \Edge(x-11)(x01)
            \Edge(x01)(x11)
            
            \Edge(x10)(x11)
            \Edge(x10)(x1-1)
        \end{scope}
    \end{tikzpicture}
    \hfill
    
    
    \caption{A graph (left), one of its maximal bicliques in black (center), and a transversal in black (right).}
    \label{fig:transversals}
\end{figure}


Our first task is to establish a family of sets that can be safely used to cover $V(G)$, which we accomplish as follows.


\begin{lemma}
    \label{lem:transversal_colorings}
    A $k$-partition $\varphi = \{\varphi_1, \dots, \varphi_k\}$ is a $k$-biclique-coloring of $G$ if and only if for every $i$, $\overline{\varphi_i} \in \trans{B}(G)$.
\end{lemma}

\begin{tproof}
$(\Rightarrow)$ Suppose that there exists some $\varphi_i$ such that $\overline{\varphi_i} \notin \trans{B}(G)$.
This implies that there exists some $B \in \biq(G)$ such that $B \cap \overline{\varphi_i} = \emptyset$ and that $B \subseteq \varphi_i$; that is, $|\varphi(B)| = 1$, which is a contradiction, since $\varphi$ is a $k$-biclique-coloring.

$(\Leftarrow)$ Let $\varphi$ be a $k$-partition of $G$ with $\overline{\varphi_i} \in \trans{B}(G)$, but suppose that $\varphi_k$ is not a $k$-biclique-coloring.
That is, there exists some maximal biclique $B \in \biq(G)$ such that $B \subseteq \varphi_i$ for some $i$.
This implies that $B \cap \overline{\varphi_i} = \emptyset$, and, therefore, $\overline{\varphi_i} \notin \trans{B}(G)$, which contradicts the hypothesis.
\end{tproof}

Simply testing for each $X \in 2^{V(G)}$ if $X \in \transs{B}(G)$ is a costly task.
A naive algorithm would check, for each $B \in \biq(G)$, if $\overline{X} \cap B \neq \emptyset$.
With $|\biq(G)| \in \bigO{n3^{\frac{n}{3}}}$ (see~\citep{gaspers} for the proof), such algorithm would take $\bigO{n2^n3^{\frac{n}{3}}}$ time.

The next Lemma, along with Lemma~\ref{lem:down_closure}, considerably reduce the complexity of enumerating $\trans{B}(G)$.
Effectively, we will enumerate $\oblq{B}(G)$ by first generating its maximal elements and then use the fact that $\oblq{B}(G)$ is closed under the subset operation, which takes $\bigOs{2^n}$ time.

\begin{lemma}
    \label{lem:complementary_obliques}
    The maximal obliques of $\Hyper{B}(G)$ are exactly the complements of the maximal bicliques of $G$.
\end{lemma}

\begin{tproof}
    $(\Rightarrow)$ Let $X \in \oblq{B}$ be a maximal oblique. By definition, there exists some $B \in \biq(G)$ such that $X \cap B = \emptyset$, which implies that $X \subseteq \overline{B}$.
    Note that, if $X \subset \overline{B}$, there is some $v \in \overline{B} \setminus X$, which implies that $(X \cup \{v\}) \cap B = \emptyset$ and that $X$ is not a maximal oblique.
    
    $(\Leftarrow)$ Let $B \in \biq(G)$. By definition, $\overline{B} \in \oblq{B}$ and must be maximal because $\{B, \overline{B}\}$ is a partition of $V(G)$.
\end{tproof}

\begin{corollary}
    \label{col:is_maximal_oblique}
    Given a graph $G = (V, E)$ and a subset $X \subseteq V(G)$, there exists an $O(n(n - |X|))$-time algorithm to determine if $X$ is a maximal oblique.
\end{corollary}

\begin{theorem}
    There is an $\bigOs{2^n}$ time algorithm for \textsc{biclique coloring}.
\end{theorem}

\begin{tproof}
    Our goal is to make use of Theorem~\ref{thm:inc_exc} to solve an instance of \textsc{exact cover}, with $A = V(G)$, $\mathcal{F} = \transs{B}(G)$ and $k$ the partition size.
    Lemma~\ref{lem:transversal_colorings} guarantees that there is an answer to our instance of \textsc{biclique coloring} if and only if there is an answer to the corresponding \textsc{exact cover} instance.
    
    To compute $\transs{B}(G)$, for each $X \in 2^{V(G)}$, we use Lemma~\ref{lem:complementary_obliques} and Corollary~\ref{col:is_maximal_oblique} to say whether or not $X$ is a maximal oblique of $\Hyper{B}(G)$.
    
    Afterwards, we compute $\oblq{B}(G)$ from its maximal elements using Lemma~\ref{lem:down_closure}, and use the fact that $\trans{B}(G) = 2^{V(G)} \setminus \oblq{B}(G)$ and complement each transversal to obtain $\transs{B}(G)$.
    Clearly, this procedure takes $\bigOs{2^n}$ to construct $\transs{B}(G)$ and an additional $\bigOs{2^n}$ to use Theorem~\ref{thm:inc_exc}.
\end{tproof}