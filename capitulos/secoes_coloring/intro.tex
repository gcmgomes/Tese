\section{Definitions and related work}

A \tdef{k-coloring} of a graph $G$ is a $k$-partition $\varphi = \{\varphi_1, \dots,\varphi_k\}$ of $V(G)$.
Each $\varphi_i$ is a \tdef{color class} and $v \in V(G)$ is \tdef{colored} with color $i$ if and only if $v \in \varphi_i$.
In a slight abuse of notation, we use $\varphi(v)$ to denote the color of $v$ and, for $X \subseteq V(G)$, $\varphi(X) = \bigcup_{v \in X} \{\varphi(v)\}$.

\subsection{Proper coloring}

A \tdef{proper k-coloring} of $G$ is a $k$-coloring such that each $\varphi_i$ is an independent set.
In the literature, proper coloring is usually referenced to as \tdef{Vertex Coloring}, a convention we also adopt.
If $G$ has a proper $k$-coloring we say that $G$ is \tdef{k-colorable}.
The smallest integer $k$ such that $G$ is $k$-colorable is called the \tdef{chromatic number} $\chi(G)$ of $G$.
The natural decision problem associated with vertex coloring simply asks whether or not a given graph is $k$-colorable.

\problem{Vertex Coloring}{A graph $G$ and a positive integer $k$.}{Is $G$ $k$-colorable?}


\begin{figure}[!htb]
    \centering
    \begin{tikzpicture}[rotate=90,scale=0.6]
            %\draw[help lines] (-5,-5) grid (5,5);
            \GraphInit[vstyle=Simple]
            \SetVertexSimple[Shape=circle, FillColor=black, MinSize=2pt]
            \tikzset{VertexStyle/.append style = {inner sep = \inners, outer sep = \outers}}
            \SetUpVertex[Ldist=3pt]
            \grWheel[prefix=a]{6}
            \AssignVertexLabel{a}{1,2,1,2,3,4}
        \end{tikzpicture}
    \caption{An optimal proper coloring.}
    \label{fig:prop_color}
\end{figure}

Determining if a given instance of \pname{Vertex Coloring} is a $\YES$ instance is a classic problem in both graph theory and algorithmic complexity, being a known $\NPc$ problem.
Some particular cases of \pname{Vertex Coloring} are still $\NPc$.
For instance, even if we fix $k=3$ or restrict the input to $K_3$-free graphs the problem does not get any easier~\citep{garey_johnson, triangle_free_coloring}.

It is worth to point out the subtle difference between the parameter $k$ being part of the input or being \tdef{fixed}.
Informally, when $k$ is fixed, we are willing to pay exponential time only on $k$ to solve our problem, whereas when $k$ is part of the input, we are not.
Note that when we fix $k$ and find an $f(k)n^{\bigO{1}}$ time algorithm, we show that the problem is in $\FPT$ when parameterized by $k$.
The fact that $3$-coloring is $\NPc$ is evidence that \pname{Vertex Coloring} parameterized by the number of colors is not in $\FPT$, otherwise we would have an $f(3)n^{\bigO{1}}$ algorithm, which would imply that $\P = \NP$. 

For an unconstrained input, \pname{Vertex Coloring} is hard to approximate to a factor of $n^{1-\epsilon}$, for any $\epsilon > 0$, unless some complexity hypothesis fail (see~\citep{color_zpp} for more on the topic).
On a brighter note, a celebrated theorem due to Brooks in~\citep{brooks_theorem} gives a nice upper bound for general graphs, and gives a natural direction for research on tighter upper bounds on graph classes.

\begin{theorem*}[Brooks' Theorem]
    For every connected graph $G$ which is neither complete nor an odd-cycle $\chi(G) \leq \Delta(G)$.
\end{theorem*}

These results motivated much of the research about \pname{Vertex Coloring}.
There are polynomial time algorithms for a myriad of different classes, including chordal, bipartite and cographs.
More generally, there are known polynomial time algorithms for \tdef{perfect graphs}~\citep{perfect_polynomial}, which is a superclass of the aforementioned ones.
$G$ is perfect if for every induced subgraph $G'$ of $G$, $\chi(G') = \omega(G')$.

More particular cases for \pname{Vertex Coloring} have also been analyzed. For instance, \cite{coloring_art} present some results for graph classes that have two connected five-vertex forbidden induced subgraphs. There are some surveys on the subject, as such we point to~\citep{coloring_survey} and~\citep{coloring_survey2} for more on the classic \pname{Vertex Coloring} problem, since it is not the focus of this thesis.


\subsection{Equitable coloring}
A $k$-coloring of an $n$ vertex graph is said to be \tdef{equitable} if for every color class $\varphi_i$, $\floor{\frac{n}{k}} \leq |\varphi_i| \leq \ceil{\frac{n}{k}}$ or, equivalently, if for, any two color classes $\varphi_i$ and $\varphi_j$, $||\varphi_i| - |\varphi_j|| \leq 1$.
If $G$ admits a proper equitable $k$-coloring, we say that $G$ is \tdef{equitably k-colorable}.
Unlike other coloring variants previously discussed, an equitably $k$-colorable graph is not necessarily equitably $(k+1)$-colorable.

As such, two different parameters are defined: the smallest integer $k$ such that $G$ is equitably $k$-colorable is called the \tdef{equitable chromatic number} $\CN{=}(G)$; the smallest integer $k'$ such that $G$ is equitably $k$-colorable for every $k \geq k'$ is the \textit{equitable chromatic threshold} $\CN{=}^*(G)$ of $G$.

As with the previous coloring problems, we define the \pname{Equitable Coloring} decision problem.

\problem{Equitable Coloring}{A graph $G$ and a positive integer $k$.}{Is $G$ equitably $k$-colorable?}


\begin{figure}[!htb]
    \centering
    \begin{tikzpicture}
        \begin{scope}[rotate=90,scale=0.6]
            %\draw[help lines] (-5,-5) grid (5,5);
            \GraphInit[vstyle=Normal]
            \SetVertexNormal[Shape=circle]
            \tikzset{VertexStyle/.append style = {inner sep = \inners, outer sep = \outers}}
            \SetVertexNoLabel
            \grStar[prefix=a]{6}
            \AssignVertexLabel{a}{2,2,2,2,2,1}
        \end{scope}
    \end{tikzpicture}
    \hfill
    \begin{tikzpicture}
        \begin{scope}[rotate=90,scale=0.6]
            %\draw[help lines] (-5,-5) grid (5,5);
            \GraphInit[vstyle=Normal]
            \SetVertexNormal[Shape=circle]
            \tikzset{VertexStyle/.append style = {inner sep = \inners, outer sep = \outers}}
            \SetVertexNoLabel
            \grStar[prefix=b]{6}
            \AssignVertexLabel{b}{3,3,4,2,2,1}
        \end{scope}
    \end{tikzpicture}
    \hfill
    
    \caption{A proper non-equitable coloring (left) and an equitable coloring (right).}
    \label{fig:eq_color}
\end{figure}


\pname{Equitable Coloring} was first discussed by~\citep{first_equitable}, with an intended application for municipal garbage collection, and later in processor task scheduling~\citep{mutual_exclusion_scheduling} and server load balancing~\citep{domain_decomposition}.

Much of the work done over \pname{Equitable Coloring} aims to prove an analogue of Brooks' theorem, known as the \tdef{Equitable coloring conjecture} (ECC).
In terms of the equitable chromatic threshold, however, we have the \tdef{Hajnal-Szemerédi theorem}~\citep{hajnal_szmeredi_theorem}.

\begin{conjecture*}[ECC]
    For every connected graph $G$ which is neither a complete graph nor an odd-hole, $\CN{=}(G) \leq \Delta(G)$.
\end{conjecture*}

\begin{theorem*}[Hajnal-Szmerédi Theorem]
    Any graph $G$ is equitably $k$-colorable if $k \geq \Delta(G) + 1$. Equivalently, $\CN{=}^*(G) \leq \Delta(G) + 1$.
\end{theorem*}

\cite{e_delta_cc} suggest that a stronger result than the Hajnal-Szmerédi theorem may be achievable, presenting some classes where the \tdef{Equitable $\Delta$-coloring conjecture} (E$\Delta$CC) holds.
Moreover, they prove that if E$\Delta$CC holds for every regular graph, then it holds for every graph.

\begin{conjecture*}[E$\Delta$CC]
    For every connected graph $G$ which is not a complete graph, an odd-hole nor $K_{2n+1, 2n+1}$, for any $n \geq 1$, $\CN{=}^*(G) \leq \Delta(G)$ holds.
\end{conjecture*}

Quite a lot of effort was put into finding classes where E$\Delta$CC holds, even with the knowledge that only proofs for regular graphs are required.
A result given by~\cite{claw_free_de_werra}, combined with Brooks' Theorem, implies that every claw-free graph is equitably $k$-colorable for every $k \geq \chi(G)$.
A very extensive survey on the subject was conducted by~\cite{equitable_survey}, where many of the results of the past 50 years were assembled.
Among the many reported results, the E$\Delta$CC is known to hold for:
bipartite graphs (with the obvious exceptions, where the ECC holds)~\citep{equitable_bipartite},
planar graphs with maximum degree at least nine~\citep{equitable_planar},
split graphs~\citep{equitable_split},
\tdef{outerplanar graphs} (planar graphs with a drawing such that no vertex is within a polygon formed by other vertices)~\citep{equitable_outerplanar},
\tdef{$d$-degenerate graphs} (graphs such that every induced subgraph has a vertex of degree at most $d$)~\citep{equitable_degenerate},
non-trivial \tdef{Kneser graphs} (complement of the intersection graph of $F \subset 2^{[n]}$, with every set of $F$ containing exactly $k$ elements, where $n > 2k$)~\citep{equitable_kneser},
interval graphs~\citep{equitable_interval}, and
some forms of graph products~\citep{equitable_interval}.
For the exact results please refer to the survey.

Almost all complexity results for \pname{Equitable Coloring} arise from a related problem, known as \pname{Bounded Coloring}, an observation given by~\cite{equitable_treewidth}.
A $k$-coloring is said to be \tdef{$l$-bounded} if for every color class $\varphi_i$, $|\varphi_i| \leq l$.
$G$ is \tdef{l-bounded k-colorable} if it admits an $l$-bounded $k$-coloring.

\problem{Bounded Coloring}{A graph $G$ and two positive integers $l$ and $k$.}{Is $G$ $l$-bounded $k$-colorable?}

\begin{observation*}
    A Graph $G$ with $n$ vertices is $l$-bounded $k$-colorable if and only if $G' = G \cup \overline{K_{lk - n}}$ is equitably $k$-colorable.
\end{observation*}


In terms of computational complexity, however, neither problem was nearly as explored as \pname{Vertex Coloring}.
Among the complexity results for \pname{Bounded Coloring} and, consequently, \pname{Equitable Coloring}, we have polynomial time solvability for split graphs~\citep{equitable_split}, complement of interval graphs~\citep{graph_partitioning1}, forests~\citep{mutual_exclusion_scheduling}, trees~\citep{equitable_trees} and complements of bipartite graphs~\citep{graph_partitioning1}.

For cographs, we have a polynomial time algorithm when $k$ is fixed, otherwise the problem is $\NPc$~\citep{graph_partitioning1}, a situation similar to that of bipartite and interval graphs~\citep{graph_partitioning1}.
A consequence of the hardness result for cographs is that \pname{Equitable Coloring} is $\NPc$ for graphs of bounded cliquewidth.

On complements of \tdef{comparability graphs} (i.e. graphs representing a valid partial ordering) however, even if we fix $l$, \pname{Bounded Coloring} is still $\NPc$~\citep{chain_antichain}.
\cite{colorful_treewidth} show that, when parameterized by treewidth, \pname{Equitable Coloring} is $\W[1]\text{-}\Hard$.
Also in terms of treewidth, \cite{equitable_treewidth} give a polynomial time algorithm for graphs of bounded treewidth.
Note that for all of the mentioned classes, \pname{Vertex Coloring} is polynomially solvable.

A summary of the known complexities is available in Table~\ref{tab:equitable_complexity}.


\begin{table}[!htb]
    \centering
    \begin{tabular}{l|l|l}
       \hline
       \hline
       Class               &  fixed $k$         & input $k$             \\
       \hline
       Trees                &  $\P$             & $\P$                  \\
       Forests              &  $\P$             & $\P$                  \\
       Bipartite            &  $\NPc$           & $\NPc$                \\
       Co-bipartite         &  $\P$             & $\P$                  \\
       Cographs             &  $\P$             & $\NPc$                \\
       Bounded Cliquewidth  &  $\NPc$                & $\NPc$                \\
       Bounded Treewidth    &  $\P$             & $\P$                  \\
       Chordal              &  $\P^*$           & $\NPc$                \\
       Block                &  $\P^*$           & $\NPc^*$              \\
       Split                &  $\P$             & $\P$                  \\
       Interval             &  $\P$             & $\NPc$                \\
       Co-interval          &  $\P$             & $\P$                  \\
       General case         &  $\NPc$           & $\NPc$                \\
       \hline
       \hline
    \end{tabular}
    \caption{Complexity results for \pname{Equitable Coloring}. Entries marked with a * are results established in this work.}
    \label{tab:equitable_complexity}
\end{table}

\subsection{Clique coloring}
A \tdef{k-clique-coloring} of $G$ is a $k$-coloring of $G$ such that no maximal clique of $G$ is entirely contained in a single color class.
We say that $G$ is \tdef{k-clique-colorable} if $G$ admits a $k$-clique-coloring.
The smallest integer $k$ such that $G$ is $k$-clique-colorable is called the \tdef{clique chromatic number} $\CN{C}(G)$ of $G$.
Much like \pname{Vertex Coloring}, there is a natural decision problem associated with this coloring variant, which we refer to as \pname{Clique Coloring}.


\problem{Clique Coloring}{A graph $G$ and a positive integer $k$.}{Is $G$ $k$-clique-colorable?}


\begin{figure}[!htb]
    \centering
    \begin{tikzpicture}[rotate=90,scale=0.6]
            %\draw[help lines] (-5,-5) grid (5,5);
            \GraphInit[vstyle=Simple]
            \SetVertexSimple[Shape=circle, FillColor=black, MinSize=2pt]
            \tikzset{VertexStyle/.append style = {inner sep = \inners, outer sep = \outers}}
            \SetUpVertex[Ldist=3pt]
            \grWheel[prefix=a]{6}
            \AssignVertexLabel{a}{1,1,1,1,1,2}
        \end{tikzpicture}
    \caption{An optimal clique coloring.}
    \label{fig:clique_color}
\end{figure}

Research on the topic is much more recent than what was done for \pname{Vertex Coloring}, with the first papers appearing in the early 1990s~\citep{first_clique_color} and interest on the subject rising in the early 2000s.
Even when $k$ is fixed, \pname{Clique Coloring} is known to be $\SiP{2}\text{-}\Complete$, as shown by~\cite{clique_coloring_complexity}, with an $\bigOs{2^n}$ algorithm being proposed by~\cite{clique_color_algorithm}.

As with \pname{Vertex Coloring}, \pname{Clique Coloring} has been studied when restricting the input graph to certain graph classes.
\cite{weakly_clique_color} investigate 2-clique-coloring in terms of \tdef{weakly chordal graphs} (graphs free of any hole or anti-hole with more than 4 vertices), giving a series of results for the general case ($\SiP{2}\text{-}\Complete$) and showing that, for some nested subclasses, there are $\NPc$ and $\P$ instances of the problem.
When dealing with \tdef{unichord-free} graphs (graphs that contain no induced cycle with a unique chord), the problem is solvable in polynomial time~\citep{unichord_coloring}.

\tdef{Circular-arc} graphs (intersection graphs of a set of arcs of a circle) are always 3-clique-colorable, with a polynomial time algorithm to determine if the input is 2-clique-colorable~\citep{clique_circular_arc}.
When the given graph is \tdef{odd-hole-free}, it is $\SiP{2}\text{-}\Complete$ to decide whether it is 2-clique-colorable or not~\citep{clique_oddhole}.
\cite{clique_coloring_few_p4} give a series of bounds on graphs that, in some sense, contain few $P_4$'s, showing that most them are either 2 or 3-clique-colorable.

For \tdef{planar graphs} (graphs that can be drawn on a plane with no crossing edges), \cite{clique_coloring_planar} show that they are 3-clique-colorable, and \cite{clique_color_perfect_np_complete} present a polynomial time algorithm to decide whether a planar graph is 2-clique-colorable or not.

Some of these classes are subclasses of perfect graphs, and a conjecture suggests that every perfect graph is 3-clique-colorable~\citep{maximal_clique_coloring}.
Also in terms of perfect graphs, it is $\NPc$ to decide whether a perfect graph is 2-clique-colorable~\citep{clique_color_perfect_np_complete}.
\cite{clique_oddhole} also give the observation that every \tdef{strongly perfect graph}~\citep{strongly_perfect} is 2-clique-colorable, a superclass of both chordal graphs and cographs.
For a summary of the mentioned results, please refer to Table~\ref{tab:clique_color_complexity}.

\begin{table}[!htb]
    \centering
    \begin{tabular}{c|c|c}
        \hline
        \hline
        Class            & $\CN{C}$             & Complexity \\
        \hline
        Cograph          & $= 2$                & $\P$\\
        Chordal          & $= 2$                & $\P$\\
        Weakly Chordal   & $\leq 3^*$           & $\SiP{2}\text{-}\Complete^\dag$\\
        Unichord-free    & $\leq 3 $            & $\P$ \\
        Circular-arc     & $\leq 3$             & $\P^\dag$ \\
        Odd-hole-free    & $\leq 3^*$           & $\SiP{2}\text{-}\Complete^\dag$\\
        Few $P_4$'s      & $\leq 2$ or $\leq 3$ & $\P$\\
        Planar           & $\leq 3$             & $\P^\dag$\\
        Perfect          & $\leq 3^*$           & $\NPc^\dag$\\ 
        Strongly Perfect & $= 2$                & $\P$\\
        General case     & Unbounded            & $\SiP{2}\text{-}\Complete$\\
        \hline
        \hline
    \end{tabular}
    \caption{Complexity and bounds for \pname{Clique Coloring}. Entries marked with a $*$ are conjectures. $\dag$ indicates results for 2-clique-colorability.}
    \label{tab:clique_color_complexity}
\end{table}

\subsection{Biclique coloring}
A \tdef{k-biclique-coloring} of $G$ is a $k$-coloring of $G$ such that no maximal biclique of $G$ is entirely contained in a single color class.
We say that $G$ is \tdef{k-biclique-colorable} if $G$ admits a $k$-biclique-coloring.
The smallest integer $k$ such that $G$ is $k$-biclique-colorable is called the \tdef{biclique chromatic number} $\CN{B}(G)$ of $G$.
Much like \pname{Clique Coloring}, there is a natural decision problem associated with this coloring variant, which we refer to as \pname{Biclique Coloring}.

\problem{Biclique Coloring}{A graph $G$ and a positive integer $k$.}{Is $G$ $k$-biclique-colorable?}


\begin{figure}[!htb]
    \centering
    \begin{tikzpicture}[rotate=90,scale=0.6]
            %\draw[help lines] (-5,-5) grid (5,5);
            \GraphInit[vstyle=Simple]
            \SetVertexSimple[Shape=circle, FillColor=black, MinSize=2pt]
            \tikzset{VertexStyle/.append style = {inner sep = \inners, outer sep = \outers}}
            \SetUpVertex[Ldist=3pt]
            \grWheel[prefix=a]{6}
            \AssignVertexLabel{a}{1,2,1,2,3,3}
        \end{tikzpicture}
    \caption{An optimal biclique coloring.}
    \label{fig:biclique_color}
\end{figure}


\pname{Biclique Coloring} is an even more recent research topic than \pname{Clique Coloring}, with the first results being a $\SiP{2}\text{-}\Cness$ proof due to~\cite{biclique_coloring_complexity} and the confirmation that verifying a solution to the problem is a $\coNP\text{-}\Complete$ task~\citep{biclique_coloring_verification}.

In terms of complexity results, very little is known about \pname{Biclique Coloring}.
For unichord-free graphs, \cite{unichord_coloring} give a polynomial time algorithm to compute $\CN{B}$ and show that the biclique chromatic number of unichord-free graphs is either equal to or one greater than the size of the largest true twin class.
\cite{biclique_coloring_verification} present a polynomial time algorithm for powers of cycles and powers of paths.
Finally, \cite{biclique_coloring_complexity} give complexity results for $H$-free graphs, for every $H$ on three vertices, being polynomial for $H \in \{K_3, P_3, \overline{P_3}\}$ and $\NPc$ for $\overline{K_3}$-free graphs;
moreover they show that the problem is $\NPc$ for diamond ($C_4$ plus one chord) free graphs and split graphs, and polynomial for \tdef{threshold graphs} (\{$2K_2,P_4,C_4$\}-free).
We summarize the presented results in Table~\ref{tab:biclique_color_complexity}.

\begin{table}[!htb]
    \centering
    \begin{tabular}{c|c|c}
        \hline
        \hline
        Class            & $\CN{B}$             & Complexity \\
        \hline
        Split            & Unbounded            & $\NPc$\\
        Threshold        & Unbounded            & $\P$\\
        Diamond-free     & Unbounded            & $\NPc$\\
        $C_n^r$          & $\leq 3$             & $\P$\\
        $P_n^r$          & $= 2$                & $\P$\\
        Unichord-free    & Bounded              & $\P$\\
        General case     & Unbounded            & $\SiP{2}\text{-}\Complete$\\
        \hline
        \hline
    \end{tabular}
    \caption{Complexity and bounds for \pname{Biclique Coloring}.}
    \label{tab:biclique_color_complexity}
\end{table}

Both \pname{Clique Coloring} and \pname{Biclique Coloring} are, actually, colorings of the hypergraphs arising from an underlying graph (a coloring of its vertices such that no hyperedge is monochromatic), which is also an $\NPc$ task.
However, in classical hypergraph coloring problems, the hyperedge family is part of the input of the problem and, as such, naively verifying a solution is polynomial on the size of the input.