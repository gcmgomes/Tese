\section{Basic Definitions}
\label{sec:basic_defs}

We denote by $[n] = \{1, \dots, n\}$.
A (\tdef{multi})\tdef{family} is a (multi)set of sets.
The \tdef{power set} $2^S$ of a set $S$ is the family of all subsets of $S$.
A \tdef{k-partition} of $S$ into $k$ sets is denoted by $S \sim \{S_1, \dots, S_k\}$ such that $S_i \cap S_j = \emptyset$ and $\bigcup_{i \leq k} S_i = S$.
A \tdef{k-(multi)cover} of $S$ is a (multi)family $\{S_1, \dots, S_k\}$ of subsets of $S$ such that $\bigcup_{i \in [k]} S_i = S$.
A (multi)family $\mathcal{F}$ satisfies the \tdef{Helly condition} or \tdef{Helly property} if and only if, for every pairwise intersecting subfamily $\mathcal{F}'$ of $\mathcal{F}$, $\bigcap_{F \in \mathcal{F}'} F \neq \emptyset$.

A \tdef{simple graph} of \tdef{order} (or \tdef{size}) $n$ is an ordered pair $G = (V(G), E(G))$, where $V(G)$ is its \tdef{vertex set} of cardinality $n$ and its \tdef{edge set}, $E(G)$, is a family of subsets of $V(G)$, each of cardinality two. A graph is \tdef{trivial} if $|V(G)| = 1$.
Instead of $\{u,v\} \in E(G)$, we identify an edge by $uv$, simply due to convenience. Moreover, when there is no ambiguity, we denote $V(G)$ by $V$, $E(G)$ by $E$, $|V|$ as $n$ and $|E|$ as $m$.

We say that two vertices $u,v \in V$ are \tdef{adjacent} or \tdef{neighbours} if $uv \in E$.
A graph $G' = (V', E')$ is a \tdef{subgraph} of $G$ if $V' \subseteq V$ and $E' \subseteq E$.
If $E' = \{uv \in E \mid u,v \in V'\}$ we say that $V'$ \tdef{induces} $G'$, $G' = G[V']$ and that $G'$ is the \tdef{induced subgraph} of $G$ by $V$.
For simplicity, we denote by $G - v$ the graph $G[V(G) \setminus \{v\}]$ and, similarly, for $S \subseteq V(G)$, $G \setminus S$ is equivalent to $G[V(G) \setminus S]$.

The \tdef{open neighbourhood}, or just \tdef{neighbourhood} of a vertex $v$ in $G$ is given by $N_G(v) = \{u \mid uv \in E(G)\}$, its \tdef{closed neighbourhood} by $N_G[v] = N_G(v) \cup \{v\}$ and its \tdef{degree} by $\deg_G(v) = |N_G(v)|$.
A vertex is \tdef{simplicial} if its neighbours are pairwise adjacent.
For a set $S \subseteq V$, we denote its open and closed neighbourhood as $N_G(S) = \bigcup_{v \in S} N_G(v) \setminus S$, $N_G[S] = N_G(S) \cup S$ and $\deg_G(S) = |N_G(S)|$, respectively.
The \tdef{complement} $\overline{G}$ of $G$ is defined as $V(\overline{G}) = V(G)$ and $E(\overline{G}) = \{uv | uv \notin E\}$.
Given a graph $G$, we denote its \tdef{maximum degree} by $\Delta(G)$ and \tdef{minimum degree} by $\delta(G)$.
A graph $G$ is \tdef{$k$-regular} if , for every $u \in V(G)$, $\deg_G(u) = k$.

Two vertices $u,v$ are \tdef{false twins} if $N_G(u) = N_G(v)$ and \tdef{true twins} if $N_G[u] = N_G[v]$. $u,v$ are of the same \tdef{type} if they are either true or false twins.
Being of the same type is an equivalence relation~\citep{neighbourhood_diversity}, and the number of different types on a graph $G$ is called its \tdef{neighbourhood diversity}, $\nd(G)$.

Two graphs $G$ and $H$ are \tdef{isomorphic} if and only if there is a bijection $f: V(G) \mapsto V(H)$ such that $uv \in E(G) \Leftrightarrow f(u)f(v) \in E(H)$.
We denote isomorphism by $G \simeq H$.
A graph $G$ is said to be \tdef{free} of a graph $H$, or $H$-free, if there is no induced subgraph $G'$ of $G$ such that $G'$ and $H$ are isomorphic.

The \tdef{path} of length $k$, or $P_k$, is a graph with $k$ vertices $v_1 \dots v_k$ such that $v_iv_j \in E(P_k)$ if and only if $j = i+1$. Moreover, we say that $v_1, v_k$ are the \tdef{extremities}, or \tdef{endvertices}, of $P_k$ and all other $v_i$ are its \tdef{inner} vertices.
The length of a path is the number of edges contained in it, that is, $P_k$ has length $k-1$.
An \tdef{induced path} of $G$ is a subgraph $G'$ of $G$ that is isomorphic to a path.
A \tdef{cycle} with  $k \geq 3$ vertices is a path with $k$ vertices plus the edge $v_kv_1$; analogously, the length of a cycle is defined as the number of edges it contains.
An \tdef{induced cycle} of $G$ is an induced subgraph $G'$ of $G$ that is isomorphic to a cycle.
A \tdef{chord} in a cycle $C$ of length at least 4 is an edge between two non-consecutive vertices of $C$.
The \tdef{girth} of $G$, denoted by $\girth(G)$, is the length of the smallest induced cycle of $G$.
A \tdef{hole} is a chordless cycle of length at least 4; it is an even-hole if it has an even number of vertices, or an odd-hole, otherwise. An \tdef{anti-hole} is the complement of a hole.
$G$ is \tdef{acyclic} if and only if there is no induced cycle in $G$.


A graph $G$ is \tdef{connected} if and only if there is an induced path between every pair $u,v \in V$.
A \tdef{connected component}, or simply a \tdef{component}, of $G$ is a maximal connected induced subgraph of $G$.
Given a graph $G$, the \tdef{distance} $\dist_G(u,v)$ between two vertices $u,v$ in $G$ is the minimum number of edges in any path between them.
If $u,v$ are in different components, we say that $\dist_G(u,v) = \infty$.
The \tdef{diameter} of a connected graph $G$ is defined as the length of the longest shortest path between any pair of vertices $u,v \in V(G)$.
The \tdef{$k$-th power} $G^k$ of a graph $G$ is the graph where $V(G^k) = V(G)$ and $E(G^k) = \{uv \mid \dist_G(u,v) \leq k\}$.
When $k = 2$, $G^2$ is also called the \textit{square} of graph $G$ and $G$ is called the \textit{square root} of $G^2$.
The class of all graphs that admit a square root is called \textit{square graphs}.
When the graph in question is clear, we will omit the $G$ subscript.

An \tdef{articulation point}, \tdef{cut point} or \tdef{cut vertex} of a connected graph $G$ is a vertex $v$ such that $G - v$ is disconnected.
A \tdef{bridge} is an edge of $G$ whose removal increases the number of connected components of $G$.
$G$ is \tdef{biconnected} if $G$ is connected and does not have a cut vertex.
A \tdef{cutset} of a connected graph $G$ is a set $S \subset V$ such that $G \setminus S$ is disconnected.
In particular, a cut vertex is a cutset of size one.

The \tdef{complete} graph $K_n$ of order $n$ is a graph where every pair of vertices is adjacent.
A \tdef{clique} of $G$ with $n$ vertices is an induced subgraph of $G$ isomorphic to $K_n$.
An \tdef{independent set} of $G$ with $n$ vertices is an induced subgraph of $G$ isomorphic to $\overline{K_n}$.
We denote by $\omega(G)$ and $\alpha(G)$ the size of the maximum induced clique and maximum independent set of a given $G$.
A \textit{vertex cover} of $G$ is a subset $U \subseteq V(G)$ such that, for every edge $uv \in E(G)$, at least one of $u,v$ belongs to $U$.
The size of a minimum vertex cover of a graph is denote by $\tau(G)$, and can be computed in $\bigO{1.2738^{\tau(G)} + \tau(G)|V(G)|}$ time.

A graph is a \textit{cluster graph} if all of its connected components are cliques.
Analogously, a graph is a \textit{co-cluster graph} if its complement is a cluster graph.
The \textit{distance to cluster} (resp. \textit{co-cluster}) of a graph $G$, denoted by $\dc(G)$ (resp. $\dcc(G)$), is the size of the smallest subset of vertices $U$ of $G$ such that $G - U$ is a (co-)cluster graph.
These parameters can be computed in $\bigO{1.92^{\dc( G )}n^2 }$
time and $\bigO{1.92^{\dcc( G )}n^2 }$ time, respectively~\citep{branching-cluster}.
It is quite easy, however, to obtain a 3-approximation for them in polynomial time, it suffices to note that a graph is a cluster graph if and only if it is $P_3$-free, so while there is some $P_3$ in the graph, it is sufficient to remove all three vertices.
The above values are examples of \tdef{structural graph parameters}.
Determining certain parameters of a generic graph $G$ is efficient (such as $\Delta(G)$ and $\delta(G)$); however, others (such as $\omega(G)$ and $\alpha(G)$) are widely believed to be hard to ascertain.

A graph $G$ is \tdef{bipartite} if $V(G) \sim \{X, Y\}$ such that both $X$ and $Y$ are independent sets.
Such property implies that a graph is bipartite if and only if it is $C_{2k+1}$-free, for any $k \geq 1$.
A \tdef{biclique} $K_{n_1,n_2}$ is a bipartite graph with $|X| = n_1$, $|Y| = n_2$ and $uv \in E(G)$ for every pair $u \in X$ and $v \in Y$.
A \tdef{star} is a biclique with $|X| = 1$ and $|Y| \geq 1$.
Clearly, we can also define \tdef{induced bicliques} and \tdef{induced stars} much like induced cliques.
A graph is \textit{multipartite} if $V(G) \sim \{X_1, \dots,  X_p\}$ and $X_i$ is an independent set for all $i$; it is a \textit{complete multipartite} graph if $uv \in E(G)$ whenever $u \in X_i$, $v \in X_j$ and $i \neq j$.

A \tdef{hypergraph} $\hypergraph = (V, \mathcal{E})$ is a natural generalization of a graph.
That is, $V(\hypergraph)$ is its vertex set and $\mathcal{E} \subseteq 2^{V}$ its \tdef{hyperedge} set~\citep{hypergraphs}.
A graph $G$ is said to be a \tdef{host} of $\mathcal{H}$ if $V(G) = V(\hypergraph)$, every hyperedge of $\hypergraph$ induces a connected subgraph of $G$ and every edge of $G$ is contained in at least one hyperedge of $\mathcal{H}$.
A hypergraph is \textit{$k$-uniform} if all of its hyperedges have the same size.

A \tdef{transversal} of a hypergraph $\hypergraph$ is a set $X \subseteq V(\hypergraph)$ such that, for every hyperedge $\varepsilon \in \mathcal{E}(\hypergraph)$, $X \cap \varepsilon \neq \emptyset$.
If $X$ is not a transversal we say that it is an \tdef{oblique}.

The \tdef{clique hypergraph} $\Hyper{C}(G)$ of a graph $G$ is the hypergraph on the same vertex set of $G$ and with hyperedge set equal to the family of maximal cliques of $G$.
Similarly, the \tdef{biclique hypergraph} $\Hyper{B}(G)$ of a graph $G$ is the hypergraph on the same vertex set of $G$ and with hyperedge set equal to the family of maximal bicliques of $G$.

A \textit{tree decomposition} of a graph $G$ is defined as the pair $\mathbb{T} = \left(T, \mathcal{B} = \{B_j \mid j \in V(T)\}\right)$, where $T$ is a tree and $\mathcal{B} \subseteq 2^{V(G)}$ is a family satisfying $\bigcup_{B_j \in \mathcal{B}} B_j = V(G)$~\citep{treewidth};
for every edge $uv \in E(G)$ there is some~$B_j$ such that $\{u,v\} \subseteq B_j$;
for every $i,j,q \in V(T)$, if $q$ is in the path between $i$ and $j$ in $T$, then $B_i \cap B_j \subseteq B_q$.
Each $B_j \in \mathcal{B}$ is called a \emph{bag} of the tree decomposition.
The \emph{width} of a tree decomposition is defined as the size of a largest bag minus one.
The \emph{treewidth} $\tw(G)$ of a graph $G$ is the smallest width among all valid tree decompositions of $G$~\citep{downey_fellows}.
If $\mathbb{T}$ is a rooted tree, by $G_x$ we will denote the subgraph of $G$ induced by the vertices contained in any bag that belongs to the subtree of $\mathbb{T}$ rooted at bag $x$.
An algorithmically useful property of tree decompositions is the existence of a so called \emph{nice tree decompositions} of width $\tw(G)$.

\begin{class_definition*}[Nice tree decomposition]
    A tree decomposition $\mathbb{T}$ of $G$ is said to be \emph{nice} if it is a tree rooted at, say, the empty bag $r(T)$ and each of its bags is from one of the following four types:
    \begin{enumerate}
        \item \emph{Leaf node}: a leaf $x$ of $\mathbb{T}$ with $B_x = \emptyset$.
        \item \emph{Introduce node}: an inner bag $x$ of $\mathbb{T}$ with one child $y$ such that $B_x \setminus B_y = \{u\}$.
        \item \emph{Forget node}: an inner bag $x$ of $\mathbb{T}$ with one child $y$ such that $B_y \setminus B_x = \{u\}$.
        \item \emph{Join node}: an inner bag $x$ of $\mathbb{T}$ with two children $y,z$ such that $B_x = B_y = B_z$.
    \end{enumerate}
\end{class_definition*}

\section{Parameterized Complexity}

We discuss problems in different complexity classes; in particular, we work with the usual classes $\P$, $\NP$,and the \tdef{polynomial hierarchy}~\citep{polynomial_hierarchy}.
We say that an algorithm is \tdef{efficient} if its running time is bounded by a polynomial on the size of the input and that a problem belonging to $\NPH$ is most likely intractable.
As such, our complexity results will be given either by efficient algorithms or polynomial reductions from $\NPH$ problems.

Besides the classic complexity classes, we also deal with the field of \tdef{parameterized complexity} (or \tdef{multivariate complexity}).
In this area, algorithms are designed and analyzed not only with respect to the size of the input object, but also with other \tdef{parameters} of the input.
Parameters come in all sorts of flavors.
Many decision problems usually have some integer quantity representing a constraint of the problem.
For instance, \pname{Vertex Cover} -- one of the classical examples of success of parameterized complexity ~~ asks for a set of size at most $k$ of vertices covering all the edges of the graph.
Other parameters are less problem specific and relate to the structure of the graph, such as diameter or maximum degree.
The most prominent of these examples, however, is the graph parameter treewidth, which played a pivotal role in the theory of graph minors.
Other previously discussed structural parameter is neighborhood diversity, distance to cluster and distance to co~cluster.

A problem is said to be \tdef{fixed-parameter tractable} (or \FPT) when \tdef{parameterized by} $k$ if there is an algorithm with running time $f(k)n^{\bigO{1}}$, where $n$ is the size of the input object.
We denote complexities of this form by $\bigOs{f(k)}$.
In fact, we shall use $\bigOs{\cdot}$ to omit polynomial factors of the running time; that is, an algorithm with complexity $2^{f(n)}\text{poly}(n)$ is said to execute in $\bigOs{2^{f(n)}}$.
In a slight abuse of notation, $k$ is simultaneously the parameter we are working with and the value of such parameter.
An instance of a parameterized algorithm is, therefore, the pair $(x,k)$, with $x$ the input object and $k$ as previously defined.
The class of all problems that admit an FPT algorithm is the class $\FPT$.
If an algorithm has running time $\bigO{n^{f(k)}}$, for some computable function on $k$,  we say it is an \XP (slicewise polynomial) algorithm, and the corresponding problem it solves is in $\XP$. 

Much like classical univariate theory, some problems do not appear to admit an \FPT algorithm for certain parameterizations. In particular, its widely believed that finding a clique of size $k$ in a graph, parameterized by $k$, is not in $\FPT$.
In an analogue to the classical case, hardness results are usually given by what are called \tdef{parameterized reductions}.

\begin{class_definition*}[Parameterized Reduction]
    A parameterized reduction from problem $\Pi$ to problem $\Pi'$ is a transformation from an instance $(x, k)$ of $\Pi$ to an instance $(x',k')$ of $\Pi'$ such that:
    \begin{enumerate}
        \item There is a solution to $(x,k)$ if and only if there is a solution to $(x',k')$;
        \item $k' \leq g(k)$ for some computable function $g$;
        \item The transformation's running time is $\bigOs{f(k)}$.
    \end{enumerate}
\end{class_definition*}

Note that the constraints imposed by parameterized reductions are quite similar to those imposed by polynomial reductions.
We ask that $k'$ does not depend on $|x|$ - which doesn't always happen with polynomial reductions - but, at the same time, allow $\FPT$ time for the transformation, instead of the more restrictive polynomial time.
These differences imply that polynomial reductions and parameterized reductions are incomparable, with some rare cases where the transformation is both polynomial and parameterized.

Unlike the theory of $\NPcness$, where most hard problems are equivalent to each other under polynomial reductions, in parameterized complexity problems seem to be distributed along a hierarchy of difficulty.
Before handling the classes themselves, we must first define the problems of parameterized complexity that play the same role as \pname{Satisfability} for the classical theory. 

The \tdef{depth} of a circuit is the length (in terms of number of gates) of the longest path from any one variable to the output.
The \tdef{weft} of a circuit is the the maximum number of gates with more than 2 input variables in any path from any one variable to the circuit's output.
The circuits with weft $t$ and depth $d$, denoted by \pname{wcs}$_{t,d}$, will be the fundamental problems of the $t$-th level of our hierarchy.

\pproblem{weighted circuit satisfability of weft $t$ and depth $d$ (\pname{wcs}$_{t,d}$)}{A Boolean circuit $C$ with $n$ variables, weft $t$ and depth $d$.}{A positive integer $k$.}{Is $C$ satisfiable with exactly $k$ variables set to $\TRUEt$?}

\begin{class_definition*}[$\W$-hierarchy]
    For $t \geq 1$, a parameterized problem $\Pi$ is in $\W[t]$ if there is a parameterized reduction from $\tsc{wsc}_{t,d}$ to it, for some $d \geq 1$. Moreover,
    \begin{equation*}
        \FPT \subseteq \W[1] \subseteq \W[2] \subseteq \cdots \subset \XP
    \end{equation*}
\end{class_definition*}


\subsection{Kernelization}

One of the broadest class of techniques to be found in the realm of computing is perhaps that of \textit{pre-processing}.
Every real system, in one way or another, employs routines that try to prune the search space or reduce the input instance as much as possible before doing any heavy lifting.
Such is the case with most optimization suites, such as CPLEX and Gurobi, where dozens upon dozens of pre-processing methods are readily available and in many cases successfully eliminate large chunks of the input before trying to solve the integer program directly.
Furthermore, in many cases, simple heuristics or algorithms with terrible worst case running times perform surprisingly well, and, in many cases, there was no theoretically sound approach to explain this phenomenon.
This lack of work on the subject is explained by the fact that, if an instance of a $\NPH$ problem can be reduced in polynomial time to one of bounded size, then $\P = \NP$~\citep{book-kernels}.
With the advent of parameterized algorithms, however, the situation is changing drastically.
Using this framework it has become possible to derive upper on lower bounds on the sizes of the instances obtained after a set of pre-processing rules have been applied.
We define the notions of kernels and kernelization below.

\begin{class_definition*}[Kernelization]
    A  \textit{kernelization algorithm} is an algorithm that takes as input an instance $(x,k)$ of a parameterized problem $\Pi$ and its output is an equivalent instance $(x',k')$ of $\Pi$ such that $|x'| \leq f(k)$ and $k' \leq g(k)$, for some pair of computable functions $f,g$; instance $(x',k')$ is called the \textit{kernel} of $(x,k)$.
\end{class_definition*}

It is not hard to see that a parameterized problem is in \FPT\ if and only if it admits a kernel.
Not all kernels are equal, and a natural desire is for the best (i.e. smallest) possible kernel.
The size of the kernel is measured by the dependency of the kernel on the parameter -- that is, a kernel that satisfies $|x'| \leq 4k$ is much better than a kernel with $|x'| \leq k^2$.
If this dependency is linear, we say that we have a \textit{linear kernel}, if $f(k)$ is a quadratic function, than the kernel is \textit{quadratic}, and so on.
Let $k$ be the natural parameter for the following problems.
Some famous examples of problems and their kernels include: \pname{Vertex Cover}, which admits a kernel of size $2k$; \pname{Max 3-Satisfability}, which has a kernel on $6k$ variables and $2k$ clauses; \pname{Independent Set} on planar graphs, with a kernel of size $4(k - 1)$;
meanwhile, for \pname{Dominating Set} on graphs of girth at least five, there is a cubic kernel, but no known subcubic one.

The bound on the instance size, however, can be exponential.
For instance, \pname{Matching Cut} parameterized by the number of edges crossing the cut does not have a polynomial kernel~\citep{matching_cut_ipec}.
For some time, there were no techniques to prove that a parameterized problem does \textit{not} admit a polynomial kernel; this changed, however with the seminal work of~\cite{BodlaenderDFH09}, where the composition and distillation techniques were first discussed, being was further deepened
by~\cite{weak_composition} and \cite{BodlaenderJK14}, where weak and cross-compositions were described.
All of these techniques, however, make use of some well established hypothesis about classical complexity classes.
For instance, distillation is based on assumption that the polynomial hierarchy does not collapse to the third level; weak-composition and cross-composition rely on the hypothesis that $\NP \subseteq \coNP/\poly$.
Despite appearing strong assumptions, if either of these hypotheses fail the implications would reverberate through much of theoretical computer science, and not only parameterized complexity.

For further reading and other more insightful discussions on the subjects of parameterized complexity and kernelization, we point to~\citep{downey_fellows},~\citep{cygan_parameterized}, and~\citep{book-kernels} from where most of the given definitions come from. 

\section{Explicit Running Time Lower Bounds}

Both the theory of \NPcness\ and \W[1]-\Hness\ give us evidence that no polynomial or \FPT\ algorithm may exist for a myriad of problems.
However, simply assuming that $\P \neq \NP$ or that $\FPT \neq \W[1]$ seems to not be enough to prove statements about asymptotic lower bounds on the running time of an algorithm.
All is not lost, but we do need to make some additional complexity assumptions.

In their groundbreaking work, \cite{eth} give many key insights and tools which have been broadly used across the field of algorithms and parameterized complexity to prove that long known algorithms are probably optimal.
Specifically, they prove what is known as the \textit{Sparsification Lemma}, described below.
A logical formula $\phi$ on $n$ variables an $m$ clauses is in \textit{Conjunctive Normal Form} (CNF) if $\phi = \bigwedge_{i=1}^m C_i$ and every $C_i$ is a disjunction of a subset of the $2n$ possible literals.
A formula is said to be in $r$-CNF if the size of each clause is no larger than $r$.

\begin{class_definition*}[Sparsification Lemma]
    For every $\epsilon > 0$ and positive
    integer $r$, there is a constant $C = \bigO{(\frac{n}{\epsilon})^{3r}}$ so that any $r$-CNF formula $F$ with $n$ variables, can be expressed as $F = \bigvee_{i=1}^t Y_i$, where $t \leq 2^{\epsilon n}$ and each $Y_i$ is an $r$-CNF formula with every variables appearing in at most $C$ clauses.
    Moreover, this disjunction can be computed by an algorithm running in time $2^{\epsilon n}n^{\bigO{1}}$.
\end{class_definition*}

Essentially, the Sparsification Lemma implies that, when performing a polynomial reduction \pname{$r$-Satisfability}, for \textit{fixed $r$}, it suffices to assume the input instance on $n$ variables has $\bigO{n}$ \textit{clauses}.
Impagliazzo and Paturi then conjecture a cornerstone of lower bound asymptotic analysis, the \textit{Exponential Time Hypothesis}, commonly referred to as \ETH, and its strong version, known as \SETH.

\begin{class_definition*}[Exponential Time Hypothesis]
    There is a real number $s$ such that \pname{3-Satisfability} cannot be solved in $2^{sn}(n + m)^{\bigO{1}}$ time.
\end{class_definition*}


\begin{class_definition*}[Strong Exponential Time Hypothesis]
    \pname{Satisfability} cannot be solved in $(2 - \epsilon)^{n}(n + m)^{\bigO{1}}$ time, for any $\epsilon > 0$.
\end{class_definition*}


It is not hard to see that if \ETH\ holds, then $\P \neq \NP$.
From the moment they were first claimed, both hypothesis have been successfully applied across the literature.
\cite{eth_survey} survey some of these results.
For instance, unless the Exponential Time Hypothesis is false, there is no algorithm running in $2^{o(n)}$ time for \pname{Vertex 3-Coloring}, \pname{Dominating Set}, \pname{Independent Set}, \pname{Vertex Cover}, nor \pname{Hamiltonian Path};
\pname{Hamiltonian Cycle} in planar graphs cannot be solved in $2^{o(\sqrt{n})}n^{\bigO{1}}$ time.
Let $k$ denote the natural parameter of each of the following problems.
In terms of $\FPT$ algorithms, the existence of $2^{o(k)}n^{\bigO{1}}$ was ruled out for \pname{Vertex Cover}, \pname{Feedback Vertex Set}, and \pname{Longest Path}, while no $2^{o(\sqrt{k})}n^{\bigO{1}}$ time algorithm exists for \pname{Vertex Cover} on planar graphs.
\ETH\ can also be used to give algorithmic lower bound to problems not in $\FPT$.
\cite{eth_survey} neither \pname{Dominating Set}, \pname{Clique}, \pname{Independent Set}, nor their multicolored versions can be solved in $f(k)n^{o(k)}$.

While most of the complexity theory community believes \ETH\ to be true, the same is not true for the Strong Exponential time Hypothesis~\citep{bad_seth}.
The implications for \SETH, however, as the name suggests, are quite powerful.
While \ETH\ is generally used to prove assertions on the exponent of the running times of many algorithms, \SETH\ allows for a \textit{much} finer-grained analysis, at the cost of much more complex reductions and arguments, specially because the hypothesis of the Sparsification Lemma \textit{are not respected by \pname{Satisfability}}.
\cite{seth_lokshtanov} give a series of reductions for many problems parameterized by treewidth.
They show that the best known algorithms parameterized by treewidth for \pname{Independent Set}, \pname{Dominating Set}, \pname{Max Cut}, \pname{Odd Cycle Transversal}, \pname{Vertex $q$-Coloring} (for any $q \geq 3$), \pname{Partition Into Triangles} cannot be improved, unless \SETH\ is false.
Recently, \cite{seth_subset_sum} proved what may surely be considered a breakthrough result: by using a hypothesis on the running time of \pname{Satsifability} (\SETH), they proved that the pseudo-polynomial dynamic programming algorithm given by \cite{bellman_dp} for \pname{Subset Sum} is optimal.


\section{Intersection Graphs}
\label{sec:intersections}

The \tdef{intersection graph} of a multifamily $\mathcal{F} \subseteq 2^S$, denoted by $G = \Omega(\mathcal{F})$ is the graph of order $|\mathcal{F}|$ and, for every $F_u, F_v \in \mathcal{F}$, $uv \in E(G) \Leftrightarrow F_u \cap F_v \neq \emptyset$.
Any $\mathcal{F}$ such that $\Omega(\mathcal{F}) \simeq G$ is a \tdef{set representation} of $G$.
A known theorem states that every graph is the intersection graph of a family of subgraphs of a graph~\citep{intersection_graphs}.

An \tdef{edge clique cover} $\mathcal{Q} = \{Q_1, \dots, Q_n\}$ of a graph $G$ is a (multi)family of cliques of $G$ such that every edge of $G$ is contained in at least one element of $\mathcal{Q}$.
The \tdef{dual edge clique cover} of a set representation $\mathcal{F} = \{F_1, \dots, F_n\}$, with $|\bigcup_{i \leq n} F_i| = m$, is defined as $\mathcal{Q}(\mathcal{F}) = \{Q_1, \dots, Q_m\}$ such that $Q_j = \{i \mid j \in F_i\}$.
The \tdef{dual set representation} of an edge clique cover $\mathcal{Q} = \{Q_1, \dots, Q_m\}$, with $|\bigcup_{j \leq m} Q_j| = n$, is $\mathcal{F}(\mathcal{Q}) = \{F_1, \dots, F_n\}$ with $F_i = \{j \mid i \in Q_j\}$.

Some interesting intersection graphs are usually defined in terms of the intersection of structures of other graphs. For instance, \tdef{line graphs} are precisely the graphs that are the intersection graphs of the edges of a graph; \tdef{clique graphs} are the intersection graphs of the maximal induced cliques of a graph.
Both of these classes, however have nice characterizations in terms of edge clique covers, which are commonly called \tdef{Krausz-type characterizations}.

\begin{class_definition*}[Line Graph]
    $G$ is a line graph if and only if there is an edge clique cover $\mathcal{Q}$ of $G$ such that both conditions hold:
        \begin{itemize}
            \item[(i)] Every vertex of $G$ appears in exactly two members of $\mathcal{Q}$;
            \item[(ii)] Every edge of $G$ is in only one member of $\mathcal{Q}$.
        \end{itemize}
\end{class_definition*}

\begin{class_definition*}[Clique Graph]
    $G$ is a clique graph if and only if it there is an edge clique cover of $G$ satisfying the Helly property.
\end{class_definition*}

\begin{figure}[!htb]
    \centering
        \begin{tikzpicture}
            \begin{scope}[rotate=-90,scale=0.5]
                \def\x{-2}
                \GraphInit[vstyle=Normal]
                \SetVertexNormal[Shape=circle, FillColor=black, LineWidth=1pt,MinSize=2pt,]
                \tikzset{VertexStyle/.append style = {inner sep = \inners, outer sep = \outers}}
                \Vertices[Lpos=270,Ldist=3pt,LabelOut=TRUE,unit=3]{circle}{a,b,c,d}
                \Edge(a)(b)
                \Edge(a)(d)
                \Edge(c)(b)
                \Edge(d)(b)
                \Edge(c)(d)
            \end{scope}
        \end{tikzpicture}
    \hfill
        \begin{tikzpicture}
            \begin{scope}[rotate=-90,scale=0.5]
                \def\x{-2}
                \GraphInit[vstyle=Normal]
                \SetVertexNormal[Shape=circle, FillColor=black, LineWidth=1pt,MinSize=2pt,]
            \tikzset{VertexStyle/.append style = {inner sep = \inners, outer sep = \outers}}
                \Vertices[Lpos=270,Ldist=3pt,LabelOut=TRUE,unit=3]{circle}{abd,bcd}
                \Edge(abd)(bcd)
            \end{scope}
        \end{tikzpicture}
    \hfill
        \begin{tikzpicture}
            \begin{scope}[rotate=315,scale=0.5]
                \def\x{-2}
                \GraphInit[vstyle=Normal]
                \SetVertexNormal[Shape=circle, FillColor=black, LineWidth=1pt,MinSize=2pt,]
            \tikzset{VertexStyle/.append style = {inner sep = \inners, outer sep = \outers}}
                \Vertices[Lpos=315,Ldist=3pt,LabelOut=TRUE,unit=3]{circle}{ab,bc,cd,ad}
                \Vertex[Lpos=270,Ldist=3pt,LabelOut=TRUE,unit=3,x=0,y=0]{bd}
                \Edge(ab)(bc)
                \Edge(ab)(ad)
                \Edge(bc)(cd)
                \Edge(cd)(ad)
                \Edge(bd)(ab)
                \Edge(bd)(ad)
                \Edge(bd)(cd)
                \Edge(bd)(bc)
            \end{scope}
        \end{tikzpicture}
    \hfill
        \begin{tikzpicture}
            \begin{scope}[rotate=90,scale=0.5]
                \GraphInit[vstyle=Normal]
                \SetVertexNormal[Shape=circle, FillColor=black, LineWidth=1pt,MinSize=2pt,]
            \tikzset{VertexStyle/.append style = {inner sep = \inners, outer sep = \outers}}
                \Vertices[Lpos=90,Ldist=3pt,LabelOut=TRUE,unit=3]{circle}{abc,dbc,bd}
                \Edges(abc,dbc,bd,abc)
            \end{scope}
        \end{tikzpicture}
    \hfill
    \caption{A graph, its clique graph, its line graph, and its star graph}
    %  Some graph
    \label{fig:my_label}
\end{figure}

The recognition of line graphs is known to be efficient~\citep{line_dynamic,line_nich,line_naor}.
For clique graphs, however, the situation was not so simple, and the complexity of clique graph recognition was left open for several years, finally being proven to be $\NPc$ by~\cite{clique_recognition} with a quite complicated argument.

Aside from the complexity point of view, many different properties of intersection graphs have been investigated in the literature.
For instance, clique-critical graphs -- graphs whose clique graph is different from the clique graph of all of its proper induced subgraphs -- have different characterizations~\citep{clique_critical_toft} and bounds~\citep{clique_critical_alcon} which were crucial in the proof of the complexity of the recognition problem.
Another common line of investigation on intersection graphs is the behaviour of iterated applications of the operators.
For instance, \cite{clique_iterated}, and \cite{clique_divergent} study iterated applications of the clique operator.
Biclique graphs -- the intersection graph of the maximal induced complete bipartite graphs of a graph -- were first characterized and studied by \cite{biclique_graph}.
Their results, however, are not very useful from the algorithmic point of view, and appear to not yield many insights on the recognition problem.
Nevertheless, they study the behaviour of biclique graphs, showing that every edge is contained either in a diamond or a 3-fan and specialize their general characterization for biclique graphs of bipartite graphs.
As was done for clique graphs, the iterated biclique operator has also been studied by Groshaus et al. in multiple papers~\citep{biclique_iterated, almost_all_biclique}, with results ranging from characterizations of divergence, divergence type verification algorithms, and other structural results.

Regarding stars, previous work handled the intersection graphs of (not necessarily maximal) substars of a tree~\citep{substar_graph} and of a star~\citep{starlike_graph}.
For the first, a minimal infinite family of forbidden induced subgraphs was given, while, for the latter, a series of characterizations were shown (including a finite family of forbidden induced subgraphs).
Stars are a particular case of bicliques, and both the biclique graph and star graph coincide for $C_4$-free graphs.
In fact, this relationship was successfully applied to determine the complexity of biclique coloring~\citep{biclique_coloring_complexity}, using a reduction from \textsc{qsat}$_2$ to star coloring (a coloring of the vertices of a graph such that no maximal star is monochromatic).
To the best of our knowledge, these are the main topics discussed in the literature that involve maximal stars in some way.
However, star graphs appear to be natural generalizations of square graphs~\cite{murty} in the sense that, when applying the squaring operation, for each vertex $v$ only the largest, non-induced star centered at $v$ is selected, and the intersection graph of these stars is generated.
On the other hand, for star graphs, every \textit{induced} maximal star is used in the construction of the intersection graph.
Despite the classes of star graphs and biclique graphs being equivalent when restricting the  pre-image domain to $C_4$-free graphs, we were unable to deepen the study biclique graphs; our efforts were hindered by some of the questions posed and developed upon in this work.


For other classical results in the area we point to~\citep{intersection_graphs}, from where most of the given definitions come from.

\section{Graph Classes}
\label{sec:graph_classes}

Most problems in graph theory can be tackled with an arbitrary input, that is, there is no particular property that we can exploit; this can happen if the considered application is too broad or little is known about its domain.
However, it might be possible to guarantee certain characteristics for the given graph, either due constraints of the application~\citep{fernando_chordal} or due to theoretical interest.
Regardless, such guarantees might be strong enough to provide an efficient algorithm to an otherwise $\NPH$ problem.
When constraining our analysis to certain graphs, we refer to the family of all graphs that satisfy the same properties as a \tdef{graph class}.
A subfamily of a class that satisfies additional properties is referred to as a \tdef{subclass}.
For (much) more on graph classes, \citep{classes_survey} give an extensive survey of much of the work done on the field until the late 1990s.

In this section, we review some of the most studied classes and some of their properties that will aid us in the design of our algorithms.

A graph is a \tdef{tree} $T$ if it is a connected acyclic graph or, equivalently, the connected graph such that, between every pair of vertices $u$ and $v$, there exists a unique path.
The vertices of degree one of a tree are called its \tdef{leaves}, and all others are \tdef{inner} nodes. A \tdef{subtree} $T'$ of a tree $T$ is a connected subgraph which, clearly, must also be a tree.
A \tdef{rooted tree} $T_v$ is a tree with a special vertex $v$, called its \tdef{root}.
Rooted trees offer a straight forward ordering of the vertices of a tree and a nice way to decompose problems into smaller instances and combine their solutions.
A \tdef{rooted subtree} $T_u$ of $T_v$ is the subgraph of $T_v$ induced by $u$ and all vertices of $T_v$ whose path to $v$ passes through $u$.
The vertices in $T_u \setminus \{u\}$ are called the \tdef{descendants} of $u$ and its neighbours are its \tdef{children}.

A \tdef{forest} is a graph where every connected component is a tree.
Many problems which are usually quite hard for general graphs, or even some classes, usually have a straightforward answer for forests, either using a greedy strategy or a slightly more sofisticated dynamic programming idea.

\begin{figure}
    \centering
    % Define style for nodes
    \begin{tikzpicture}[scale=0.7]
    %\draw[help lines] (-5,-5) grid (5,5);
        \def\x{-2}
        \GraphInit[vstyle=Normal]
        \SetVertexNormal[Shape=circle, FillColor=black, MinSize=2pt]
        \tikzset{VertexStyle/.append style = {inner sep = \inners, outer sep = \outers}}
        \Vertex[x=0,y=1,Lpos=270,NoLabel=true]{a}
        \Vertex[x=-1,y=3,Lpos=90,NoLabel=true]{b}
        \Vertex[x=1,y=2,Lpos=90,NoLabel=true]{c}
        \Vertex[x=-3,y=2,Lpos=90,NoLabel=true]{d}
        \Vertex[x=-2,y=-1,Lpos=90,NoLabel=true]{e}
        \Vertex[x=1,y=-1,Lpos=90,NoLabel=true]{f}
        \Vertex[x=-4,y=-1,Lpos=90,NoLabel=true]{g}
        \Vertex[x=-2,y=-3,Lpos=90,NoLabel=true]{h}
        \Vertex[x=1,y=-3,Lpos=90,NoLabel=true]{i}
        \Vertex[x=3,y=-3,Lpos=90,NoLabel=true]{j}
        \Vertex[x=2.5,y=3,Lpos=90,NoLabel=true]{k}
        
        
        \Edge(a)(b)
        \Edge(a)(c)
        \Edge(k)(c)
        \Edge(a)(d)
        \Edge(a)(e)
        \Edge(a)(f)
        \Edge(e)(g)
        \Edge(e)(h)
        \Edge(f)(i)
        \Edge(j)(i)
        
        
    \end{tikzpicture}
    \caption{A tree.}
    %  Some graph
    \label{fig:some_tree}
\end{figure}


\tdef{Chordal graphs} have many nice properties that enable the computation of different graph parameters in polynomial time~\citep{golumbic}. A \tdef{perfect elimination ordering} of a graph $G$ is an ordering $v_1, \dots, v_n$ of its vertices such that for the graph $G[\{v_i, \dots, v_n\}]$, $v_i$ is a simplicial vertex. As the name implies, chordal graphs are exactly the graphs where every cycle of size at least 4 has at least one chord. 

\begin{class_definition*}[Chordal Graph]
    For a graph $G$, the following properties are equivalent:
    \begin{itemize}
        \item[(i)] $G$ is chordal;
        \item[(ii)] It is $C_k$-free, for any $k \geq 4$;
        \item[(iii)] Every minimal cutset is a clique;
        \item[(iv)] It is the intersection graph of subtrees of a tree;
        \item[(v)] There is a perfect elimination ordering of its vertices.
    \end{itemize}
\end{class_definition*}

\begin{figure}[!htb]
    \centering
    % Define style for nodes
    
        \begin{tikzpicture}
            \begin{scope}[rotate=90,scale=1]
                %\draw[help lines] (-4,-4) grid (4,4);
                \GraphInit[vstyle=Simple]
                \SetVertexSimple[Shape=circle, FillColor=black, MinSize=2pt]
                \tikzset{VertexStyle/.append style = {inner sep = \inners, outer sep = \outers}}
                \Vertex[x=0,y=2]{a}
                \Vertex[x=-1,y=3]{b}
                \Vertex[x=1,y=3]{c}
                \Vertex[x=0,y=4]{d}
                
                \Vertex[x=-1,y=1]{e}
                \Vertex[x=1,y=1]{f}
                
                \Vertex[x=-2,y=0]{h}
                \Vertex[x=0,y=0]{i}
                \Vertex[x=-1,y=-1]{j}
                
                \Vertex[x=2,y=2]{k}
                
                
                \Edge(a)(b)
                \Edge(a)(c)
                \Edge(a)(d)
                \Edge(b)(c)
                \Edge(b)(d)
                \Edge(d)(c)
                
                \Edge(c)(f)
                \Edge(c)(k)
                \Edge(k)(f)
                
                \Edge(a)(i)
                \Edge(a)(f)
                \Edge(i)(f)
                
                \Edge(e)(h)
                \Edge(e)(i)
                \Edge(e)(j)
                \Edge(h)(j)
                \Edge(i)(j)
            \end{scope}
        \end{tikzpicture}
    \hfill
        \begin{tikzpicture}
            \begin{scope}[rotate=90,scale=1]
                %\draw[help lines] (-4,-4) grid (4,4);
                \GraphInit[vstyle=Simple]
                \SetVertexSimple[Shape=circle, FillColor=black, MinSize=2pt]
                \tikzset{VertexStyle/.append style = {inner sep = \inners, outer sep = \outers}}
                
                \Vertex[x=-1.5,y=0]{c1}
                \Vertex[x=-0.5,y=0]{c2}
                \Vertex[x=0,y=1]{c3}
                \Vertex[x=0.5,y=2]{c4}
                \Vertex[x=0,y=3]{c5}
                \Vertex[x=1.5,y=2]{c6}
                
                \Edge(c1)(c2)
                \Edge(c2)(c3)
                \Edge(c3)(c4)
                \Edge(c4)(c5)
                \Edge(c4)(c6)
            \end{scope}
        \end{tikzpicture}
    \hfill
    \caption{A chordal graph and its clique tree.}
    %  Some graph
    \label{fig:some_chordal}
\end{figure}

For a chordal graph $G$, its \tdef{clique tree} is a tree $\mathcal{T}(G)$ such that: its vertex set, each of which is called a \tdef{bag}, is the set of maximal cliques of $G$, and, for every vertex $v$ of $G$, the set of bags which contains $v$ induces a subtree of $\mathcal{T}(G)$. It can be shown that such a tree satisfies property \textit{(iv)}.
For more on clique trees and other chordal graph properties please refer to~\citep{clique_tree}.

Not surprisingly, many subclasses of chordal graphs have also been studied, since even forests are chordal graphs.
A \tdef{block graph} is a chordal graph where every minimal cutset is a single vertex.
An \tdef{interval graph} is the intersection graph of a set of intervals over the real numbers.
A \tdef{split graph} is a graph whose vertex set can be partitioned into a clique and an independent set.

\tdef{Cographs} are the graphs $G$ such that either $G$ or its complement is disconnected. At first glance, such property may not seem very helpful to the algorithm designer, but it is equivalent to a very nice recursive definition, first given in~\citep{cographs}.
Given two graphs $G$ and $H$, we define their \tdef{disjoint union} as the graph $G \cup H$ with $V(G \cup H)= V(G) \cup V(H)$ and $E(G \cup H) = E(G) \cup E(H)$, and their \tdef{join} as the graph $G \otimes H$ with vertex set is $V(G \otimes H)= V(G) \cup V(H)$ and edge set $E(G \otimes H)= E(G) \cup E(H) \cup \{uv \mid u \in V(G), v \in V(H)\}$.

\begin{class_definition*}[Cograph]
    For a graph $G$, the following properties are equivalent:
    \begin{itemize}
        \item[(i)] $G$ is a cograph;
        \item[(ii)] $G$ is $P_4$-free;
        \item[(iii)] $G$ can be constructed from isolated vertices by successively applying disjoint union and join operations.
    \end{itemize}
\end{class_definition*}

\begin{figure}[!htb]
    \centering
    \begin{tikzpicture}[scale=0.7]
            \GraphInit[vstyle=Simple]
            \SetVertexSimple[Shape=circle, FillColor=black, MinSize=2pt]
            \tikzset{VertexStyle/.append style = {inner sep = \inners, outer sep = \outers}}
            \Vertices[unit=3]{circle}{a,b,c,d,e,f}
            \Vertex[x=0,y=0]{g}
            
            \Edge(b)(a)
            \Edge(c)(b)
            
            \Edge(e)(f)
            \Edge(e)(d)
            \Edge(f)(d)
            
            \Edge(g)(a)
            \Edge(g)(b)
            \Edge(g)(c)
            \Edge(g)(d)
            \Edge(g)(e)
            \Edge(g)(f)
        \end{tikzpicture}
    \caption{A cograph.}
    \label{fig:some_cograph}
\end{figure}

A graph $G$ is \tdef{distance-hereditary} if and only if for every $u,v$ and in every induced subgraph $H$ that preserves the connectivity between $u$ and $v$, $\dist_H(u,v) = \dist_G(u,v)$~\citep{distance_hereditary}.
A graph is \tdef{ptolemaic} if, for any $u,v,w,x$, $\dist(u,v) \dist(w,x) \leq \dist(u,w) \dist(v,x) + \dist(u,x) \dist(v,w)$~\citep{ptolemaic}.
Cographs and ptolemaic graphs are special cases of distance-hereditary graphs. In particular, ptolemaic graphs are exactly the graphs that are chordal and distance-hereditary.

\section{Coloring Problems}

A \tdef{k-coloring} of a graph $G$ is a $k$-partition $\varphi = \{\varphi_1, \dots,\varphi_k\}$ of $V(G)$.
Each $\varphi_i$ is a \tdef{color class} and $v \in V(G)$ is \tdef{colored} with color $i$ if and only if $v \in \varphi_i$.
In a slight abuse of notation, we use $\varphi(v)$ to denote the color of $v$ and, for $X \subseteq V(G)$, $\varphi(X) = \bigcup_{v \in X} \{\varphi(v)\}$.

\subsection{Proper Coloring}

A \tdef{proper k-coloring} of $G$ is a $k$-coloring such that each $\varphi_i$ is an independent set.
In the literature, proper coloring is usually referenced to as \tdef{vertex coloring}, a convention we also adopt.
If $G$ has a proper $k$-coloring we say that $G$ is \tdef{k-colorable}.
The smallest integer $k$ such that $G$ is $k$-colorable is called the \tdef{chromatic number} $\chi(G)$ of $G$.
The natural decision problem associated with vertex coloring simply asks whether or not a given graph is $k$-colorable.

\problem{vertex coloring}{A graph $G$ and a positive integer $k$.}{Is $G$ $k$-colorable?}


\begin{figure}[!htb]
    \centering
    \begin{tikzpicture}[rotate=90,scale=0.6]
            %\draw[help lines] (-5,-5) grid (5,5);
            \GraphInit[vstyle=Simple]
            \SetVertexSimple[Shape=circle, FillColor=black, MinSize=2pt]
            \tikzset{VertexStyle/.append style = {inner sep = \inners, outer sep = \outers}}
            \SetUpVertex[Ldist=3pt]
            \grWheel[prefix=a]{6}
            \AssignVertexLabel{a}{1,2,1,2,3,4}
        \end{tikzpicture}
    \caption{An optimal proper coloring.}
    \label{fig:prop_color}
\end{figure}

Determining if a given instance of \textsc{vertex coloring} is a $\YES$ instance is a classic problem in both graph theory and algorithmic complexity, being a known $\NPc$ problem.
Some particular cases of \textsc{vertex coloring} are still $\NPc$.
For instance, even if we fix $k=3$ or restrict the input to $K_3$-free graphs the problem does not get any easier.

It is worth to point out the subtle difference between the parameter $k$ being part of the input or being \tdef{fixed}.
Informally, when $k$ is fixed, we are willing to pay exponential time only on $k$ to solve our problem, whereas when $k$ is part of the input, we are not.
Note that when we fix $k$ and find an $f(k)n^{\bigO{1}}$ time algorithm, we show that the problem is in $\FPT$ when parameterized by $k$.
The fact that $3$-coloring is $\NPc$ is evidence that \textsc{vertex coloring} parameterized by the number of colors is not in $\FPT$, otherwise we would have an $f(3)n^{\bigO{1}}$ algorithm, which would imply that $\P = \NP$. 
For the remainder of this work, when dealing with fixed $k$, we will include the fixed parameter in the name of the problem, such as in \textsc{vertex $k$-coloring} or \textsc{weighted circuit satisfability of weft $t$ and depth $d$}.

For an unconstrained input, \textsc{vertex coloring} is hard to approximate to a factor of $n^{1-\epsilon}$, for any $\epsilon > 0$, unless some complexity hypothesis fail (see~\citep{color_zpp} for more on the topic).
On a brighter note, a celebrated theorem due to Brooks in~\citep{brooks_theorem} gives a nice upper bound for general graphs, and gives a natural direction for research on tighter upper bounds on graph classes.

\begin{theorem*}[Brooks' Theorem]
    For every connected graph $G$ which is neither complete nor an odd-cycle $\chi(G) \leq \Delta(G)$.
\end{theorem*}

These results motivated much of the research about \textsc{vertex coloring}.
There are polynomial time algorithms for a myriad of different classes, including chordal, bipartite and cographs.
More generally, there are known polynomial time algorithms for \tdef{perfect graphs}~\citep{perfect_polynomial}, which is a superclass of the aforementioned ones.
$G$ is perfect if for every induced subgraph $G'$ of $G$, $\chi(G') = \omega(G')$.

More particular cases for \textsc{vertex coloring} have also been analyzed. For instance, \citep{coloring_art} presents some results for graph classes that have two connected five-vertex forbidden induced subgraphs. There are some surveys on the subject, as such we point to~\citep{coloring_survey} and \citep{coloring_survey2} for more on the classic \textsc{vertex coloring} problem, since it is not the focus of this thesis.


\subsection{Equitable Coloring}
A $k$-coloring of an $n$ vertex graph is said to be \tdef{equitable} if for every color class $\varphi_i$, $\floor{\frac{n}{k}} \leq |\varphi_i| \leq \ceil{\frac{n}{k}}$ or, equivalently, if for, any two color class $\varphi_i$ and $\varphi_j$, $||\varphi_i| - |\varphi_j|| \leq 1$.
If $G$ admits a proper equitable $k$-coloring, we say that $G$ is \tdef{equitably k-colorable}.
Unlike other coloring variants previously discussed, an equitably $k$-colorable graph is not necessarily equitably $(k+1)$-colorable.

As such, two different parameters are defined: the smallest integer $k$ such that $G$ is equitably $k$-colorable is called the \tdef{equitable chromatic number} $\CN{=}(G)$; the smallest integer $k'$ such that $G$ is equitably $k$-colorable for every $k \geq k'$ is the \textit{equitable chromatic threshold} $\CN{=}^*(G)$ of $G$.

As with the previous coloring problems, we define the \textsc{equitable coloring} decision problem.

\problem{equitable coloring}{A graph $G$ and a positive integer $k$.}{Is $G$ equitably $k$-colorable?}


\begin{figure}[!htb]
    \centering
    \begin{tikzpicture}
        \begin{scope}[rotate=90,scale=0.6]
            %\draw[help lines] (-5,-5) grid (5,5);
            \GraphInit[vstyle=Normal]
            \SetVertexNormal[Shape=circle]
            \tikzset{VertexStyle/.append style = {inner sep = \inners, outer sep = \outers}}
            \SetVertexNoLabel
            \grStar[prefix=a]{6}
            \AssignVertexLabel{a}{2,2,2,2,2,1}
        \end{scope}
    \end{tikzpicture}
    \hfill
    \begin{tikzpicture}
        \begin{scope}[rotate=90,scale=0.6]
            %\draw[help lines] (-5,-5) grid (5,5);
            \GraphInit[vstyle=Normal]
            \SetVertexNormal[Shape=circle]
            \tikzset{VertexStyle/.append style = {inner sep = \inners, outer sep = \outers}}
            \SetVertexNoLabel
            \grStar[prefix=b]{6}
            \AssignVertexLabel{b}{3,3,4,2,2,1}
        \end{scope}
    \end{tikzpicture}
    \hfill
    
    \caption{A proper non-equitable coloring (left) and an equitable coloring (right).}
    \label{fig:eq_color}
\end{figure}


\textsc{equitable coloring} was first discussed by~\citep{first_equitable}, with an intended application for municipal garbage collection, and later in processor task scheduling~\citep{mutual_exclusion_scheduling} and server load balancing~\citep{domain_decomposition}.

Much of the work done over \textsc{equitable coloring} aims to prove an analogue of Brooks' theorem, known as the \tdef{Equitable coloring conjecture} (ECC).
In terms of the equitable chromatic threshold, however, we have the \tdef{Hajnal-Szemerédi theorem}~\citep{hajnal_szmeredi_theorem}.

\begin{conjecture*}[ECC]
    For every connected graph $G$ which is neither a complete graph nor an odd-hole, $\CN{=}(G) \leq \Delta(G)$.
\end{conjecture*}

\begin{theorem*}[Hajnal-Szmerédi Theorem]
    Any graph $G$ is equitably $k$-colorable if $k \geq \Delta(G) + 1$. Equivalently, $\CN{=}^*(G) \leq \Delta(G) + 1$.
\end{theorem*}

\citep{e_delta_cc} suggest that a stronger result than the Hajnal-Szmerédi theorem may be achievable, presenting some classes where the \tdef{Equitable $\Delta$-coloring conjecture} (E$\Delta$CC) holds.
Moreover, they prove that if E$\Delta$CC holds for every regular graph, then it holds for every graph.

\begin{conjecture*}[E$\Delta$CC]
    For every connected graph $G$ which is not a complete graph, an odd-hole nor $K_{2n+1, 2n+1}$, for any $n \geq 1$, $\CN{=}^*(G) \leq \Delta(G)$ holds.
\end{conjecture*}

Quite a lot of effort was put into finding classes where E$\Delta$CC holds, even with the knowledge that only proofs for regular graphs are required.
A result given by~\citep{claw_free_de_werra}, combined with Brooks' Theorem, implies that every claw-free graph is equitably $k$-colorable for every $k \geq \chi(G)$.
Unfortunately, we were unable to verify whether this result is algorithmic and, in the affirmative, the algorithm's complexity.
However, due to the broadness of the class, we believe that, even if it exists, the algorithm has a higher complexity than the ones we describe.

A very extensive survey on the subject was conducted by~\citep{equitable_survey}, where many of the results of the past 50 years were assembled.
Among the many reported results, the E$\Delta$CC is known to hold for:
bipartite graphs (with the obvious exceptions, where the ECC holds),
trees,
split graphs,
planar graphs,
\tdef{outerplanar graphs} (planar graphs with a drawing such that no vertex is within a polygon formed by other vertices),
\tdef{low degeneracy graphs} (graphs such that every subgraph has a vertex with a small degree),
\tdef{Kneser graphs} (complement of the intersection graph of $F \subset 2^{[n]}$, with every set of $F$ containing exactly $k$ elements),
interval graphs,
random graphs and
some forms of graph products.
For the exact results please refer to the survey.

Almost all complexity results for \textsc{equitable coloring} arise from a related problem, known as \textsc{bounded coloring}, an observation given by~\citep{equitable_treewidth}.
A $k$-coloring is said to be \tdef{$l$-bounded} if for every color class $\varphi_i$, $|\varphi_i| \leq l$.
$G$ is \tdef{l-bounded k-colorable} if it admits an $l$-bounded $k$-coloring.

\problem{bounded coloring}{A graph $G$ and two positive integers $l$ and $k$.}{Is $G$ $l$-bounded $k$-colorable?}

\begin{observation*}
    A Graph $G$ with $n$ vertices is $l$-bounded $k$-colorable if and only if $G' = G \cup \overline{K_{lk - n}}$ is equitably $k$-colorable.
\end{observation*}


In terms of computational complexity, however, neither problem was nearly as explored as \textsc{vertex coloring}.
Among the complexity results for \textsc{bounded coloring} and, consequently, \textsc{equitable coloring}, we have polynomial time solvability for split graphs~\citep{equitable_split}, complement of interval graphs~\citep{graph_partitioning1}, forests~\citep{mutual_exclusion_scheduling}, trees~\citep{equitable_trees} and complements of bipartite graphs~\citep{graph_partitioning1}.

For cographs, we have a polynomial time algorithm when $k$ is fixed, otherwise the problem is $\NPc$~\citep{graph_partitioning1}, a situation similar to that of bipartite and interval graphs~\citep{graph_partitioning1}.
A consequence of the hardness result for cographs is that \textsc{equitable coloring} is $\NPc$ for graphs of bounded cliquewidth.

On complements of \tdef{comparability graphs} (graphs representing a valid partial ordering) however, even if we fix $l$, \textsc{bounded coloring} is still $\NPc$~\citep{chain_antichain}.
\citep{colorful_treewidth} show that, when parameterized by treewidth, \textsc{equitable coloring} is $\W[1]\text{-}\Hard$.
Also in terms of treewidth, \citep{equitable_treewidth} give a polynomial time algorithm for graphs of bounded treewidth.
Note that for all of the mentioned classes, \textsc{vertex coloring} is polynomially solvable.

A summary of the known complexities is available in Table~\ref{tab:equitable_complexity}.


\begin{table}[!htb]
    \centering
    \begin{tabular}{l|l|l}
       \hline
       \hline
       Class               &  fixed $k$         & input $k$             \\
       \hline
       Trees                &  $\P$             & $\P$                  \\
       Forests              &  $\P$             & $\P$                  \\
       Bipartite            &  $\NPc$           & $\NPc$                \\
       Co-bipartite         &  $\P$             & $\P$                  \\
       Cographs             &  $\P$             & $\NPc$                \\
       Bounded Cliquewidth  &  ?                & $\NPc$                \\
       Bounded Treewidth    &  $\P$             & $\P$                  \\
       Chordal              &  $\P^*$           & $\NPc$                \\
       Block                &  $\P^*$           & $\NPc^*$              \\
       Split                &  $\P$             & $\P$                  \\
       Interval             &  $\P$             & $\NPc$                \\
       Co-interval          &  $\P$             & $\P$                  \\
       General case         &  $\NPc$           & $\NPc$                \\
       \hline
       \hline
    \end{tabular}
    \caption{Complexity results for \textsc{equitable coloring}. Entries marked with a * are results established in this work.}
    \label{tab:equitable_complexity}
\end{table}

\subsection{Clique Coloring}
A \tdef{k-clique-coloring} of $G$ is a $k$-coloring of $G$ such that no maximal clique of $G$ is entirely contained in a single color class.
We say that $G$ is \tdef{k-clique-colorable} if $G$ admits a $k$-clique-coloring.
The smallest integer $k$ such that $G$ is $k$-clique-colorable is called the \tdef{clique chromatic number} $\CN{C}(G)$ of $G$.
Much like \textsc{vertex coloring}, there is a natural decision problem associated with this coloring variant, which we refer to as \textsc{clique coloring}.


\problem{clique coloring}{A graph $G$ and a positive integer $k$.}{Is $G$ $k$-clique-colorable?}


\begin{figure}[!htb]
    \centering
    \begin{tikzpicture}[rotate=90,scale=0.6]
            %\draw[help lines] (-5,-5) grid (5,5);
            \GraphInit[vstyle=Simple]
            \SetVertexSimple[Shape=circle, FillColor=black, MinSize=2pt]
            \tikzset{VertexStyle/.append style = {inner sep = \inners, outer sep = \outers}}
            \SetUpVertex[Ldist=3pt]
            \grWheel[prefix=a]{6}
            \AssignVertexLabel{a}{1,1,1,1,1,2}
        \end{tikzpicture}
    \caption{An optimal clique coloring.}
    \label{fig:clique_color}
\end{figure}

Research on the topic is much more recent than what was done for \textsc{vertex coloring}, with the first papers appearing in the early 1990s~\citep{first_clique_color} and interest on the subject rising in the early 2000s.
Even when $k$ is fixed, \textsc{clique $k$-coloring} is known to be $\SiP{2}\text{-}\Complete$, as shown by~\citep{clique_coloring_complexity}, with an $\bigOs{2^n}$ algorithm being proposed by~\citep{clique_color_algorithm}.

As with \textsc{vertex coloring}, \textsc{clique $k$-coloring} has been studied when restricting the input graph to certain graph classes.
\citep{weakly_clique_color} investigated 2-clique-coloring in terms of \tdef{weakly chordal graphs} (graphs free of any hole or anti-hole with more than 4 vertices), giving a series of results for the general case ($\SiP{2}\text{-}\Complete$) and showing that, for some nested subclasses, there are $\NPc$ and $\P$ instances of the problem.
When dealing with \tdef{unichord-free} graphs (graphs that contain no induced cycle with a unique chord), the problem is solvable in polynomial time~\citep{unichord_coloring}.

\tdef{Circular-arc} graphs (intersection graphs of a set of arcs of a circle) are always 3-clique-colorable, with a polynomial time algorithm to determine if the input is 2-clique-colorable~\citep{clique_circular_arc}.
When the given graph is \tdef{odd-hole-free}, it is $\SiP{2}\text{-}\Complete$ to decide whether it is 2-clique-colorable or not~\citep{clique_oddhole}.
\cite{clique_coloring_few_p4} give a series of bounds on graphs that, in some sense, contain few $P_4$'s, showing that most them are either 2 or 3-clique-colorable.

For \tdef{planar graphs} (graphs that can be drawn on a plane with no crossing edges), \citep{clique_coloring_planar} show that they are 3-clique-colorable, and \citep{clique_color_perfect_np_complete} present a polynomial time algorithm to decide whether a planar graph is 2-clique-colorable or not.

Some of these classes are subclasses of perfect graphs, and a conjecture suggests that every perfect graph is 3-clique-colorable~\citep{maximal_clique_coloring}.
Also in terms of perfect graphs, it is $\NPc$ to decide whether a perfect graph is 2-clique-colorable (\cite{clique_color_perfect_np_complete}).
\citep{clique_oddhole} also give the observation that every \tdef{strongly perfect graph} (see~\citep{strongly_perfect}) is 2-clique-colorable, a superclass of both chordal graphs and cographs.
For a summary of the mentioned results, please refer to Table~\ref{tab:clique_color_complexity}.

\begin{table}[!htb]
    \centering
    \begin{tabular}{c|c|c}
        \hline
        \hline
        Class            & $\CN{C}$             & Complexity \\
        \hline
        Cograph          & $= 2$                & $\P$\\
        Chordal          & $= 2$                & $\P$\\
        Weakly Chordal   & $\leq 3^*$           & $\SiP{2}\text{-}\Complete^\dag$\\
        Unichord-free    & $\leq 3 $            & $\P$ \\
        Circular-arc     & $\leq 3$             & $\P^\dag$ \\
        Odd-hole-free    & $\leq 3^*$           & $\SiP{2}\text{-}\Complete^\dag$\\
        Few $P_4$'s      & $\leq 2$ or $\leq 3$ & $\P$\\
        Planar           & $\leq 3$             & $\P^\dag$\\
        Perfect          & $\leq 3^*$           & $\NPc^\dag$\\ 
        Strongly Perfect & $= 2$                & $\P$\\
        General case     & Unbounded            & $\SiP{2}\text{-}\Complete$\\
        \hline
        \hline
    \end{tabular}
    \caption{Complexity and bounds for \textsc{clique coloring}. Entries marked with a $*$ are conjectures. $\dag$ indicates results for 2-clique-colorability.}
    \label{tab:clique_color_complexity}
\end{table}

\subsection{Biclique Coloring}
A \tdef{k-biclique-coloring} of $G$ is a $k$-coloring of $G$ such that no maximal biclique of $G$ is entirely contained in a single color class.
We say that $G$ is \tdef{k-biclique-colorable} if $G$ admits a $k$-biclique-coloring.
The smallest integer $k$ such that $G$ is $k$-biclique-colorable is called the \tdef{biclique chromatic number} $\CN{B}(G)$ of $G$.
Much like \textsc{clique coloring}, there is a natural decision problem associated with this coloring variant, which we refer to as \textsc{biclique coloring}.

\problem{biclique coloring}{A graph $G$ and a positive integer $k$.}{Is $G$ $k$-biclique-colorable?}


\begin{figure}[!htb]
    \centering
    \begin{tikzpicture}[rotate=90,scale=0.6]
            %\draw[help lines] (-5,-5) grid (5,5);
            \GraphInit[vstyle=Simple]
            \SetVertexSimple[Shape=circle, FillColor=black, MinSize=2pt]
            \tikzset{VertexStyle/.append style = {inner sep = \inners, outer sep = \outers}}
            \SetUpVertex[Ldist=3pt]
            \grWheel[prefix=a]{6}
            \AssignVertexLabel{a}{1,2,1,2,3,3}
        \end{tikzpicture}
    \caption{An optimal biclique coloring.}
    \label{fig:biclique_color}
\end{figure}


\textsc{biclique coloring} is an even more recent research topic than \textsc{clique coloring}, with the first results being a $\SiP{2}\text{-}\Cness$ proof due to~\citep{biclique_coloring_complexity} and the confirmation that verifying a solution to the problem is a $\coNP\text{-}\Complete$ task (\cite{biclique_coloring_verification}).

In terms of complexity results, very little is known about \textsc{biclique coloring}.
For unichord-free graphs, \citep{unichord_coloring} give a polynomial time algorithm to compute $\CN{B}$ and show that the biclique chromatic number of unichord-free graphs is either equal to or one greater than the size of the largest true twin class.
\citep{biclique_coloring_verification} present a polynomial time algorithm for powers of cycles and powers of paths.
Finally, \citep{biclique_coloring_complexity} give complexity results for $H$-free graphs, for every $H$ on three vertices, being polynomial for $H \in \{K_3, P_3, \overline{P_3}\}$ and $\NPc$ for $\overline{K_3}$-free graphs;
moreover they show that the problem is $\NPc$ for diamond ($C_4$ plus one chord) free graphs and split graphs, and polynomial for \tdef{threshold graphs} (\{$2K_2,P_4,C_4$\}-free).
We summarize the presented results in Table~\ref{tab:biclique_color_complexity}.

\begin{table}[!htb]
    \centering
    \begin{tabular}{c|c|c}
        \hline
        \hline
        Class            & $\CN{B}$             & Complexity \\
        \hline
        Split            & Unbounded            & $\NPc$\\
        Threshold        & Unbounded            & $\P$\\
        Diamond-free     & Unbounded            & $\NPc$\\
        $C_n^r$          & $\leq 3$             & $\P$\\
        $P_n^r$          & $= 2$                & $\P$\\
        Unichord-free    & Bounded              & $\P$\\
        General case     & Unbounded            & $\SiP{2}\text{-}\Complete$\\
        \hline
        \hline
    \end{tabular}
    \caption{Complexity and bounds for \textsc{biclique coloring}.}
    \label{tab:biclique_color_complexity}
\end{table}

Both \textsc{clique coloring} and \textsc{biclique coloring} are, actually, colorings of the hypergraphs arising from an underlying graph (a coloring of its vertices such that no hyperedge is monochromatic), which is also an $\NPc$ task.
However, in classical hypergraph coloring problems, the hyperedge family is part of the input of the problem and, as such, naively verifying a solution is polynomial on the size of the input.

\section{Cut Problems}

A \tdef{cut} of a graph $G = (V, E)$ is a bipartition of its vertex set $V(G)$ into two non-empty sets, denoted by $(A,B)$.
The set of all edges with one endpoint in $A$ and the other in $B$ is the \tdef{edge cut}, or the set of \tdef{crossing edges}, of $(A,B)$.
A \tdef{matching cut} is a (possibly empty) edge cut that is a matching, that is, such that its edges are pairwise vertex-disjoint. Equivalently, $(A, B)$ is a matching cut of $G$ if and only if every vertex is incident to at most one crossing edge of $(A, B)$~\citep{matching_cut_graham, chvatal_matching_cut}, that is, it has at most one neighbor across the cut.
Figure~\ref{fig:matching_cut} gives an example of a graph with a matching cut.


\begin{figure}[!htb]
        \centering
        \begin{tikzpicture}[rotate = 0]
                %\draw[help lines] (-5,-5) grid (5,5);
                \GraphInit[unit=3,vstyle=Normal]
                \SetVertexNormal[Shape=circle, FillColor = black, MinSize=3pt]
                \tikzset{VertexStyle/.append style = {inner sep = \inners, outer sep = \outers}}
                \SetVertexNoLabel
                
                \begin{scope}[rotate=90, shift={(0, 2)}]
                    \tikzset{VertexStyle/.append style = {shape = rectangle, inner sep = 2pt}}
                    \grComplete[RA=1]{2}
                \end{scope}
                \begin{scope}[rotate = 90, shift={(0, -1)}]
                    \grComplete[RA=1,prefix=b]{2}
                \end{scope}
                \begin{scope}
                    \tikzset{VertexStyle/.append style = {shape = rectangle, inner sep = 2pt}}
                    \Vertex[x=-1, y=0]{f}
                \end{scope}
                \Vertex[x=0, y=0]{g}
                \Edges(g,b0,a0,f,a1,b1,g,f)
        \end{tikzpicture}
        \caption{Example of a matching cut. Square vertices would be assigned to $A$, circles to $B$.\label{fig:matching_cut}}
    \end{figure}


Motivated by an open question posed by \cite{matching_cut_ipec} during the presentation of their article,  we investigate a natural generalization that arises from this alternative definition
For a positive integer $d \geq 1$, a \tdef{$d$-cut} is a a cut $(A, B)$ such that each vertex has at most $d$ neighbors across the partition, that is, every vertex in $A$ has at most $d$ neighbors in $B$, and vice-versa. Note that a $1$-cut is a matching cut.
As expected, not every graph admits a $d$-cut, and the \pname{$d$-Cut} problem is the problem of, for a fixed integer $d \geq 1$, deciding whether or not an input graph $G$ has a $d$-cut.

When $d=1$, we refer to the problem as \pname{Matching Cut}.
Graphs with no matching cut first appeared in Graham's manuscript~\cite{matching_cut_graham} under the name of \textit{indecomposable graphs}, presenting some examples and properties of decomposable and indecomposable graphs, leaving their recognition as an open problem.
In answer to Graham's question, Chv\'atal~\cite{chvatal_matching_cut} proved that the problem is \NP-hard for graphs of maximum degree at least four and polynomially solvable for graphs of maximum degree at most three; in fact, as shown by Moshi~\cite{matching_cut_moshi}, every graph of maximum degree three and at least eight vertices has a matching cut.

Chvátal's results spurred a lot of research on the complexity of the problem~\citep{matching_cut_ipec,matching_cut_structural,matching_cut_tcs, matching_cut_diameter, matching_cut_planar, matching_cut_series_parallel, stable_cutset_line_graphs}.
In particular, \cite{matching_cut_planar} showed that \pname{Matching Cut} remains \NP-hard for planar graphs of maximum degree four and for planar graphs of girth five;
\cite{stable_cutset_line_graphs} gave an \NP-hardness reduction for bipartite graphs of maximum degree four;
\cite{matching_cut_diameter} proved that \pname{Matching Cut} is \NP-hard for graphs of diameter at least three, and presented a polynomial-time algorithm for graphs of diameter at most two.
Beyond planar graphs, Bonsma's work~\citep{matching_cut_planar} also proves that the matching cut property is expressible in monadic second order logic and, by Courcelle's Theorem~\citep{courcelle_theorem}, it follows that \pname{Matching Cut} is $\FPT$ when parameterized by the treewidth of the input graph; he concludes with a proof that the problem admits a polynomial-time algorithm for graphs of bounded cliquewidth.

\cite{matching_cut_tcs} noted that Chv\'atal's original reduction also shows that, unless the Exponential Time Hypothesis fails, there is no algorithm solving \pname{Matching Cut} in time $2^{o(n)}$ on $n$-vertex input graphs.
Also in~\citep{matching_cut_tcs}, the authors provide a first branching algorithm, running in time $\bigOs{2^{n/2}}$, a single-exponential $\FPT$ algorithm when parameterized by the vertex cover number $\tau(G)$, and an algorithm generalizing the polynomial cases of line graphs~\citep{matching_cut_moshi} and claw-free graphs~\citep{matching_cut_planar}.
\cite{matching_cut_tcs} also asked for the existence a single-exponential algorithm parameterized by treewidth.
In response, \cite{matching_cut_structural} provided a $\bigOs{12^{\tw(G)}}$ algorithm for \pname{Matching Cut} using nice tree decompositions, along with $\FPT$ algorithms for other structural parameters, namely neighborhood diversity, twin-cover, and distance to split graph.

The natural parameter -- the number of edges crossing the cut -- has also been considered.
Indeed, \cite{marx_treewidth_reduction} tackled the \pname{Stable Cutset} problem, to which \textsc{Matching Cut} can be easily reduced via the line graph, and through a breakthrough technique showed that this problem is $\FPT$ when parameterized by the maximum size of the stable cutset.
Recently, \cite{matching_cut_ipec} improved on the results of \cite{matching_cut_tcs}, providing an exact exponential algorithm for \textsc{Matching Cut} running in  time $\bigOs{1.3803^n}$, as well as $\FPT$ algorithms parameterized by the distance to a cluster graph and the distance to a co-cluster graph, which improve the algorithm parameterized by the vertex cover number, since both parameters are easily seen to be smaller than the vertex cover number.
For the distance to cluster parameter, they also presented a quadratic kernel; while for a combination of treewidth, maximum degree, and number of crossing edges, they showed that no polynomial kernel exists unless $\NP \subseteq \coNP/\poly$.

A problem  closely related to \pname{$d$-Cut} is that of \pname{Internal Partition}, first studied by \cite{internal_partition_thomassen}.
In this problem, we seek a bipartition of the vertices of an input graph such that every vertex has at least as many neighbors in its
own part as in the other part. Such a partition is called an \tdef{internal partition}.
Usually, the problem is posed in a more general form: given functions $a,b: V(G) \rightarrow \mathbb{Z}_+$, we seek a bipartition $(A,B)$ of $V(G)$ such that every $v \in A$ satisfies $\dgr_A(v) \geq a(v)$ and every $u \in B$ satisfies $\dgr_B(u) \geq b(u)$, where $\dgr_A(v)$ denotes the number of neighbors of $v$ in the set $A$. Such a partition is called an \tdef{$(a,b)$-internal partition}.

Originally, Thomassen asked in~\citep{internal_partition_thomassen} whether for any pair of positive integers $s,t$, a graph $G$ with $\delta(G) \geq s + t + 1$ has a vertex bipartition $(A,B)$ with $\delta(G[A]) \geq s$ and $\delta(G[B]) \geq t$.
\cite{internal_partition_stiebitz} answered that, in fact, for any graph $G$ and any pair of functions $a,b: V(G) \rightarrow \mathbb{Z}_+$ satisfying $\dgr(v) \geq a(v) + b(v) + 1$ for every $v \in V(G)$, $G$ has an $(a,b)$-internal partition.
Following Stiebitz's work, \cite{internal_partition_triangle_free} showed that if $G$ is triangle-free, then the pair $a,b$ only needs to satisfy $\dgr(v) \geq a(v) + b(v)$.
More recently, \cite{internal_partition_c4_free} proved that, if $G$ is $\{C_4, K_4, \text{diamond}\}$-free, then $\dgr(v) \geq a(v) + b(v) - 1$ is enough.
Furthermore, they also showed, for any pair $a,b$, a family of graphs such that $\dgr(v) \geq a(v) + b(v) - 2$ for every $v \in V(G)$ that do not admit an $(a,b)$-internal partition.

It is conjectured that, for every positive integer $r$, there exists some constant $n_r$ for which every $r$-regular graph with more than $n_r$ vertices has an internal partition~\citep{DeVos09,internal_partition_regular6} (the conjecture for $r$ even appeared first in~\citep{internal_partition_regular3_4}).
The cases $r \in \{3,4\}$ have been settled by \cite{internal_partition_regular3_4}; the case $r=6$ has been verified by \cite{internal_partition_regular6}.
This latter result implies that every 6-regular graph of sufficiently large size has a 3-cut.