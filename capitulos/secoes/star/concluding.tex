\section{Concluding Remarks}

This chapter introduced the class of star graphs -- the intersection graphs of the induced maximal stars of some graph.
We presented various results, beginning with a quadratic bound on the size of minimal pre-images.
Then, a Krausz-type characterization for the class was presented, which yielded membership of the recognition problem in \textsf{NP}.
We also presented a series of properties the members of the class must satisfy, such as being biconnected, that every edge must belong to some triangle, and present a bound on the diameter of the star graph based on the diameter of the pre-image, which implies that the diameter of the iterated star graph converges to either three or four;  when restricting the analysis to star-critical pre-images, we completely characterize the structures that the pre-image must have in order for the star graph to have a degree two vertex.
We present a monotonicity theorem for star-critical pre-images, which allowed us to compute all 267 star graphs on at most eight vertices, as well as all 978612 star-critical pre-images of these graphs.

We leave two main open questions.
The first, the complexity of the recognition problem, is perhaps the most challenging; for example, the complexity of the clique graph recognition problem was left open for many decades, only being settled recently~\citep{clique_recognition} through a series of non-intuitive gadgets.
The second is the optimality of the quadratic bound, or if it can be improved.
A complete characterization of both critical and non-critical vertices seems the biggest obstacle to obtain a linear bound on the size of critical pre-images; all our attempts to solve this task were thwarted by the amount of cases the analysis usually boils down to.
Despite our special interests on these questions, many different directions are available for investigation, such as on iterated applications of the star operator or relationships with other classes.